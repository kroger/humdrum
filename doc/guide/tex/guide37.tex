% This file was converted from HTML to LaTeX with
% Tomasz Wegrzanowski's <maniek@beer.com> gnuhtml2latex program
% Version : 0.1
\documentclass{article}
\begin{document}



  
  
    
      
      
      
    
  



\\
\\

\section*{Chapter37}


[\textit{Previous Chapter}]
[\textit{Contents}]
[\textit{Next Chapter}]


\section*{Electronic Music Editing}



The preparation of encoded musical materials for processing
is an essential part of computer-assisted musicology.
Without electronic data to process, none of the
manipulations described in this book would be possible.
The encoding and editing of musical data has been
an activity that has occupied various scholars since the
earliest forays into computer-assisted musicology (e.g., Bronson, 1959).

\par 
In this chapter we identify and describe some of the tasks
and issues involved in preparing electronic musical
documents in the Humdrum format.
In general, the following discussion pertains to the
production of electronic documents representing musical
score information.
However, the basic procedures are applicable to
any kind of data -$\,$- from sound recordings to
historical choreographies of ballets.


\subsection*{The Process of Electronic Editing}

\par 
Electronic editing entails translating one representation
into another representation.
In many cases, an electronic edition is prepared from
existing source documents -$\,$- such as printed scores or
manuscripts.
In other cases, an electronic edition may be prepared
from another electronic form -$\,$- such as MIDI.
When translating between two formats, it is important to be
familiar with both representations.
Descriptions of a number of popular electronic music formats
may be found in Eleanor Selfridge-Field's encyclopedic
\textit{Handbook of Musical Codes}
(1997).

\par 
The principal processes involved in electronic editing
include identifying possible sources and alternative versions,
selecting and encoding the material,
proofing the material (using both automated and manual methods),
adding reference and editorial information,
writing appropriate research notes,
generating possible analytic information,
and resolving issues related to copyright, distribution, and data integrity.


\subsection*{Establishing the Goal}

\par 
Encoding and editing music is a time-consuming and
labor-intensive process.
Before starting to work on a electronic edition,
it is appropriate to ask: what is the goal?
What purpose will be served by the edition?
Is the objective to provide a performance copy?
A historically accurate document?
A document suitable for analysis?
If so, what sort of analysis?

\par 
In general, it is wise to avoid trying to cater to all
needs at once.
There is nothing wrong with creating a narrowly-defined
document that serves a modest goal.
For example, if our intention is to pursue a study
of rhythm, there is little need to encode data that is
extraneous to this goal.
Accordingly we might omit stem-direction, beaming,
textual underlay, and even pitch information.

\par 
Note that it is always possible to incorporate additional
information at a later date.
Additional information can be inserted into
a Humdrum encoding by using the
\textbf{assemble}
and
\textbf{cleave}
commands.
In general, it is important that the encoding of electronic
documents not consume all of a researcher's efforts and resources.
The most common problem that beset early projects in
computational musicology was that researchers rarely
got past the stage of inputting data.
Early researchers typically ran out of time, money or
enthusiasm before they could turn to
\textit{using}
the materials they had input.


\subsection*{Documenting Encoded Data}

\par 
More important than generating a complete and accurate database is
generating complete and accurate documentation so that future
users of the information understand the limitations of the materials.
Whether or not the electronic edition is "complete,"
the materials are almost useless without proper
documentation about what information is present,
what information has been omitted, what sources were used,
and what interpretations have been made.
As long as the electronic materials are well documented,
users can make effective use of whatever you create.

\par 
An ethnomusicologist may have inaccurately transcribed pitches,
but even approximate pitch information can be used for a study
of pitch contour.
Of course, it is better to have clean data than messay data,
but even messy data can be highly useful if the nature of the mess
is well understood by researchers.
In fact, most historical information is a mess.

\par 
Concretely, Humdrum provides several reference records that allow
electronic editors to identify what information has been encoded
and what information has been omitted or interpreted.


\subsection*{Sources}

\par 
In light of the editorial goal, you can procede to
select source materials for encoding.
In many cases, this will involve selecting printed musical scores.
Begin by spending time with the materials.
What special problems or challenges are raised?
What sort of Humdrum representation would be suitable
for the proposed edition?
Are there special notational symbols that suggest
a new representation should be designed?
Can special representational problems be accommodated
within an existing Humdrum representation?
How would users be made aware of extensions or
modifications to the representation?

\par 
In the case of musical scores, there is rarely a single
source or \textit{Urtext}.
In most cases, multiple sources are available.
How do you want to approach the problem of variant sources?
One approach is to select a single source and remain
true to it.
A second approach is to identify the principal variant sources
and encode each independently.
A third approach is to encode several variants within a
single document and provide Humdrum "versions"
that allow users to select particular "readings"
from a single file.
A fourth approach is to create your own "critical"
edition -$\,$- based on a close re-examination of the
original materials.

\par 
An important consideration will be the copyright status of
the materials you examine.
Some materials are old enough that their copyrights have
expired.
Encoding older materials raises two difficulties.
First, older sources are typically not the best quality
editions available.
Second, access to printed versions of older sources may be
difficult for other scholars using your data.
Some publishers (such as Dover Publications) specialize
in reprinting musical scores whose copyrights have lapsed.
Although these sources may not be the best available,
they are often  convenient because other scholars can easily
purchase the modern Dover reprints that correspond
to your electronic edition.

\par 
The best materials are usually under copyright.
If you encode this material without permission,
you will never be able to distribute the encodings -$\,$-
even on a non-profit basis.
Most modern publishers profess an interest in electronic publishing,
but (at the time of this writing) few publishers have any pertinent
expertise, and fewer yet are willing to encourage or allow
the preparation of electronic documents
from their publication catalogue by third parties.
It is likely to take decades before the financial and
legal issues are satisfactorily addressed.

\par 
In some cases, you will be translating from one electronic
format to another -$\,$- such as from MIDI to
\texttt{**kern}.
Once again, be conscious of any copyright issues
that are raised.
Few experiences are more discouraging than discovering
that your work cannot be distributed because you
failed to consider seriously the copyright issues involved.


\subsection*{Selecting a Sample from Some Repertory}

\par 
Quite often, there are insufficient resources
to encode an entire corpus.
Many repertories are simply too large to consider
creating an exhaustive electronic edition.
Nevertheless, you might choose to encode a
subset or selection of works from the given corpus.

\par 
A common expectation is that studying a sample subset of works
will inform us about the corpus as a whole.
That is, it is commonly expected that studying a number of works
by, say, Offenbach, the scholar may be able to identify general
characteristics of Offenbach's writing.
In order for this assumption to be valid, statisticians tell us
that the sample of works must be a
\textit{representative}
sample -$\,$- free of all sorts of possible selection biases.

\par 
From a research perspective, the best sample is a
\textit{random sample.}
Suppose, for example, that there are 2,000 items in a
particular repertory, but you are able to encode
just 100 items.
You might be tempted to select the first 100 items,
or to select a subset of items that share some common feature
that makes the collection seem "coherent."

\par 
However, picking and choosing what to encode will prevent
researchers from being able to draw general conclusions
about the repertory as a whole.
For example, selecting the first 100 items will typically
bias the sample to early works in a composer's career.
Selecting 100 random items provides a very
good statistical sample of a population of 2,000 items.

\par 
Unfortunately, since random sampling has little precedence
in traditional humanities research, many scholars are resistent
to the idea.
The value of random sampling has been established beyond
a doubt by statisticians.
This is not the place to rehearse the detailed arguments.
Simply take my word for it:
if you can't encode a complete corpus,
the very best solution is to select a random sample.

\par 
In making such a random sample, it is essential to resist
the temptation to select a "random" sample "by eye."
Establish a truly random procedure (such as flipping coins
or using a random number table) and methodically follow
the procedure.

\par 
Incidentally, it is common to run out of resources
before completely encoding the selected materials.
As a result, you may end up encoding only half or
two-thirds of the projected materials.
If you began encoding the materials in (say)
chronological order, then the resulting database will
be biased toward the early works of the repertory.
In order to avoid introducing an unwanted bias,
it is also prudent to encode the selected materials in a
random order.


\subsection*{Encoding}

\par 
Once you have established your materials and
have decided on the type of encoding,
you can go ahead and begin encoding the documents in random order.
Use whatever resources are available to you.
These might include scanning software,
MIDI performance capture, or the Humdrum
\textbf{encode}
command.
Begin by encoding a sample section or sections.
Spend some time determining ways to increase your
productivity.

\par 
As you encode the selected materials, editorial problems or questions will inevitably arise.
As you gain experience, you may realize that earlier
encoding practices were not the best.
You may want to return to these problems and encode
them in a different manner.
Be sure to keep notes -$\,$- either pencil marks on
a page, or local comments in a file -$\,$- so that you
can easily revisit these problem sites later.
Again, it is valuable to encode works in random order in order to
avoid possible confounds arising from editorial experience.
That is, you don't want a scholar's conclusions about differences
between early works and late works to be merely an artifact
of the electronic editor's increasing experience.

\par 
Typically, it is more efficient to encode individual parts
and then assemble all parts into a single full score.


\subsection*{Transposing Instruments}

\par 
In the case of the \texttt{**kern} representation, all parts are
represented at concert pitch.
It is typically easier to encode the parts as written and then
transpose the result using the Humdrum
\textbf{trans}
command.
For example, material for B-flat trumpet or B-flat clarinet
can be transposed using the following command:

\par 

\texttt{trans -d -1 -c -2}


\par 
In the case of clarinet in A, a suitable transposition would be:

\par 

\texttt{trans -d -2 -c -3}


\par 
The
\textbf{trans}
command adds a transposition interpretation to the output in order
to identify that the material has been shifted.
In the \texttt{**kern} representation, transposed instruments must be explicitly
identified using a special "transposing-instrument interpretation"
(see \textit{Humdrum Reference Manual} -$\,$- Section 3 for details).
A suitable interpretation can be created by adding the
upper-case letter `I' prior to the `T' in the appropriate tandem interpretation.
In the case of a horn in F for example, the transpostion interpretation
would be modified from

\par 

\texttt{*Trd-4c-7}


\par 
to:

\par 

\texttt{*ITrd-4c-7}



\subsection*{Instrument Identification}

\par 
Humdrum provides standardized instrumentation indicators.
Three different types of indication are appropriate:
(1) the instrument name as indicated in the source,
(2) standardized instrument name, and (3) instrument class.
Standardized instrument names can be found in
\textbf{Appendix II}.
For example, the standard indicator for "harpsichord"
is
\texttt{*Icemba}.

\par 
Standardized instrument class designators include \texttt{*ICklav} for keyboard instruments
and \texttt{*ICidio} for percussion instruments, etc.,
and instrument grouping designators -$\,$- such as \texttt{*IGripn} for
\textit{ripieno}
instruments and \texttt{*IGacmp} for accompaniment instruments.
These instrument class designators can also be found in
Appendix II.

\par 
In addition, the original instrument name (as found in the score)
should also be encoded as a Humdrum local comment.


\subsection*{Leading Barlines}

\par 
Humdrum tools prefer to have explicit information indicating
the beginning of the first measure.
If a file does not begin with an anacrusis ("pickup")
then it is appropriate to encode an "invisible" first barline.
For a hypothetical file containing five spines,
we would need to insert the following line just before the first
note(s) in the work:

\texttt{=1-=1-=1-=1-=1-}


\par 
Recall that the common system for representing barlines makes a
distinction between the logical
\textit{function}
of a barline and it's visual or
\textit{orthographic}
appearance.
For example, the common system for barlines distinguishes between double
barlines whose function is to indicate the end of a work or movement,
and double barlines that simply delineate sections within the course
of a work or movement.
It is possible for a barline at the
end of the work to be "functionally" a double barline,
yet appear visually as a single barline.

\par 
\textit{Functional double barlines}
are encoded with a double equals sign (==) whether or not they
are visually rendered as double barlines.
\textit{Functional single barlines}
are encoded with a single equals sign (=) whether or not they
are visually rendered as single barlines.

\par 
The specific visual appearance may be encoded following the equals
sign(s).
The vertical line (|) represents a `thin' line and the exclamation
mark (!) represents a `thick' line.
A typical final double bar would be encoded:

\par 

\texttt{==|!}


\par 
Most mid-movement double bars are encoded with two thin lines
and so would be encoded:

\par 

\texttt{=||}


\par 
A common encoding error is to render mid-movement double
barlines as
\textit{functional}
rather than
\textit{orthographic}
double-bars.


\subsection*{Ornamentation}

\par 
The \texttt{**kern} representation makes a distinction between
whole-tone and semitone trills and mordents.
Typically, each ornament must be examined manually and the
correct code selected.

\par 
In some cases, the size of the trill or mordent will be ambiguous
and so some sort of editorial decision will be necessary.
One possibility is to add the kern `x' signifier immediately following the `T' or `t'.
This indicates that the trill size is an
"editorial interpretation."

\par 
The \texttt{**kern} representation treats appoggiaturas in a
special way.
In general, \texttt{**kern} is oriented to representing things
in a manner closer to how they sound.
Consequently, appoggiaturas are encoded as they would be
logically performed.
For example, a quarter-note preceded by an appoggiatura (small note)
would be performed as two eighth-notes.
Similarly, a dotted quarter-note preceded by an appoggiatura
would be performed as a quarter-note followed by an eighth-note.

\par 
All appoggiaturas must be re-encoded in a
way that reflects their likely performance.
At the same time, the two notes forming the appoggiatura
must be marked in the kern representation:
the initial note of the appoggiatura is marked by the upper-case
letter `\texttt{P}' and the final (second) note of the appoggiatura
is marked by a lower-case letter `\texttt{p}'.


\subsection*{Editing Sections}

\par 
It is helpful to break-up large works/movements into
smaller sections that can be labelled.
In a binary work, for example, it may be useful
to label the `A' and `B' sections.
In a sonata-allegro work, it may be useful to label the
introduction, exposition, development, recapitulation, etc.
Some works include explicitly notated labels.
These labels may be traditional, e.g. "Coda,"
or they may reflect programatic descriptions,
such as the section entitled
\textit{Il canto degl'uccelli}
[The song of the birds] in Vivaldi's
\textit{The Four Seasons.}

\par 
Where appropriate, suitable section labels should be created
and encoded using the Humdrum Section Label designator.
Remember that section labels can include the space character:

\par 

\texttt{*>1st Theme}


\par 
If you include section labels, you must also include a Humdrum
"Expansion List" to indicate how the sections are
connected.
The Humdrum
\textbf{thru}
command causes a through-composed version of a file to be
generated according to the expansion list.
For example, an expansion list for a simple binary work may be
encoded as:

\par 

\texttt{*>[A,B]}


\par 
Remember that expansion lists ought to be encoded prior
to the first section label.

\par 
Whenever a work/movement includes repeats or Da Capos,
section labels and expansion lists must be encoded.
In some cases, there is more than one way of interpreting
how to realize the repeats.
The most "conventional" realization should be encoded
with the
\textit{unnamed expansion list.}
This will specify the default expansion using the Humdrum
\textbf{thru}
command.
Suppose for example, that you are encoding a typical
minuet and trio.
The conventional performance practice involves repeating
all sections of both the minuet and trio, but then avoiding
the repeats in the minuet following the Da Capo.
A suitable expansion list might be:

\par 

\texttt{*>[Minuet,Minuet,Trio,Trio,,Minuet]}


\par 
An alternative expansion list might be encoded as follows
(notice the expansion-list-label \textit{ossia}):

\par 

\texttt{*>ossia[Minuet,Minuet,Trio,Trio,,Minuet,Minuet]}



\subsection*{Editorialisms in the \textit{**kern} Representation}

\par 
Humdrum provides several ways of encoding editorialisms.
These include editorial footnotes, local comments,
global comments, interpretation data,
\textit{sic}
and
\textit{ossia}
designations, version labels, sectional labels,
and expansion lists.

\par 
The
\texttt{**kern}
representation provides several
special-purpose signifiers to help make explicit various
classes of editorial
amendments, interpretations, or commentaries.
Five types of editorial signifiers are available:
(1) \textit{sic} (information is encoded literally, but is questionable)
signified by the \texttt{Y} character;
(2) \textit{invisible symbol} (Unprinted note, rest or barline,
but logically implied) signified by the \texttt{y} character;
(3) \textit{editorial interpretation}, (a "modest" editorial act
of interpretation -$\,$- such as the interpretation of accidentals in
\textit{musica ficta}) signified by the \texttt{x} character;
(4) \textit{editorial intervention} (a "significant" editorial
intervention) signified by the \texttt{X} character;
(5) \textit{footnote} (accompanying local or global comment provides
a text commentary pertaining to specified data token)
signified by \texttt{?}.

\par 
One of the most onerous impositions of the \texttt{**kern}
representation is the requirement that the music be
interpreted into a coherent spine organization.
Why not avoid interpreting the voicings?

\par 
The answer to this question is that editorial interventions
are often essential clarifications that make a document useable.
Without voicing information, users would be unable to calculate
melodic intervals, for example.
Without melodic intervals, it may be impossible to search
for themes, motives, and other patterns.
Editorial interpretations are not simply unwarranted obfuscations.
This does not mean that interpretations are "correct"
and so it may be necessary to provide several alternative or
plausible interpretations of an artifact.

\par 
One of the advantages of computers is that it is possible
for documents to undergo continuous revision.
In research, it is common for documents to be reinterpreted, annotated,
or recast in light of newly found documents.

\par 
The kern `\texttt{x}' signifies an "editorial interpretation"
-$\,$-  that the immediately preceding signifier is interpreted.
The kern `\texttt{xx}' also signifies an editorial interpretation
where the immediately preceding data token is interpreted.
The kern `\texttt{X}' signifies an "editorial intervention"
-$\,$-  that the immediately preceding signifier is an editorial
addition.
The kern `\texttt{XX}' also signifies an editorial intervention
where the immediately preceding data token is an editorial addition.
The kern `\texttt{y}' designates a invisible symbol -$\,$- such as
an unprinted note or rest that is logically implied.
The kern `\texttt{Y}' signifies an editorial
\textit{sic}
marking -$\,$- that the information is encoded literally,
but is questionable.
The kern `\texttt{?}' signifies an editorial footnote where the
immediately preceding signifier has an accompanying editorial
footnote (located in a comment record).
The kern `\texttt{??}' signifies an editorial footnote where
the immediately preceding data token has an accompanying
editorial footnote (located in a comment record).


\subsection*{Adding Reference Information}

\par 
Reference information must be added to each file.
This information provides "library-type" information
about the composer, date of composition, place of composition,
copyright notice, etc.

\par 
As many reference records should be added as possible since
these are immensely useful to Humdrum users.
Essential reference records include the following:

\texttt{!!!COM:}composer's name
\texttt{!!!CDT:}composer's dates
\texttt{!!!OTL:}title (in original language)
\texttt{!!!OMV:}movement number (if appropriate)
\texttt{!!!OPS:}opus number (if appropriate)
\texttt{!!!ODT:}date of composition
\texttt{!!!OPC:}place of composition
\texttt{!!!YEP:}publisher of electronic edition
\texttt{!!!YEC:}date \& owner of electronic copyright
\texttt{!!!YER:}date electronic edition released
\texttt{!!!YEM:}copyright message
\texttt{!!!YEN:}country of copyright
\texttt{!!!EED:}electronic editor
\texttt{!!!ENC:}encoder of document
\texttt{!!!EEV:}electronic edition version
\texttt{!!!EFL:}file number, e.g. 1 or 4 (1/4)
\texttt{!!!VTS:}checksum validation number (see below)
\texttt{!!!AMT:}metric classification
\texttt{!!!AIN:}instrumentation


\par 
Where appropriate, the following reference records should
also be included:

\texttt{!!!CNT:}composer's nationality
\texttt{!!!XEN:}title (English translation)
\texttt{!!!OPR:}title of larger (or parent) work
\texttt{!!!ODE:}dedication
\texttt{!!!OCY:}country of composition
\texttt{!!!PPR:}first publisher
\texttt{!!!PDT:}date first published
\texttt{!!!PPP:}place first published
\texttt{!!!SCT:}scholarly catalogue name \& number
\texttt{!!!SMA:}manuscript acknowledgement
\texttt{!!!AFR:}form of work
\texttt{!!!AGN:}genre of work
\texttt{!!!AST:}style of period

\par 
In general, place essential reference records at the beginning
of a document.
These will include the composer, title of the work, etc.
Less important reference records should be placed at the
end of the file.
Minimizing the number of reference records at the beginning
of a file makes it more convenient for users looking at
the contents of a Humdrum file.

\par 
Refer to the \textit{Humdrum Reference Manual} for
further information about the types and format for different
reference records.


\subsection*{Proof-reading Materials}

\par 
Once you have encoded your document, you should create a
error-checking strategy.
The Humdrum
\textbf{humdrum}
command can be used to identify whether the final encoded
output conforms to the Humdrum syntax:

\par 

\texttt{humdrum full.krn}


\par 
Use the Humdrum
\textbf{proof} \textbf{-k}
command to identify any syntactical errors in any encoded
\texttt{**kern} data:

\par 

\texttt{proof -k full.krn}



\par 
One of the best ways to ensure that musical data makes sense
is to listen to it.
The Humdrum
\textbf{midi}
and
\textbf{perform}
commands can be used to listen to your data.
The \textbf{-c} option for \textbf{midi} causes
the Humdrum data to be displayed while the
MIDI data is performed.
This can further help in locating errors.

\par 

\texttt{midi -c full.krn | perform}


\par 
The
\textbf{perform}
command allows you to
\textit{pause}
(press the space bar), to
\textit{move}
to a particular measure
(type a measure number followed by enter),
to increase (type <) or decrease (type >) the
\textit{tempo,}
and to
\textit{return}
to the beginning of the score
(type enter).
There are many other functions within the
\textbf{perform}
command; refer to the \textit{Humdrum Reference Manual} -$\,$- section 4
for further details.

\par 

\subsection*{Data Integrity Using the VTS Checksum Record}

\par 
When using electronic documents, it is often useful to
modify the document for some purpose.
After a while, the user will become confused about the
status of a document.
Is this the original distribution file?
Did I make some modification to this file
that I've forgotten about?
Has someone tampered with this data?

\par 
Humdrum provides a means for ensuring that a particular
file is what it purports to be.
The
\textbf{veritas}
command provides a formal means for verifying that a given
Humdrum file is identical to the original distribution file
and has not been modified in some way.

\par 
The
\textbf{veritas}
command works by looking for a VTS reference record
in the file.
It then calculates a "checksum" for the file
(excluding the VTS record itself) and compares this value
with the encoded VTS value.
If these values differ, a warning is issued that the file has
been modified in some way.

\par 
Once you are certain that an encoded Humdrum file is completely
finished, you should calculate a "checksum" value to be
encoded in a Humdrum "VTS" reference record.

\par 
In order to calculate the checksum value for a given file,
use the following command:

\par 



\par 
Open the original file and move to the bottom of the document.
Then read in the calculated checksum value.
Finally, insert the `\texttt{!!!VTS: }' reference record designator.

\par 
You can check that everything is fine by invoking the
\textbf{veritas}
command:

\par 



\par 
The command will complain only if the VTS checksum value
does not correspond to the computed checksum for the file.
Finally, be sure to include the checksum value in an index
or README file for the distribution.
This provides a public venue for users to determine
whether the VTS record itself has not been modified.


\subsection*{Preparing a Distribution}

\par 
Finally, you may want to prepare the material you have
encoded for public distribution.
Rename the score files and collect them into a coherent repertory.
If your data is encoded in the \texttt{**kern} format,
be sure to use the \texttt{.krn} file extension.
Place all resulting Humdrum files in a single directory.

\par 
Create a \texttt{README} file similar to others in Humdrum data
distributions.
The file should contain a repertory title,
a brief paragraph describing the historical background
for the works, a paragraph describing
the personnel involved in the production,
a copyright and license notice, and a table of contents.
Avoid tabs in this file, and ensure that no line is greater
than 80-characters in length.

\par 
It is wise to also add a \texttt{LICENSE} file that reiterates
whatever licensing agreement is entailed
for the distributed data.


\subsection*{Electronic Citation}

\par 
Electronic editions of music might be cited in printed or other
documents by including the following information.
The "author" (e.g. \textbf{!!!COM:}), the "title" -$\,$-
either original title (\textbf{!!!OTL:}) or translated title
(\textbf{!!!XEN:}).
The editor (\textbf{!!!EED:}), published (\textbf{!!!YEP:}),
date of publication and copyright owner (\textbf{!!!YED:}),
and electronic version (\textbf{EEV:}).
In addition, a full citation ought to include the
validation checksum (\textbf{!!!VTS:}).
The validation number will allow others to verify that
a particular electronic document is precisely the one cited.
A sample citation to an electronic document might be:

\par 

Franz Liszt, Hungarian Rhapsody No. 8 in F-sharp minor (solo piano).
Amsterdam: Rijkaard Software Publishers, 1994; H. Vorisek (Ed.),
Electronic edition version 2.1, checksum 891678772.




\subsection*{Reprise}

\par 
In this chapter we have reviewed the principal issues involved
in preparing electronic music documents in Humdrum.

\\
\begin{itemize}
\item 

\textbf{Next Chapter}
\item 

\textbf{Previous Chapter}
\item 

\textbf{Table of Contents}
\item 

\textbf{Detailed Contents}
\\\\

� Copyright 1999 David Huron
\end{document}
