% This file was converted from HTML to LaTeX with
% Tomasz Wegrzanowski's <maniek@beer.com> gnuhtml2latex program
% Version : 0.1
\documentclass{article}
\begin{document}



  
  
    
      
      
      
    
  



\\
\\

\section*{Chapter34}


[\textit{Previous Chapter}]
[\textit{Contents}]
[\textit{Next Chapter}]


\section*{Serial Processing}



Humdrum provides a handful of specialized tools for serial
and serial-inspired analytic processing.
In this chapter we introduce the
\textbf{reihe},
\textbf{pcset},
\textbf{iv},
\textbf{pf}
and
\textbf{nf}
commands
These commands reveal their greatest power when used in conjunction
with Humdrum tools we have already encountered -$\,$-
such as
\textbf{context},
\textbf{humsed}
and
\textbf{patt}.

\par 
The chapter culminates with a script that automatically identifies
12-tone row variants in complex orchestral scores.
The general approach is instructive for applications
beyond serial analysis.


\subsection*{Pitch-Class Representation}

\par 
In set theoretic applications it is common to use pitch-class
representations.
The
\textbf{pc}
command can be used to transform pitch-related representations
(such as
\texttt{**pitch},
\texttt{**freq}
and
\texttt{**kern})
to a conventional pitch-class
notation where pitch-class C is represented by the value zero.
With the
\textbf{-a}
option,
\textbf{pc}
will generate outputs where the pc values `10' and `11' are
rendered by the alphabetic characters `A' and `B' respectively.
Using the alpha-numeric pc representation is recommended;
it proves to be especially convenient for searching tasks since,
otherwise, the characters 1 and 0 do not uniquely specify a single pitch-class type.


\subsection*{The \textit{pcset} Command}

\par 
A common set theoretic task is identifying occurrences of various
pitch-class set forms.
Figure 34.1 identifies the set forms for several sample vertical sonorities.
These forms are identified using standard numerical designations
(see Forte, 1973; Rahn, 1980).
Set forms are insensitive to transposition, pitch-inversion, and pitch spelling
so all major and minor triads
are identified as pitch-class set 3-11.
Similarly, the dominant seventh chord and `Tristan' chord are similarly related
by inversion so both are identified as pc set 4-27.
\\\\
Figure 34.1.  Examples of PC set forms.



The
\textbf{pcset}
command identifies pitch-class sets from
\texttt{**pc}
or
\texttt{**semits}
input.
Illustrated below are the corresponding
\texttt{**kern},
\texttt{**pc}
and
\texttt{**pcset}
representations
for Example 34.1.

\texttt{**kern\texttt{**pc\texttt{**pcset
\texttt{4c\texttt{0\texttt{1-1
\texttt{4c 4d\texttt{0 2\texttt{2-2
\texttt{4c 4e 4g\texttt{0 4 7\texttt{3-11
\texttt{4e 4g 4cc\texttt{4 7 0\texttt{3-11
\texttt{4g 4cc 4ee\texttt{7 0 4\texttt{3-11
\texttt{4c 4e- 4g\texttt{0 3 7\texttt{3-11
\texttt{4c 4e 4g 4b-\texttt{0 4 7 10\texttt{4-27
\texttt{4c\# 4e\# 4g\# 4b\texttt{1 5 8 11\texttt{4-27
\texttt{4f 4b 4dd\# 4gg\#\texttt{5 11 3 8 \texttt{4-27
\texttt{*-\texttt{*-\texttt{*-}


\par 
Suppose we wanted to identify the pc sets for successive vertical
sonorities in the first movement of Webern's Opus 24 concerto.
First, we translate the input to a pitch-class representation,
and then we apply the
\textbf{pcset}
command:

\par 

\texttt{pc opus24 | pcset}


\par 
Of course this command will only identify the set forms for pitches
that have concurrent attacks.
If any pitch is sustained,
\textbf{pcset}
won't know that some null tokens indicate sustained pitch activity.
We can rectify this by using the
\textbf{ditto}
command
(Chapter 15)
to fill-out the null tokens:

\par 

\texttt{pc opus24 | ditto -s \^{}= | pcset}



\par 
If we wanted, we could assemble the resulting
\texttt{**pcset}
spine to the original input.
This would allow us to search for particular patterns that
are coordinated with certain pitch-class sets.
For example, we might be interested in comparing the pitch-class
sets that coincide with the beginnings of slurs/phrases
versus those pitch-class sets coinciding with the ends of slurs/phrases.
First we generate the \texttt{**pcset} spine:

\par 

\texttt{pc opus24 | ditto -s \^{}= | pcset > opus24.pcs}


\par 
Then we assemble this spine to the original input score:

\par 

\texttt{assemble opus24 opus24.pcs > opus24.all}


\par 
Now we can search for data records containing phrase (`\{\}') or slur ('()')
markers.
Using
\textbf{yank}
\textbf{-m ... -r 0}
rather than
\textbf{grep}
assures that the output retains the Humdrum syntax (see
Chapter 12).
Maintaining the Humdrum syntax will allow us to use
\textbf{extract}
to isolate just the \texttt{**pcset} data.
Finally, we create an inventory of the pc sets.
The process is repeated -$\,$- once for beginning slurs/phrases, and
once for ends of slurs/phrases.

\par 

\texttt{yank -m '[\{(]' -r 0 opus24.all | extract -i '**pcset' $\backslash$

| rid -GLId | sort | uniq -c}

\texttt{yank -m '[)\}]' -r 0 opus24.all | extract -i '**pcset' $\backslash$

| rid -GLId | sort | uniq -c}



\par 
Two pitch-class set inventories will be generated:
one inventory for the beginnings of phrases/slurs
and one for phrase/slur endings.


\par 
Incidentally, the
\textbf{pcset}
command supports a
\textbf{-c}
option that can be used to generate the set
\textit{complement}
rather than the principal set form.




\subsection*{Prime Form and Normal Form}



\par 
The 3-11 set form designates both the major and minor chords
(since they are symmetrical).
In order to distinguish symmetrical forms, it is sometimes
useful to represent pitch-class sets using either
\textit{prime form}
(the
\textbf{pf}
command)
or
\textit{normal form}
(the
\textbf{nf}
command).


\par 
Suppose we wanted to count the proportion of phrase endings in music
by Alban Berg where the phrase ends on either a major or minor chord.
First, we locate all works composed by Berg:

\par 

\texttt{BERG=`find /scores -type f -exec grep -l '!!!COM.*Berg,' "\{\}" ";"`}


\par 
Let's put all the Berg works in a single temporary file:

\par 

\texttt{cat \$BERG > AllBerg}


\par 
Next we generate the normal set forms:

\par 

\texttt{pc AllBerg | ditto -s \^{}= | nf > AllBerg.nf}


\par 
Assemble the
\texttt{**nf}
spine with the original scores:

\par 

\texttt{assemble AllBerg AllBerg.nf > AllBerg.all}


\par 
Now we're ready to count the number of phrases that match the
appropriate patterns.
First, count the total number of phrases:

\par 

\texttt{grep -c '\}' AllBerg.all}


\par 
Count the number of phrases that end with a major chord:

\par 

\texttt{grep -c '\}.*$\backslash$t(047)' AllBerg.all}


\par 
And count the number of phrases that end with a minor chord:

\par 

\texttt{grep -c '\}.*$\backslash$t(037)' AllBerg.all}



\subsection*{Interval Vectors Using the \textit{iv} Command}

\par 
Interval vectors identify the frequency of occurrence of various
interval-classes for a given pitch-class set.
The
\textbf{iv}
command generates the six-element interval vector for any of several
types of inputs -$\,$- including semitones
(\texttt{**semits}),
pitch-class
(\texttt{**pc}),
normal form
(\texttt{**nf}),
prime form
(\texttt{**pf}),
and pitch-class set
(\texttt{**pcset}).
The following example shows several different pitch-class sets,
their corresponding pitch-class sets and (right-most spine), the
associated interval vector.

\texttt{\texttt{**pc\texttt{**pcset\texttt{**name**iv
\texttt{0\texttt{1-1\texttt{tone\texttt{<000000>
\texttt{0 2\texttt{2-2\texttt{major second<010000>
\texttt{0 3 7\texttt{3-11\texttt{minor triad<001110>
\texttt{0 4 7\texttt{3-11\texttt{major triad<001110>
\texttt{0 4 7 10\texttt{4-27\texttt{dominant seventh<012111>
\texttt{1 5 8 11\texttt{4-27\texttt{dominant seventh<012111>
*-*-\texttt{*-\texttt{*-}



\par 
Suppose we wanted to determine whether Arnold Schoenberg tended
to use simultaneities that have more semitone (interval-class 1)
relations and fewer tritone (interval-class 6) relations.
As before, we might translate his scores to pitch-class notation,
fill-out the sonorities using \textbf{ditto}, and then determine the
associated interval vectors for each sonority.
Interval vectors without semitone relations will have a zero in
the first vector position (i.e., <0.....>) whereas interval vectors without
tritone relations will have a zero in the last position (i.e., <.....0>).

\par 

\texttt{pc schoenberg* | ditto -s \^{}= | iv | grep -c '<0.....>'}
\\
\texttt{pc schoenberg* | ditto -s \^{}= | iv | grep -c '<.....0>'}



\subsection*{Segmentation Using the \textit{context} Command}

\par 
So far, we have processed only "vertical" sets
of concurrent pitches.
In set-theory analyses, there are many other important ways
of "segmenting" the musical pitches into pitch-class sets.
As we saw in
Chapter 19,
the
\textbf{context}
command provides a useful way of grouping together successive
data tokens.




\par 
Suppose, for example, we wanted to analyze set forms in Claude Debussy's
\textit{Syrinx}
for solo flute.
The opening measures are shown in Example 34.1.
\\\\
\textbf{Example 34.1.}  From Claude Debussy, \textit{Syrinx} for flute.



There are a number of ways we might want to try segmenting the
melodic line.
One possibility is to regard slurs or phrases as indicating
appropriate groups.
Recall that the
\textbf{-b}
and
\textbf{-e}
options for
\textbf{context}
are used to specify regular expressions that match the
beginning and end (respectively) of the context group:
We can invoke an appropriate
\textbf{context}
command, translate the output to a pitch-class representation,
and then use the
\textbf{pcset}
command to identify the set names:

\par 

\texttt{context -b '[\{(]' -e '[\})]' syrinx | pc | pcset}


\par 
Perhaps we might consider gathering groups of three successive notes together,
and then generating an inventory of the set forms associated
with such a segmentation:

\par 

\texttt{context -n 3 -o '[=r]' syrinx | pc | pcset | rid -GLId $\backslash$

| sort | uniq -c}




\par 
Another possibility is to treat rests as segmentation boundaries.

\par 

\texttt{context -e r syrinx | pc | pcset}


\par 
When a work consists of more than one instrument or part,
useful segmentations can be made by extracting each instrument
individually, using
\textbf{context}
to generate musically-pertinent sets, and then assembling all
of the
\texttt{**pcset}
spines into a single file.


\subsection*{The \textit{reihe} Command}

\par 
Twelve-tone music raises additional analysis issues.
Variants of a tone-row can be generated using the
\textbf{reihe}
command.
Given some input,
\textbf{reihe}
will output a user-specified transformation.
Options are provided for prime transpositions (\textbf{-P} option),
for inversions (\textbf{-I} option), for retrogrades (\textbf{-R} option)
and for retrograde-inversions (\textbf{-RI} option).

\par 
Inputs do not have to be 12-tone rows.
The 5-tone row used in Igor Stravinsky's "Dirge-Canons" from
\textit{In Memoriam Dylan Thomas}
is as follows:

\texttt{**pc
\texttt{2
\texttt{3
\texttt{6
\texttt{5
\texttt{4
\texttt{*-}



\par 
The following command will generate a prime transposition of the
tone-row so that it begins on pitch-class 6:

\par 

\texttt{reihe -P 6 memoriam}


\par 
The result is:

\texttt{**pc
\texttt{6
\texttt{7
\texttt{10
\texttt{9
\texttt{8
\texttt{*-}



\par 
Generating the inversion beginning at pitch-class 2 would be carried
out using the following command.

\par 

\texttt{reihe -a -I 2 memoriam}


\par 
The
\textbf{-a}
option causes the values `10' and `11' to be rendered alphabetically
as `A' and `B'.

\par 
The
\textbf{reihe}
command also provides a
\textit{shift}
operation (\textbf{-S}) that is useful for shifting the serial order
of data tokens forward or backward.
Consider the following command:

\par 

\texttt{reihe -S -1 memoriam}


\par 
This shifts all of the data tokens back one position so the
data begins with the second value in the input, and the first
value is moved to the end:

\texttt{**pc
\texttt{3
\texttt{6
\texttt{5
\texttt{4
\texttt{2
\texttt{*-}



\par 
The shift option for
\textbf{reihe}
can be used to shift
\textit{any}
type of data -$\,$- not just pitches of pitch-classes.
For example, one might use the shift option to rotationally
permute dynamic markings, text, durations, articulation marks,
or any other type of Humdrum data.
In
Chapter 38
we will see how the shift option for
\textbf{reihe}
can be effectively used in many applications apart from
serial analysis.


\subsection*{Generating a Set Matrix}

\par 
The first step in automated row-finding is to generate a
set matrix of all the set variants.
Typically, the user begins with a hypothesized tone row.
Suppose the tone-row was stored in a file called
\texttt{primerow}.
From this we can generate the entire set matrix.
There is a Humdrum
\textbf{matrix}
command that automatically generates a set matrix,
but let's create our own script to see how this can be done.

\par 
The following script uses the
\textbf{reihe}
command to generate each set form.
Each form is stored in a separate file with names such
as \texttt{I8} and \texttt{RI3}.
There are two noteworthy features to this script.
Notice that the alphanumeric system (\textbf{-a} option) is used
-$\,$-  so the values `A' and `B' are used rather than `10' and `11';
this will facilitate searching.
Also notice that our script provides an option that allows us to specify
\textit{partial}
rows:
that is, we can store (say) only the first 5 notes in each tone
row file.
This feature will also prove useful when doing an automatic search.

\par 


\\
\texttt{\#\#\#\#\#\#\#\#\#\#\#\#\#\#\#\#\#\#\#\#\#\#\#\#\#\#\#\#\#\#\#\#\#\#\#\#\#\#\#\#\#\#\#\#\#\#\#\#\#\#\#\#\#\#\#\#\#\#\#\#\#\#\#\#\#\#\#
\\
\#                              MATRIX
\\
\# This script generates a tone-row matrix for a specified prime row.
\\
\# The -n option is used to specify the number of pitches to be out-
\\
\# put in each row-file (e.g. the first 7 pitches of a 12-tone row).
\\
\#
\\
\# Usage: matrix -n N primerowfile
\\
\#
\\
if [ "x\$1" != "x-n" ]
\\
then
\\
	echo "-n option must be specified."
\\
	echo "USAGE:   matrix -n number primerowfile"
\\
	exit
\\
fi
\\
if [ ! -f \$3 ]
\\
then
\\
	echo "File \$3 not found."
\\
	echo "USAGE:   matrix -n number row-file"
\\
	exit
\\
fi
\\
\# Generate the primes, inversions, retrograde, etc:
\\

\\
X=0
\\
while [ \$X -ne 12 ]
\\
do
\\
	reihe -a -P \$X \$3 | rid -GLId | head -\$2 > P\$X
\\
	reihe -a -I \$X \$3 | rid -GLId | head -\$2 > I\$X
\\
	reihe -a -R \$X \$3 | rid -GLId | head -\$2 > R\$X
\\
	reihe -a -RI \$X \$3 | rid -GLId | head -\$2 > RI\$X
\\
	let X=\$X+1
\\
done}


\par 
For any given input, the above script produces 48 short files
named P0, P1, ... I0, I1 ... R0, R1 ... RI10, RI11.


\subsection*{Locating and Identifying Tone-Rows}

\par 
Each of the row variant files can be used as a template for the
\textbf{patt}
command (see
Chapter 21).
The following "rowfind" script shows how the Humdrum tools can be
coordinated to carry out an automatic search and identification
of tone row variants for some score.

\par 
The first part of the script simply checks to ensure that
all of the row variant files are present:

\par 

\\
\texttt{\#\#\#\#\#\#\#\#\#\#\#\#\#\#\#\#\#\#\#\#\#\#\#\#\#\#\#\#\#\#\#\#\#\#\#\#\#\#\#\#\#\#\#\#\#\#\#\#\#\#\#\#\#\#\#\#\#\#\#\#\#\#\#\#\#\#\#
\\
\#                              ROWFIND
\\
\# 						
\\
\# This script carries a preliminary tone-row search in a specified
\\
\# score.  It assumes that a complete set of set-variant files exists
\\
\# in the current directory, named P0-P11, I0-I11, R0-R11, and RI0-RI11.
\\
\#
\\
\# This script puts a file named "analysis" which may be assembled
\\
\# with the original input file.
\\
\#
\\
\# Invoke:
\\
\#       rowfind scorefile
\\
\#
\\
\# Check that the specified input file exists:
\\
if [ ! -f \$1 ]
\\
then
\\
	echo "rowfind: ERROR: Input score file \$1 not found."
\\
	exit
\\
fi
\\
\# Also check that the row-variant files exist:
\\
X=11
\\
while [ \$X -ne -1 ]
\\
do
\\
	if [ ! -f P\$X ] 
\\
	then
\\
		echo "rowfind: ERROR: Row file P\$X not found."
\\
		exit
\\
	fi
\\
	if [ ! -f I\$X ] 
\\
	then
\\
		echo "rowfind: ERROR: Row file I\$X not found."
\\
		exit
\\
	fi
\\
	if [ ! -f R\$X ] 
\\
	then
\\
		echo "rowfind: ERROR: Row file R\$X not found."
\\
		exit
\\
	fi
\\
	if [ ! -f RI\$X ] 
\\
	then
\\
		echo "rowfind: ERROR: Row file RI\$X not found."
\\
		exit
\\
	fi
\\
	let X=\$X-1
\\
done}


\par 
The following two lines of the script prepare the input score for searching.
Specifically, the score is transformed to pitch-class notation
(using \textbf{pc}) and then all rests are changed to null tokens
using the
\textbf{humsed}
command.
Notice the use of the
\textbf{-a}
option for
\textbf{pc}
in order to use the alpha-numeric pitch-class representation.

\par 

\texttt{pc -at \$1 > temp.pc
\\
humsed 's/r/./g' temp.pc > score.tmp}


\par 
The main searching task is done by \textbf{patt}.
The
\textbf{patt}
command is executed 48 times -$\,$- once for each row variant.
The
\textbf{-t}
(tag) option is used so that a \texttt{**patt} output is generated.
Each time a match is made the appropriate name (e.g. P4) is output in the spine.
The
\textbf{-s}
option is used to skip barlines and null data records when matching patterns.
The
\textbf{-m}
option invokes the multi-record matching mode -$\,$- which allows
\textbf{patt}
to recognize row statements where several nominally successive pitches
are collapsed into a vertical chord:

\par 

\texttt{\# Search for instances of each tone-row variant.
\\
X=0
\\
while [ \$X -ne 12 ]
\\
do
\\

\\
patt -s '=|\^{}$\backslash$.($\backslash$t$\backslash$.)*\$' -f P\$X -m score.tmp -t P\$X $\backslash$
\\

\\
| extract -i '**patt' > P\$X.pat
\\

\\
patt -s '=|\^{}$\backslash$.($\backslash$t$\backslash$.)*\$' -f I\$X -m score.tmp -t I\$X $\backslash$
\\

\\
| extract -i '**patt' > I\$X.pat
\\

\\
patt -s '=|\^{}$\backslash$.($\backslash$t$\backslash$.)*\$' -f R\$X -m score.tmp -t R\$X $\backslash$
\\

\\
| extract -i '**patt' > R\$X.pat
\\

\\
patt -s '=|\^{}$\backslash$.($\backslash$t$\backslash$.)*\$' -f RI\$X -m score.tmp -t RI\$X $\backslash$
\\

\\
| extract -i '**patt' > RI\$X.pat
\\

\\
let X=\$X+1
\\

\\
done}


\par 
Each of the above 48
\textbf{patt}
searches resulted in a separate temporary output file.
It would be convenient to reduce all 48 \texttt{**patt} spines into a single
aggregate spine.
This can be done using the
\textbf{assemble}
and
\textbf{cleave}
commands:

\par 

\texttt{\# Now we have a lot of files to assemble:
\\
assemble P*.pat > prime.pat
\\
cleave -d ' ' -i '**patt' -o '**rows' prime.pat > analysis.1
\\

\\
assemble I*.pat > inversion.pat
\\
cleave -d ' ' -i '**patt' -o '**rows' inversion.pat > analysis.2
\\

\\
assemble R*.pat > retro.pat
\\
cleave -d ' ' -i '**patt' -o '**rows' retro.pat > analysis.3
\\

\\
assemble RI*.pat > retroinv.pat
\\
cleave -d ' ' -i '**patt' -o '**rows' retroinv.pat > analysis.4
\\

\\
assemble analysis.[1-4] > temp
\\
cleave -d ' ' -i '**rows' -o '**rows' temp > analysis.out
\\

\\
\# Finally, clean up some temporary files:
\\

\\
rm [PRI][0-9].pat [PRI]1[01].pat RI[0-9]*.pat temp.pat
\\
rm analysis.[1-4] temp temp.pc score.tmp}


\par 
There are a few subtleties and problems that deserve mention about our
\textbf{rowfind}
script.
In general, shorter patterns are easier to find than longer patterns.
Since row statements tend to be unique after the first 4 or 5 notes,
it is preferable to clip the row patterns used as templates.
Reducing the length of the templates can lead to "false hits"
-$\,$-  but these tend to be infrequent
and are easily recognized.

\par 
Applied to an entire multi-part score,
\textbf{rowfind}
may miss concurrent row statements due to interposed notes appearing in
an irrelevant instrument or part.
This problem can be avoided by first extracting individual parts and
running
\textbf{rowfind}
on each part separately.
(The results can then be amalgamated using \textbf{assemble} and \textbf{cleave}.)
On the other hand, searching instruments separately can mean that row
statements crossing between instruments may be missed.
This problem can be addressed by extracting pairs and groups of instruments
and analyzing them together.

\par 
For a complex work like Webern's Opus 24 Concerto, this strategy of analyzing
both individual instruments and groups of instruments works very well.



\subsection*{Reprise}

\par 
In this chapter we have discussed several tools related to set theory
analysis.
These include the
\textbf{pc}
(pitch-class) command, the
\textbf{pcset}
command (for identifying set-forms), and the
\textbf{reihe}
command (for generating set variants).

\par 
We have seen how general tools like
\textbf{context}
can be used to carry out segmentation of some score.
Similarly, we have seen how the
\textbf{patt}
command can be used to identify tone-row statements.

\par 
Two scripts were described in this chapter:
\textbf{matrix}
and
\textbf{rowfind.}
These demonstrated how the tools may be coordinated to
carry out various automated processes.

\par 
This chapter has only scratched the surface regarding the
types of pertinent serial-related manipulations that might be pursued.
For example, much more sophisticated approaches to segmentation
can be created by using some of the layer techniques described
in the
next Chapter.
Similarly, the pattern searches could easily be expanded to look at other
parameters typical of "complete serialism" -$\,$- such as durations,
dynamics, articulation marks, and so on.

\\
\begin{itemize}
\item 

\textbf{Next Chapter}
\item 

\textbf{Previous Chapter}
\item 

\textbf{Table of Contents}
\item 

\textbf{Detailed Contents}
\\\\

� Copyright 1999 David Huron
\end{document}
