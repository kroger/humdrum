% This file was converted from HTML to LaTeX with
% Tomasz Wegrzanowski's <maniek@beer.com> gnuhtml2latex program
% Version : 0.1
\documentclass{article}
\begin{document}



  
  
    
      
      
      
    
  




\section*{Bibliography}


\\\\
Alphonce, B.
Music analysis by computer.
\textit{Computer Music Journal,}
Vol. 4, No. 2 (1980) pp. 26-35.
\\\\
Alphonce, B.
Computer applications: Analysis and modeling.
\textit{Music Theory Spectrum,}
Vol. 11, No. 1 (1989) pp. 49-59.
\\\\
Baroni, M. \& Callegari, L. (eds.)
\textit{Musical Grammars and Computer Analysis.}
Firenze: Leo S. Olschki Editore, 1984.
\\\\
Bauer-Mengelberg, S.
The Ford-Columbia input language.
In B. Brook (ed.),
\textit{Musicology and the Computer,}
New York: City University of New York, 1970; pp. 53-56.
\\\\
Brinkman, A.
\textit{Pascal Programming for Music Research.}
Chicago: University of Chicago Press, 1990.
\\\\
Bronson, B.
Toward the comparative analysis of British-American folk tunes.
\textit{Journal of American Folklore,}
Vol. 72, No. 284 (1959) pp. 165-191.
\\\\
Cope, D.
\textit{New Music Notation.}
Dubuque, Iowa: Kendall/Hunt Publishing Co., 1976.
\\\\
Erickson, R.
\textit{DARMS; A Reference Manual.}
Binghamton, NY: typescript, 1976.
\\\\
Dannenberg, R.
Music representation issues, techniques, and systems.
\textit{Computer Music Journal,}
Vol. 17, No. 3 (1993) pp. 20-30.
\\\\
Hall, T.
DARMS: The A-R Dialect.
In E. Selfridge-Field (ed.),
\textit{Beyond MIDI: The Handbook of Musical Codes.}
Cambridge, MA: MIT Press, 1997; pp. 193-214.
\\\\
Halperin, D.
Afterword: Guidelines for new codes.
In E. Selfridge-Field (ed.),
\textit{Beyond MIDI: The Handbook of Musical Codes.}
Cambridge, MA: MIT Press, 1997; pp. 573-580.
\\\\
Haus, G. (ed.)
\textit{Music Processing.}
Madison, WI: A-R Editions, Inc., and
Oxford: Clarendon Press, 1993.
\\\\
Hewlett, W. B.
A base-40 number-line representation of musical pitch notation.
\textit{Musikometrika,}
Vol. 4 (1992) pp. 1-14.
\\\\
Hewlett, W. B.
The representation of musical information in machine-readable format.
\textit{Directory of Computer Assisted Research in Musicology 1987.}
Menlo Park: Center for Computer Assisted Research in the Humanities, 1987; pp. 1-22.
\\\\
Hewlett, W. B. \& Selfridge-Field, E. (eds.)
\textit{Directory of Computer Assisted Research in Musicology 1988.}
Menlo Park: Center for Computer Assisted Research in the Humanities, 1988.
\\\\
Hewlett, W. B. \& Selfridge-Field, E. (eds.)
\textit{Computing in Musicology; A Directory of Research;}
Menlo Park: Center for Computer Assisted Research in the Humanities, 1989.
\\\\
Hewlett, W. B. \& Selfridge-Field, E. (eds.)
\textit{Computing in Musicology; A Directory of Research;}
Volume 6.
Menlo Park: Center for Computer Assisted Research in the Humanities, 1990.
\\\\
Hewlett, W. B. \& Selfridge-Field, E. (eds.)
\textit{Computing in Musicology; A Directory of Research;}
Volume 7.
Menlo Park: Center for Computer Assisted Research in the Humanities, 1991.
\\\\
Hewlett, W. B. \& Selfridge-Field, E. (eds.)
\textit{Computing in Musicology; An International Directory of Applications;}
Volume 8.
Menlo Park: Center for Computer Assisted Research in the Humanities, 1992.
\\\\
Howard, J.
Plaine and Easie Code: A Code for Music Bibliography.
In E. Selfridge-Field (ed.),
\textit{Beyond MIDI: The Handbook of Musical Codes.}
Cambridge, MA: MIT Press, 1997; pp. 343-361.
\\\\
Huron, D.
Error categories, detection, and reduction in a musical database.
\textit{Computers and the Humanities,}
Vol. 22, No. 2 (1988) pp. 253-264.
\\\\
Huron, D.
Characterizing musical textures.
\textit{Proceedings of the 1989 International Computer Music Conference,}
San Francisco: Computer Music Association, 1989; pp. 131-134.
\\\\
Huron, D.
The avoidance of part-crossing in polyphonic music:
Perceptual evidence and musical practice.
\textit{Music Perception,}
Vol. 9, No. 1 (1991) pp. 93-104.
\\\\
Huron, D.
Tonal consonance versus tonal fusion in polyphonic sonorities.
\textit{Music Perception,}
Vol. 9, No. 2 (1991) pp. 135-154.
\\\\
Huron, D.
Design principles in computer-based music representation.
In  A. Marsden \& A. Pople (eds.),
\textit{Computer Representations and Models in Music.}
London: Academic Press, 1992; pp. 5-59.
\\\\
Huron, D. \& Royal, M.
What is melodic accent?
Converging evidence from musical practice.
\textit{Music Perception,}
Vol. 13, No. 4 (1996) pp. 498-516.
\\\\
Kippen, J. \& Bernard, B.
Modelling music with grammars: Formal language representation in the Bol processor.
In A. Marsden \& A. Pople (eds.),
\textit{Computer Representations and Models in Music.}
London: Academic Press, 1992; pp. 207-238.
\\\\
Kornst�dt, A.
SCORE-to-Humdrum: A graphical environment for musicological
analysis.
\textit{Computing in Musicology,}
Vol. 10 (1995-96) pp. 105-122.
\\\\
Marsden, A. \& Pople, A. (eds.)
\textit{Computer Representations and Models in Music.}
London: Academic Press, 1992.
\\\\
Moles, A.
\textit{Information Theory and Esthetic Perception.}
Urbana: University of Illinois Press, 1968.
\\\\
Moore, F. R.
\textit{Elements of Computer Music.}
Englewood Cliffs, NJ: Prentice Hall, 1990.
\\\\
Nettheim, N.
On the accuracy of musical data, with examples from Gregorian
Chant and German folksong.
\textit{Computers and the Humanities,}
Vol. 27, (1993) pp. 111-120.
\\\\
Newcomb, S.
Standard music description language complies with hypermedia
standard.
\textit{IEEE Computer,}
Vol. 24, No. 7 (1991) pp. 76-79.
\\\\
Newcomb, S., Kipp, N. \& Newcomb, V.
The HyTime hypermedia time-based document structuring language.
\textit{Communications of the ACM,}
Vol. 34, No. 11 (1991) pp. 67-83.
\\\\
Orpen, K. \& Huron, D.
Measurement of similarity in music:
A quantitative approach for non-parametric representations.
\textit{Computers in Music Research,}
Vol. 4 (1992) pp. 1-44.
\\\\
�' Maidin, D.
Representation of music scores for analysis.
In  A. Marsden \& A. Pople (eds.),
\textit{Computer Representations and Models in Music.}
London: Academic Press, 1992; pp. 67-93.
\\\\
Overill, R. E.
On the combinatorial complexity of fuzzy pattern matching
in music analysis.
\textit{Computers and the Humanities,}
Vol. 27 (1993) pp. 105-110.
\\\\
Pope, S. T.
Music notations and the representation of musical structure and knowledge.
\textit{Perspectives of New Music,}
Vol. 24 (1986) pp. 156-189.
\\\\
Prather, R. \& Elliot, R.
SML: A structured musical language.
\textit{Computers and the Humanities,}
Vol. 22 (1988) pp. 137-151.
\\\\
Rahn, J.
\textit{Basic Atonal Theory.}
New York: Longman Inc., 1980.
\\\\
Rastall, R.
\textit{The Notation of Western Music.}
London: J. M. Dent \& Sons Ltd., 1983.
\\\\
Roads, C.
An overview of music representations.
In M. Baroni \& L. Callegari (eds.)
\textit{Musical Grammars and Computer Analysis.}
Firenze: Leo S. Olschki Editore, 1984, pp. 7-37.
\\\\
Schaffrath, H.
\textit{The Essen Folksong Collection}
[database].
D. Huron (ed.).
Stanford, CA: Center for Computer Assisted Research in the Humanities, 1995.

\\\\
Selfridge-Field, E. (ed.)
\textit{Beyond MIDI: The Handbook of Musical Codes.}
Cambridge, MA: MIT Press, 1997.
\\\\
Simpson, J. \& Huron, D.
The perception of rhythmic similarity: A test of a modified
version of Johnson-Laird's theory.
\textit{Canadian Acoustics,}
Vol. 21, No. 3 (1993) pp. 89-90.
\\\\
Smith, L.
\textit{SCORE -$\,$- A Musician's Approach to Computer Music.}
\textit{Journal of the Audio Engineering Society,}
Vol. 20 (1972) pp. 7-14.
\\\\
Straus, J. N.
\textit{Introduction to Post-Tonal Theory.}
Englewood Clifs, N.J.: Prentice Hall, 1990.
\\\\
Taylor, W. M.
\textit{Humdrum Graphical User Interface.}
MA Thesis, Queen's University of Belfast, 1996.
\\\\
Thompson, W. F. \& Stainton, M.
Using Humdrum to analyze melodic structure: An assessment
of Narmour's implication-realization model.
\textit{Computing in Musicology,}
Vol. 10 (1995-96) pp. 24-33.
\\\\
Vercoe, B.
\textit{Csound: A Manual for the Audio Processing System and Supporting Programs}
\textit{with Tutorials.}
2nd rev. ed.
Cambridge, MA: Massachusetts Institute of Technology Media Lab, 1993.
\\\\
Vivaldi, A.
\textit{"The Four Seasons" and Other Violin Concertos in Full Score.}
E. Selfridge-Field (ed.).
New York: Dover Publications, Inc., 1995.
\\\\
Vivaldi, A.
\textit{The Twelve Violin Concertos Opus 8 including "The Four Seasons"}
[database].
E. Selfridge-Field \& D. Huron (eds.).
Stanford CA: Center for Computer Assisted Research in the Humanities, 1996.
\\\\
Vollsnes, A. \& Lande, T.S.
\textit{Music Encoding and Analysis in the MUSIKUS System.}
Oslo: University of Oslo, Dept. of Informatics, 1988.
\\\\
Wenker, J.
A foundation for computational musicology.
In L. Mitchell (ed.)
\textit{Computer in the Humanities,}
Edinburgh: University Press, 1974, pp. 267-280.
\\\\
Wiggens, G., Miranda, E., Smaill, A. \& Harris, M.
A framework for the evaluation of music representation systems.
\textit{Computer Music Journal,}
Vol. 17, No. 3 (1993) pp. 31-42.


\\
\begin{itemize}
\item 

\textbf{Next Appendix}
\item 

\textbf{Table of Contents}
\item 

\textbf{Detailed Contents}
\\\\

� Copyright 1999 David Huron
\end{document}
