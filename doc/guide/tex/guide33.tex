% This file was converted from HTML to LaTeX with
% Tomasz Wegrzanowski's <maniek@beer.com> gnuhtml2latex program
% Version : 0.1
\documentclass{article}
\begin{document}



  
  
    
      
      
      
    
  



\\
\\

\section*{Chapter33}


[\textit{Previous Chapter}]
[\textit{Contents}]
[\textit{Next Chapter}]


\section*{Word Sounds}



In addition to the meanings conveyed by words, words also
provide distinctive sounds that can prove to be musically useful.
Poets and composers often arrange or choose texts so that the sequence of
sounds create alliteration, onomatopoeia, rhythm, rhyme and other sonorous effects.
This chapter introduces the
\texttt{**IPA}
scheme for representing speech sounds.
This representation provides a companion to the
\texttt{**text}
and
\texttt{**silbe}
representations discussed in
Chapter 27.
Various sonorous processes are illustrated.


\subsection*{The **IPA Representation}

\par 
The Humdrum
\texttt{**IPA}
scheme provides a way to represent the International Phonetic Alphabet.
The \texttt{**IPA} scheme is based on the
transliteration scheme developed by
linguist Evan Kirshenbaum.
The scheme is suitable for representing the basic phonemes found in
most of the world's languages.
The table below summarizes the \texttt{**IPA}
mappings for various phonemes.

@\textbf{schwa}� as in (unaccented) \textit{banana, collide, alone} or (accented) \textit{humdrum}
V\textbf{schwa} (IPA symbol: $\backslash$(an); as in the British pronunciation of \textit{hut}
R\textbf{R}$\backslash$(dd; as in \textit{burn, operation, dirt, urgent}
\&short \textbf{a} (IPA symbol: $\backslash$(ae); as in \textit{mat, map, mad, gag, snap, patch}
A\textbf{�} (IPA symbol: \textit{a}); as in \textit{bother, cot}, and, with most American speakers,
\textit{father, cart}
a\textbf{a$\backslash$(de}; \textit{father} as pronounced by speakers who do not rhyme it with \textit{bother}.
Eshort \textbf{e} (IPA symbol: $\backslash$(*e or E); as in \textit{get, bed, peck, edge}
ilong \textbf{e} (IPA symbol: e$\backslash$(ma); as in \textit{beat, greed, evenly, easy}
Ishort \textbf{i} (IPA symbol: I or $\backslash$(*i); as in \textit{tip, banish, active}
o\textbf{o} as in \textit{oboe, trombone, banjo}
O\textbf{o$\backslash$(de} (IPA symbol: o or upside-down `c'); as in \textit{law, all, shawm}
W\textbf{oe} digraph (IPA symbol: $\backslash$(oe); as in the French \textit{boeuf}, German \textit{H�lle}
u\textbf{�}; as in \textit{rule, youth, few, ooze}
U\textbf{u$\backslash$(de'} (IPA symbol: $\backslash$(*u or U or $\backslash$(*w);  as in \textit{pull, wood, book}
y\textbf{ue}; as in the German \textit{f�llen}, \textit{h�bsch}, or French \textit{rue}
\textit{vowel}~following a vowel* indicates a vowel or diphthong pronounced with open
nasal passages; as in the French "un bon vin blanc" (\texttt{W~ bo~ va~ blA~})

b\textbf{b} (IPA symbol: b or c$\backslash$(hc'); as in \textit{beam, cabin, rob}
d\textbf{d}; as in \textit{deed, dulcimer, adder}
f\textbf{f}; as in \textit{fugue, staff}, \textit{forte}
g\textbf{g}; as in \textit{guitar, fagot, gig}
h\textbf{h}; as in \textit{hear, ahead, horn}
k\textbf{k}; as in \textit{cook, take, score, ache}
x\textbf{K} (IPA symbol: k); as in the German \textit{ich, Buch}
l\textbf{l}; as in \textit{libretto, Lully, pool}
m\textbf{m}; as in \textit{music, limb, hymn}
n\textbf{n}; as in \textit{no, instrument, blown}
N\textbf{eng} (IPA symbol: `n' with a tail); as in \textit{sing, fingering, ink}
p\textbf{p}; as in \textit{piano, beeper, lip}
r\textbf{r}; as in \textit{reed, organ, car}
s\textbf{s}; as in \textit{soprano, cymbal, source, bass}
S\textbf{sh} ["esh"] (IPA symbol: $\backslash$(It$\backslash$(Im$\backslash$(Ib'); as in \textit{sharp, crescendo, special, percussion}
t\textbf{t}; as in \textit{tempo, tie, attacca, minuet}
T\textbf{th} ["thorn"] (IPA symbol: $\backslash$(*h); as in \textit{thin, path, ether}
D\textbf{th} ["eth"] (IPA symbol: \textit{d~) as in \textit{then, rhythm, smooth}
v\textbf{v}; as in \textit{voice, vivace, live}
w\textbf{w}; as in \textit{we, away}
j\textbf{j}; as in \textit{yes, Johann, cue, onion}
z\textbf{z}; as in \textit{zone, raise, xylophone, jazz}
Z\textbf{zh} ["yogh"$\backslash$(sc]; as in \textit{measure, vision, azure}
\textit{consonant}-following a consonant (l-, n-, m-, or N-)** indicates a consonant preceded
by a \textit{schwa} that is pronounced as an independent syllable; as in \textit{battle,
mitten, eaten
\textit{consonant};following a consonant,�� indicates that the front of the tongue is
positioned as in the beginning of the word `yard'

\^{}preceding phoneme is palatalized
$\backslash$'primary stress (should precede stressed sound)
,secondary stress (should precede stressed sound)
\%silence signifier


\textit{Summary of \textit{**IPA} Signifiers}


\par 
\begin{itemize}
\item 
The IPA \textit{schwa} is notated as an upside-down `e'.
\item 
The IPA symbol consists of a \textit{schwa} with a hook.
\item 
The IPA \textit{yogh} is written like a flat-topped numeral `3' that has
been lowered in height.
\item 
In IPA such vowels are marked by the presence of a tilde
above the vowel.
\item 
In IPA such consonants are marked by the presence of
a vertical bar below the consonant.
\item 
The IPA symbol consists of a superscript letter `j' either following or
hooked beneath the consonant.
\end{itemize}

\par 
Humdrum does not provide a tool for translating from
\texttt{**text}
or
\texttt{**silbe}
representations
to
\texttt{**IPA}.
However, there are a number of commercial text-to-phoneme translators available for most
common languages.


\subsection*{Alliteration}


\par 
A common sonorous use of words is found in alliteration where
several successive words commence with the same sound.
A famous example of alliteration is found at the beginning of William Shakespeare's
\textit{Tempest}:

\texttt{**text**IPA
\texttt{Fullful
\texttt{fathomf\&D@m
\texttt{fivefAiv
\texttt{thyDAi
\texttt{fatherfADR
\texttt{lieslAiz
\texttt{*-*-}


\par 
Given an \texttt{**IPA} input,
occurrences of alliteration can be found by first isolating the
initial phoneme for each word using \textbf{humsed}.
This task requires some additional knowledge about using
\textbf{humsed}.
Both
\textbf{sed}
and
\textbf{humsed}
provide a "back reference" construction that allows users
to manipulate a matched expression without knowing the precise
matched sequence of characters.
The expression to be matched is indicated via parentheses preceded
by back-slash characters, i.e., \texttt{$\backslash$(} and \texttt{$\backslash$)}.
Several such expressions can be defined and each successive expression is
internally labelled with an integer beginning with 1.
The marked expression can then be "back-referenced"
by using the integer label preceded by a back-slash.
Hence, \texttt{$\backslash$1} refers to the first referenced expression.
Consider, by way of illustration, the following command:

\par 

\texttt{sed 's/\^{}$\backslash$(.$\backslash$).*/$\backslash$1/'}


\par 
In this command, the referenced expression consists of the period (match any single character).
Notice that this expression is back-referenced in the replacement string -$\,$- \texttt{$\backslash$1}.
In other words, this
\textbf{sed}
command carries out the following operation:
find a single character at the beginning of a line followed by zero or more characters.
Replace this entire string by just the first character in the line.

\par 
Let's now use this back-reference technique in our alliteration search.
First we extract the
\texttt{**IPA}
spine and use
\textbf{humsed}
to eliminate all but the first character in each data record:

\par 

\texttt{extract -i '**IPA' Tempest | humsed 's/\^{}$\backslash$(.$\backslash$).*/$\backslash$1/'}


\par 
The result is:

\texttt{**IPA
\texttt{f
\texttt{f
\texttt{f
\texttt{D
\texttt{f
\texttt{*-}


\par 
We can now amalgamate successive initial phonemes by using the
\textbf{context}
command.
Suppose we are interested in identifying alliterations where three
or more words begin with the same initial phoneme.
For this, we would use the \textbf{-n 3} option for \textbf{context}.
Having amalgamated three phonemes on each data record we can use
\textbf{humsed}
to eliminate the spaces between the multiple stops:

\par 

\texttt{extract -i '**IPA' Tempest | humsed 's/$\backslash$(.$\backslash$).*/$\backslash$1/' $\backslash$

| context -n 3 | humsed 's/ //g'}



\par 
The revised output is:

\texttt{**IPA
\texttt{fff
\texttt{ffD
\texttt{fDf
\texttt{Dfl
\texttt{.
\texttt{.
\texttt{*-}


\par 
Now we need to identify any data records that contain three
identical sigifiers.
Once again, we can use the back-reference feature for \textbf{humsed}.

\par 

\texttt{extract -i '**IPA' Tempest | humsed 's/$\backslash$(.$\backslash$).*/$\backslash$1/' $\backslash$

| context -n 3 | humsed 's/ //g'; s/$\backslash$(.$\backslash$)$\backslash$1$\backslash$1/allit: $\backslash$1/'}



\par 
The resulting output is:

\texttt{**IPA
\texttt{allit: f
\texttt{ffD
\texttt{fDf
\texttt{Dfl
\texttt{.
\texttt{.
\texttt{*-}


\par 
Let's add one further refinement which illustrates yet another feature
provided by
\textbf{sed}
and \textbf{humsed}.
Recall that operations such as substitions (\texttt{s}) and deletions (\texttt{d})
can be preceded by a regular expression that limits the operation only
to those lines that match the expression.
For example, the command \texttt{sed '/=/s/[0-9]//g'} will eliminate all
numbers found on lines containing an equals sign.
The leading regular expression can be followed by an exclamation mark
which reverses the sense of the action.
For example, the command \texttt{sed '/=/!s/[0-9]//g'} will eliminate all
numbers \textit{except} those found on lines containin an equals sign.

\par 
This feature can be usefully applied in our alliteration task to
eliminate all other data in our spine except alliteration markers.
Our final revised pipeline transforms non-alliteration data to null tokens:

\par 

\texttt{extract -i '**IPA' Tempest | humsed 's/$\backslash$(.$\backslash$).*/$\backslash$1/' $\backslash$

| context -n 3 | humsed 's/ //g'; s/$\backslash$(.$\backslash$)$\backslash$1$\backslash$1/allit: $\backslash$1/; $\backslash$
\\
/allit/!s/.*/./'}



\par 
The final output is:

\texttt{**IPA
\texttt{allit: f
\texttt{.
\texttt{.
\texttt{.
\texttt{.
\texttt{.
\texttt{*-}



\subsection*{Classifying Phonemes}

\par 
Linguists have devised innumerable ways for classifing phonemes.
For example, phenomes such as \textit{f}, \textit{s}, \textit{sh}, \textit{th}, \textit{v}, etc. are
classified as fricatives.
Bi-labial plosives include the \textit{p} and \textit{b} sounds.
The \textit{m} and \textit{n} sounds are classified as nasals.
And so on.

\par 
For some tasks, it is often useful to reduce the phonemes
to phonetic classes.
For example, in our example from Shakespeare's \textit{Tempest}, the `th' in `thy'
is part of the fricative alliteration.

\par 
In
Chapter 22
we saw how
\textbf{humsed}
can be used to classify things.
A simple reassignment script can be defined which collapses
the various phonemes into a smaller set of phonetic classes.
For example, a suitable script might contain the following assignments:

\par 

\texttt{s/[bdtk]//g
\\
s/[mn]//g
\\
s/[fsSTDv]//g}
\\
etc.


\par 
Classifying phonemes in this way will allow us to broaden our searches
for alliterative passages.


\subsection*{Properties of Vowels}

\par 
Vowels are particular important in music since notes can be
sustained only by increasing the duration of the vowels.
In speech, more time is occupied by vowel sounds than by consonants.
In vocal music, an even greater proportion of the time is taken up by vowels.
This means that the quality of the vowels can have a marked
impact on the overall timbre or tone color of a work.
For example, a long note sustained with an `ee' can sound
very different from the same note sung with a sustained `ah'.

\par 
Like consonants, vowels can also be classified in many ways.
One common classifying dimension is the front/back distinction.
The vowel `oo' is a front position vowel whereas the vowel `ah'
is a back position vowel.
Similarly, vowels can be classified according to the high/low distinction.
The vowel `ee' is a high vowel whereas the vowel `oh' is a low vowel.

\par 
A preponderance of high vowels is often associated with sarcasm, irony
or humor.
Taunting sounds made by children ("nya, nya ...")
commonly use high vowels mixed with nasals.
Similarly, high/nasal vocal sounds are often used by comedians
and actors to produce a `funny' voice.


\par 
Suppose we want to test the idea that a certain piece in a Gilbert and
Sullivan operetta exhibits a preponderance of high vowels.
We might begin by creating a re-assignment file where
the estimated height of each vowel is given a value between 1 (low)
and 10 (high).
We can estimate the overall average vowel height for a piece by
averaging these values together.
The basic pipeline will extract the pertinent \texttt{**IPA}
spine, eliminate all
non-vowel phonemes, add spaces between each vowel, and then assign
estimated heights to each vowel.
Finally, non-data records are eliminated using
\textbf{rid}
and the data values averaged using the
\textbf{stats}
command:

\par 

\texttt{extract -i '**IPA' Penzance | humsed 's/[\^{}@VR\&AaEiIoOWuUy]//'$\backslash$

| humsed 's/./\& /g; s/  / /; s/ \$//' | humsed -f vowel.map$\backslash$
\\
| rid -GLId | stats}



\par 
This procedure can be repeated for several movements or pieces to
provide a contrast for the piece of interest.


\subsection*{Vowel Coloration}

\par 
When translating vocal texts from one language to another,
it is often difficult for translators to preserve the vowel coloration.
A vocal work can be considerably maligned if a prominent
(high/long) note is changed from an `ah' sound to an `ee' sound.

\par 
Suppose we want to determine which of several English translations
of a song by Schubert best preserves the vowel coloration.
As above, let's limit our notion of coloration to vowel height.
(Of course any other similarity mapping or dimension can be used.)
As in our Gilbert and Sullivan example, we could simply compare the
overall vowel height for the original Schubert song with each of
the translations.
However, not all notes are equally important.
In the first instance, the vowels on longer sustained notes will be
more noticeable than the vowels attending shorter notes.
A simple remedy is to use the
\textbf{timebase}
command to expand the input so that longer notes are proportionally
more influential in our measure of overall vowel height.
We can use
\textbf{ditto}
to repeat sustained vowels:

\par 

\texttt{timebase -t 16 Schubert | extract -i '**IPA' $\backslash$

| humsed 's/[\^{}@VR\&AaEiIoOWuUy]//' | humsed 's/./\& /g; $\backslash$
\\
s/  / /; s/ \$//' | humsed -f vowel.map | rid -GLId | stats}



\par 
Since translators have plenty of other issues to consider when translating a
vocal text, we might focus our comparisons solely on a small collection of
especially important notes.
We might for example use a longer value for \textbf{timebase}.
Alternatively, we might use the Humdrum
\textbf{accent}
command (described in
Chapter 25)
to identify notes have a particularly high noticeability.



\subsection*{Rhymes and Rhyme Schemes}

\par 
Rhymes are common poetic devices throughout the world's cultures.
Rhymes involve the use of similar or identical word-final phonemes.
Typically, rhymes are based on the final phonemes of phrase-terminating words,
but rhymes commonly occur in mid-phrase and other positions in poetry from
various cultures.
Consider the rhymes in the following traditional nonsense verse:

\par 

We're all in the dumps,
\\
For diamonds are trumps,
\\
The kittens are gone to St. Paul's
\\
The babies are bit,
\\
The moon's in a fit
\\
And the houses are built without walls.


-Anon. "We're All in the Dumps" From The Faber Book of Nonsense Verse, Geoffrey Grigson (ed.) 1979; p.65


\par 
Suppose we want to automatically identify the rhyme scheme for this (or some other) text.
Our first order of business is to identify phrase-terminating points.
Let's assume we already have some phrase indicators (via curly braces \{\}).
Our input might begin as follows:

\texttt{**text**IPA
\texttt{We're\{wRr
\texttt{allAl
\texttt{inIn
\texttt{theD@
\texttt{dumps,d@mps\}}
etc.


\par 
Using \textbf{extract}, \textbf{context} and \textbf{rid}
we can isolate each poetic phrase:

\par 

\texttt{extract -i '**IPA poem | context -b \{ -e \} | rid -GLId}


\par 
The result is as follows:

\par 

\texttt{\{wRr Al In D@ d@mps\}
\\
\{foR dAim@nds Ar tr@mps\}
\\
\{D@ kIt@ns Ar gAn tu seint pAUls\}
\\
\{D@ beibiz Ar bIt\}
\\
\{D@ munz In @ fIt\}
\\
\{\&nd D@ h\&uz@z Ar bIlt wITAut wAUls\}}


\par 
The rhyming portion of words typically consist of a final vowel
plus any subsequent consonants.
We can isolate these phonemes using \textbf{sed}.
Notice our use of back reference to preserve the final phonemes:

\par 

\texttt{ ... | sed 's/.*$\backslash$([@VR\&AaEiIoOWuUy][\^{}@VR\&AaEiIoOWuUy]*\}\$$\backslash$)/$\backslash$1/'}


\par 
The resulting output is:

\texttt{@mps\}
\texttt{@mps\}
\texttt{Uls\}
\texttt{It\}
\texttt{It\}
\texttt{Uls\}}


\par 
A little further processing can remove the closing braces using
\textbf{sed}, and eliminate the duplicate lines using \textbf{sort} and \textbf{uniq}.

\par 

\texttt{ ... | humsed 's/\}//' | sort | uniq}


\par 
The output can then be changed into a set of substitutions for a
\textbf{humsed}
script.
A suitable file would contain the following substitutions:

\par 

\texttt{s/.*@mps\$/A/
\\
s/.*It\$/B/
\\
s/.*Uls\$/C/
\\
s/.*/./}


\par 
This script will label all words ending with "umps" to `A'.
Word ending with "its" will be labelled `B', and so on.
All other words will be output as null tokens.
Using this script, a suitable pipeline for processing our
original file would be as follows:

\par 

\texttt{extract -i '**IPA poem | context -b \{ -e \} | humsed 's/\}//' $\backslash$

| humsed -f rhyme | rid -GLId}



\par 
The corresponding output would indicate the rhyme scheme for this poem:

\texttt{A
\texttt{A
\texttt{B
\texttt{C
\texttt{C
\texttt{B


\par 
Note that the entire analytic procedure can be placed in a shell script and
applied to any input containing \texttt{**IPA} text.
The following script adds a number of refinements.

\par 
\texttt{\# RHYME
\\
\#
\\
\# This script determines the rhyme scheme for an input file containing
\\
\# an **IPA spine.  This script assumes that the input contains curly
\\
\# braces indicating phrase endings.
\\
\#
\\
\# USAGE:  rhyme 
\\

\\
extract -i '**IPA' \$1 | extract -f 1 | context -b \{ -e \} | rid -GLId $\backslash$
\\

\\
| sed 's/.*$\backslash$([@VR\&AaEiIoOWuUy][\^{}@VR\&AaEiIoOWuUy]*\}\$$\backslash$)/$\backslash$1/' $\backslash$
\\
| sort | uniq | sed 's/\^{}/s$\backslash$/.*/; s/\$/$\backslash$/XXX$\backslash$//' $\backslash$
\\
| awk 'BEGIN \{alphabet[1]="A"; alphabet[2]="B"; alphabet[3]="C";
\\

\\
alphabet[4]="D"; alphabet[5]="E"; alphabet[6]="F";
\\
alphabet[7]="G"; alphabet[8]="H"; alphabet[9]="I";
\\
alphabet[10]="J"; alphabet[11]="K"; alphabet[12]="L";
\\
alphabet[13]="M"; alphabet[14]="N"; alphabet[15]="O";
\\
alphabet[16]="P"; alphabet[17]="Q"; alphabet[18]="R";
\\
alphabet[19]="S"; alphabet[20]="T"; alphabet[21]="U";
\\
alphabet[22]="V"; alphabet[23]="W"; alphabet[24]="X";
\\
alphabet[25]="Y"; alphabet[26]="Z"; alphabet[27]="ERROR"\}
\\
\{temp=\$0
\\
gsub("XXX",alphabet[NR],temp)
\\
print temp
\\
\}' > rhyme.sed.\$\$
\\

\\

\\
extract -i '**IPA' \$1 | extract -f 1 | context -b \{ -e \} | rid -GLId $\backslash$
\\

\\
| sed -f rhyme.sed.\$\$
\\

\\
rm rhyme.sed.\$\$}



\subsection*{Reprise}

\par 
By focusing on phonetic signifiers, the \texttt{**IPA}
representation provides opportunities
for analyzing many sonorous aspects of vocal sounds -$\,$-
including alliteration, vowel coloration, rhyme, and other effects.
Although we did not illustrate it in this chapter, the
\texttt{**IPA}
representation can be used in conjunction witht the
\texttt{**silbe}
representation to characterize
complex aspects of rhythm and rhyme in vocal texts.

\\
\begin{itemize}
\item 

\textbf{Next Chapter}
\item 

\textbf{Previous Chapter}
\item 

\textbf{Table of Contents}
\item 

\textbf{Detailed Contents}
\\\\

� Copyright 1999 David Huron
\end{document}
