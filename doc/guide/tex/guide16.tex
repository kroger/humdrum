% This file was converted from HTML to LaTeX with
% Tomasz Wegrzanowski's <maniek@beer.com> gnuhtml2latex program
% Version : 0.1
\documentclass{article}
\begin{document}



  
  
    
      
      
      
    
  



\\
\\

\section*{Chapter16}


[\textit{Previous Chapter}]
[\textit{Contents}]
[\textit{Next Chapter}]


\section*{The Shell (II)}



In
Chapter 8
we introduced some of the shell special characters.
By way of review, we learned that the shell interprets the
octothorpe (\#) as the beginning of a comment.
By itself, the asterisk (*) is "expanded" by the shell to
the names of all files in the current directory.
When linked with other characters, such as \texttt{A*} or \texttt{*B},
the shell expands the expression to the names of all files beginning
with A or ending with B.
The greather-than sign (>) directs the output to a named file.
The vertical bar or pipe (|) allows the output of one command
to be directed to the input of the following command.
When placed at the end of a command line,
the ampersand (\&) causes the shell to execute the command as
a background process, and immediately returns a prompt so
the user can execute other commands.
The semicolon (;) indicates the end of a command;
this allows more than one command to be placed on a single line.
The backslash ($\backslash$) escapes the special meaning of the immediately
following character so it is treated literally.
The single quote or apostrophe (') can escape the special meaning
of all characters up to the appearance of a matching single quote.

\par 
In this chapter we will continue to describe shell special characters
and identify their functions.
In addition, we will learn about the shell
\textit{alias}
function.


\subsection*{Shell Special Characters}

\par 
The remaining special shell characters include the following:
the dollars sign (\$), the greve (`),
the less-than sign (<), the question mark (?),
and the double quote (").
We'll consider the function of each of these characters in turn.


\subsection*{Shell Variables}

\par 
Like any programming language,
the shell allows information to be stored and retrieved through
shell \textit{variables}.
Variables can be given all sorts of names, such as \texttt{value},
\texttt{Meter}, \texttt{A34x} and \texttt{BARLINE3}.
In order to retrieve information from a variable,
the variable name is preceded by a dollars sign.
For example, the string \texttt{\$VARIABLE} means "the current value of
the variable named \texttt{VARIABLE}.
Suppose you had a file named \$FILE in the current directory
(\$FILE is a legitimate filename on UNIX systems).
If you type:

\par 

\texttt{sort \$FILE}


\par 
The shell will assume that there is a variable named FILE,
and retrieve its contents.
Since the contents are likely to be empty, the above command
is identical to typing:

\par 

\texttt{sort}


\par 
In order to sort the file named \$FILE, the dollars sign would need to be escaped:

\par 

\texttt{sort $\backslash$\$FILE}


\par 
Depending on the type of shell, variables can be assigned numerical
or string values in various ways.
For most shells, variables can be assigned using the equals sign
(with no intervening spaces).
For example, the integer 7 can be assigned to the variable X as follows:

\par 

\texttt{X=7}


\par 
Or the string "hello" can be assigned to a variable by placing
the string in quotation marks:

\par 

\texttt{X="hello"}


\par 
Single quotation marks can also be used:

\par 

\texttt{X='hello'}


\par 
If you had a file named \texttt{hello} in the current directory,
and if the variable \texttt{X} had been assigned as above,
then the following command
would sort this file:

\par 

\texttt{sort \$X}



\subsection*{The Shell Greve}

\par 
It is often useful to be able to save the results of
some operation in a shell variable.
Suppose for example, that we want to sort a file containing the word \texttt{zebra}.
But we're not certain what file (or files) contain this word.
Manually, we would need to carry out two operations.
First we would search for any file(s) containing the word:

\par 

\texttt{grep -l zebra *}


\par 
We might find that the word "zebra" appears in the
files \texttt{animals} and \texttt{mammals}.
Having determine what files to sort, now we would actually
carry out the appropriate sort command:

\par 

\texttt{sort animals mammals}


\par 
If we found that word "zebra" occurred in 50 files,
then typing the appropriate sort command would require
a lot of typing.
Alternatively, we could use a shell variable to store the results
of the first command, and then retrieve the filenames in the second command.
For this, we must use the greve character (\texttt{`}).
UNIX shells will execute whatever command(s) appear
between two greve characters;
the result of the operation can then be treated as a string
which may be assigned to a shell variable or used in some other way.
In the following commands, the filenames produced by the
\textbf{grep}
command are assigned to a shell variable named \texttt{FILES}.
In the subsequent command a dollars sign instructs the shell
to retrieve the contents of this variable:

\par 

\texttt{FILES=`grep -l zebra *`}
\\
\texttt{sort \$FILES}


\par 
Alternatively, we can avoid the \texttt{FILES} variable altogether,
and execute the following command:

\par 

\texttt{sort `grep -l zebra *`}


\par 
The shell interprets the above command as follows:
First it recognizes the presence of the command delineated by greves.
This command is executed
\textit{before}
the
\textbf{sort}
command.
The
\textbf{grep -l}
command will generate as output a string of filenames.
This output will replace the material delineated by greves.
Finally, the
\textbf{sort}
command will be executed -$\,$- using the filenames generated by the \textbf{grep}
command.

\par 
This command structure is useful in a variety of circumstances.
For example, suppose we wanted to identify any encoded works that are
composed by Josquin and are also in triple meter:

\par 

\texttt{grep -l '!!!COM: Josquin' `grep -l '!!!AMT:.*triple' *`}


\par 
Here we have imbedded one
\textbf{grep}
"inside" another.
Remember that the command delineated by the greve is executed first.
In this case, we begin by searching all of the files in the current
directory for an \texttt{AMT} reference record containing the
keyword "\texttt{triple}."
The
\textbf{-l}
option causes the output to consist of only filenames.
Then the second
\textbf{grep}
is executed.
It looks for files that contain a \texttt{COM} reference record
containing the keyword "\texttt{Josquin}."
But this second
\textbf{grep}
only searches those filenames passed to it by the first \textbf{grep}.
In other words, the composer search is restricted to only those
files that have a triple meter designation.


\par 
Consider another way of using the greve structure.
Suppose we have a file named \texttt{opus16}.
We would like to know what other works contain the same
instrumentation as \texttt{opus16}, but we've forgotten
what the precise instrumentation is.
We can first seach \texttt{opus16} for the instrumentation
data (encoded in the \texttt{AIN:} reference record),
and then search for this information in all files in
the current directory.
This task can be carried out using a single command line:

\par 

\texttt{grep -l `grep '!!!AIN:' opus16` *}


\par 
In this example, the imbedded command provides the regular expression
rather than the files to be searched.


\subsection*{Single Quotes, Double Quotes}

\par 
In
Chapter 8
we learned that single quotation marks can be used to
escape the special meanings of reserved shell characters -$\,$- such
as * and \$.
Double quotation marks (\texttt{"}) have a similar effect with one
important exception.
The dollars sign continues to retain its special meaning
inside double quotes.

\par 
The
UNIX
\textbf{echo}
command causes information to be printed or displayed.
Consider the following three commands:

\par 
\texttt{

echo \$A
\\
echo "\$A"
\\
echo '\$A'

}

\par 
In the first and second commands, the shell looks for a variable
named \texttt{A} and attempts to echo the contents of this variable
on the display.
Unless \texttt{A} happens to be a defined shell variable,
only an empty line will be displayed.
In the third command, the string \texttt{\$A} is treated literally,
and is echoed back to the display.
There are circumstances where the double quotes are more useful,
but for most casual users, the single quotes provide the best
means for disengaging the meanings of special characters.


\subsection*{Using Shell Variables}

\par 
Let's consider an example where shell variables
prove to be useful in Humdrum processing.
Suppose for some score that we want to change the stem-directions
in measures 34 through 38 from up-stems to down-stems.
First, we need to establish the line number corresponding to
the beginning of measure 34 and the line number corresponding
to the end of measure 38 (i.e. beginning of measure 39).
In the following script,
\textbf{grep}
is used to assign these line numbers to the shell variables \texttt{\$A}
and \texttt{\$B}.

\par 

\texttt{A=`grep -n \^{}=34`}
\\
\texttt{B=`grep -n \^{}=39`}


\par 
Now we can construct an appropriate
\textbf{humsed}
command.
Recall that each substitute (\texttt{s}) command in
\textbf{humsed}
can be preceded by a range indication.
In the following command, the \texttt{\$A} and \texttt{\$B} variables
convey the appropriate range to each substitution.
This means that the substitutions are limited to the line numbers
ranging between \texttt{\$A} and \texttt{\$B}.

\par 

\texttt{humsed "\$A,\$Bs/$\backslash$/XXX/g; \$A,\$Bs///$\backslash$/g; \$A,\$Bs/XXX/$\backslash$//g"} \textit{inputfile}


\par 
Notice that we have used double quotes (") rather than single quotes.
The quotation marks are necessary to pass all three substitutions
as an argument to \textbf{humsed}.
Using singe quotes, however, would have caused \texttt{\$A} and \texttt{\$B}
to be treated as literal strings rather than shell variables.


\subsection*{Aliases}

\par 
An alias is an alternative name for something.
The shell provides a way of defining aliases,
and these aliases can prove very convenient.

\par 
Consider, by way of example, the following common
pipeline:

\par 

\texttt{sort inputfile | uniq -c | sort -n}


\par 
In
Chapter 17
we will see that this is a useful
way for generating inventories.
Typically, this sequence occurs at the end of a pipeline
where some preliminary processing has taken place, such as:

\par 

\texttt{timebase -t 8 input | ditto | hint | rid -GLI $\backslash$

| sort | uniq -c | sort -n}



\par 
Since the construction \texttt{sort | uniq -c | sort -n} is
so common, we might want to define an alias for it.
To do so, we simply execute the
\textbf{alias}
command.
In this case, we've defined a new command called \texttt{inventory}:

\par 

\texttt{alias inventory="sort | uniq -c | sort -n"}


\par 
Having defined this alias, we can now make use of it.
Any time we type the word \texttt{inventory},
the shell will expand it to "\texttt{sort | uniq -c | sort -n}".
The above command can be shortened as follows:

\par 

\texttt{timebase -t 8 input | ditto | hint | rid -GLI | inventory}


\par 
Another common task is eliminating barlines.
Frequently, we need to use the construction:

\par 

\texttt{grep -v \^{}=}


\par 
Actually, this is not the most prudent construction.
Depending on the spines present in a document,
sometimes barlines will be mixed
with null tokens in other spines that do not encode explicit barlines.
E.g.

\par 

\texttt{$\backslash$.=23=23..=23}


\par 
A more careful way of eliminating barlines would use the
following regular expression:

\par 

\texttt{egrep -v '\^{}($\backslash$.	)*='}


\par 
That is, eliminate all lines that either begin with an equals-sign,
or have one or more leading null tokens followed by a token
with a leading equals-sign.
Since this is somewhat complicated to remember,
we might alias it.
In the following command, we have created a new command
called \texttt{nobarlines}:

\par 

\texttt{alias nobarlines='egrep -v '\^{}($\backslash$.	)*='}



\par 
In Humdrum, a good use of aliases is to define commonly used
regular expressions.
Consider the regular expression used to define tandem interpretations
that encode meter signatures.
Here we are searching for an asterisk at the beginning of a line,
followed by the upper-case letter `M' followed by a digit,
followed by zero or more digits, followed by a slash,
followed by a digit:

\par 

\texttt{grep '\^{}$\backslash$*M[0-9][0-9]*/[0-9]' inputfile}


\par 
Actually, this regular expression will fail to find any meter signature
that is not in the first spine.
A more circumspect regular expression will include the possibility
of a leading tab:

\par 

\texttt{grep '	*$\backslash$*M[0-9][0-9]*/[0-9]' inputfile}


\par 
Since this is a cumbersome regular expression,
it can help to provide an alias.
Here we have aliased the regular expression to the name \texttt{metersig}:

\par 

\texttt{alias metersig="'	*$\backslash$*M[0-9][0-9]*/[0-9]'"}


\par 
Now we can search for meter signatures as follows:

\par 

\texttt{grep metersig inputfile}




\subsection*{Reprise}

\par 
In this chapter we have discussed how the shell interprets
the dollars sign (\$), the greve (`), and the double quote (").
When followed by printable characters, the dollars sign is interpreted
as designating the value of a shell variable.
Any command enclosed between two greve characters is executed
by the shell first, and the returned output of the command is available
as an input parameter to some other command.
Like single quotes, double quotes can be used to escape special shell
characters; however, an important difference is that the dollars-sign
retains its special meaning within the double quotes.
This allows shell variables to be embedded into text strings.

\par 
We have also learned that the shell \textbf{alias} command can be
used to provide a convenient short-hand or way of abbreviating
a complex pipeline or regular expression into a single
user-defined keyword.

\\
\begin{itemize}
\item 

\textbf{Next Chapter}
\item 

\textbf{Previous Chapter}
\item 

\textbf{Table of Contents}
\item 

\textbf{Detailed Contents}
\\\\

� Copyright 1999 David Huron
\end{document}
