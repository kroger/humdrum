% This file was converted from HTML to LaTeX with
% Tomasz Wegrzanowski's <maniek@beer.com> gnuhtml2latex program
% Version : 0.1
\documentclass{article}
\begin{document}



  
  
    
      
      
      
    
  



\\
\\

\section*{Chapter26}


[\textit{Previous Chapter}]
[\textit{Contents}]
[\textit{Next Chapter}]


\section*{Moving Signifiers Between Spines}



In many types of tasks is it useful to be able
to transfer signifiers from one spine to another.
In this chapter we will discuss two commands.
The
\textbf{rend}
command makes it possible to split characters in a single spine
and distribute them across two or more spines.
The
\textbf{cleave}
command does the reverse:
it allows characters that are distributed
across two or more spines to be gathered into a single new spine.
These two commands provide opportunities to create Humdrum
spines that contain precisely the information of interest to the user.


\subsection*{The \textit{rend} Command}

\par 
The
\textbf{rend}
command allows a Humdrum spine to be broken apart into two or more spines.
Different pieces of information can be distributed to the individual
output spines.
Consider for example the following spine containing
\texttt{**pitch}
data:
\\\\

\texttt{**pitch
Ab3
F\#4
C5
*-}

The
\textbf{rend}
command might be used to structure this as three independent spines:
\\\\

\texttt{**octave**note**accidental
3Abb
4F\#\#
5C.
*-*-*-}

The operation of
\textbf{rend}
requires a reassignment file where each line contains an
output exclusive interpretation, followed by a tab, followed by
a regular expression.
In the above case, the reassignment file (\texttt{reassign})
contained the following:

\texttt{**octave[0-9]
**note[A-Gb\#x]
**accidental[b\#x]}


\par 
The first line tells
\textbf{rend}
that any signifiers matching the character-class 0-9 should be
output in a spine labelled \texttt{**octave}.
The second line causes signifiers matching the upper-case letters A to G
and the \texttt{b}, \texttt{\#} and \texttt{x} signifiers to be output
in a spine labelled \texttt{**note}.
The third line causes signifiers matching
just \texttt{b}, \texttt{\#} and \texttt{x} to be output
in a spine labelled \texttt{**accidental}.

\par 
The above output was generated by invoking the following command:

\par 

\texttt{rend -i '**pitch' -f reassign} \textit{inputfile}


\par 
Note that the
\textbf{-i}
and
\textbf{-f}
options are mandatory.
The
\textbf{-i}
option tells
\textbf{rend}
which input spines to process and the
\textbf{-f}
option tells
\textbf{rend}
the name of the file containing the spine-reassignments.

\par 
The
\textbf{rend}
command is typically paired with a subsequent
\textbf{cleave}
command.


\subsection*{The \textit{cleave} Command}

\par 
The
\textbf{cleave}
command amalgamates concurrent data tokens in two or more
spines into a single data spine.
In effect,
\textbf{cleave}
does the opposite of
\textbf{rend}.
By way of example,
\textbf{cleave}
can be used to transform the following:

\texttt{**A**B**C
abc
ABC
*-*-*-}

into:

\texttt{**new
abc
ABC
*-}

Specifically, the above "cleaving" would be done using
the following command:

\par 

\texttt{cleave -i '**A,**B,**C' -o '**new'} \textit{inputfile}


\par 
Both the
\textbf{-i}
and
\textbf{-o}
options are mandatory.
The
\textbf{-i}
option tells
\textbf{cleave}
which exclusive interpretations should be cleaved together.
In the above case, we have provided a list of three types of data.
The
\textbf{-o}
option tells
\textbf{cleave}
what to call the resulting cleaved spine.
In this case, we've simply called the result \texttt{**new}.

\par 
Suppose that we would like to automatically add key-velocities
to some
\texttt{**MIDI}
data that reflect the normal accents arising from the meter.
For example, in 4/4 meter, we would like the first note in
each measure to be strongest, the third beat to be next
most strongest and so on.
Recall that \texttt{**MIDI} data tokens consist of three
elements: (1) the duration in MIDI clock ticks,
(2) the key number (also on/off indication), and
(3) the key velocity.
In order to add accents, we need to change the key velocity values.
The maximum MIDI key velocity value is 127;
the minimum value is 0; and the default value is 64.

\par 
Consider the following hypothetical input file:

\texttt{**kern
*M4/4
=1-
4c
8d
8e
8f
8g
8a
8b
=2
2cc
*-}

The
\textbf{metpos}
command can be used to identify the metric position of various
note onsets.
Before using
\textbf{metpos}
however, we must use the
\textbf{timebase}
command to create an isorhythmic record structure.
Since the shortest note is an eighth-note, the appropriate command is
as follows.

\par 

\texttt{timebase -t 8 scale > scale.tb}


\par 
Using the timebased output, we can then invoke the
\textbf{metpos}
command:

\par 

\texttt{metpos scale.tb > scale.met}


\par 
The resulting output contains both the original input (on the left)
and the metric position spine (on the right):

\texttt{**kern**metpos
*M4/4*M4/4
*tb8*tb8
=1-=1-
4c1
.4
8d3
8e4
8f2
8g4
8a3
8b4
=2=2
2cc1
.4
.3
.4
*-*-}

Let's now eliminate the null data tokens introduced by
\textbf{timebase}.
Using
\textbf{humsed},
we delete each data record beginning with a period
character:

\par 

\texttt{humsed '/\^{}$\backslash$./d' scale.met > scale.tmp}


\par 
Next, we can use the
\textbf{recode}
command to change the metric position values to appropriate MIDI
key velocities.
We might use the following reassignment file (named \texttt{accent}):

==1100
==280
==360
==440
elseerror


\par 
In applying \textbf{recode} we will take care to avoid processing measure
numbers using the
\textbf{-s}
(skip) option:

\par 

\texttt{recode -f accent -s \^{}= -i '**metpos' scale.tmp > scale.acc}


\par 
The output will now appear as follows:

\texttt{**kern**metpos
*M4/4*M4/4
*tb8*tb8
=1-=1-
4c100
8d60
8e40
8f80
8g40
8a60
8b40
=2=2
2cc100
*-*-}

Now we can use the
\textbf{midi}
command to generate \texttt{**MIDI} data:


\par 
\texttt{midi scale.acc > scale.mid}

\par 

The result is given below:

\texttt{**MIDI**metpos
*Ch1*
*M4/4*M4/4
*tb8*tb8
=1-=1-
72/60/64100
72/-60/64 72/62/6460
36/-62/64 36/64/6440
36/-64/64 36/65/6480
36/-65/64 36/67/6440
36/-67/64 36/69/6460
36/-69/64 36/71/6440
=2=2
36/-71/64 36/72/64100
144/-72/64.
*-*-}

Before using
\textbf{cleave}
to join the new key velocity values to the \texttt{**MIDI} data
we need to delete the current key-down velocities.
These are the values `64' preceding the tab character.
The
\textbf{humsed}
command can be used as follows:

\par 

\texttt{humsed 's/64}\textit{tab}/\textit{tab}\texttt{/' scale.mid > scale.tmp}


\par 
The modified output will now be:

\texttt{**MIDI**metpos
*Ch1*
*M4/4*M4/4
*tb8*tb8
=1-=1-
72/60/100
72/-60/64 72/62/60
36/-62/64 36/64/40
36/-64/64 36/65/80
36/-65/64 36/67/40
36/-67/64 36/69/60
36/-69/64 36/71/40
=2=2
36/-71/64 36/72/100
144/-72/.
*-*-}

Finally, we use
\textbf{cleave}
to add the new key-down velocities.

\par 

\texttt{cleave -i '**MIDI,**metpos' -o '**MIDI' scale.tmp > scale.mid}


\par 
The final output is:

\texttt{**MIDI
*
*M4/4
*tb8
=1-=1-
72/60/100
72/-60/64 72/62/60
36/-62/64 36/64/40
36/-64/64 36/65/80
36/-65/64 36/67/40
36/-67/64 36/69/60
36/-69/64 36/71/40
=2=2
36/-71/64 36/72/100
144/-72/
*-}



\subsection*{Creating Mixed Representations}

\par 
For some analytic tasks it is often useful to generate a special representation
that combines all of the elements or types of data of interest to the researcher.
For example, suppose we were working on a model of melodic organization that
reduced melodies to three types of information:
relative-duration context, gross pitch height, and scale step.
Sample data tokens for our representation and their meanings are given in
the following table.
Notice that the order of signifiers is important:
\\\\

\textbf{token\textbf{meaning}

\texttt{LSLHto}long-short-long rhythm, high pitch, tonic
\texttt{LLSLsd}long-long-short rhythm, low pitch, subdominant
\texttt{MLSMlt}medium-long-short rhythm, medium pitch, leading tone
\texttt{r}rest

In
Chapter 22
we learned how to use
\textbf{recode}
to classify various numerical ranges and
\textbf{humsed}
to classify non-numeric data.
We already know how to create the elements of our
new representation.

\par 
The scale degree information can be created by using
\textbf{deg}
and
\textbf{humsed}
can be used to transform the signifiers as in the following \texttt{degree}
file:

\par 

\texttt{s/1.*/to/
\\
s/2.*/st/
\\
s/3.*/me/
\\
s/4.*/sd/
\\
s/5.*/do/
\\
s/6.*/sm/
\\
s/7.*/lt/}


\par 
We can classify the pitch ranges into high, medium, and low
using the
\textbf{semits}
command, followed by \textbf{recode}.
For example, we could transform the
\texttt{**semits}
data using the following reassignment file:

\texttt{<0L
>16H
>=0M
elser}


\par 
Durations can be similarly classified into
long (L), medium (M), and short (S) using the
\textbf{dur}
command, followed by \textit{recode}.

\texttt{>1.0L
>0.5M
>0S}


\par 
Using
\textbf{context} \textbf{-n 3}
we could then create contextual `triples'
so that data records contain three durations.
Suppose also that we have used
\textbf{sed}
to change the names of the exclusive interpretations
so they are more appropriate.
As a result we have three spines that, when assembled together
are organized as in Example 26.1

\par 
\textbf{Example 26.1}

\par 

\texttt{**rhythm**range**scale-step
L S LHto
L L SLsd
M L SMlt
rrr
*-*-*-}

We need to process the first spine with
\textbf{humsed}
again to eliminate the spaces in the multiple stops.
The rhythm spine would be processed as follows:

\par 

\texttt{humsed 's/ //g' rhythm > rhythm.new}


\par 
We could assemble these spines using the
\textbf{assemble}
command:

\par 

\texttt{assemble rhythm.new range scale.step > newfile}


\par 
Finally we can use
\textbf{cleave}
to amalgamate all of the data into a single final spine.

\par 

\texttt{cleave -i '**rhythm,**range,**scale-step' -o '**complex' $\backslash$

newfile > output}




\par 
Having created our new representation, we can continue
to process this new data with the various Humdrum tools.
For example, we could generate inventories that answer
questions such as "How often does a high subdominant note in
a long-short-long rhythmic follow a low submediant in a long-long-short context?


\par 
A similar approach can be used to address other questions,
such as whether large leaps involving chromatically-altered tones
tend to have a longer duration on the altered tone.
Etc.



\subsection*{Reprise}

\par 
In this chapter we have seen how
\textbf{rend}
and
\textbf{cleave}
can be used to take bits and pieces of signifiers from
potentially many spines, and assemble a composite Humdrum spine that
contains precisely the information of interest.
Before amalgamating spines, you can use the
\textbf{humsed}
command to translate the characters/signifiers
so that you use your preferred way of representing something.

\\
\begin{itemize}
\item 

\textbf{Next Chapter}
\item 

\textbf{Previous Chapter}
\item 

\textbf{Table of Contents}
\item 

\textbf{Detailed Contents}
\\\\

� Copyright 1999 David Huron
\end{document}
