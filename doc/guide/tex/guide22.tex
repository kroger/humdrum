% This file was converted from HTML to LaTeX with
% Tomasz Wegrzanowski's <maniek@beer.com> gnuhtml2latex program
% Version : 0.1
\documentclass{article}
\begin{document}



  
  
    
      
      
      
    
  



\\
\\

\section*{Chapter22}


[\textit{Previous Chapter}]
[\textit{Contents}]
[\textit{Next Chapter}]


\section*{Classifying}



Many of the most important analytic tasks involve classifying
or categorizing various things.
In this chapter we will discuss two general approaches to classifying:
\textit{parametric}
classifying and
\textit{non-parametric}
classifying.
In the first instance, we will see how numerical data can be
categorized according to arithmetic ranges.
We will then revisit the
\textbf{humsed}
command and learn how it can be used to classify different types
of non-numeric data tokens.


\subsection*{The \textit{recode} Command}

\par 
Suppose that we have a Humdrum spine that contains numerical
information representing the moment-to-moment heart-rate of a listener.
Heart rate is related to arousal level
and so we might use our data to identify passages that tend to arouse listeners.
Since the average heart-rate of listeners differs,
we are interested primarily in the rate-of-change.
We can use the
\textbf{xdelta}
command to calculate the differences in heart-rate
between successive values.

\par 

\texttt{xdelta -s = heart.dat > changes}


\par 
The example below displays the input (left) spine and the corresponding
output (right) spine for the above command:

\texttt{**heart*Xheart
=133=133
550
561
55-1
=134=134
583
56-2
55-1
=135=135
572
55-2
561

=136=136
55-1
605
622
=137=137
61-1
59-2
590
=138=138
*-*-}


\par 
A certain amount of heart-rate variation is to be expected
because of monitoring equipment and other variables.
So we are primarily interested in large changes of heart-rate,
such as the change occurring in measure 136.
The
\textbf{recode}
command allows us to classify numerical data according to value or range.
In the above case, we may be interested in identifying acceleration
or decelerations that exceed some threshold.
The
\textbf{recode}
command requires that the user supply a
\textit{reassignment file}
that specifies how numerical values are to be reassigned.
In our heart-rate application, we might create the following
reassignment file, named \texttt{reassign}.
Reassignment files obey the following syntax:
for each line,
\textit{conditions}
are given on the left followed by a single tab,
followed by a
\textit{reassignment string.}

\texttt{>3+event
<-3-event
else.}


\par 
The above reassignment file may be interpreted as follows:
if the numerical value is greater than 3,
then output the string "\texttt{+event}";
if the numerical value is less than -3,
then output the string "\texttt{-event}";
otherwise output a string consisting of an isolated period (\texttt{.}).
We can invoke an appropriate command as follows:

\par 

\texttt{recode -f reassign -i '**Xheart' -s \^{}= changes}


\par 
The
\textbf{-f}
option is required, and is used to identify the file containing the
reassignment information.
The
\textbf{-i}
option is also required, and is used to identify the
exclusive interpretation for the data to be processed.
The
\textbf{-s}
option tells
\textbf{recode}
to skip records matching some specified regular expression -$\,$-
in this case, to skip barlines.
Finally, "\texttt{changes}" is the name of our input file.

\par 
The result of applying this process to the right-most spine in
the above example is given below:

\texttt{*Xheart
=133
\texttt{.
\texttt{.
\texttt{.
=134
\texttt{.
\texttt{.
\texttt{.
=135
\texttt{.
\texttt{.
\texttt{.
=136
\texttt{.
+event
\texttt{.
=137
\texttt{.
\texttt{.
\texttt{.
=138
*-}


\par 
Notice that we have used
\textbf{recode}
to drastically reduce the volume of data by
transforming the input into a set of more basic cateogires.

\par 
Having constructed our new output spine,
we can further process this information in various ways.
For example, we might assemble this spine to our original musical score.
Then we might then use
\textbf{grep -n}
to located any points in the score where a heart-rate related
event has occurred.

\par 
Permissible relational operators used by
\textbf{recode}
include the following:

==equals
!=not equal
<less than
<=less than or equal
>greater than
>=greater than or equal
elsedefault relation


\par 
Conditions are tested in the order given in the reassignment file.
Thus if a numerical value satisfies more than one condition,
only the first string replacement is made.
Consider the following reassignment file:

\texttt{<=0LOW
>100HIGH
>0MEDIUM}


\par 
The order of specification is important here.
If the \texttt{MEDIUM} condition was specified prior to the \texttt{HIGH}
condition, then all values greater than one hundred would be
categorized as \texttt{MEDIUM} rather than as \texttt{HIGH}.
Only a single
\textbf{else}
condition is allowed in a reassignment file;
when it is present, the else statement should appear
as the last reassignment.


\subsection*{Classifying Intervals}

\par 
The
\textbf{recode}
command has innumerable applications.
Suppose we wanted to determine how frequently ascending melodic leaps
are followed by a descending step.
Let's consider two different ways of distinguishing steps and leaps:
a "semitone" method and a "diatonic" method.
In the first method, we might define a step interval as either one or two
semitones.
Our reassignment file (dubbed "\texttt{reassign}") might appear as follows:

\texttt{>=3up-leap
>0up-step
==0unison
>=-2down-step
<=-3down-leap}


\par 
Given this reassignment file, we can now begin our processing.
In the first method, we translate to semitone data using
\textbf{semits},
translate to semitone-differences using
\textbf{xdelta},
and then classify into five interval types using
\textbf{recode}.
The
\textbf{context}
\textbf{-n 2}
command will create pairs of interval types, then
\textbf{rid},
\textbf{sort}
and
\textbf{uniq -c}
are used to generate an
inventory.
Finally, we use
\textbf{grep}
to identify what happens following ascending leaps:

\par 

\texttt{semits melody | xdelta -s = | recode -f reassign $\backslash$

-i '**Xsemits' -s = | context -n 2 | rid -GLId | sort $\backslash$
\\
| uniq -c | grep 'up-leap .*\$'}



\par 
An alternative way of distinguishing steps from leaps
is by diatonic interval.
For example, we might consider a diminished third to be a leap,
while an augmented second may be considered a step.
In this case, we can use the
\textbf{mint}
command to determine the melodic interval size; the
\textbf{-d}
option limits the output to diatonic intervals and excludes
the interval quality (perfect, major, minor, etc.).
The appropriate reassignment file would be:

\texttt{>=3up-leap
==2up-step
==1unison
==-2down-step
<=-3down-leap}


\par 
The appropriate command pipe would be:

\par 

\texttt{mint melody | xdelta -s = | recode -f reassign -i '**mint' $\backslash$

-s = | context -n 2 | rid -GLId | sort | uniq -c $\backslash$
\\
| grep 'up-leap .*\$'}




\subsection*{Clarinet Registers}

\par 
Consider another use of the
\textbf{recode}
command.


Imagine that we wanted to arrange Claude Debussy's
\textit{Syrinx}
for soprano clarinet instead of flute.
Our principle concern as arranger is determining what key
would be especially well suited to the clarinet.
Tone color is particularly important for this piece.
The clarinet has four fairly distinctive tessituras as
shown in Example 22.1.
These are the
\textit{chalemeau}
register (dark and rich), the
\textit{clarion}
register (bright and clear), the
\textit{altissimo}
register (very high and piercing), and the
\textit{throat}
register (weak and breathy).
\\\\
\textbf{Example 22.1.}  Clarinet registers (notated at concert pitch).



\textit{chalemeau\textit{throat}\textit{clarion}\textit{altissimo}




\par 
Suppose we wanted to pick a key that satisfies two conditions:
(1) it is not out of range for the clarinet, and
(2) it minimizes the number of notes played in the throat register.
We can use
\textbf{recode}
to classify all pitches according to the following reassignments:

\texttt{>=30too-high
>=23altissimo
>=8clarion
>=5throat-register
>=-10chalemeau
elsetoo-low}


\par 
Now we simply explore various transpositions using
\textbf{trans}
and create an inventory of pitch types.
For Debussy's \textit{Syrinx}, the minimum number of throat tones
(without exceeding the clarinet's range) occurs when we
transpose down a major sixth:

\par 

\texttt{trans -d -5 -c -9 syrinx | semits | recode -f reassign $\backslash$

-i '**semits' -s = | rid -GLId | sort | uniq -c}




\subsection*{Open and Close Position Chords}


\par 
Inputs to the
\textbf{recode}
command can be quite sophisticated.
Consider, for example, the task of
classifying chords as "open" or "close" position.
According to one definition,
a chord is said to be in "open" position when the
the interval separating the soprano and tenor voices is
an octave or greater.
One music theorist has claimed that close position chords are
more common than open position.
How might we test this?

\par 
In determining an appropriate sequence of Humdrum commands,
it is often helpful to work backwards from our goal.
We'd like to end up with a spine that simply encodes the
words "open" or "close" for each sonority.
This classification will be based on the distance separating
the soprano and tenor voices.
Our reassignment file might be as follows:

\texttt{<=12close}
\texttt{>12open}


\par 
We will need to extract the soprano and tenor voices,
translate the pitch representation to
\texttt{**semits}
and use
\textbf{ydelta}
to calculate the semitone distance between the two voices.
In the following set of commands, we have also added the
\textbf{ditto}
command to ensure that there are semitone values for each sonority.

\par 

\texttt{extract -i '*Itenor,*Isopran'} \textit{inputfile}\texttt{ | semits -x | ditto $\backslash$

| ydelta -s = -i '**semits' | recode -f reassign $\backslash$
\\
-i '**Ysemits' -s = > tempfile

grep -c 'open' tempfile
\\
grep -c 'close' tempfile}


\par 
The
\textbf{grep -c}
commands tell us whether open position sonorities
are more common than close position sonorities.


\subsection*{Flute Fingering Transitions}

\par 
There is no fixed limit to the length of a reassignment file.
Consider for example, the following file named \texttt{map}.
Each \texttt{**semits} value from C4 (0) to C7 (36) has been
assigned to a schematic representation of flute fingerings.
The letter `X' indicates a closed key, whereas the letter `O'
indicates an open key.
The first letter pertains to the left thumb;
the next group of four letters pertain to the ensuing fingers
of the left hand;
the final group of letters pertain to the right-hand fingers.
The little finger of the right hand is able to play three keys
(labelled X, Y, and Z).
Fingerings are shown only for the first octave (from C4 to C5):

\texttt{<0out-of-range
\texttt{==0\texttt{X-XXXO-XXXZ
\texttt{==1\texttt{X-XXXO-XXXY
\texttt{==2\texttt{X-XXXO-XXXO
\texttt{==3\texttt{X-XXXO-XXXX
\texttt{==4\texttt{X-XXXO-XXOX
\texttt{==5\texttt{X-XXXO-XOOX
\texttt{==6\texttt{X-XXXO-OOXX
\texttt{==7\texttt{X-XXXO-OOOX
\texttt{==8\texttt{X-XXXX-OOOX
\texttt{==9\texttt{X-XXOO-OOOX
\texttt{==10\texttt{X-XOOO-XOOX
\texttt{==11\texttt{X-XOOO-OOOX
\texttt{==12\texttt{O-XOOO-OOOX}

etc.

\texttt{elserest}


\par 
Suppose we wanted to determine what kinds of fingering
\textit{transitions}
occur in Joachim Quantz's flute concertos.
Since instrument fingerings are insensitive to enharmonic spelling,
an appropriate input representation would be \texttt{**semits}.
Having used
\textbf{recode}
to translate the pitches to fingerings,
we can then use
\textbf{context -n 2}
to generate diads of successive finger combinations.

\par 

\texttt{semits con* | recode -f map -s = | context -n 2 -o = > fingers}


\par 
For example, if our input contains the pitch G5 followed by B4,
the appropriate data record in the \texttt{fingers} file would be the
following Humdrum double-stop:

\par 

\texttt{X-XXXO-OOOX X-XOOO-OOOX}


\par 
We could create an inventory of finger transitions by continuing the
processing:

\par 

\texttt{rid -GLI fingers | sort | uniq -c | sort -n}



\par 
We could create a similar reassignment file containing fingers pertaining
to the pre-Boehm flute.
Suppose the revised reassignment file was called \texttt{premodern}.
We could determine how the finger transitions differ between
the pre-Boehm traverse flute and the modern flute.
In
Chapter 29
we will see how the
\textbf{diff}
command can be used to identify
differences between two spines.
This will allow us to identify specific places in the score
where Baroque and modern fingerings differ.

\par 
The
\textbf{recode}
command can be used for innumerable other kinds of classifications.
For example,
\texttt{**kern}
durations might be expressed in seconds
(using the
\textbf{dur}
command), and the elapsed times then classified as
\textit{long}, \textit{short} and \textit{medium} (say).
Sound pressure levels (in decibels) might be classified
as dynamic markings (\textbf{\textit{ff}}, \textbf{\textit{mf}}, \textbf{\textit{mp}}, \textbf{\textit{pp}}, etc.),
and so on.


\subsection*{Classifying with \textit{humsed}}

\par 
The
\textbf{recode}
command is restricted to classifying numerical data only.
For many applications, it is useful to be able to classify data
according to non-numerical criteria.
As we saw in
Chapter 14,
stream editors such as
\textbf{sed}
and
\textbf{humsed}
provide automated substitution operations.
Such string substitutions can be used for non-parametric classifying.
We can illustrate this with
\textbf{humsed.}


\par 
Suppose we wanted to classify various flute finger-transitions
as either \textit{easy}, \textit{moderate} or \textit{difficult}.
For example, F4 to G4 is an easy fingering,
E5 to A5 is a moderate fingering, whereas C5 to D5
is difficult.
As before, it is best to use a semitone representation
so we don't need to consider differences in enharmonic pitch spelling.
We can use the
\textbf{semits}
command to transform all pitches.
Then we can use
\textbf{context -n 2}
to generate pairs of successive pitches as double-stops.
We can then create a
\textbf{humsed}
script file (let's call it \texttt{difficulty})
containing substitutions such as the following:

\texttt{s/5 7/easy/}[i.e. F4 to G4]
\texttt{s/16 21/moderate/}[i.e. E5 to A5]
\texttt{s/12 14/difficult/}[i.e. C5 to D5]
etc.


\par 
We can apply the script as follows:

\par 

\texttt{humsed -f difficulty sonata*}


\par 
Since there are a large number of possible pitch transitions,
our script file is apt to be especially large.
However, notes an octave apart have a high likelihood of
having identical fingerings on the modern flute.
A more succinct
\textbf{humsed}
script would deal with fingering transitions rather than pitch transitions.

\par 

\\
\texttt{s/X-XXXO-XOOX X-XXXO-OOOX/easy/
\\
s/X-XXXO-XXOX X-XXOO-OOOX/moderate/
\\
s/O-XOOO-OOOX X-OXXO-XXXO/difficult/}
\\
etc.


\par 
The three substitutions shown above apply to many more pitch
transitions than the original transitions F4-G4, E5-A5, and C5-D5.
The above three substitutions apply also to F5-G5, F5-G4, F4-G5,
E4-A4, E4-A5, and E5-A4.

\par 
Having created a file classifying all fingering transitions as
"easy," "moderate" or "difficult,"
we can characterize our Quantz flute concertos using the following pipeline:

\par 

\texttt{semits Quantz* | recode -f map -s = | context -n 2 -o = $\backslash$

| humsed -f difficulty}



\par 
The output will be a single spine that classifies the difficulty of
all fingering transitions.


\subsection*{Classifying Cadences}

\par 
Consider another application where we use
\textbf{humsed}
to classify cadences.
Suppose we have Roman-numeral harmonic data (as provided by the
\texttt{**harm}
representation).
In the case of Bach's chorale harmonizations, for example,
cadences are clearly evident by the presence of pauses
(designated by the semicolon).
We can easily create a spine that identifies only cadences.
Consider a suitable reassignment file (dubbed \texttt{cadences}):

\par 

\\
\texttt{s/V I;/authentic/
\\
s/V7 I;/authentic/
\\
s/V i;/authentic/
\\
s/V7 i;/authentic/
\\
s/IV I;/plagal/
\\
s/iv i;/plagal/
\\
s/iv I;/plagal/
\\
s/V vi;/deceptive/
\\
s/V VI;/deceptive/}
\\

\\
etc.
\\

\\
\texttt{s/\^{}[IiVv].*\$/./}


\par 
(The precise file will depend on your preferred way of
labeling cadences.)
Remember that, unlike the
\textbf{recode}
command, all of the substitutions in a
\textbf{humsed}
or
\textbf{sed}
script are applied to every input line.
The final substitution causes any record beginning
with either an \textit{i}, \textit{i}, \textit{v} or \textit{V} to be
changed to a null data token.
In effect, any progression that is not deemed to be an authentic,
plagal or deceptive cadence is transformed to a null data record.
Using the above reassignment file, we could create a cadence
spine using the following pipeline:

\par 

\texttt{extract -i '**harm' chorales | context -o = -n 2 $\backslash$

| humsed -f cadences | sed 's/$\backslash$*$\backslash$*harm/**cadences/'}



\par 
We first extract the \texttt{**harm} spine using
\textbf{extract}.
We then generate a sequence of two-chord progressions using
\textbf{context}
-$\,$-  taking care to omit barlines (\texttt{-o =}).
We then use
\textbf{humsed}
to run the script of cadence-name substitutions.
Finally, we use the
\textbf{sed}
command to change the name of the exclusive interpretation from \texttt{**harm}
to something more suitable -$\,$- \texttt{**cadences}.

\par 
Many more sophisticated variants of this sort of procedure may be used.
For example, one could first classify harmonies more broadly.
In so-called "functional" harmony, for example,
supertonic chords in first inversion are normally considered to be
subdominant functions.
One could construct a whole series of re-write rules that classify
harmonies in a variety of ways.


\subsection*{Orchestration}

\par 
One of the simplest classifications in a musical score is
whether or not an instrument is sounding or resting.
Suppose we extracted the viola part from Beethoven's Symphony No. 1.
We might use the
\textbf{ditto}
command to ensure that each data record encodes either a note, rest,
or barline:

\par 

\texttt{extract -i '*Iviola' symphony1 | ditto -s =}


\par 
Let's append to this pipeline a
\textbf{humsed}
command that makes two string substitutions.
The first substitution replaces all data records containing the lower-case
letter \texttt{r} (i.e., rests) with the string \texttt{-viola}.
The second substitution changes any record that does not begin with
either a minus sign or an equals sign to the string \texttt{+viola}.
In effect, we've transformed the viola part so that all data tokens
encode either \texttt{+viola}, \texttt{-viola} or are barlines.

\par 

\texttt{extract -i '*Iviola' symphony1 | ditto -s = $\backslash$

| humsed 's/.*r.*/-viola/; /s/\^{}[\^{}-=].*\$/+viola/' > viola}



\par 
Now imagine that we repeat this process for every instrument in
Beethoven's Symphony No. 1.
In each case, we substitute the name of the instrument (preceded by
a plus-sign or minus-sign) for the various note or rest tokens.

\par 

\texttt{extract -i '*Iflt' symphony1 | ditto -s = $\backslash$

| humsed 's/.*r.*/-flt/; /s/\^{}[\^{}-=].*\$/+flt/' > flt}

\texttt{extract -i '*Ioboe' symphony1 | ditto -s = $\backslash$

| humsed 's/.*r.*/-oboe/; /s/\^{}[\^{}-=].*\$/+oboe/' > oboe}

\texttt{extract -i '*Iclars' symphony1 | ditto -s = $\backslash$

| humsed 's/.*r.*/-clars/; /s/\^{}[\^{}-=].*\$/+clars/' > clars}

\texttt{extract -i '*Ifagot' symphony1 | ditto -s = $\backslash$

| humsed 's/.*r.*/-fagot/; /s/\^{}[\^{}-=].*\$/+fagot/' > fagot}


\par 
etc.


\par 
When we are finished, we reassemble all of the transformed parts into a complete score.

\par 

\texttt{assemble cbass cello viola violn2 violn1 tromb tromp fagot $\backslash$

clars oboe flt > orchestra}



\par 
We now have a file that contains data records that look something
like the following excerpt:

\texttt{+cbass+cello+viola+violn+violn-tromb-tromp+fagot-clars+oboe+flt
+cbass+cello-viola-violn+violn-tromb-tromp+fagot-clars+oboe+flt
+cbass+cello+viola+violn+violn-tromb-tromp+fagot-clars+oboe+flt
+cbass+cello-viola-violn+violn-tromb-tromp+fagot-clars+oboe+flt
-cbass-cello+viola+violn+violn-tromb-tromp-fagot-clars+oboe+flt
-cbass-cello-viola-violn+violn-tromb-tromp-fagot-clars+oboe+flt
=131=131=131=131=131=131=131=131=131=131=131
+cbass+cello+viola+violn+violn-tromb-tromp+fagot-clars+oboe+flt
+cbass+cello-viola-violn+violn-tromb-tromp+fagot-clars+oboe+flt
-cbass-cello+viola+violn+violn-tromb-tromp-fagot-clars+oboe+flt
-cbass-cello-viola-violn+violn-tromb-tromp-fagot-clars+oboe+flt
+cbass+cello+viola+violn+violn-tromb-tromp+fagot-clars+oboe+flt
+cbass+cello-viola+violn+violn-tromb-tromp+fagot-clars+oboe+flt}

etc.


\par 
The first sonority indicates that all of the string instruments are playing,
that the brass are inactive, and that all of the woodwinds are sounding
with the exception of the clarinet.


\par 
A representation such as the above provides an opportunity to
study instrumental combinations in Beethoven's orchestration.
For example, the following command will count the number of sonorities
where the oboe and bassoon sound concurrently:

\par 

\texttt{grep -c '+fagot.*+oboe' orchestra}


\par 
It is better to express this count as a proportion of the total work.
We can count the total number of sonorities in the work
by omitting any leading plus or minus sign:

\par 

\texttt{grep -c 'fagot.*oboe' orchestra}



\par 
How often are the oboe and bassoon resting at the same time?

\par 

\texttt{grep -c '-fagot.*-oboe' orchestra}



\par 
Excluding \textit{tutti} sections, do the trumpet and flute tend to "repell"
each others' presence?

\par 

\texttt{grep '$\backslash$-' orchestra | grep -c '+tromp.*-flt' orchestra}
\\
\texttt{grep '$\backslash$-' orchestra | grep -c '+tromp.*+flt' orchestra}
\\
\texttt{grep '$\backslash$-' orchestra | grep -c '-tromp.*-flt' orchestra}
\\
\texttt{grep '$\backslash$-' orchestra | grep -c '-tromp.*+flt' orchestra}



\par 
When all of the woodwinds are playing, which of the remaining instruments is
Beethoven most likely to omit from the texture?


\texttt{grep '+fagot.*+clars.*+oboe.*+flt' orchestra | grep -c '-cbass'}
\\
\texttt{grep '+fagot.*+clars.*+oboe.*+flt' orchestra | grep -c '-cello'}
\\
\texttt{grep '+fagot.*+clars.*+oboe.*+flt' orchestra | grep -c '-viola'}
\\
\texttt{grep '+fagot.*+clars.*+oboe.*+flt' orchestra | grep -c '-violn'}
\\
etc.



\par 
Many refinements can be added to this basic approach.
For example, instead of classifying instruments as simply being
"present" or "absent,"
we might distinguish various registers for each instrument -$\,$- as we
did with the clarinet when describing \textbf{recode}.
We could then determine whether Beethoven tends to link, say,
activity in the chalemeau register of the clarinet with low register
activity in the strings.

\par 
Further refinements might include relating orchestration to structural
aspects of the music.
For example, we might use
\textbf{yank}
to extract sections of movements;
we could then compare possible differences
of orchestration between the first and second themes, for example.
Similarly, we could reduce instruments to instrument classes,
and examine how brass, woodwinds, strings, and percussion in general
are related.



\subsection*{Reprise}

\par 
A large number of analytic tasks simply involve classifying things.
In general, two sorts of classifying methods can be distinguished:
(1) a numerical or
\textit{parametric}
classification can be used to reassign various ranges of numerical
values into a finite set of classes or categories;
(2) a
\textit{non-parametric}
classification maps one set of words or terms into a second (usually smaller)
set of words (used to label various classes or categories).
In this chapter, we have seen that,
for any Humdrum representation, parametric classification can be done
using the
\textbf{recode}
command and non-parametric classification can be achieved using the
\textit{substitution}
operation provided by the
\textbf{humsed}
command.

\\
\begin{itemize}
\item 

\textbf{Next Chapter}
\item 

\textbf{Previous Chapter}
\item 

\textbf{Table of Contents}
\item 

\textbf{Detailed Contents}
\\\\

� Copyright 1999 David Huron
\end{document}
