% This file was converted from HTML to LaTeX with
% Tomasz Wegrzanowski's <maniek@beer.com> gnuhtml2latex program
% Version : 0.1
\documentclass{article}
\begin{document}



  
  
    
      
      
      
    
  



\\
\\

\section*{Chapter36}


[\textit{Previous Chapter}]
[\textit{Contents}]
[\textit{Next Chapter}]


\section*{Sound and Spectra}



Music is a sonic art and no analytic toolkit would be
complete without considering the representation and manipulation of
sound-related information.
In this chapter we introduce some special-purpose tools
related to sound analysis, sound synthesis,
and auditory perception.
We have already encountered the
\texttt{**freq}
and
\texttt{**cents}
representations in
Chapter 4.
Much of this chapter will center on the
\texttt{**spect}
representation.
Three tools will be discussed in connection with \texttt{**spect}:
the
\textbf{spect},
\textbf{mask}
and
\textbf{sdiss}
commands.
\textbf{Spect}
accesses a database of analyzed instrument
tones to generate harmonic spectra for all notes for various
orchestral instruments over their complete ranges;
\textbf{mask}
can be used to modify a spectrum so that
masked frequencies are attenuated in a manner that simulates human hearing;
\textbf{sdiss}
characterizes the degree of sensory dissonance for
arbitrary sonorities.

\par 
In addition, we will consider how Humdrum data can be
used in conjunction with non-Humdrum tools -$\,$- such
as digital sound editors, spectral analysis tools,
and general signal processing software,
In particular, we will discuss the
\textbf{kern2cs}
command which generates score data for the popular
\textit{Csound} digital sound synthesis language.


\subsection*{The \textit{**spect} Representation}

\par 
A useful predefined sound-related representations that
in Humdrum is the \texttt{**spect} scheme.
The
\texttt{**spect}
representation is used to represent successive
acoustic spectra.
Each data record represents a complete spectrum specified as a set
of concurrent discrete frequency components.
Each frequency component in the spectrum is represented
by a pair of numerical values separated by a semicolon (;).
These paired values encode the frequency and amplitude for a single
spectral component.
Frequency values are positive values representing \textit{hertz}.
Amplitude values are positive values representing the sound pressure level
in decibels (dB SPL).
Most sonorities consist of more than one pure tone component,
so \texttt{**spect} data records typically encode a number of multiple stops.

\par 
Example 36.1 shows a sample document containing five spectra and a barline.
The first data record encodes an ambient spectrum ("silence")
represented by the upper-case letter `\texttt{A}'.
Following this are two spectra, each consisting of three spectral
components:
the first spectrum consists of a 261 Hz tone at 47 dB SPL,
as well as frequencies at 523 Hz and 785 Hz at 57 dB SPL
and 35 dB SPL, respectively.
Following the barline are two data records that represent two different
amalgamations of the preceding two three-component spectra.
Notice that these two spectra are identical;
only the order of the components differs.
In the \texttt{**spect} representation there is no special
requirement that the spectral components be encoded in any particular order.
However, it is often convenient to have the components assembled
from left to right in ascending frequency order.
This can be achieved by passing an input through the
\textbf{spect}
command.

\par 
\textbf{Example 36.1}

\texttt{**spect
\texttt{A
\texttt{261;47 523;57 785;35
\texttt{330;57 659;35 989;27
\texttt{=1
\texttt{261;47 523;57 785;35 330;57 659;35 989;27
\texttt{261;47 330;57 523;57 659;35 785;35 989;27}



\subsection*{The SHARC Database and \textit{spect} Command}

\par 
More commonly,
the
\textbf{spect}
command is used to generate a
\texttt{**spect}
(acoustic spectral data) output from a
\texttt{**semits}
score input.
The
\textbf{spect}
command recognizes instrument tandem interpretations (e.g., *Iclarinet)
and fetches a corresponding spectral data file (\texttt{clarinet.spe}).
These files are derived from the SHARC database of music instrument
spectra created by Gregory Sandell (1991).
The file contains precise spectral measurements for recordings
of the instrument playing each note throughout the instrument's range.
Suppose that the input to
\textbf{spect}
contains the note F\#5 for oboe (i.e., semits value 18).
Then
\textbf{spect}
will retrieve the spectral information for a recording of an oboe playing
F\#5 and add it to the composite spectrum for the particular sonority.

\par 
If more than one instrument is playing concurrently, then
\textbf{spect}
will generate an output record representing the aggregate of all
the spectral components generated by all of the sounding instruments.

\par 
The SHARC database includes most orchestral instruments
including piccolo, E-flat clarinet, contrabassoon, etc.
The database also includes selected Medieval instruments
such as the soprano crumhorn and alto shawm.
Timbres for string instruments are distinguished according to
different playing methods including arco, vibrato, non-vibrato,
pizzicato, mute, and martello.


\subsection*{The \textit{mask} Command}

\par 
Masking is the tendency for sounds to obscure one another.
In many cases, masking may cause a sound to become completely inaudible.
This means that the physical presence of a frequency component
in a spectrum does not necessarily mean that the component
is audible to the listener.
Especially in the case of complex orchestral sonorities,
many of the acoustically present components are irrelevant
for human listeners.
A clear demonstration of this effect is evident in digital sound
recordings that have been processed using the
JPEG compression scheme.
JPEG-encoded audio is indistinguishable from the
original uncompressed audio, yet the scheme eliminates
those aspects of the sound which are masked.

\par 
The Humdrum
\textbf{mask}
command implements a common masking algorithm.
It accepts as input any \texttt{**spect} data,
and for each sonority modifies the spectrum so that
masked frequencies are attenuated accordingly.
So-called "forward" and "backward" masking are not
taken into account in this utlity.
No options are provided, and the command is invoked
as follows:

\par 

\texttt{mask} \textit{inputfile}\texttt{ > }\textit{outputfile}


\par 
Both the output and input to the
\textbf{mask}
command are \texttt{**spect} representations.
A tandem interpretation (\texttt{*masked}) is added to the output
to indicate that the sonorities already reflect the influence
of masking.
Some sound utilities (such as the \textbf{sdiss} command) already
take into account the effects of masking, and so an input
containing the \texttt{*masked} interpretation will cause
an error to be generated.
Similarly, the \textbf{mask} command itself will generate an
error if the input sonorities have already been modified
using the \textbf{mask} command.


\subsection*{The \textit{sdiss} Command}

\par 
A great deal of research has been carried out over the centuries
concerning the nature of consonance and dissonance.
This complex subject remains something of an enigma.
The perception of consonance or dissonance is known to be
affected by a number of factors, including past musical
experience and cultural milieu.
Perceptions of dissonance are even known to be influenced by the
personality of the listener.


\par 
Research by Donald Greenwood, Reiner Plomp, Wim Levelt, and others
has established that one aspect of dissonance perception
is related to the physiology of the ear.
This aspect of dissonance is referred to as low-level or
\textit{sensory dissonance}.

\par 
The
\textbf{sdiss}
command implements a measurement method for sensory
dissonance described by Kameoka and Kuriyagawa (1969a/b).
(The Humdrum
\textbf{sdiss}
command itself was written by Keith Mashinter.)
The
\textbf{sdiss}
command characterizes the degree of sensory dissonance for
successive vertical sonorities or acoustical moments.
The command accepts as input one or more \texttt{**spect} spines and
produces a single
\texttt{**sdiss}
spine as output.
For each \texttt{**spect} data record,
\textbf{sdiss}
produces a single numerical value representing the
aggregate sensory dissonance.
The greater the output value, the greater the dissonance.

\par 
Example 36.2 illustrates some sample inputs and outputs for
\textbf{sdiss}.
The left-most spine provides double-stops for
\texttt{**kern}
data for violin.
The middle spine provides corresponding \texttt{**spect} data
using the
\textbf{spect}
command.
The right-most spine shows the result of passing the \texttt{**spect}
data through the
\textbf{sdiss}
command.

\par 
\textbf{Example 36.2}

\texttt{**kern**spect**sdiss
\texttt{*Ivioln*Ivioln*Ivioln
\texttt{4c 4e..
\texttt{4G 4d..
\texttt{4f 4g..
\texttt{4e 4g..
\texttt{*-*-*-}


\par 
Note that sensory dissonance is known to be influenced by the
number of complex tones in the sonority.
That is, three-note sonorities are virtually always more
dissonant than three-note sonorities, etc.
However, it is known that increasing the number of notes in a
chord can sometimes reduce the perceived dissonance.
For example, the dyad of a major seventh generally sounds
more dissonant than the major-major-seventh (four-note) chord.
Consequently, it is problematic to compare sensory dissonance
values for sonorities consisting of different numbers of complex tones.
Further problems with the Kameoka and Kuriyagawa measurement method
are described in Mashinter (1995).


\subsection*{Connecting Humdrum with Csound -$\,$- the \textit{kern2cs} Command}

\par 
Apart from generating and processing acoustic spectra,
it is often convenient to be able to listen to the data.
Generating sounds from descriptions of acoustic spectra
cannot be done using MIDI synthesizers.
The sounds can be heard only by doing direct computer
sound synthesis using an audio-rate digital-to-analog converter.
A number of popular computer sound synthesis languages exist,
such as \textit{Csound} developed by Barry Vercoe (1993).
Most of these languages are inspired by the
\textit{Music 5}
language developed by Max Mathews (1969).

\par 
Typically, these languages divide the task of sound synthesis
into two representations called the \textit{score} and the \textit{orchestra}.
The
\textit{orchestra}
is an executable program, whereas the
\textit{score}
is a set of note- or event-related data that is "performed"
by the orchestra.
Typically, the
\textit{score}
consists of a series of note-records where each data record defines
several attributes for a single note.
Common attributes include the frequency (or pitch), amplitude,
duration, onset time, attack/decay envelope, spectral content, etc.
Example 36.3 shows a sample Csound score corresponding
to the opening measures of a Mozart clarinet trio.

\textbf{Example 36.3}  W.A. Mozart \textit{Clarinet Quintet}.





\texttt{; W.A. Mozart, Second trio from Clarinet Quintet
\texttt{f1 0 512 10 5 3 1 ; three harmonics in waveform table
\texttt{t0 96; tempo of 96 beats per minute



\texttt{; Instrument \#1
\texttt{; insttimedurationslur pitchvolstac
\texttt{i1.010.0000.50019.000.21.0; measure 1
\texttt{i1.010.5000.50039.040.21.0
\texttt{i1.011.0000.50039.070.21.0
\texttt{i1.011.5000.50039.040.21.0
\texttt{i1.012.0001.000310.000.21.0
\texttt{i1.013.0000.50039.070.21.0; measure 2
\texttt{i1.013.5000.50039.040.21.0
\texttt{i1.014.0000.50039.020.21.0
\texttt{i1.014.5000.50039.050.21.0
\texttt{i1.015.0001.00029.090.21.0

\texttt{; Instrument \#2
\texttt{i2.012.0001.00008.090.21.0
\texttt{i2.013.0001.00008.090.21.0; measure 2
\texttt{i2.015.0001.00008.090.21.0

\texttt{; Instrument \#3
\texttt{i3.012.0001.00008.040.21.0
\texttt{i3.013.0001.00008.040.21.0; measure 2
\texttt{i3.015.0001.00008.060.21.0

\texttt{; Instrument \#4
\texttt{i4.012.0001.00008.010.21.0
\texttt{i4.013.0001.00008.010.21.0; measure 2
\texttt{i4.015.0001.00007.110.21.0}�


� Example is modified from David Bainb�e (1997).

\par 
\textit{Csound}
is able to generate traditional 16-bit digital audio output.
It can also be used to generate AIFF files
(audio information file format) for greater portability.
\textit{Csound}
provides several other utilities for sound analysis,
including Fourier analysis and linear predictive coding.


\subsection*{Sound Analysis}

\par 
Humdrum does not provide any sound analysis tools \textit{per se}.
As we noted,
\textit{Csound}
provides utilities for Fourier analysis and linear predictive coding.
A wealth of software exists for sound analysis,
loudness estimation, mixing, sound card input/output,
CD audio input/output, and other sound-related applications.
(See Tranter, 1996 for a sampling of such multimedia applications.)
Other analysis methods are available through general-purpose 
signal analysis software such as
\textit{matlab}, \textit{mathematica}, and \textit{maple}.
Custom software has also been written by Humdrum users,
such as Kyle Dawkins Humdrum synchronization of CD audio disks.

\par 
Sound synthesis and analysis software has a rapid rate of
development.
Users should consult recent audio and multi-media resources
for up-to-date information.



\subsection*{Reprise}

\par 
In this chapter we have seen that Humdrum score-related data can be
transformed into spectral information using the \textbf{spect}
command.
This allows us to reconstitute a score as a sequence of sonorous
spectra -$\,$- which might be used for studies in timbre or orchestration.
The \textbf{mask} tool can be used to revise a spectral description
so that it reflects how listeners hear rather than the actual
acoustical information present.
The \textbf{sdiss} command can be used to characterize successive
spectra in terms of the estimated sensory dissonance.

\par 
We have also seen that Humdrum data can be connected to other
sound-related software, such as \textit{Csound}.
Since Humdrum data consists of simple ASCII text,
it is generally easy to write filters that allow the data
to be imported to a wide variety of existing sound analysis
and synthesis software.

\\
\begin{itemize}
\item 

\textbf{Next Chapter}
\item 

\textbf{Previous Chapter}
\item 

\textbf{Table of Contents}
\item 

\textbf{Detailed Contents}
\\\\

� Copyright 1999 David Huron
\end{document}
