% This file was converted from HTML to LaTeX with
% Tomasz Wegrzanowski's <maniek@beer.com> gnuhtml2latex program
% Version : 0.1
\documentclass{article}
\begin{document}



  
  
    
      
      
      
    
  



\\
\\

\section*{Chapter18}


[\textit{Previous Chapter}]
[\textit{Contents}]
[\textit{Next Chapter}]


\section*{Fingers, Footsteps and Frets}



Throughout this book, our examples have tended to rely on
just a handful of Humdrum representations -$\,$- \texttt{**kern}
in particular.
Of course the Humdrum syntax provides opportunities for an
unlimited number of representations.
In this chapter, we will consider some less common representations.
Most of the chapter will deal with the
\texttt{**fret}
representation -$\,$- a pre-defined Humdrum
representation for fretted instrument tablatures.
However, we will begin with a grab-bag of unorthodox representations.


\subsection*{Heart Beats and Other Esoterica}

\par 
The Humdrum syntax provides a framework within which different
symbol systems can be defined.
Each symbol system or representation scheme is denoted by a unique
exclusive interpretation.
There is no restriction on the number of schemes that can be used
or created by the user.
Each time you create a new exclusive interpretation,
all of the alphanumeric characters can be redefined.
A representation scheme springs into existence simply
by encoding some data.

\par 
In research activities, it is common to create representation
schemes for a specific task.
A user might define a spine that represents the heart-rate
(in beats per minute) of a music listener:

\texttt{**cardio
\texttt{76
\texttt{76
\texttt{74
\texttt{73
\texttt{73
\texttt{*-}

A scheme might be created to represent different types of
contrapuntal motion:

\texttt{**motion
\texttt{similar
\texttt{contrary
\texttt{parallel
\texttt{oblique
\texttt{*-}


\par 
Chords might be classified -$\,$- using words:

\texttt{**chords
\texttt{minor
\texttt{.
\texttt{augmented
\texttt{major
\texttt{*-}


\par 
Or using abbreviations:

\texttt{**chords
\texttt{m
\texttt{.
\texttt{A
\texttt{M
\texttt{*-}


\par 
Fingerings might be represented.
Each hand may have a separate spine:

\texttt{**finger**finger
\texttt{*left*right
\texttt{.1
\texttt{1 52
\texttt{.3
\texttt{.5
\texttt{1 52 5
\texttt{*-*-}


\par 
Or the hands might be combined in a single spine:

\texttt{**finger
\texttt{R1
\texttt{L1 L5 R2
\texttt{R3
\texttt{5
\texttt{L1 L5 R2 R5
\texttt{*-}


\par 
Time-scales might be large:

\texttt{**Periods
\texttt{Medieval
\texttt{Renaissance
\texttt{Baroque
\texttt{Classical
\texttt{Romantic
\texttt{*-}


\par 
Or miniscule:

\texttt{**milliseconds
\texttt{0.03
\texttt{0.8
\texttt{23.2
\texttt{31.6
\texttt{*-}


\par 
A user might define a highly refined special-purpose representation.
For example, the following scheme is fashioned after the Benesh
dance notation:

\texttt{!! Kellom Tomlinson's\texttt{Gavot of 1720.
\texttt{!! Transcribed from\texttt{Feuillet's notation.
\texttt{**kern\texttt{**Benesh
\texttt{*MM120\texttt{*MM120
\texttt{*M2/2\texttt{*M2
\texttt{*e:\texttt{*
\texttt{!! First Couplet
\texttt{!\texttt{! Half Coupee
\texttt{4.gg\texttt{ | |u| | |
\texttt{.\texttt{\%|+| | | |
\texttt{.\texttt{ |=|v| | |
\texttt{.\texttt{(| | | | |
\texttt{.\texttt{(|-| | | |\{
\texttt{8ff\#\texttt{ |\_| | | |2
\texttt{=1\texttt{-$\,$-$\,$-$\,$-$\,$-$\,$-$\,$-$\,$-$\,$-$\,$-$\,$-
\texttt{2ee\texttt{m| | | | |\}
\texttt{!\texttt{! Bound
\texttt{.\texttt{\_|\^{}| | | |\{
\texttt{.\texttt{(|+| | | |
\texttt{4.b\texttt{-| | | | |
\texttt{.\texttt{(|+| | | |\}
\texttt{.\texttt{\_| | | | |\{
\texttt{8a\texttt{\_|+|+| | |
\texttt{=2\texttt{-$\,$-$\,$-$\,$-$\,$-$\,$-$\,$-$\,$-$\,$-$\,$-$\,$-
\texttt{!\texttt{! Bouree
\texttt{4g\texttt{ | | | | |
\texttt{.\texttt{ |=| | | |
\texttt{4e\texttt{ |=| | | |\}
\texttt{.\texttt{\_|=| | | |
\texttt{4g\texttt{ |o| | | |
\texttt{!\texttt{! Bouree
\texttt{4a\texttt{\_|+|+| | |
\texttt{=3\texttt{-$\,$-$\,$-$\,$-$\,$-$\,$-$\,$-$\,$-$\,$-$\,$-$\,$-
\texttt{*-\texttt{*-}


\par 
Depending on the task, user-defined schemes can be either
carefully designed, or "throw-away" concoctions
created for momentary purposes.
The
\textit{Humdrum Reference Manual}
provides detailed advice on how to go about designing
special-purpose Humdrum representations.

\par 
For the remainder of this chapter we will examine a
pre-defined representation scheme for fretted instrument tablatures.
Although not all users will be interested in the \texttt{**fret}
representation, it provides an instructive contrast to the score-
and MIDI-based representations that we have relied on for most
of the examples in this book.


\subsection*{The \textit{**fret} Representation}

\par 
The
\texttt{**fret}
representation is a pre-defined Humdrum scheme that provides a comprehensive
system for representing performance aspects for fretted instruments.
The \texttt{**fret}
scheme is suitable for representing tablature information for
most fretted instruments, such as various guitars,
lute, mandore, theorbo, chitarrone, mandoline, banjo, dulcimer, and
even viols.
The
\texttt{**fret}
interpretation is not limited to equal-temperament tuning,
and so can be used to represent non-Western fretted instruments,
such as the \textit{oud} and the \textit{sitar}.

\par 
The
\texttt{**fret}
representation is performance-oriented rather than notationally-oriented.
Thus
\texttt{**fret}
is not suitable for distinguishing different visual renderings -$\,$-
such as differences between traditional French or German lute tablatures.
Some other Humdrum representation should be used if the user's
goal is to distinguish different forms of visual signifiers.

\par 
The basic pitches produced by fretted instruments depend on three factors:
(1) the relative tuning of the strings with respect to each other,
(2) the absolute overall tuning of the instrument,
and (3) the position of the frets.
Three tandem interpretations allow the user to specify each of
these aspects.

\par 
The absolute tuning of an instrument is indicated by encoding
the pitch of the lowest string using the \texttt{*AT:} tandem interpretation.
For the common six-string guitar, the lowest pitch is normally tuned
to E2, and so would be encoded with the following tandem interpretation:

\par 

\texttt{*AT:E2}


\par 
The \texttt{*AT:} interpretation makes use of
\texttt{**pitch}-type
pitch designations and may also include cents deviation.
For example, an instrument tuned 45 cents sharp might be represented as
\texttt{*AT:E2+45}.
Encoding the absolute tuning is optional with
\texttt{**fret};
when absent, a default tuning of E2 will be assumed
by various processing tools.

\par 
A second tandem interpretation (\texttt{*RT:}) specifies the relative
tuning as well as the number and arrangement of strings.
Some instruments pair strings together in close physical proximity
so that two strings are treated by the performer as a single
virtual "string."
Such paired strings are referred to as
\textit{courses.}
For example, the 12-string guitar is constructed using 6 courses,
and is played much like a 6-string guitar -$\,$- except that two strings
sound together, rather than a single string.

\par 
The \texttt{*RT:} tandem interpretation encodes the relative
tuning of each string by specifying the number of semitones
above the lowest string -$\,$- where each course is delineated by a colon (:).
In Example 18.1 three sample tunings are shown.
Example (a) defines the most common relative tuning for the six-string
guitar.
Successive strings are tuned 0, 5, 10, 15, 19, and
24 semitones above the lowest string.
\\
\\
\textbf{Example 18.1.  Sample Tunings for Fretted Instruments.}

\par 
(a) Common 6-string guitar.


\\
\\
\texttt{*AT:E2}
\\
\texttt{*RT:0:5:10:15:19:24}


(b) Common 12-string guitar.


\\
\\
\texttt{*AT:E2}
\\
\texttt{*RT:0,12:5,17:10,22:15,27:19,19:24,24}


(c) Vieil accord lute.


\\
\texttt{*AT:G2}
\\
\texttt{*RT:0,12:5,17:10,22:14,14:19,19:24,24}


\par 
Example (b) defines the most common relative tuning for the 12-string
guitar.
The six courses are delineated by colons and the tuning of strings
withint courses are delineated by commas.
In this case, the lower four courses consists of two strings tuned an
octave apart, whereas the upper two courses consist of paired unison strings.

\par 
Example (c) shows the most common tuning of the 6-course
\textit{lute} -$\,$- a tuning referred to as the so-called
\textit{vieil accord}: G2, C3, F3, A3, D4, and G4.
During the first half of the 16th century, it was common
to tune the lower three courses in octaves.

\par 
For non-Western and other instruments, it is possible to encode non-integer
semitone values for various strings, such as a string tuned 9.91 semitones
above the lowest string.

\par 
In addition to the absolute and relative tunings,
\texttt{**fret}
also allows the user to specify the tuning of successive frets
using the \texttt{FT:} tandem interpretation.
In Western instruments, frets are normally placed in semitone increments.
For a 12-fret instrument, this semitone arrangement may be explicitly
represented using the following tandem interpretation:

\par 

\texttt{*FT:1,2,3,4,5,6,7,8,9,10,11,12}


\par 
Each successive numerical value indicates the number of semitones
above the open string for successive fret positions.
The interpretation begins with the tuning of the first fret
rather than the tuning of the open string.
The above interpretation is similar to the
\textit{default fret tuning}
-$\,$-  which is an increase of precisely one-$\,$- itone for each successive fret.
The default fret tuning is not limited to 12 frets as in the above example.
An instrument constructed with nine 1/4-tone fret positions can be encoded
as follows:

\par 

\texttt{*FT:.5,1,1.5,2,2.5,3,3.5,4,4.5}


\par 
The only restriction imposed by \texttt{*FT:} is that all strings must have
identical fret distances.
That is, if the first fret is positioned 1 semitone above the open
string, then this relative pitch arrangement must be true of all strings.

\par 
The
\texttt{**fret}
representation distinguishes three types of data tokens:
tablature-tokens, rests, and barlines.
\textit{Tablature-tokens}
encode information regarding the fret/finger positions,
the manner by which individual strings are plucked (or bowed),
pitch-bending, vibrato, damping, harmonics, and other effects.
The actions of individual fingers can also be represented.
Each tablature-token consists of a several subtokens in the
form of Humdrum multiple-stops.
Subtokens are delimited by spaces and represent individual courses/strings.
A six-string (or six-course) instrument will require six subtokens
in each tablature-token.
For example, the following tablature token encodes the plucking of
the first and sixth string:

\par 

\texttt{| - - - - |}


\par 
Subtokens consist of up to five component elements:
(1) the string/course status, (2) fret position,
(3) bowing/strumming, (4) finger action, and (5) percussive effects.
In addition, the tablature-token can encode bowing and strumming information.

\par 
In the
\texttt{**fret}
representation, the status of a string/course can occupy one of sixteen states.
An
\textit{inactive}
string is signified by th-nus sign (\texttt{-}).
An ordinary
\textit{plucked}
string is represented by the vertical line (\texttt{|}).
Plucking near the bridge (\textit{plucked ponticello}) is represented by
the slash character (\texttt{/}).
Plucking near the tone-hole (\textit{plucked sul tasto}) is represented by
the backslash character (\texttt{$\backslash$}).
The repeated
\textit{plucked-tremolo}
(commonly used on the mandoline) is
represented using the octothorpe or hash character (\texttt{\#}).
\textit{Pizzicato}
is represented by the small letter `\texttt{z}'.
Normal bowing of a string is represented by the plus sign (\texttt{+});
\textit{ponticello}
bowing is represented by the open parenthesis `\texttt{(}' whereas
\textit{sul tasto}
bowing is represented by the closed parenthesis `\texttt{)}'.
\textit{Spiccato}
(bouncing the bow) is represented by the open curly brace `\texttt{\{}'.
\textit{Col legno}
(using the wood of the bow) is represented by the closed
curly brace '\texttt{\}}'.
\textit{Tremolo bowing}
is represented by the ampersand (\texttt{\&}).
\textit{Natural harmonics}
and
\textit{artificial harmonics}
are represented by the lower-case `\texttt{o}' and upper-case `\texttt{O}'
respectively.
String
\textit{ringing}
is denoted by the colon (\texttt{:}), and the
\textit{damping}
of a string is denoted by the small letter `\texttt{x}'.

\par 
By way of illustration, the following tablature-token represents
a six-string or six-course instrument, where the first through sixth
strings are respectively (1 and 2) plucked, (3) damped, (4) bowed,
(5) plucked sul tasto, (6) inactive.

\par 

\texttt{| | x + $\backslash$ -}


\par 
Note that the layout of the strings in a tablature-token always
corresponds to the tuning specified in the relative-tuning interpretation.
In most representations, the lower-pitched strings will be
toward the left side of the tablature token.

\par 
\textit{Fret-position}
information is indicated through the use of numbers,
with the first fret signified by the number `1'.
Fret-position numbers are encoded immediately to the right of their
respective string/course.
For example, the following tablature-token encodes a six-string/course
instrument in which the second and third strings are both stopped at the
second fret.

\par 

\texttt{| |2 |2 | | |}


\par 
Example 18.2 shows a sample passage for guitar with a corresponding
\texttt{**fret} representation displayed beneath.
The \texttt{**fret} representation does not encode duration
information.
It is common to join the \texttt{**fret} spine with a
\texttt{**recip}
spine representing the nominal duration data.
In example 18.2 a
\texttt{**kern}
spine is also shown indicating
the pitches in the \texttt{**fret} representation.
\\\\
\textbf{Example 18.2.}  J.S. Bach, \textit{Anna Magdalena Bach Notebook} Menuet II.  Guitar arr.



\texttt{\texttt{**recip\texttt{**kern\texttt{**fret
\texttt{**\texttt{*AT:G2
\texttt{**\texttt{*RT:0,12:5,17:10,22:14,14:19,19:24,24
\texttt{*M3/4\texttt{*\texttt{*M3/4
\texttt{=1\texttt{=1\texttt{=1
\texttt{4\texttt{E e g\texttt{- |4 - - - |0
\texttt{8\texttt{c\texttt{- : : |3 : :
\texttt{8\texttt{d\texttt{- : : : |0 x
\texttt{8\texttt{D d e\texttt{- |2 : : |2 :
\texttt{8\texttt{f\texttt{- : : : |3 :
\texttt{=2\texttt{=2\texttt{=2
\texttt{4\texttt{E e g\texttt{- |4 : : : |0
\texttt{4\texttt{c\texttt{- : : |3 : :
\texttt{4\texttt{c\texttt{- : : |3 : x
\texttt{=3\texttt{=3\texttt{=3
\texttt{4\texttt{F f a\texttt{- |5 : : : |2W
\texttt{8\texttt{f\texttt{- : : : |3 :
\texttt{8\texttt{g\texttt{- : : : : |0
\texttt{8\texttt{a\texttt{- : : : : |2
\texttt{8\texttt{b\texttt{- : : : : |4
\texttt{=4\texttt{=4\texttt{=4
\texttt{2\texttt{E e cc\texttt{- |4 : : : |5v
\texttt{*-\texttt{*-\texttt{*-}


\par 
The \texttt{**fret}
representation also provides several short-hand abbreviations
for common ornaments and effects.
Trills are indicated by the letters `t' (one semitone) and `T'
(two semitones).
Mordents are indicated by the letters `m' (one semitone) and `D'
(two semitones).
Inverted mordents are indicated by the letters `w' (one semitone) and `W'
(two semitones).
Turns are indicated by the letters `S' and `\$' (for the inverted
"Wagnerian" turn).
Two types of vibrato are distinguished: `v' for transverse vibrato
and `V' for lateral vibrato.
Pitch bending is signified by the tilde (~).

\par 
Apart from tablature-tokens,
\texttt{**fret} also permits the encoding of rests and barlines.
Rests tokens are denoted simply by the lower-case letter `r'.
Barlines are represented using the "common system" for barlines
used by \texttt{**kern} and other representations.


\subsection*{Additional Features of \textit{**fret}}

\par 
\textit{Bowing-direction}
and
\textit{strumming}
information is prepended to the beginning of the tablature-token.
The direction of bowing/strumming is encoded using the left
and right angle brackets:
\texttt{>}
means to bow/strum from the strings on the left side of the
representation toward the strings on the right side of the representation.
(On most instruments this means strumming
"downward" -$\,$- from the lowest- to the highest-pitched strings.)
The left angle bracket:
\texttt{<}
means to strum in the opposite direction.
A rough indication of the speed of bowing/strumming can be
represented by duplicating these signifiers.
For example,
\texttt{>>}
means a slower "downward" bow/strum, and
\texttt{<<<}
means an especially slow "upward" bow/strum.
The percent sign (\%) is used to signify the so-called
\textit{rasgueado}
-$\,$-  or flamboyant Spanish strum.
Once again these signifiers appear at the beginning of a tablature-token
-$\,$-  whenever they are encoded.
Strumming all 6 open strings downward on a commonly-tuned guitar
is represented as:

\par 

\texttt{*AT:E2
\\
*RT:0:5:10:15:19:24
\\
>| | | | | |}


\par 
Notice that there is no space between the right angle bracket and
the first vertical bar.

\par 
The
\texttt{**fret}
representation also permits the optional encoding of
\textit{fingering}
information.
For the plucking-hand (normally right hand),
traditional musical abbreviations are used:
\textit{P} (pollex) for the thumb, \textit{I} (index) for the index finger,
\textit{M} (medius) for the middle finger, \textit{A} (annularis)
for the ring finger, and \textit{Q} (quintus) for the little finger.
In addition, the lower-case letter \textit{p} is used to signify
the palm of the hand.
Note that these letters are applied only to the `plucking' hand.
In the case of the `fret-board' hand,
the lower-case letters
\textit{a-e}
are used to denote the thumb, index finger,
middle finger, ring finger, and little fingers, respectively.
Like the fret information, fingering information is encoded
immediately to the right of the string to which the information applies.
By way of illustration, the finger actions used in the above example
may be made explicit as follows:

\par 

\texttt{>|P |2bP |2cP |P |P |P}


\par 
The strum is carried out by the thumb, while the index and middle
fingers of the fret-hand stop the second and third courses/strings
at the second fret.
In the following continuation of this representation,
the first course/string is replucked by the thumb.
With the exception of the second and third courses/strings,
the other strings are allows to ring.

\par 

\texttt{>|P |2bP |2cP |P |P |P
\\
>|P xIM xIM : : :}


\par 
Notice that in damping the vibrations of the second and third strings,
both the index and middle fingers of the `pluck' hand are used on
both strings.

\par 
On rare occasions, guitarists will substitute fingers on the fret-board
while a string remains sounding.
The following example illustrates such a finger-substitution where
the middle finger is replaced by the ring finger:

\par 

\texttt{| |2b |2c | | |
\\
: :2b :2d : : :}


\par 
Note that in the \texttt{**fret}
representation, no special signifiers are provided for
so-called `hammer-on' or (ascending-slur),
nor for the so-called `pull-off' or (descending-slur).
During the ascending-slur, the sound is produced simply
by engaging the next fret.
This can be represented in \texttt{**fret}
by using the "let ring" signifier (:) in conjunction
with the appropriate fret notation.
The descending-slur can be similarly notated.

\par 
Four types of "percussion effects"
can be represented using \texttt{**fret}.
The two most common
\textit{tambours}
involve tapping on the bridge (represented by the lower-case letter `u')
and tapping on the strings near the bridge (represented by the
upper-case letter `U').
A simple `tap' on the top-plate is represented by the lower-case letter `y',
whereas a lower-pitched `thump' on the top-plate is represented by
the upper-case letter `Y'.
When sounded alone, these signifiers appear on a line by themselves.
When sounded in conjunction with a plucked or (uncommonly) bowed string,
these signifiers appear at the beginning of the tablature-token.

\par 
The complete system of signifiers used by
\texttt{**fret}
is summarized in Table 18.1.

\par 
\textbf{Table 18.1.}  Signifiers used by \textit{**fret}.

\textbf{Fret-board (left) Hand}
1first fret position
2second fret position, ...
11eleventh fret position, etc.
0open string (not necessarily sounded)
~bend up in pitch
vvibrato (transverse)
Vvibrato (lateral)

ttrill (1 fret distance)
Ttrill (2 frets distance)
mmordent (1 fret distance)
Dmordent (2 frets distance)
winverted mordent (1 fret distance)
Winverted mordent (2 frets distance)
Sturn
\$inverted (Wagnerian) turn

athumb (of fret hand)
bindex finger (of fret hand)
cmiddle finger (of fret hand)
dring finger (of fret hand)
elittle finger (of fret hand)
nno finger (of fret hand)

\textbf{Pluck (right) Hand}
-unplucked or unactivated string
|plucked string (normal)
/plucked string -$\,$- near bridge (ponticello)
$\backslash$plucked string -$\,$- near tone-hole (sul tasto)
\#tremolo (plucked, ala mandoline)
zpizzicato
:let string ring
xdamp string
onatural harmonic
Oartificial harmonic

+bow (normal)
(bow -$\,$- near bridge (ponticello)
)bow -$\,$- toward fret-board (sul tasto)
\{spiccato
\}col legno (with wood of the bow)
\&tremolo (bowed)

>strum from low notes to high notes (= down-bow)
<strum from high notes to low notes (= up-bow)
>>slower down-strum; slower down-bow
>>slower up-strum; slower up-bow
>>>very slow down-strum; very slow down-bow
<<<very slow up-strum; very slow up-bow
\%rasgueado (Spanish strum)

Ppollex: thumb (of pluck hand)
Iindex: index finger (of pluck hand)
Mmedius: middle finger (of pluck hand)
Aannularis: ring finger (of pluck hand)
Qquintus: little finger (of pluck hand)
ppalm (of pluck hand)
Nno finger (of pluck hand)

utambour (tap on bridge)
Utambour (tap on strings near bridge)
y`tap' on top-plate
Y`thump' on top-plate


\textit{Summary of \textit{**fret} Signifiers}


\par 
A number of pitch-related Humdrum commands 
accept \texttt{**fret} encoded data as inputs, including
\textbf{cents},
\textbf{freq},
\textbf{kern},
\textbf{pitch},
\textbf{semits},
\textbf{solfg},
and
\textbf{tonh}.



\subsection*{Reprise}


\par 
In this chapter we have tried to reinforce the lesson that
\texttt{**kern}
is only one of an unbounded number of existing and possible Humdrum
representations.
As a Humdrum user, you are free to concoct your own representations
to better address the kinds of information you are interested in
manipulating.
As long as the resulting representation conforms to the Humdrum syntax,
the most important Humdrum tools can still be used to manipulate
your data.

\\
\begin{itemize}
\item 

\textbf{Next Chapter}
\item 

\textbf{Previous Chapter}
\item 

\textbf{Table of Contents}
\item 

\textbf{Detailed Contents}
\\\\

� Copyright 1999 David Huron
\end{document}
