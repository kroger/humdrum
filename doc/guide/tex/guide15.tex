% This file was converted from HTML to LaTeX with
% Tomasz Wegrzanowski's <maniek@beer.com> gnuhtml2latex program
% Version : 0.1
\documentclass{article}
\begin{document}



  
  
    
      
      
      
    
  



\\
\\

\section*{Chapter15}


[\textit{Previous Chapter}]
[\textit{Contents}]
[\textit{Next Chapter}]


\section*{Harmonic Intervals}



In
Chapter 11
we examined Humdrum tools related to
melodic pitch intervals.
This chapter returns to the discussion of intervals by focussing on
two tools pertaining to characterizing harmonic intervals:
the
\textbf{hint}
and
\textbf{ydelta}
commands.


\subsection*{Types of Harmonic Intervals}

\par 
Harmonic intervals identify pitch distances between nominally
concurrently-sounding notes.
Example 15.1 illustrates five types of harmonic intervals.
The most obvious harmonic intervals are those that occur between
pitches that have concurrent onsets.
We might call these
\textit{explicit harmonic intervals.}
An example of an explicit harmonic interval is the perfect octave
in the first sonority of Example 15.1
\\\\
\textbf{Example 15.1}  Types of harmonic intervals.




\texttt{**kern**kern
\texttt{*M4/4*M4/4
\texttt{=1-=1-
\texttt{8c2cc
\texttt{8d.
\texttt{4e.
\texttt{8f8r
\texttt{8r8a
\texttt{8d8r
\texttt{8r8b
\texttt{=2=2
\texttt{2.c 2.e2.g 2.cc
\texttt{*-*-}

Less obvious harmonic intervals arise when tone onsets are
not synchronous -$\,$- yet both pitches are sustained simultaneously
for a period of time.
We might call these
\textit{passing harmonic intervals.}
Two examples are evident in Example 15.1:
the minor seventh between the D and the sustained C,
and the minor sixth between E and C.
Most music theorists would argue that the minor sixth
interval is more important than the minor seventh.
Compared with the D, the following E has a longer duration and coincides
with a stronger metrical position.
Hence the interval of the minor sixth is likely to be more
perceptually salient than the minor seventh interval.
We might call these intervals
\textit{strong passing intervals}
and
\textit{weak passing intervals}
respectively.


\par 
Another type of harmonic interval may be deemed to occur
even when there are no concurrently sounding pitches.
In Example 15.1, the alternating F->A and D->B imply
subdominant and dominant harmonic functions and so suggest
the intervals of a major third followed by a major sixth.
We might call these intervals
\textit{implicit harmonic intervals.}
The harmonic status of the pitch sequence A->D is more contentious;
many theorists would argue that no harmonic interval is implied
either because of the weak->strong metric relationship
or because the harmonic interval is less stereotypic of the
given the cadential clich�.
Once again, implicit harmonic intervals might be distinguished
as weak or strong.

\par 
A further complication in identifying harmonic intervals
arises when there are more than two pitches involved.
In the final four-note chord of Example 15.1, what harmonic intervals
are implied?
Example 15.2 shows three possible break-downs of the harmonic relationships.
\textit{Stacked harmonic intervals}
include only the intervals between successive pitches ordered from
low to high.
Four pitches thus generate three intervals.
\textit{Bass-related harmonic intervals}
include all intervals calculated with respect to the lowest pitch.
Four pitches again leads to three (different) intervals.
This latter way of characterizing intervals is the most similar
to figured bass notation.
\textit{Permuted harmonic intervals}
include all intervals that can be calculated between all notes
of a sonority.
Four pitches leads to six intervals.

\\\\
\textbf{Example 15.2}  Interpreting Interval Content in Chords
\\

\\

Stacked IntervalsBass-related IntervalsPermuted Intervals


\par 
A final issue in characterizing harmonic intervals is whether or not the
presence of unisons is important.
In most applications (such as figured bass) the presence of unisons
is unimportant.
In other applications (such as tabulating instrument doubling in orchestration)
the occurrence of unisons is noteworthy.

\par 
By way of summary, we have distinguished five types
of harmonic intervals, and three measurement methods:
explicit harmonic intervals, implicit harmonic intervals (both strong and weak),
plus passing harmonic intervals (again both strong and weak varieties).
Measurement methods include (1) stacked harmonic intervals,
(2) bass-related harmonic intervals, and (3) permuted harmonic intervals.

\par 
As in the case of melodic intervals, harmonic intervals can be
calculated according to a variety of units -$\,$- including diatonic intervals,
semitones, cents, frequency, and even cochlear coordinates.
We will consider just two tools for calculating harmonic intervals:
\textbf{hint}
and
\textbf{ydelta.}


\subsection*{Harmonic Intervals Using the \textit{hint} Command}

\par 
The Humdrum
\textbf{hint}
command calculates harmonic intervals for pitch-related representations
such as
\texttt{**kern},
\texttt{**pitch},
\texttt{**solfg},
and
\texttt{**Tonh}.
As in the case of the
\textbf{mint}
command, output intervals are expressed as a combination of diatonic
interval size plus interval quality (such as `perfect fourth'
and `minor ninth').

\par 
In the default operation,
\textbf{hint}
calculates only explicit harmonic intervals;
for sonorities containing more than two pitches,
only stacked harmonic intervals are calculated.
The output from the
\textbf{hint}
command always consists of a single
\texttt{**hint}
spine.
Any number of spines may be present in the input,
but only pitch-related spines are processed.
Given the default invocation, the output corresponding to Example 15.1
is as follows:
\\\\

\texttt{**hint
\texttt{*M4/4
\texttt{=1-
\texttt{P8
\texttt{-
\texttt{-
\texttt{-
\texttt{-
\texttt{-
\texttt{-
\texttt{=2
\texttt{M3 m3 P4
\texttt{*-}

Notice that sonorities that contain only a single pitch result in
the outputting of a hyphen (\texttt{-}).
The hyphen indicates that pitched material is present, but there
are no explicit harmonic intervals.
Input records that contain no pitch tokens result in the outputting
of a null token (.).
If a single duplicated pitch is present, then the output will
indicate a perfect unison (P1).

Unisons can be suppressed from the output via the
\textbf{-u}
option for \textbf{hint}.


\par 
When more than two pitches are present in a sonority,
\textit{permuted harmonic intervals}
can be calculated by invoking the
\textbf{-a}
option (i.e.
\textit{all}
intervals).
For example, with the
\textbf{-a}
option, the final chord in Example 15.1 would produce the
following output:
\\\\

\texttt{**hint
\texttt{*all
\texttt{M3 P5 P8 m3 m6 P4
\texttt{*-}

Notice the presence of the \texttt{*all} tandem interpretation in
the above output.
This interpretation is added to the output in order to  warn
users that the representation should not be
interpreted as stacked intervals.

\par 
\textit{Bass-related harmonic intervals}
can be calculated with the
\textbf{-l}
option.
In this case, harmonic intervals are calculated with respect to
the lowest pitch in the sonority.
This option is helpful in determining figured bass.
For the final chord in Example 12.1 the corresponding output
would be


\par 

\texttt{**hint
\\
M3 P5 P8
\\
*-}


\par 
Two further options for
\textbf{hint}
allow the user to tailor how the intervals are represented.
The
\textbf{-c}
option causes compound intervals such as a minor tenth (m10)
to be output as non-compound equivalents (m3).
This means that the interval of an octave (P8) will be rendered
as a unison (P1).
The
\textbf{-d}
option suppresses the outputting of interval qualities and results
in only diatonic interval sizes being output.
Again, this option is helpful in determining figured bass.
The command:

\par 

\texttt{hint -clud}


\par 
will produce the following output for the final major chord
in Example 12.1:

\par 

\texttt{**hint
\\
3 5
\\
*-}



\subsection*{Propagating Data Using the \textit{ditto} Command}

\par 
In the default operation,
\textbf{hint}
calculates intervals only between pitches that are explicitly
present in an input data record.
This means that passing intervals are not calculated.

\par 
In order to generate passing intervals,
we will make use of the Humdrum
\textbf{ditto}
command.
The
\textbf{ditto}
command replaces null tokens with the
previous non-null data token in the same spine.
Suppose we had an arbitrary input such as the following:
\\\\

\texttt{**flip**flop
\texttt{Axyz
\texttt{.jkl
\texttt{..
\texttt{Babc
\texttt{..
\texttt{C.
\texttt{..
\texttt{*-*-}

The effect of
\textbf{ditto}
would be the following:
\\\\

\texttt{**flip**flop
\texttt{Axyz
\texttt{Ajkl
\texttt{Ajkl
\texttt{Babc
\texttt{Babc
\texttt{Cabc
\texttt{Cabc
\texttt{*-*-}

Each null token has been replaced by the preceding data token
within the spine.

\par 
Consider the effect of
\textbf{ditto}
on the \texttt{**kern} data in Example 15.1:

\par 

\texttt{ditto -p example15.1}


\par 
The following output results:
\\\\

\texttt{**kern**kern
\texttt{*M4/4*M4/4
\texttt{=1-=1-
\texttt{8c2cc
\texttt{8d(2cc)
\texttt{4e(2cc)
\texttt{8f8r
\texttt{8r8a
\texttt{8d8r
\texttt{8r8b
\texttt{=2=2
\texttt{2.c 2.e2.g 2.cc
\texttt{*-*-}

Notice that the half-note C5 has been repeated.
The
\textbf{-p}
option has caused each repetition of the data token to be placed in
parentheses so they can be easily recognized.
By using
\textbf{ditto,}
we have transformed previously passing intervals into explicit
sonorities whose intervals can now be identified by
\textbf{hint.}
We can combine the two commands in a single pipeline:

\par 

\texttt{ditto example15.1 | hint}


\par 
The resulting output for Example 15.1 includes the
two passing intervals (m7 and m6) in the first measure:
\\\\

\texttt{**hint
\texttt{*M4/4
\texttt{=1-
\texttt{P8
\texttt{m7
\texttt{m6
\texttt{-
\texttt{-
\texttt{-
\texttt{-
\texttt{=2
\texttt{M3 m3 P4
\texttt{*-}

The
\textbf{ditto},
command provides two additional options that are worthy
of note: the
\textbf{-s}
and
\textbf{-c}
options.
The
\textbf{-s}
option allows
\textbf{ditto}
to skip or ignore the presence of certain data records.
Suppose, for example, that we had a barline in the
midst of some null tokens:
\\\\

A
.
=
.

Often, we would like to propagate certain data tokens
\textit{around}
some other types of data tokens, so the result might be:
\\\\

A
A
=
A

By providing
\textbf{ditto}
with a suitable regular expression, we can have the data token `A'
skip over the barline:

\par 

\texttt{ditto -s \^{}=}


\par 
Without this option, the final data token in the above example
would be an equals-sign rather than the token `A'.

\par 
The
\textbf{-c}
option for
\textbf{ditto}
allows the user to selectively identify which characters are
propagated.
For example, the following command will cause only the lower-case
letters `a' and `b' to be propagated:

\par 

\texttt{ditto -c ab}


\par 
This feature allows users to replicate only certain kinds of data -$\,$-
such as pitches, durations, dynamic marks, etc.

\par 
As we will see in future chapters, the
\textbf{ditto}
command proves useful in a wide variety of situations
apart from calculating intervals.


\subsection*{Using the \textit{ditto} and \textit{hint} Commands}

\par 
Let's pause and consider some of the ways we might
use the
\textbf{ditto}
and
\textbf{hint}
commands.
First, let's determine if some input contains a
particular interval.
Are there any augmented sixth intervals in Bach's two-part
inventions?
The following commands look for explicit and passing
sixths respectively.
Notice the use of the
\textbf{-c}
option so octave equivalents will also be identified:

\par 

\texttt{hint -c inventio* | grep A6}
\\
\texttt{ditto -s \^{}= inventio* | hint -c | grep A6}



\par 
Are there any diminished octave intervals between any
two concurrent notes in any of Beethoven's piano sonatas?

\par 

\texttt{ditto -s \^{}= sonatas* | hint -a | grep d8}


\par 
In orchestral works, some pairs of instruments are more
likely than others to double each other at the unison or octave.
What proportion of the intervals formed by the oboe and flute
notes are doubled?
Since we are looking for a proportion, we need to make two
counts: the total number of (explicit) intervals formed by the oboe and
flute, and the number of those intervals that are octave equivalents.
(We will assume that there is only one oboe and one flute part in the
file \texttt{Rimsky-K}:)

\par 

\texttt{extract -i '*Ioboe,*Iflts' Rimsky-K | hint -c | rid -GLId $\backslash$

| grep -c P1}

\\
\texttt{extract -i '*Ioboe,*Iflts' Rimsky-K | hint -c | rid -GLId $\backslash$

| grep -c [MmPAd]}



\par 
The second
\textbf{grep}
counts the total number of intervals by looking for all of the
interval qualities (major, minor, perfect, etc.)

\par 
Suppose we have extracted two horn parts from an orchestral score.
Are octave intervals between the horns more likely to occur on the
dominant pitch or the tonic pitch?

\par 

\texttt{solfa horns > temp1}
\\
\texttt{hint horns > temp2}
\\
\texttt{assemble temp1 temp2 | grep -c \^{}do.*P8}
\\
\texttt{assemble temp1 temp2 | grep -c \^{}so.*P8}



\subsection*{Determining Implicit Harmonic Intervals}

\par 
Recall that
\textit{implicit harmonic intervals}
may be deemed to occur between tones that don't actually
sound at the same time.
This arises when one part has a rest while the other part
is sounding.
Note that if we could eliminate rest tokens, then we could use
\textbf{ditto}
to repeat previous pitch tokens in place of the rests and so
generate implicit harmonic intervals.

\par 
The
\textbf{humsed}
command (described in
Chapter 14)
is well suited to this task.
We want to transform any data token containing the letter `r'
to a null token.
Consider the following substitution:

\par 

\texttt{humsed 's/.*r.*/.    ./' example15.1}


\par 
Unfortunately, this isn't quite right.
The above substitution will find any data record containing the letter `r'
and transform the entire record to a single null data record.
We need to address individual data tokens.
In this example, rest tokens may be in two possible positions:
the first token in the record or the last token in the record.
We need two different regular expressions to address each
of these conditions.
First, a regular expression to identify rests in the first token:

\par 

\texttt{/\^{}[\^{}   ]*r[\^{}   ]*   /}


\par 
(That is, the beginning of the record (\^{}), followed by zero or more
instances of any character other than a tab ([\^{}   ]*),
followed by the letter `r', followed by zero or more instances
of any character other than a tab ([\^{}   ]*), followed by a tab.)

\par 
Similarly, our second regular expression identifies rests in the last token:

\par 

\texttt{/   [\^{}   ]*r[\^{}   ]*\$/}


\par 
Now we can eliminate rest tokens using the following two
substitution commands within a single invocation of \textbf{humsed}:

\par 

\texttt{humsed 's/\^{}[\^{}  ]*r[\^{}  ]*  /.  /; s/  [\^{}  ]*r[\^{}  ]*\$/  ./' example15.1}


\par 
The following output results:
\\\\

\texttt{**kern**kern
\texttt{*M4/4*M4/4
\texttt{=1-=1-
\texttt{8c2cc
\texttt{8d.
\texttt{4e.
\texttt{8f.
\texttt{.8a
\texttt{8d.
\texttt{.8b
\texttt{=2=2
\texttt{2.c 2.e2.g 2.cc
\texttt{*-*-}

If we now apply
\textbf{ditto}
and recalculate the intervals, the resulting output will identify
some implicit intervals as well:

\par 

\texttt{humsed 's/\^{}[\^{}  ]*r[\^{}  ]*  /.  /; s/  [\^{}  ]*r[\^{}  ]*\$/  ./' example15.1 $\backslash$

| ditto -p}



\par 
Below we see the output assembled with the output from the corresponding
\textbf{hint}
command:
\\\\

\texttt{**kern**kern**hint
\texttt{*M4/4*M4/4*M4/4
\texttt{=1-=1-=1-
\texttt{8c2ccP8
\texttt{8d(2cc)m7
\texttt{4e(2cc)m6
\texttt{8f(2cc)P5
\texttt{(8f)8aM3
\texttt{8d(8a)P5
\texttt{(8d)8bM6
\texttt{=2=2=2
\texttt{2.c 2.e2.g 2.ccM3 m3 P4
\texttt{*-*-*-}


\par 

\subsection*{The \textit{ydelta} Command}


\par 
Often it is useful to represent intervals by the number of
semitones (or some other numerical value).
We might begin by using the
\textbf{semits}
command to translate Example 15.1 to a
\texttt{**semits}
representation.

\par 

\texttt{semits example15.1}


\par 
The resulting output would be as follows:
\\\\

\texttt{**semits**semits
\texttt{*M4/4*M4/4
\texttt{=1-=1-
\texttt{012
\texttt{2.
\texttt{4.
\texttt{5r
\texttt{r9
\texttt{2r
\texttt{r11
\texttt{=2=2
\texttt{0 47 12
\texttt{*-*-}

Numerical differences for values on a single data record
can be computed using the
\textbf{ydelta}
command.
The
\textbf{ydelta}
command is comparable to
\textbf{xdelta},
however, numerical differences are calculated between simultaneous numerical
values (delta-\textit{y}) rather than between successive numerical
values (delta-\textit{x}).

\par 
Like the
\textbf{hint}
command,
\textbf{ydelta}
always outputs a single spine.
The user must specify which input spines are to be processed using the
\textbf{-i}
option.
In the following command, only **semits input is to be processed:

\par 

\texttt{semits example15.1 | ydelta -s = -i '**semits'}


\par 
The
\textbf{-s}
option allows the user to identify data records to be
\textit{skipped}
while processing.
In this case, the regular expression `=' is used to identify barlines,
so measure numbers will be excluded when processing the numerical data.

\par 
The above command yields the following output:
\\\\

\texttt{**Ysemits
\texttt{*
\texttt{=1-
\texttt{12
\texttt{.
\texttt{.
\texttt{.
\texttt{.
\texttt{.
\texttt{.
\texttt{=2
\texttt{4 7 12
\texttt{*-}

Notice that
\textbf{ydelta}
prepends the upper-case letter `Y' to the given input interpretation.
All output values are calculated with respect to the lowest value
in the current data record.
Hence, the `\texttt{4 7 12} in the last data record means that there
are pitches 4 semitones above the lowest note, 7 semitones above
the lowest note, and 12 semitones above the lowest note.
(If necessary, the lowest or offset value for each record can be
output in square brackets using the
\textbf{-o}
option.)

\par 
Like the
\textbf{hint}
command,
\textbf{ydelta}
calculates numerical intervals only when more than one value is
present on a given input data record.
As in the case of
\textbf{hint,}
we might use the
\textbf{ditto}
command to propagate pitch values -$\,$- replacing all the null data tokens.
A suitable command would be:

\par 

\texttt{semits example15.1 | ditto -s = | ydelta -s = -i '**semits'}


\par 
The resulting output would be:
\\\\

\texttt{**Ysemits
\texttt{*
\texttt{=1-
\texttt{12
\texttt{10
\texttt{8
\texttt{.
\texttt{.
\texttt{.
\texttt{.
\texttt{=2
\texttt{4 7 12
\texttt{*-}



\subsection*{More Examples Using the \textit{ydelta} Command}

\par 
What is the average semitone distance between the cantus and
altus voices in Lassus motets?
We can answer this question by first extracting the appropriate
voices and translating to a semitone representation.

\par 

\texttt{extract -f 1,2 motet* | semits > temp1}


\par 
Using
\textbf{ditto}
we can expand the pitched material so that concurrently-sounding
tones will generate explicit intervals.
We then use
\textbf{ydelta}
to calculate the actual semitone interval distances.
The
\textbf{rid}
command can be used to eliminate non-data records, and the
\textbf{grep -v}
command can be used to further eliminate barlines.
Finally, we can calculate the mean interval distance using the
\textbf{stats}
command:

\par 

\texttt{ditto -s = temp1 | ydelta -s = -i '**semits' | rid -GLId $\backslash$

| grep -v = | stats}



\par 
Suppose we have a two-part input.
Are there tritone intervals (explicit and passing) that are not
spelled as either an augmented fourth or diminished fifth?
We can answer this question by using both
\textbf{hint}
and
\textbf{ydelta}
and a suitable sequence of
\textbf{grep}
commands.
The
\textbf{ditto}
command is used to ensure that both explicit and passing intervals
are generated.

\par 

\texttt{ditto -s = 2part | semits | ydelta -s = -i '**semits' > temp1}
\\
\texttt{ditto -s = 2part | hint > temp2}
\\
\texttt{assemble temp1 temp2 | rid -GLId | grep \^{}6 | grep -v A4 $\backslash$

| grep -v d5}



\par 
Notice the use of
\textbf{grep -v}
to first exclude any records that match an augmented fourth,
and then to exclude any remaining records that match a diminished fifth.

\par 


\subsection*{Reprise}

\par 
Harmonic intervals can be measured in a variety of ways.
They can be characterized as diatonic qualities such
as minor sevenths or augmented sixths.
They can be measured in terms of semitone distance -$\,$- or
even in cents or hertz (frequency difference).
Only the diatonic size may be of interest (e.g., "a fifth"),
and compound intervals (e.g., major tenth) can be expressed
by their non-compound equivalents (major third).

\par 
At least five types of harmonic intervals can be distinguished including
\textit{explicit harmonic intervals},
\textit{strong passing intervals},
\textit{weak passing intervals},
\textit{strong implicit harmonic intervals}
and
\textit{weak implicit harmonic intervals}.
In addition, we distinguished three different ways of characterizing
harmonic intervals:
\textit{stacked harmonic intervals},
\textit{bass-related harmonic intervals}
and
\textit{permuted harmonic intervals}.

\par 
In this chapter we have seen how to use the
\textbf{hint}
command to calculate these various kinds of intervals.
We have also seen how
\textbf{ydelta}
can be used to measure purely numerical distances between concurrent values.

\\
\begin{itemize}
\item 

\textbf{Next Chapter}
\item 

\textbf{Previous Chapter}
\item 

\textbf{Table of Contents}
\item 

\textbf{Detailed Contents}
\\\\

� Copyright 1999 David Huron
\end{document}
