% This file was converted from HTML to LaTeX with
% Tomasz Wegrzanowski's <maniek@beer.com> gnuhtml2latex program
% Version : 0.1
\documentclass{article}
\begin{document}



  
  
    
      
      
      
    
  



\\
\\

\section*{Chapter28}


[\textit{Previous Chapter}]
[\textit{Contents}]
[\textit{Next Chapter}]


\section*{Dynamics}



In this chapter we introduce three pre-defined representations
pertaining to musical dynamics.
One representation (**dyn) represents dynamic markings
as they appear visually in a printed score.
A second representation (**dynam) represents a rationalized
interpretation of the notated dynamic markings in a score.
A third scheme (**dB) represents continuous dynamic levels
in decibels.


\subsection*{The \textit{**dynam} and \textit{**dyn} Representations}

\par 
Musical scores commonly contain dynamic markings that include
both written text (such as "\textit{subito forte}" and "\textit{dimin.}")
and graphic representations (such as hairpin or wedge-shaped crescendo markings).
Unfortunately, traditional dynamic markings are often confusing
or ambiguous.
Consider, for example, the following sequence of dynamic markings
from a Beethoven piano sonata:

\par 


\textbf{\textit{pp\textbf{\textit{cresc. \textbf{\textit{cresc. \textbf{\textit{p    \textbf{\textit{cresc.     \textbf{\textit{pp}}



\par 
What are we to make of these markings?
Does the music gradually \textit{crescendo} from \textit{pianissimo} to \textit{piano}?
Does this initial \textit{crescendo} occur in two distinct phases or does the
repetition of the term "\textit{cresc.}" merely indicate a continuation
of a single \textit{crescendo}?
Does this \textit{crescendo} move to a dynamic level above \textit{piano}
and abruptly reduce to \textit{piano}?
Does the final \textit{crescendo} begin at a \textit{piano} level and get louder -$\,$-
followed by a relatively abrupt reduction to \textit{pianissimo}?
Or does the final \textit{crescendo} begin below \textit{piano} and gradually
reach \textit{pianissimo}?
Such ambiguities are rampant in printed musical scores.
We can examine the accompanying musical context to help us resolve questions
of interpretation, but computers are unable to bring such sophistication 
to the task.

\par 
Humdrum provides two pre-defined representations for score-related
dynamic markings.
One representation
(\texttt{**dyn})
attempts to represent
the dynamic markings as they appear in a visual rendering of a score.
That is, \texttt{**dyn} represents the visual or orthographic information.
A second representation
(\texttt{**dynam})
provides a "rationalized"
or canonical means for interpreting score-related dynamic indications.
Users will want to choose one or the other representation depending on
the analytic task being pursued.

\par 
In the first instance, \texttt{**dynam} uses standardized data tokens to represent
particular dynamic levels.
Table 28.1 shows the standard representations for \texttt{**dynam}.
For example, the token \texttt{pp} represents the concept of pianissimo,
even if the visual rendering may be \textbf{\textit{pp}} or
\textit{pianissimo} or \textit{pianiss.}, etc.

\par 
\textbf{Table 28.1.}

ppiano
pppianissimo
ppptriple piano
ppppquadruple piano, [etc]
fforte
fffortissimo
ffftriple forte
ffffquadruple forte, [etc]
mpmezzo-piano
mfmezzo-forte
ssubito (suddenly), e.g. spp (\textit{subito pianissimo}), sf (\textit{subito forte})
zsforzando = fp (forte-piano)
<begin crescendo
>begin diminuendo
(continuing crescendo
)continuing diminuendo
[end crescendo
]end diminuendo
Xexplicit interpretation (not indicated in the score)
xpublished interpretation (indicated in the score, often in parentheses)
rrest (silence)
vnotated accent or stress


\par 
In the case of crescendo and diminuendo markings,
\texttt{**dynam}
requires an explicit interpretation of where the dynamic
marking begins and ends.
The beginning of a crescendo is indicated by the less-than sign (\texttt{<}).
The end of the crescendo is marked by the open square bracket (\texttt{]}).
Between the beginning and end points,
\textit{continuation signifiers}
are encoded.
For crescendos, continuations are indicated using the open parenthesis;
for diminuendos, continuations are indicated using the closed parenthesis.

\par 
In the \texttt{**dynam} representation,
no distinction is made for various ways a composer might indicate a crescendo or
a diminuendo.
For example, it doesn't matter whether a diminuendo is notated as
\textit{dim.}, \textit{dimin.}, \textit{diminuendo}, \textit{decres.}, \textit{decresc.},
\textit{decresendo}, \textit{calando}, \textit{morendo}, \textit{se perdant},
\textit{c�d�z}, \textit{gradually quieter},
or via a hairpin or wedge graphic diminuendo.
All are represented by \texttt{>} ... \texttt{)} ... \texttt{]}.

\par 
The \texttt{**dynam} representation also requires explicit resolution
of possibly ambiguous dynamic markings.
In many cases, the user will be required to add dynamic markings that
are only implicit in the original score.
Interpreted dynamics are preceded by the upper-case letter \texttt{X},
so an interpreted diminuendo will be
represented by \texttt{X>} ... \texttt{X)} ... \texttt{X]}.

\par 
Often published editions will include dynamic markings that have
been introduced by the editor.
In scholarly publications these editorialisms are indicated
in parentheses or square brackets.
Such interpreted dynamics are preceded by the lower-case letter \texttt{x}.

\par 
The use of the
\texttt{**dynam}
representation is illustrated in
Example 28.1.
\\\\
\textbf{Example 28.1}



This example might be encoded as follows:

\texttt{**dynam**kern
*staff1*staff1
==
p2c 2e 2g 2cc
<.
(2G 2d 2g 2b
==
(2A 2c 2e 2a
[.
pp2E 2B 2e 2g
==
*-*-}


\par 
The \texttt{**dynam} encoding is interpreted as follows:
the level begins \textit{piano} with a crescendo beginning
prior to the second chord;
the crescendo continues until after the third chord
and then the level abruptly drops to \textit{pianissimo} with
the onset of the fourth chord.
Notice that dynamic markings are "read from left-to-right";
that is, we presume that the crescendo begins \textit{piano} and
that the \textit{pianissimo} is an abrupt reduction in level,
rather than presuming that the crescendo builds to the \textit{pianissimo}
level and so there is an abrupt reduction in level after the initial \textit{piano}
just before the cresendo begins.
In short, a crescendo (or diminuendo) marking is always assumed to
increase (or decrease) the dynamic level from the preceding indication.

\par 
If appropriate, a user can render implicit dynamic shading explicitly.
For example, a user might choose to re-code Example 28.1 as either

\texttt{**dynam**kern
*staff1*staff1
==
p2c 2e 2g 2cc
<2G 2d 2g 2b
==
[2A 2c 2e 2a
X>.
X].
pp2E 2B 2e 2g
==
*-*-}


\par 
or

\texttt{**dynam**kern
*staff1*staff1
==
p2c 2e 2g 2cc
Xppp.
<2G 2d 2g 2b
==
(2A 2c 2e 2a
[.
pp2E 2B 2e 2g
==
*-*-}


\par 
Notice that null data records may be inserted as necessary to clarify
the moment of dynamic change.
The
\texttt{**dynam}
representation makes use of the common system
for representing barlines.


\subsection*{The \textit{**dyn} Representation}

\par 
The
\texttt{**dyn}
representation provides a method for representing
the orthographic appearance of notated dynamic markings.
Unlike \texttt{**dynam}, the \texttt{**dyn} representation distinguishes
between different ways of identifying a dynamic marking.
For example, \textit{dim.}, \textit{dimin.}, \textit{diminuendo}, \textit{decres.},
\textit{decresc.}, \textit{decresendo}, are all regarded as different
from each other.
Composers often have idiosyncratic ways of writing dynamic markings.
As a result, the specific terms used may have repercussions,
for example, in resolving cases of disputed composership.
In some circumstances, it is thought that individual composers
distinguish the terms in their own minds.
For example, a composer might use \textit{decrescendo} as a general
term to indicate a temporary descending dynamic shape,
whereas \textit{diminuendo} might have a more specific meaning
of a `dying away' or `fade-out' gesture.

\par 
In the **dyn representation the horizontal position of dynamic markings
is indicated in quarter-durations with respect to the previous barline.
This number appears prior to the dynamic signifier,
hence \texttt{4.1f} means a \textit{forte} (e.g., \textbf{\textit{f}}) marking just after
the horizontal position of the fourth quarter in the measure.
The vertical position of dynamic markings is indicated with respect to
the middle line of a corresponding staff;
this number appears in curly braces.

<begin wedge-graphic crescendo marking
>begin wedge-graphic diminuendo marking
[terminate wedge-graphic crescendo marking
]terminate wedge-graphic diminuendo marking
(continuing wedge-graphic crescendo
)continuing wedge-graphic diminuendo
\{...\}vertical position (in staff-line steps from mid-line)
\#...size of marking (in staff-line steps)
:number:density of dashed lines in strokes per quarter-duration
/.../wedge opening size (in staff-line steps)
rrest (silence)
Hmarking appears in square brackets


\par 
By way of illustration, consider Example 28.2.
\\\\
\textbf{Example 28.2:}} Arnold Schoenberg, \textit{Three Piano Pieces, Op. 11, No. 2}, excerpt.



Using the \texttt{**dyn} representation, Example 28.2 might be encoded as follows:

\texttt{**dyn**kern**kern**kern
*staff1/2*staff2*staff1*staff1
**clefF4*clefG2*clefG2
**M12/8*M12/8*M12/8
====
0.8f\{-4\}8F\# 8en8r2.ffn
\texttt{.8A\#8r.
\texttt{.8Dn 8cn[8cc\#.
\texttt{.8F\#8cc\#].
\texttt{.8AAn 8Gn8ccn.
\texttt{.8C\#8bn.
\texttt{.8FFn 8E-8b-[4.aan
\texttt{.8AAn8ddn.
\texttt{.8DD- 8C-8b-.
\texttt{.8FFn4.dd8aa]
\texttt{.8BBB- 8AA-.4ff\#
\texttt{.8DDn..
***v*v
===
\texttt{.8r8r
1.4fp\{-4.5\}\{8d-\{8gn 8ccn 8ffn
1.6>\{-4.3\}/1.5/..
)8fn8bn 8een
2.4]\{-4.3\}..
2.5pp\{-5\}24A-4.cn 4.fn 4.b\}
\texttt{.24d-.
2.8>/1.4/24Dn.
)8.GG.
4.2]16BB-\}.
\texttt{.8r8r
4.4fp\{-4.5\}\{8An\{8e- 8a- 8dd-
4.6>/1.5/..
)8d-8gn 8ccn
5.2]..
5.4pp\{-4.5\}24F\#4.B- 4.en 4.a\}
5.7>/1.4/24Bn.
)24BB-.
)8.DD.
6.8]16GG\}.
===
*-*-*-}


\par 
The \texttt{*staff1/2} tandem interpretation indicates that the dynamic markings
pertain to both staffs 1 and 2, however all vertical
\texttt{**dyn}
distance measures are encoded with respect to staff 1.
(Reversing the numerical order -$\,$- \texttt{*staff2/1} -$\,$- would cause
all distances to be measured with respect to staff 2.)
The token \texttt{0.8f\{-4\}} means that the signifier \textbf{\textit{f}}} is located
0.8 quarter-duration spaces from the beginning of the bar and
4 staff-line steps below the center line of staff 1.
The token \texttt{1.6>\{-4.3\}/1.5/} means that a wedge diminuendo marking begins
1.6 quarter-durations from the beginning of the bar;
the size of the opening of the wedge is 1.5 staff-line steps wide and
the center of the opening is located 4.3 staff-line steps below the center
line for staff 1.
The token \texttt{2.4]\{-4.3\}} means that a wedge diminuendo marking ends
2.4 quarter-durations from the beginning of the bar;
the tip of the wedge converges at a point 4.3 staff-line steps below
the center line for staff 1.
Changing this value allows tilted wedges to be represented.


\subsection*{The \textit{**dB} Representation}

\par 
The
\texttt{**dB}
representation provides a way to represent
intensity in decibels.
Decibels can be expressed in relative or absolute terms.
Absolute values are represented according to sound pressure level (SPL).
An absolute representation is indicated by the presence of the \texttt{*SPL}
tandem interpretation.
Zero decibels (SPL) corresponds roughly to the quietest sound
detectable under ideal circumstances.
A quiet room is roughly 40 dB in intensity;
a conversation produces roughly 70 dB,
a vacuum cleaner produces roughly 80 dB,
a noisy factory produces roughly 90 dB,
and a passing loud motorcycle generates roughly 100 dB (SPL).

\par 
The \texttt{**dB} representation provides a convenient way
to represent sound intensity in a numerical form.
A numerical representation allows us to carry out a
variety of calculations and comparisons.


\subsection*{The \textit{db} Command}

\par 
The
\textbf{db}
command translates dynamic markings to dynamic level expressed
in decibels;
specifically,
\textbf{db}
translates from the \texttt{**dynam} representation to
\texttt{**dB} representation.
By default,
\textbf{db}
uses the following mapping:

dynamic leveldB SPL

fffffff115
ffffff110
fffff105
ffff100
fff90
ff80
f75
mf70
mp65
p60
pp55
ppp50
pppp45
ppppp40
pppppp35
ppppppp30
v+5


\par 
Notice the presence of the
\textit{accent}
signifier (\texttt{v});
the assigned value (\texttt{+5}) means that any encoded accents will
receive a decibel level 5 dB higher than the basic sound pressure level
at that point in the score.
For example, an explicitly accented note occurring in a \textit{fortissimo}
passage will be assigned a value of 85 dB SPL.

\par 
Users can define other mappings by using the
\textbf{-f}
option for \textbf{db}.
With
\textbf{-f}
the user provides a filename that contains the non-default mapping values.
The format for this file is the same as that shown in the above table.
Each table entry specifies a dynamic marking, followed by a tab, followed
by a numerical value.

\par 
In the case of crescendo and diminuendo markings, the
\textbf{db}
command attempts to interpolate a series of values between any preceding
and subsequent dynamic markings.
The following example shows a pianissimo marking at the beginning of
measure 5;
a crescendo marking spans all of measure 6, and a mezzo-forte marking
appears in measure 7.
The right-most spine shows the corresponding output generated by the
\textbf{db}
command.
It shows an interpolation between the two dynamic levels.

\texttt{**dynam**dB
*SPL*
=5=5
pp55
..
..
..
=6=6
<58
(61
..
(64
]67
=7=7
..
mf70
..}


\par 
The interpolation begins with the crescendo indicator and increments
for each continuation signifier (i.e., the open parentheses).
Interpolations are linear and continue up to the crescendo termination signifier.
The size of the increment value depends on starting and ending dynamic levels
as well as the number of crescendo-continuation signifiers.
In the above case four pertinent
crescendo signifiers separate the pianissimo and mezzo-forte markings;
each of these records has been incremented by 3 decibels.
Where necessary, decimal values are output.
Notice that null tokens (such as those in the middle of measure 6)
are ignored in the calculation.


\subsection*{Processing Dynamic Information}

\par 
The
\texttt{**dB}
representation can be used to assist a number
of tasks related to musical dynamics.
Suppose, for example, that we want to compare the average overall
dynamic levels for two arabesques:

\par 

\texttt{extract -i '**dynam' arabesque1 | db | rid -GLId | stats}
\\
\texttt{extract -i '**dynam' arabesque2 | db | rid -GLId | stats}



\par 
Similarly, we might compare the overall dynamic levels between
two sections of a single work.
Perhaps we wish to know whether the exposition is on average louder
than the development section:

\par 

\texttt{yank -s Exposition -r 1 symphony3 | extract -i '**dynam' $\backslash$

| db | rid -GLId | stats}

\texttt{yank -s Development -r 1 symphony3 | extract -i '**dynam' $\backslash$

| db | rid -GLId | stats}




\par 
Does a work tend to begin quietly and end loudly, or vice versa?
Here we might compare the first 10 measures with the final 10 measures.
Notice the use of
\textbf{ditto}
to increase the number of values participating in the calculation of the
average dynamic level:

\par 

\texttt{yank -n = -r 1-10 janacek | extract -i '**dynam' $\backslash$

| ditto -s = | db | rid -GLId | stats}

\texttt{yank -n = -r '\$-10-\$' janacek | extract -i '**dynam' $\backslash$

| ditto -s = | db | rid -GLId | stats}




\par 
Suppose we want to determine whether there is an association between
dynamic levels and pitch height for Klezmer music.
That is, does the music tend to be quieter for lower pitches and
louder for higher pitches?
A straightforward way to determine this is to compare dynamic level
with pitch height -$\,$- represented in semitones
(\texttt{**semits}).
The
\textbf{correl}
command can then be used to measure Pearson's coefficient of correlation.
If there is a relationship between pitch height and dynamic level
then the correlation should be positive.


\par 

\texttt{semits klezmer | correl -s \^{}= -m}


\par 
This command assumes an input consisting of two spines -$\,$- one pitch-related
and a \texttt{**dB} spine.
The
\textbf{-s}
option for
\textbf{correl}
is used to skip barlines so bar numbers aren't included in the calculation.
The
\textbf{-m}
option for
\textbf{correl}
disables the "matched pairs" criterion.
Normally, if a number is found in one spine but not the other then
\textbf{correl}
will complain and terminate.
With the
\textbf{-m}
option, each encoded pitch need not have a corresponding dynamic level indication
and vice versa.

\par 
Similarly, we could use this same approach to determine whether there is
a relationship between duration and dynamic level.
Are longer notes more likely to be louder in Klezmer music?

\par 

\texttt{dur klezmer | correl -s \^{}= -m}



\par 
A variation on this procedure might be to restrict the comparison over a specified
pitch range.
For example, one might think that higher pitches tend to be louder but that
lower pitches are neither softer nor louder than usual.
In order to test this view we can use the
\textbf{recode}
command to reassign "low" pitches to a single value.
By way of illustration, the reassignment might presume that below G4 (semits=7)
there is no relationship between pitch height and dynamic level.
We might recode all values lower than 7 to a unique string (such as `XXX') and
then use
\textbf{grep -v}
to eliminate these notes from a subsequent correlation:

\par 

\texttt{extract -i '**kern' klezmer | semits recode > temp1
\\
extract -i '**dB' klezmer > temp2
\\
assemble temp1 temp2 | grep -v 'XXX' | correl -s \^{}= -m}



\subsection*{Terraced Dynamics}

\par 
Suppose we want to identify whether various works exhibit "terraced"
or "graduated" dynamics.
In the case of terraced dynamics, we would expect to see many relatively
abrupt dynamic contrasts, such as alternations between \textit{forte} and \textit{piano}.
There are several ways of approaching this question.
One approach might translate
\texttt{**dynam}
data to
\texttt{**dB}
data and then calculate the average (or maximum) changes in dynamic level.
If a work contains many crescendos and diminuendos markings,
then most of the changes in \texttt{**dB} values will be small.
Conversely, alternations between contrasting dynamic levels will cause
the average decibel differences to be larger.
The
\textbf{xdelta}
command can be used to calculate the changes in dynamic level.
Notice that it is important to avoid using the
\textbf{ditto}
command since repeated dynamic level values will cause the average
dynamic difference to approach zero.

\par 

\texttt{extract -i '**dynam' haendel | db | xdelta -a -s = | rid -d $\backslash$

| stats}



\par 
Another approach to this problem might be to count the number of dynamic contrasts,
avoiding the use of the
\textbf{db}
command.
In the following pipeline, we use
\textbf{context}
to generate pairs of dynamic markings, and then use
\textbf{grep}
to count the number of alternations between \textbf{\textit{f}} and \textbf{\textit{p}}.

\par 

\texttt{extract -i '**dynam' haendel | grep -v '[][()=rX]' | rid -d $\backslash$

| context -n 2 | grep -c 'f p'}

\\
\texttt{extract -i '**dynam' haendel | grep -v '[][()=rX]' | rid -d $\backslash$

| context -n 2 | grep -c 'p f'}




\subsection*{Dynamic Swells}

\par 
Conceptually, crescendos and diminuendos can be paired to form one of two dynamic
gestures.
A "swell" gesture consists of a crescendo followed by a diminuendo.
Conversely, a "dip" gesture would consist of a diminuendo followed by a
crescendo.
Musical intuition would suggest that swell gestures are more common than
dip gestures.
We could test this view as follows:

\par 

\texttt{extract -i '**dynam' grieg | grep -v '[][()=rX]' | rid -d $\backslash$

| context -n 2 | grep -c '< >'}

\texttt{extract -i '**dynam' grieg | grep -v '[][()=rX]' | rid -d $\backslash$

| context -n 2 | grep -c '> <'}




\subsection*{MIDI Dynamics}

\par 
Dynamic level data is not always easily available.
One possible source is to translate MIDI key-velocity data to
an estimated decibel value.
Actual sound pressure levels will depend on the timbre of the MIDI
sounds, the specific pitch played, and the volume on the output amplifier.
Nevertheless, a rough estimate of sound pressure level may be useful
for various analytic tasks.
Recall that in the
\texttt{**MIDI}
representation, key-velocity data is encoded
as the final number in three-number tokens where numbers are separated by slashes.
The first value in the triplet is elapsed clock ticks and the second value
is the MIDI key number (positive for key-on events, negative for key-off events).
By way of reminder, the following example shows three
\texttt{**kern}
notes with a corresponding **MIDI representation.

\texttt{**kern**MIDI
**Ch1
4c72/60/64
4d72/-60/64 72/62/64
4e72/-62/64 72/64/64
.72/-64/64
*-*-}


\par 
In order to translate to a **dB representation, we must first isolate the
key velocity values for key-on events.
The following
\textbf{humsed}
command simply eliminates all data up to (and including) the last slash character:

\par 

\texttt{extract -i '**MIDI' mono\_input | humsed 's/.*$\backslash$///'}


\par 
This will leave us with just the key-down velocity data.
Let's suppose that the following rough decibel equivalents are established:

key velocityapproximate dB SPL
12785
10080
9077
8074
7070
6065
5060
4053
3044
2032
1021
110
00


\par 
An appropriate reassignment file for
\textbf{recode}
would begin as follows:

\texttt{>=12785
>=10080
>=9077
>=8074
>=7070}
etc.


\par 
The completed translation would be accomplished by the following pipeline:

\par 

\texttt{extract -i '**MIDI' mono\_input | humsed 's/.*$\backslash$///' $\backslash$

| recode -f reassign | sed 's/**MIDI/**dB/'}



\par 
Notice the use of the
\textbf{sed}
command to replace the **MIDI interpretation by a **dB interpretation.



\subsection*{Reprise}

\par 
In this chapter we have introduced three representations related
to musical dynamics.
The
\texttt{**dyn}
representation allows us to encode dynamic markings
as they appear visually in a printed score.
Unfortunately, traditional notated dynamic markings are often confusing
or ambiguous.
In order to facilitate some types of analytic processing it is useful
to generate a more rationalized interpretation of the dynamics
of a work.
The
\texttt{**dynam}
representation provides a canonical scheme
for representing basic notated dynamic markings
where ambiguities are resolved by explicitly interpreting
the meaning of dynamic markings.
A third scheme
(\texttt{**dB})
provides a scheme for representing continuous
dynamic levels in decibels.
We have seen that the
\textbf{db}
command (which translates from \texttt{**dynam} to \texttt{**dB})
allows us to pose and answer a variety of questions related to the
dynamic organization of music.

\\
\begin{itemize}
\item 

\textbf{Next Chapter}
\item 

\textbf{Previous Chapter}
\item 

\textbf{Table of Contents}
\item 

\textbf{Detailed Contents}
\\\\

� Copyright 1999 David Huron
\end{document}
