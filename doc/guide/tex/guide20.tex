% This file was converted from HTML to LaTeX with
% Tomasz Wegrzanowski's <maniek@beer.com> gnuhtml2latex program
% Version : 0.1
\documentclass{article}
\begin{document}



  
  
    
      
      
      
    
  



\\
\\

\section*{Chapter20}


[\textit{Previous Chapter}]
[\textit{Contents}]
[\textit{Next Chapter}]


\section*{Strophes, Verses and Repeats}



We often tend to think of musical information as a linear stream of
successive events.
However, there are many circumstances where musical
information exhibits more complex structures.
These include such structural devices as repeats, da capos,
first and second ending, multiple verses, alternative or
\textit{ossia}
passages, different performance renditions,
divergent sources, and competing editions or versions.

\par 
This chapter describes the basic Humdrum mechanisms
for representing non-linear musical structures.
The two critical mechanisms are the Humdrum
\textit{section}
and \textit{strophe}.
We will encounter examples using the
\textbf{yank},
\textbf{thru},
and
\textbf{strophe}
commands.

\par 

\subsection*{Section Labels}

\par 
Musical scores are often notated to take advantage of
repetitions in the music.
Devices such as repeat marks, \textit{Da Capo}, \textit{Dal Segno}, \textit{Codas},
and other mechanisms make it possible to represent a musical
work in an abbreviated format.
Humdrum provides corresponding mechanisms that allow works to
be represented in succinct ways.

\par 
Humdrum files may be logically divided into segments or passages
by encoding Humdrum
\textit{section labels.}
A section label is a type of tandem interpretation that consists
of a single asterisk, followed by a greater-than sign,
followed by a keyword that labels the section.
The following are examples of section labels.

\par 

\texttt{
*>Coda
\\
*>1st Ending
\\
*>Refrain
\\
*>Exposition>2nd Theme
}


\par 
Notice that spaces can appear in section labels -$\,$- as in \texttt{1st Ending}.
Sections begin with a section label and generally end when another
section label is encountered.
Sections also end whenever all spines are assigned new
exclusive interpretations, or all spines terminate.
If there is more than one spine present in a passage,
identical section labels must appear concurrently in all spines.


\subsection*{Expansion Lists}

\par 
Rather than encode multiple copies of a passage,
a single instance may be encoded and labelled as a section.
The complete version of the work can be reconstructed by referring to an
\textit{expansion list.}
An expansion list is another tandem interpretation that contains
an ordered list of section labels.
The list is specified in square brackets.
Like section labels, expansion lists begin with an
asterisk followed by a greater-than sign.
In effect, the expansion list indicates how the abbreviated
file should be expanded to a full-length encoding.
Consider the following expansion list:

\par 

\texttt{
*>[verse1,refrain,verse2,refrain]
}


\par 
This list indicates that the abbreviated file contains
(at least) three sections, labelled "\texttt{verse1},"
"\texttt{verse2}" and "\texttt{refrain}."
When the file is expanded, the "\texttt{refrain}" section should
be repeated following each verse.


\subsection*{Using \textit{yank} to Extract Sections}


\par 
We encountered the
\textbf{yank}
command earlier in
Chapter 12.
Recall that
\textbf{yank}
can be used to extract material by
\textit{section}
using the
\textbf{-s}
option.
For example, if the appropriate section is labelled, we might
extract the coda of a work as follows:

\par 

\texttt{yank -s Coda -r 1} \textit{file}


\par 
Recall that the
\textbf{-r}
option is mandatory with \textbf{yank};
in this case, it identifies the
\textit{first}
occurrence of a section labelled \texttt{Coda}.


\subsection*{Using the \textit{thru} Command to Expand Encodings}

\par 
The Humdrum
\textbf{thru}
command expands
\textit{abbreviated format}
representations to a so-called
\textit{through-composed format}
in which repeated passages are expanded according to an expansion list.
When the
\textbf{thru}
command is invoked, it eliminates any expansion lists present in
the input;
in addition,
\textbf{thru}
places a \texttt{*thru} tandem interpretation in all spines
immediately following each instance of an exclusive interpretation in the input.
This marks the file as being in a through-composed format.
Any other \texttt{*thru} tandem interpretations encountered in the input
are subsequently discarded.
As a result, running a file through
\textbf{thru}
twice will not result in further changes to the file.


\subsection*{Alternative Versions}

\par 
For works encoded in an abbreviated format, it is not always useful
to expand it according to a single fixed recipe.
Depending on the performance practice, individual performer, or edition,
certain repeats may be avoided, passages may be added, or
material eliminated altogether.
In short, several different versions or interpretations of
the overall organization of a work may exist.


\par 
Humdrum provides a mechanism by which several alternative versions
of the overall organization of a work may co-exist in the same file.
This is achieved simply by encoding more than one expansion list.
In order to distinguish different versions,
each expansion list is given a unique \textit{version label}.

Consider the following expansion lists:

\par 

\texttt{
*>Gould1982[A,A,B]
\\
*>Landowska[A,A,B,B]
}



\par 
Here we see two expansion lists, one carries the version label \texttt{Gould1982}
and the other is labelled version \texttt{Landowska}.
These expansion lists might encode different interpretations
of the repeats in a rounded binary form -$\,$- Landowska performed the
second repeat whereas Gould (1982) did not.
When the
\textbf{thru}
command is invoked, the user can specify which
\textit{version}
is intended using the
\textbf{-v}
option.
The appropriate through-composed expansion will be output.

\par 
The following example illustrates the use of the
\textbf{thru}
command in selecting particular versions of data in a file.
Three sections are encoded in the file -$\,$- labelled A, B and C.
Each section in this example contains just a single data record.
Three expansion lists are encoded:
one is unlabelled, a second is labelled \texttt{long} and a third is
labelled \texttt{weird}.
\\\\

\texttt{**example\texttt{**example
\texttt{*>[A,B,A,C]\texttt{*>[A,B,A,C]
\texttt{*>long[A,A,B,A,C]\texttt{*>long[A,A,B,A,C]
\texttt{*>weird[C,A,C]\texttt{*>weird[C,A,C]
\texttt{*>A\texttt{*>A
\texttt{data-A\texttt{data-A
\texttt{*>B\texttt{*>B
\texttt{data-B\texttt{data-B
\texttt{*>C\texttt{*>C
\texttt{data-C\texttt{data-C
\texttt{*-\texttt{*-}

Consider the following command:

\par 

\texttt{thru -v weird file}


\par 
The corresponding "through-composed" output would be as follows:
\\\\

\texttt{**example\texttt{**example
\texttt{*thru\texttt{*thru
\texttt{*>C\texttt{*>C
\texttt{data-C\texttt{data-C
\texttt{*>A\texttt{*>A
\texttt{data-A\texttt{data-A
\texttt{*>C\texttt{*>C
\texttt{data-C\texttt{data-C
\texttt{*-\texttt{*-}

Notice that all expansion-list records have been eliminated from
the output.
A \texttt{*thru} tandem interpretation has been added to all output spines
immediately following the exclusive interpretation.
Also notice that there are now two sections in the output sharing
the same label (\texttt{*>C}).
This duplication of section-labels is not permitted in abbreviated-format
encodings and can only occur in through-composed documents.

\par 
Without the
\textbf{-v}
option,
\textbf{thru}
expands the abbreviated file according to the
\textit{unlabelled}
(default) expansion list.
So the following command would result in an output consisting
of section A, followed by section B, followed by section A (again),
followed by section C:

\par 

\texttt{thru file}



\subsection*{Section Types}

\par 
Suppose we had two different theorists -$\,$- Smith and Jones -$\,$-
who had analyzed the same work differently.
Smith thinks there are basically two sections in the work,
whereas Jones argues that there are essentially three sections.
Humdrum permits alternative schemes of section labels
to coexist in a file by allowing the user to designate section
\textit{types}.
A section label is considered to have a "type" when more
than one greater-than sign (\texttt{>}) is present in the label.
Consider the following example of sections defined by Smith and Jones:
\\\\

\texttt{**Example
\texttt{*>Smith>A
\texttt{*>Jones>A
\texttt{data1
\texttt{*>Jones>B
\texttt{data2
\texttt{*>Smith>B
\texttt{data3
\texttt{*>Jones>C
\texttt{data4}
\texttt{*-

Both Smith and Jones label the work as beginning with section `A'.
Later Jones's `B' section begins; then Smith's `B' section;
then Jones's `C' section.
Note that Smith's `B' section also contains the material Jones
has identified as section `C'.

\par 
Normally, the
\textbf{yank}
command extracts a labelled section up to the next occurrence
of a section label.
However, the
\textbf{-t}
option causes
\textbf{yank}
to ignore all section labels except for a specified type.
We could extract Smith's `B' section by using the
\textbf{-t}
option to limit extraction to "Smith"-type section labels:

\par 

\texttt{yank -t Smith -s B}


\par 
This command would produce the following output:
\\\\

\texttt{**Example
\texttt{*>Smith>B
\texttt{data3
\texttt{*>Jones>C
\texttt{data4}
\texttt{*-

Without the
\textbf{-t}
option,
\textbf{yank}
will simply extract material up to the occurrence of the next
section label.
Note that section types can be used to define innumerable
alternative organizations for a single document.


\subsection*{Hierarchical Sections}

\par 
For many applications, it is useful to define "nested" structures
where two or more sections form part of a larger section.
Humdrum section labels allow users to distinguish hierarchical \textit{levels}.
Levels are indicated by the number of greater-than signs
following the section type.
Consider the following:
\\\\

\texttt{**Example
\texttt{*>Form>Exposition
\texttt{data1
\texttt{*>Form>>1st Theme
\texttt{data2
\texttt{*>Form>>2nd Theme
\texttt{data3
\texttt{*>Form>Development
\texttt{data4
\texttt{*>Form>Recapitulation
\texttt{*>Form>>1st Theme
\texttt{data5
\texttt{*>Form>>2nd Theme
\texttt{data6
\texttt{*>Form>Coda
\texttt{data7
\texttt{*-

All of the above section labels are identified as type \texttt{Form}.
However, two levels are distinguished (denoted by \texttt{>} and \texttt{>>}).
Subsections are specified by increasing the number of greater-than signs,
hence \texttt{2nd Theme} is a subsection.
When
\textbf{yank}
is invoked, it will extract the identified section up to the
next section of comparable level.
The operation is illustrated in the following sample commands:
indicating the first and second themes.

\texttt{yank -t Form -s '1st Theme' -r 1(}extracts up to \texttt{>Form>>2nd Theme})
\texttt{yank -t Form -s '2nd Theme' -r 1(}extracts up to \texttt{>Form>Development})
\texttt{yank -t Form -s 'Exposition' -r 1(}extracts up to \texttt{>Form>Development})



\par 
For example, the second theme from the recapitulation can be extracted
as follows:

\par 

\texttt{yank -t Form -s '2nd Theme' -r 2}


\par 
Alternatively:

\par 

\texttt{yank -t Form -s Recapitulation} \textit{file}\texttt{ | yank -t Form -s '2nd Theme' -r 1}



\subsection*{Using the \textit{yank} and \textit{thru} Commands}

\par 
Section labels can be used in a wide number of applications.
By way of illustration, here are a few pipeline processes
involving section labels.
First, we might ask the question -$\,$- how does the user
know what sections labels are present in a document?
This is a task for \textbf{grep}:

\par 

\texttt{grep '\^{}$\backslash$*>'} \textit{file}


\par 
This command will also output any expansion-lists.
If we want to restrict our output to identifying which
\textit{versions}
are available for a document we would look for the presence
of square brackets:

\par 

\texttt{grep '\^{}$\backslash$*>.*$\backslash$[.*$\backslash$]'} \textit{file}



\par 
How many notes are there in the exposition?

\par 

\texttt{yank -t Form -s Exposition -r 1} \textit{file}\texttt{ | census}}



\par 
How many phrases are there in the development?

\par 

\texttt{yank -t Form -s Development -r 1} \textit{file}\texttt{ | grep -c '\{'}



\par 
Extract the figured bass for the third recitative:

\par 

\texttt{yank -s Recitativo -r 3} \textit{file}\texttt{ | extract -i '**B-num'}



\par 
Compare the estimated key for the second theme in the exposition
versus the estimated key for the second theme in the recapitulation:

\par 

\texttt{yank -t Form -s '2nd Theme' -r 1} \textit{file}\texttt{ | key}
\\
\texttt{yank -t Form -s '2nd Theme' -r 2} \textit{file}\texttt{ | key}



\par 
Determine the nominal (non-rubato) duration of Gould's performance
of the work:

\par 

\texttt{thru -v Gould1982} \textit{file}\texttt{ | extract -i '**kern' | extract -f 1 $\backslash$

| dur -d | rid -GLId | grep -v '\^{}=' | stats | grep -i total}



\par 
Perform the first three measures from the second section of a binary form:

\par 

\texttt{yank -s B} \textit{file}\texttt{ | yank -o = -r 1-3 | midi | perform}



\subsection*{Strophic Representations}

\par 
Section labels and versions allow Humdrum users to select
alternative groups of (horizontal) records within a Humdrum file or document.
In other circumstances it is useful to be able to select
alternative (vertical) paths within a file.
Strophic representations may be conceived as
"alternative concurrent paths" through a Humdrum document.
Examples of alternative concurrent representation paths might
include (1) texts for different verses of a song,
(2) alternative renditions of the same passage (such as
\textit{ossia}
passages),
or (3) differing editorial interpretations of a given note or sequence of notes.

\par 
Structurally, strophic data must begin from a single common spine,
split apart into two or more alternative spines, and then rejoin
to form a single spine.
Since the strophes split from a common spine, they all necessarily
begin by sharing the same exclusive interpretation.
Different exclusive interpretations may be introduced in the strophic
passage -$\,$- provided all strophic spines end up sharing the
same data type just prior to being rejoined.

\par 
The beginning of a strophic passage is signalled by the presence of a
\textit{strophic passage initiator}
-$\,$-  a single asterisk followed by the keyword "strophe"
(\texttt{*strophe}).
The end of a strophic passage is signalled by the
\textit{strophic passage terminator}
-$\,$-  a single asterisk followed by the upper-case letter `S' followed
by a minus sign (\texttt{*S-}).
Each spine within the strophic passage begins with a
\textit{strophe label}
and ends with a
\textit{strophe end indicator}
(\texttt{*S/fin}).
Strophe labels may consist of either alphanumeric names, or numbers.
Numerical labels should be used when the strophic data imply some
sort of order, such as verses in a song.
Alphanumeric labels are convenient for distinguishing different editions or
\textit{ossia}
passages.
The following example encodes a melodic phrase containing four numbered verses
from "Das Wandern" from
\textit{Die Schoene Muellerin}
by Schubert:
\\\\

\texttt{

\texttt{!! Franz\texttt{Schubert,\texttt{`Das Wandern' from "Die Schoene Muellerin"


\texttt{**kern\texttt{**silbe
\texttt{*k[b-e-]\texttt{*Deutsch
\texttt{*\texttt{*solo
\texttt{*>[1,1,1,1]\texttt{*>[1,1,1,1]
\texttt{*>1\texttt{*>1
\texttt{*\texttt{*strophe
\texttt{*\texttt{*\^{}
\texttt{*\texttt{*\^{}\texttt{*\^{}
\texttt{*\texttt{*S/1\texttt{*S/2\texttt{*S/3\texttt{*S/4
\texttt{8f\texttt{Das\texttt{Vom\texttt{Das\texttt{Die
\texttt{=5\texttt{=5\texttt{=5\texttt{=5\texttt{=5
\texttt{8f\texttt{Wan-\texttt{Was-\texttt{sehn\texttt{Stei-
\texttt{8b-\texttt{-dern\texttt{-ser\texttt{wir\texttt{-ne
\texttt{8a\texttt{ist\texttt{ha-\texttt{auch\texttt{selbst,
\texttt{8ee-\texttt{des\texttt{-ben\texttt{den\texttt{so
\texttt{=6\texttt{=6\texttt{=6\texttt{=6\texttt{=6
\texttt{(16dd\texttt{M�l-\texttt{wir's\texttt{R�-
\texttt{16ff)\texttt{|\texttt{|\texttt{|\texttt{|
\texttt{(16dd\texttt{-lers\texttt{ge-\texttt{-dern\texttt{sie
\texttt{16b-)\texttt{|\texttt{|\texttt{|\texttt{|
\texttt{8f\texttt{Lust,\texttt{-lernt,\texttt{ab,\texttt{sind,
\texttt{8dd\texttt{das\texttt{vom\texttt{den\texttt{die
\texttt{=7\texttt{=7\texttt{=7\texttt{=7\texttt{=7
\texttt{(8.cc\texttt{Wan-\texttt{Was-\texttt{R�-\texttt{Stei-
\texttt{16a)\texttt{|\texttt{|\texttt{|\texttt{|
\texttt{8b-\texttt{-dern!\texttt{-ser!\texttt{-dern!\texttt{-ne!
\texttt{8r\texttt{\%\texttt{\%\texttt{\%\texttt{\%
\texttt{*\texttt{*S/fin\texttt{*S/fin\texttt{*S/fin\texttt{*S/fin
\texttt{*\texttt{*v\texttt{*v\texttt{*v\texttt{*v
\texttt{*\texttt{*S-
\texttt{*-\texttt{*-

}
Notice that this file contains a single section labelled `1'
and that an expansion list occurs near the beginning of the file
that indicates section 1 is to be repeated 4 times in total.

\par 
The strophic passage pertains only to the spine marked
\texttt{**silbe}.
The
\texttt{**silbe}
representation pertains to syllabic text
encoding and is a pre-defined representation in Humdrum.
The \texttt{**silbe} representation is discussed in
Chapter 27.
Following the strophic passage indicator (\texttt{*strophe}),
the spine is split apart until the required number of verses
are generated.
Then each spine is labelled with its own strophe label.
Since the verses have an order, it is appropriate to label them
with numbers: \texttt{*S/1, *S/2,} and so on.
The individual verses are terminated with strophe end indicators
(\texttt{*S/fin}), the spines rejoin, and then a strophic passage
terminator (\texttt{*S-}) marks the end of the strophic passage.


\subsection*{The \textit{strophe} Command}

\par 
The Humdrum
\textbf{strophe}
command can be used to isolate or extract selective strophic data.
The
\textbf{-x}
option for
\textbf{strophe}
allows the user to extract a particular labelled strophe.
Consider, for example the effect of the following command:

\par 

\texttt{strophe -x 3 schubert}


\par 
Using the above data, the result is:
\\\\

\texttt{!! Franz Schubert, `Das Wandern' from "Die Schoene Muellerin"


**kern**silbe
*k[b-e-]*Deutsch
*>[1,1,1,1]*>[1,1,1,1]
**solo
*>1*>1
8fDas
=5=5
8fsehn
8b-wir
8aauch
8ee-den
=6=6
(16ddR�-
16ff)|
(16dd-dern
16b-)|
8fab,
8ddden
=7=7
(8.ccR�-
16a)|
8b--dern!
8r\%
*-*-}

Notice that all of the tandem interpretations related to the
strophe organization are eliminated from the output.


\par 
Suppose that we wanted to create a through-composed version
of the entire work.
We would expect as output, just two spines -$\,$- the
\texttt{**kern}
spine and the \texttt{**silbe} spine.
First, we need to create the full length version using the
\textbf{thru}
command.
This will take the default expansion list, and repeat the
appropriate section for each successive verse.

\par 

\texttt{thru schubert}


\par 
The effect of this will be to simply repeat section 1 four times.
However, each repetition will contain all four verses.
We can use the
\textbf{strophe}
command to eliminate the unwanted verse texts at each verse.
When no option is given,
\textbf{strophe}
operates by preserving strophes in numerical order.
That is, when it encounters the first strophic section it will
preserve strophe \#1 (\texttt{*S/1});
then when it encounters the next strophic section it will
preserve strophe \#2 (\texttt{*S/2}).
And so on.
In summary, the follow command will create a proper through-composed
rendition of the Schubert lieder illustrated above.

\par 

\texttt{thru schubert | strophe}


\par 
Incidentally, the input passage need not necessary begin with strophe \#1.
The
\textbf{strophe}
command will adapt to the input, and use the lowest previously unencountered
strophe number.


\subsection*{Using the \textit{strophe} and \textit{thru} Commands}


\par 
As noted, the strophe technique can be used to encode
different editorial interpretations of a single work.
Suppose for example that we had two editions of the Bach
chorale harmonizations: Erk and Reimenschneider.
We could select the Erk edition as follows:

\par 

\texttt{strophe -x Erk chorale166}



\par 
In a strophic song, suppose we would like to
compare the number of syllables in the first and second verses.
We begin by selecting the appropriate verse,
extract the syllable spine, eliminate all non-data records,
eliminate any other special signifiers (like barlines),
and finally count the number of remaining records.
We repeat this procedure for both verses:

\par 

\texttt{strophe -x 1} \textit{file}\texttt{ | extract -i '**silbe' | rid -GLId $\backslash$

| grep -v [=$\backslash$|\%] | wc -l}

\texttt{strophe -x 2} \textit{file}\texttt{ | extract -i '**silbe' | rid -GLId $\backslash$

| grep -v [=$\backslash$|\%] | wc -l}



\par 
(In the \texttt{**silbe} representation, the vertical bar (|) and
the percent sign (\%) have special meanings so the
\textbf{grep -v}
is used to eliminate them along with barlines.)



\subsection*{Reprise}

\par 
Between stophes and sections, highly non-linear musical documents
can be constructed.
We have seen how section labels can be defined,
how lists of sections ("expansion lists")
can be constructed and expanded to through-composed formats using the
\textbf{thru}
command.
An unlabelled expansion list is the default version.
Other versions have labelled expansion lists.

\par 
Several different
\textit{types}
of section labels can coexist in the same document and the
\textbf{yank}
command can be instructed to ignore all sections other than
a certain type via the
\textbf{-t}
option.

\par 
The basic ideas introduced in this chapter are summarized in the
following table.
\\\\

sectionpassage defined by a section label, ends with occurrence of
section label of identical or higher level
section labeltandem interpretation beginning: \texttt{*>} and not containing
square brackets
section typefirst part of section label: \texttt{*>\textit{type\texttt{>}
expansion listtandem interpretation beginning \texttt{*>} and containing a list of
section labels in square brackets, e.g. \texttt{*>[A,B,A]}
versiona labelled expansion list, e.g. \texttt{*>ternary[A,B,A]}
levelhierarchical level of a section, designed by the number of `>'
following the section type, e.g. \texttt{*>type>>>name} is lower
than \texttt{*>type>name}
abbreviated formatHumdrum document encoded using expansion lists
through-composedHumdrum document encoded without expansion lists
\texttt{thru}command to create a through-composed document from an
abbrevatiated format
\texttt{thru -v}command to create a particular version of a through-composed
document
\texttt{yank -s}command to extract sections
\texttt{yank -t -s}command to extract sections limited to sections of a particular type
strophe1. alternative spine path, 2. command for extracting a particular
strophe
strophic passage initiatortandem interpretation indicating the beginning of a strophe (\texttt{*strophe})
strophic passage terminatortandem interpretation indicating the end of a strophe (\texttt{*S-})
strophe labeltandem interpretation labelling one of several alternative spine-
paths, begins \texttt{*S/}
strophe end indicatortandem interpretation indicating the end of some spine path,
e.g. \texttt{*S/fin}


\textit{Summary of terms related to sections and strophes}


\par 
In
Chapter 37
we will see further examples of how sections and
strophes are especially useful when producing electronic editions.

\\
\begin{itemize}
\item 

\textbf{Next Chapter}
\item 

\textbf{Previous Chapter}
\item 

\textbf{Table of Contents}
\item 

\textbf{Detailed Contents}
\\\\

� Copyright 1999 David Huron
\end{document}
