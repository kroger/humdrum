% This file was converted from HTML to LaTeX with
% Tomasz Wegrzanowski's <maniek@beer.com> gnuhtml2latex program
% Version : 0.1
\documentclass{article}
\begin{document}



  
  
    
      
      
      
    
  



\\
\\

\section*{Chapter27}


[\textit{Previous Chapter}]
[\textit{Contents}]
[\textit{Next Chapter}]


\section*{Text and Lyrics}



Musical texts include lyrics, librettos, stage directions, recitativo
and other collections of words.
It goes without saying that words provide important information
related to semantics, imagery, similes, allusion, irony, parody, word painting,
and emotion.
In addition, words can add to the sonorous dimension of music,
including rhyme schemes, word-rhythms, and phonetic and syllabic effects
such as alliteration, vowel-coloration.

\par 
Humdrum provides three pre-defined representations pertinent
to text or lyrics.
The
\texttt{**text}
representation can be used to represent \textit{words};
the
\texttt{**silbe}
representation can be used to represent \textit{syllables};
and the
\texttt{**IPA}
representation can be used to represent \textit{phonemes}
(via the International Phonetic Alphabet).
Discussion of the \texttt{**IPA} representation will be delayed
until
Chapter 34.
In this chapter we will look at various representational and processing
issues related to the manipulation of words and syllables.


\subsection*{The \textit{**text} and \textit{**silbe} Representations}


\par 
Syllable- and word-oriented representations are
illustrated in the following excerpt from a motet by Byrd (Example 27.1).
The encoded Humdrum data includes three spines:
\texttt{**text}, \texttt{**silbe} and \texttt{**kern}.
Normally, only the \texttt{**silbe} and \texttt{**kern} data would be
encoded -$\,$- since the \texttt{**text} spine can be generated
from the \texttt{**silbe} representation.
\\\\
\textbf{Example 27.1.}  From William Byrd, \textit{Why Do I use my paper ink and pen.}



\texttt{!!!COM: Byrd, William
!!!OTL: Why do I use my paper ink and pen.


**text**silbe**kern
*LEnglish*LEnglish*
=11=11=11
\texttt{..2r
WhyWhy2g
=12=12=12
dodo2b-
II2a
=13=13=13
useuse4g
mymy4g
paperpa-[2dd
=14=14=14
\texttt{.|2dd]
\texttt{.-per2cc
=15=15=15
inkink2.ff
andand(4ee
=16=16=16
\texttt{.|4dd
\texttt{.|4cc
\texttt{.|4b-
\texttt{.|4cc
=17=17=17
pen,pen,1dd
=18=18=18
andand1dd
=19=19=19
pen,pen,1dd
=20=20=20
\texttt{..2r
andand2ff
=21=21=21
callcall2.ee
mymy4dd
=22=22=22
witswits1cc\#
=23=23=23
toto2ee
counselcoun-[2aa
=24=24=24
\texttt{.-sel4aa]
whatwhat4ff
toto(8ee
\texttt{.|8dd
=25=25=25
\texttt{.|4cc
\texttt{.|4dd
\texttt{.|4.ee
\texttt{.|8dd
=26=26=26
\texttt{.|4ee
\texttt{.|4ee
\texttt{.|2dd)
=27=27=27
say,say,1cc\#
=28=28=28
*-*-*-}


Note that all three representations in Example 27.1 make use of
the common system for representing barlines.
In the
\texttt{**text}
representation tokens represent individual words.
In some scores, several words will be associated with a single moment
(or pitch), as in the case of
\textit{recitativo}
passages.
Multi-word tokens are encoded as Humdrum multiple-stops with a space
separating each word on a record.

\par 
In the
\texttt{**silbe}
representation tokens represent individual syllables.
In \texttt{**silbe} the hyphen (-) is used explicitly to signify syllable
boundaries and the tilde (~) is used to signify boundaries
between hyphenated words (necessarily also a syllable boundary).
In other words, four types of syllables are distinguished by \texttt{**silbe}:
(1) a single-syllable word,
(2) a word-initiating syllable,
(3) a word-completing syllable, and
(4) a mid-word syllable.
The following table illustrates how these signifiers are used:
\\\\
\textbf{Table 26.1.}

\textit{text}a single-syllable word
\textit{text-}a word-initiating syllable
\textit{-text}a word-completing syllable
\textit{-text-}a mid-word syllable
\textit{text~}a single-syllable word beginning a hyphenated multi-word
\textit{~text}a single-syllable word completing a hyphenated multi-word
\textit{~text~}a single-syllabe word continuing a hyphenated multi-word
\textit{~text-}a word-initating syllable continuing a hyphenated multi-word
\textit{-text~}a word-completing syllable -$\,$- part of a hyphenated multi-word

Both the \texttt{**text} and \texttt{**silbe} representations are able to
distinguish different tones of voice such as spoken voice, whispered voice,
laughing voice, emotional voice,
\textit{Sprechstimme}
and humming.
In addition, there are signifiers for indicating untexted laughter
and untexted sobs or crys.
Some sample signifiers are shown in Table 26.2
\\\\
\textbf{Table 26.2.}

\texttt{A-Z}upper-case letters A to Z
\texttt{a-z}lower-case letters a-z
\texttt{(}open parenthesis
\texttt{)}closed parenthesis
\texttt{\{}beginning of phrase
\texttt{\}}end of phrase
\texttt{\%}silence (rest) token (character by itself)
\texttt{M}humming voice (character by itself)
\texttt{[}beginning of spoken voice
\texttt{[[}beginning of whisper
\texttt{]}end of spoken voice
\texttt{]]}end of whisper
\texttt{<}beginning of \textit{Sprechstimme}
\texttt{>}\textit{end of Sprechstimme}
\texttt{\#}beginning of laughing voice
\texttt{\#\#}end of laughing voice
\texttt{@}laughter (no text)
\texttt{\&}sob or cry (no text)
\texttt{\$}beginning of emotional voice
\texttt{\$\$}end of emotional voice
\texttt{*}follows stressed word (\texttt{**test}) or stressed syllable (\texttt{**silbe})


\textit{Signifiers common to \textbf{**text} and \textbf{**silbe}}


\par 

\subsection*{The \textit{text} Command}

\par 
In most notated music, lyrics are written using a syllabic representation
rather than a word-oriented representation.
The \texttt{**silbe} representation is typically a better representation
of the score than \texttt{**text}.
However, for many analytic applications, words often prove
to be more convenient.
The Humdrum
\textbf{text}
command can be used to translate \texttt{**silbe} data to \texttt{**text} data.
In general, syllabic information is useful for addressing questions related to
rhythm and rhyme, whereas text information is more useful for addressing
questions related to semantics, metaphor, word-painting, etc.

\par 
Invoking the
\textbf{text}
command is straightforward:

\par 

\texttt{text inputfile > outputfile}



\par 
A simple text-related task might be looking for occurrences of a particular
word, such as the German "Liebe" (love).
If the lyrics are encoded in the
\texttt{**text}
representation, then a simple
\textbf{grep}
will suffice:

\par 

\texttt{grep -n 'Liebe' schubert}


\par 
Recall that the
\textbf{-n}
option gives the line number of any occurrences found.
If the input is encoded in the
\texttt{**silbe}
representation,
then the output of
\textbf{text}
can be piped to \textbf{grep}:

\par 

\texttt{extract -i '**silbe' schubert | text | grep -n 'Liebe'}



\par 
Given a \texttt{**silbe} input, a inventory of words can be generated using
\textbf{sort}
and
\textbf{uniq}
in the usual way:

\par 

\texttt{extract -i '**silbe' song | text | rid -GLId | sort | uniq}


\par 
Frequently, it is useful to search for a group of words
rather than individual words.
Suppose we are looking for the phrase "white Pangur."
The
\textbf{context}
command can be used to amalgamate words as multiple stops.
If we are looking for a phrase consisting of just two words,
we might use the
\textbf{-n 2}
option for \textbf{context}:

\par 

\texttt{text barber | context -n 2 | grep -i 'white Pangur'}



\par 
Alternatively, we might amalgamate words so they form sentences,
or at least phrases.
Puntuation marks provide a convenient marker for
ending the amalgamation process carried out by \textbf{context}.
In the following command, we have defined a regular expression
with a character-class containing all of the puntuation marks.
The output from this command will display all punctuated phrases
(one per line) that contain the phrase "white Pangur."

\par 

\texttt{text | context -e '[.,;?!]' | grep -i 'white Pangur'}



\subsection*{The \textit{fmt} Command}

\par 
Another common task is simply to provide a readable text of the
text or lyrics of a work.
Given a \texttt{**text} representation, we can use the
\textbf{rid}
command to eliminate all records except non-null data records.
This will result in a list of words -$\,$- one word per line.
UNIX provides a simple text formatter called
\textbf{fmt}
that will assemble words or lines into a block text where
all output lines are roughly the same width.



Consider the Gregorian chant
\textit{A Solis Ortus}
from the
\textit{Liber Usualis}
(shown in Example 27.2.)
\\\\
\textbf{Example 27.2.}  Beginning of chant \textit{A Solis Ortus}.



The Latin text for this chant can be formatted as follows:

\texttt{extract -i '**silbe' chant12 | text | rid -GLId | fmt -50}


\par 
The
\textbf{-50}
option tells
\textbf{fmt}
to place no more than 50 characters per line.
The default line-length is 72 characters.
The above pipeline produces the following output:

\par 

\\
\texttt{A solis ortus cardine ad usque terrae limitem,
\\
Christum canamus principem, natum Maria Virgine.
\\
Beatus auctor saeculi servile corpus induit: ut
\\
carne carnem liberans, ne perderet quos condidit.
\\
Castae parentis viscera cae lestis intratgratia:
\\
venter puellae bajulat secreta, quae non noverat.
\\
Domus pudici pectoris tem plum repente fit Dei:
\\
intacta nesciens virum, concepit alvo filium.}


\par 
Another useful output would have the text arranged with one sentence
or phrase on each line.
As before we can use the
\textbf{context}
command with the
\textbf{-e}
option to amalgamate words, where each amalgamated line ends with
a punctuation mark:

\par 

\texttt{extract -i '**silbe' chant12 | text | context -e '[.,;:?!]' $\backslash$

| rid -GLId}



\par 
The corresponding output is:

\par 

\\
\texttt{A solis ortus cardine ad usque terrae limitem,
\\
Christum canamus principem,
\\
natum Maria Virgine.
\\
Beatus auctor saeculi servile corpus induit:
\\
ut carne carnem liberans,
\\
ne perderet quos condidit.
\\
Castae parentis viscera cae lestis intratgratia:
\\
venter puellae bajulat secreta,
\\
quae non noverat.
\\
Domus pudici pectoris tem plum repente fit Dei:
\\
intacta nesciens virum,
\\
concepit alvo filium.}


\par 
Yet another way of arranging the text output would be to parse
the text according to explicit phrase marks in the
\texttt{**kern}
data.
This will require a little more work, but it's worth going through the steps
since the same process can be applied to any representation.
First, we will need to transfer the end-of-phrase signifier (`\texttt{\}}') from
the \texttt{**kern} spine to the \texttt{**silbe} spine.
This transfer entails four steps.
(1) Extract the monophonic \texttt{**kern} spine and eliminate all data
signifiers except closing curly braces (`\texttt{\}}').
Store the result in a temporary file:

\par 

\texttt{extract -i '**kern' chant12 | humsed 's/[\^{}\}]*//; s/\^{}\$/./' $\backslash$

> temp1}



\par 
Notice that
\textbf{humsed}
has been given two substitution commands.
The first eliminates all data signifiers except the close curly brace.
The second substitution transforms empty output lines to null data records
by adding a single period.

\par 
(2) Extract the \texttt{**silbe} spine, translate it to \texttt{**text}
and store the result in another temporary file:

\par 

\texttt{extract -i '**silbe' chant12 | text > temp2}


\par 
(3) Assemble the two temporary files together and use the
\textbf{cleave}
command to join the end-of-phrase marker to the syllable representation.

\par 

\texttt{assemble temp1 temp2 | cleave -i '**kern,**text' $\backslash$

-o '**text' > temp3}



\par 
With this cleaved data we can now use the
\textbf{context}
command to amalgamate phrase-related text.
Finally,
\textbf{rid}
is used to eliminate everything but non-null data records.

\par 

\texttt{context -o = -e \} temp3 | rid -GLId}


\par 
The result is as follows:

\texttt{A solis ortus cardine \}
\\
ad usque terrae limitem, \}
\\
Christum canamus principem, \}
\\
natum Maria Virgine. \}
\\
Beatus auctor saeculi \}
\\
servile corpus induit: \}
\\
ut carne carnem liberans, \}
\\
ne perderet quos condidit. \}
\\
Castae parentis viscera \}
\\
cae lestis intratgratia: \}
\\
venter puellae bajulat \}
\\
secreta, quae non noverat. \}
\\
Domus pudici pectoris \}
\\
tem plum repente fit Dei: \}
\\
intacta nesciens virum, \}
\\
concepit alvo filium. \}}


\par 
We could clean up the output by using the
\textbf{sed}
command to remove the trailing closed curly brace.
We simple add the following to the pipeline:

\par 

\texttt{ . . . | sed 's/\}//'}


\par 
You might have noticed that each of the above phrases seems to consist
of eight syllables.
We can confirm this by returning to the syllabic rather than
word-oriented output.
For the above command sequence, simply omit the
\textbf{text}
command and replace \texttt{**text} with \texttt{**silbe}.
The revised script becomes:

\par 

\texttt{extract -i '**kern' chant12 | humsed 's/[\^{}\}]*//; s/\^{}\$/./' $\backslash$

> temp1}

\\
\texttt{extract -i '**silbe' chant12 > temp2}
\\
\texttt{assemble temp1 temp2 | cleave -i '**kern,**silbe' $\backslash$

-o '**silbe' > temp3}

\texttt{context -o = -e \} temp3 | rid -GLId | sed 's/\}//'}


\par 
The corresponding output is:

\texttt{A so/- -lis or/- -tus car/- -di- -ne
\\
ad us- -que ter/- -rae li/- -mi- -tem,
\\
Chri/- -stum ca- -na/- -mus prin/- -ci- -pem, 
\\
na/- -tum Ma- -ri/- -a Vir/- -gi- -ne.
\\
Be- -a/- -tus au/- -ctor sae/- -cu- -li
\\
ser- -vi/- -le cor/- -pus in/- -du- -it:
\\
ut car/- -ne car/- -nem li/- -be- -rans, 
\\
ne per/- -de- -ret quos con/- -di- -dit.
\\
Ca/- -stae pa- -ren/- -tis vis/- -ce- -ra
\\
cae/ le/- -stis in/- -trat- -gra/- -ti- -a:
\\
ven/- -ter pu- -el/- -lae ba/- -ju- -lat 
\\
se- -cre/- -ta, quae non no/- -ve- -rat.
\\
Do/- -mus pu- -di- -ci pe/- -cto- -ris
\\
tem/ plum re- -pen/- -te fit De/- -i:
\\
in- -ta/- -cta ne/- -sci- -ens vi/- -rum, 
\\
con- -ce/- -pit al/- -vo fi/- -li- -um.}



\par 
If we are looking for vocal texts that exhibit a recurring rhythm,
we might make a simple addition to the above script.
Instead of outputting the actual syllables in each phrase,
we would output a count of the number of syllables in each phrase.
The standard
\textbf{awk}
utility allows us to write simple in-line programs.
The following
\textbf{awk}
script simply outputs the number of fields (white-space separated text)
in each input line:

\par 

\texttt{awk '\{print NF\}'}


\par 
If we add this to the end of our command sequence, then
the output would simply be a sequence of numbers -$\,$- where each
number indicates the number of syllables in successive phrases.
In the case of
\textit{O Solis Ortus}
our output would consist of a series of 8s indicating that
each phrase contains precisely eighth syllables.

\par 
By way of summary, we can generalize the above process so that
syllable/phrase schemes can be generated for any syllable-related input.
The following script counts the number of
syllables in successive phrases for a single input file.

\par 

\\
\texttt{\# SYLLABLE - count the number of syllables in each phrase
\\
\#
\\
\# Usage: syllable filename [ > outputfile]
\\
\#
\\
extract -i '**kern' \$1 | humsed 's/[\^{}\}]*//; s/\^{}\$/./' > temp1
\\
extract -i '**silbe' \$1 > temp2
\\
assemble temp1 temp2 | cleave -i '**kern,**silbe' -o '**silbe' $\backslash$
\\

\\
| context -o = -e \} | rid -GLId | sed 's/\}//' | awk '\{print NF\}'
\\

\\
rm temp[12]}


\par 
Variations on this theme abound.
For example, if we wish to determine the number
of syllables between successive punctuation marks,
the following pipeline could be used:

\par 

\texttt{extract -i '**silbe' | context -o = -e '[.,;:?!]' $\backslash$

| rid -GLId | awk '\{print NF\}'}




\subsection*{Rhythmic Feet in Text}

\par 
Another question related to rhythm is to identify rhythmic patterns.
Once again, we might look at the chant
\textit{O Solis Ortus.}
Below we have recoded the syllables in each phrase,
where the value \texttt{0} indicates an unstressed syllable and \texttt{1}
indicates a stressed syllable:

\par 

\\
\texttt{0 1 0 1 0 1 0 0
\\
0 0 0 1 0 1 0 0
\\
1 0 0 1 0 1 0 0 
\\
1 0 0 1 0 1 0 0
\\
0 1 0 1 0 1 0 0
\\
0 1 0 1 0 1 0 0
\\
0 1 0 1 0 1 0 0 
\\
0 1 0 0 0 1 0 0
\\
1 0 0 1 0 1 0 0
\\
1 1 0 1 0 1 0 0
\\
1 0 0 1 0 1 0 0 
\\
0 1 0 0 0 1 0 0
\\
1 0 0 0 0 1 0 0
\\
1 0 0 1 0 0 1 0
\\
0 1 0 1 0 0 1 0 
\\
0 1 0 1 0 1 0 0}


\par 
The above output was generated using the
\textbf{humsed}
command.
Any syllable containing a trailing asterisk (\texttt{*}) is re-written
as a `1', otherwise as a `0'.

\par 

\texttt{ . . . | humsed 's/[\^{} ][\^{} ]*$\backslash$*/1/g; s/[\^{}1][\^{}1]*\$/0/g'}


\par 
With the above output, we can generate an inventory of phrase-related
text-rhythms.

\par 

\texttt{ . . . | sort | uniq -c | sort}


\par 
With the following results:

\par 

\\
\texttt{5	0 1 0 1 0 1 0 0
\\
4	1 0 0 1 0 1 0 0
\\
2	0 1 0 0 0 1 0 0
\\
1	0 1 0 1 0 0 1 0
\\
1	1 1 0 1 0 1 0 0
\\
1	1 0 0 0 0 1 0 0
\\
1	1 0 0 1 0 0 1 0
\\
1	0 0 0 1 0 1 0 0}


\par 
We can create a summary rhythmic pattern by adding together the
values in each column -$\,$- that is, counting the number of accented
syllables that occur in each syllable position within the phrase.
We can isolate each column using the UNIX
\textbf{cut}
command;
\textbf{cut}
is analogous to the Humdrum
\textbf{extract}
command.
Fields are delineated by white space (tabs or spaces).
For example,
\textbf{cut -f 1}
will isolate the first column of numbers.
We can then pipe the results to the
\textbf{stats}
utility in order to calculate the numerical total.
For example,

\par 

\texttt{ . . . | cut -f 1 | stats | grep 'total'}
\\
\texttt{ . . . | cut -f 2 | stats | grep 'total'}
\\
\texttt{ . . . | cut -f 3 | stats | grep 'total'}
\\
etc


\par 
For the chant
\textit{O Solis Ortus}
the results are as follows:

\par 

\texttt{7 9 0 13 0 14 2 0}


\par 
This means that there are seven stressed syllables in the first
syllable position of the phrase,
nine stressed syllables in the second syllable position, and so on.
These results suggest the following
rhythmic structure: medium-strong-weak-strong-weak-strong-weak-weak.
By way of conclusion, it appears that this work has a strongly
rhythmic text structure -$\,$- implying that this `chant' might have
been sung rhythmically.


\subsection*{Concordance}

\par 
A traditional text-related reference tool is the
\textit{concordance.}
Concordances allow users to look up a word, to see the word
in the context of several preceding and following words,
and provide detailed information about the location of the
word in some repertory or corpus.


Suppose, for example, that we wanted to create a concordance for
the lyrics in Samuel Barber's songs.
We would like to create a file that has a structure such as shown in
Table 26.3 below.
The first column identifies the filename.
The second column identifies the bar number in which the keyword
occurs.
The third column gives a five-word context where the middle
word (in bold) identifies the keyword.

\textbf{Table 26.3.}
\par 

\texttt{chant294\texttt{ut possim \textbf{cantare}\texttt{, Alleluia: gaudebunt}
\texttt{chant297\texttt{mea, dum \textbf{cantavero}\texttt{ tibi: Alleluia,}
\texttt{chant271\texttt{Cantate Domino \textbf{canticum}\texttt{ novum Alleluia:}
\texttt{chant544\texttt{Cantate Domino \textbf{canticum}\texttt{ novum, quia}
\texttt{chant2410\texttt{Cantate Domino \textbf{canticum}\texttt{ novum: quia}
\texttt{chant4214\texttt{totus non \textbf{capit}\texttt{ orbis, in}
\texttt{chant475\texttt{et exaltavit \textbf{caput}\texttt{ ejus; et}
\texttt{chant121\texttt{solis ortus \textbf{cardine}\texttt{ ad usque}
\texttt{chant144\texttt{arrisit orto \textbf{caritas}\texttt{: Maria, dives}
\texttt{chant127\texttt{induit: ut \textbf{carne}\texttt{ carnem liberans,}
\texttt{chant585\texttt{et in \textbf{carne}\texttt{ mea videbo}
\texttt{chant127\texttt{ut carne \textbf{carnem}\texttt{ liberans, ne}
\texttt{chant146\texttt{sola quae \textbf{casto}\texttt{ potes fovere}
\texttt{chant173\texttt{et discerne \textbf{causam}\texttt{ meam de}
\texttt{chant212\texttt{Dominus a \textbf{cena}\texttt{, misit aquam}
etc.


We would also like to provide a \textbf{grep}-like search tool so
users can search for particular keywords.

\par 
The following script will generate our concordance file.
For each file specified in the input, we extract the \texttt{**silbe}
spine and store it.
We then process this spine no less than three times.
In the first pass, we translate from the \texttt{**silbe}
to the \texttt{**text} representation,
and generate a context of 5 words (\textbf{-n 5}) making sure to omit
barlines (\textbf{-o =}).
We also pad the amalgamated line with three null tokens (\textbf{-p 3})
so the context is centered near the third word in the sequence.
In the second pass, we generate a new spine
(\texttt{**nums})
that contains only bar numbers.
The
\textbf{ditto}
command is used to ensure that every data record contains a bar number.
To ensure that pick-up bars are numbered with the value 0, we've
used
\textbf{humsed}
to replace any leading null-tokens with the number 0.
In the third pass, we replace every data token with the name of the file.
Finally, we assemble all three of these spines, eliminate everything
but data records, and also eliminate lines that don't contain any text.
All of this processing is carried out in a while-loop that cycles
through all of the files provided when the command is invoked.

\par 

\\
\texttt{while [ \$\# -ne 0 ]
\\
do
\\

\\
extract -i '**silbe' \$1 > temp1
\\
text temp1 | context -o = -n 5 -p 3 > temp2
\\
num -n = -a '**nums' temp1 | extract -i '**nums' $\backslash$
\\

\\
| ditto | humsed 's/$\backslash$./0/' > temp3
\\

\\
humsed "s/.*/\$1/" temp1 > temp4
\\
assemble temp4 temp3 temp2 | rid -GLId | sed '/.*	$\backslash$.\$/d'
\\
shift;
\\

\\
done
\\
rm temp[1-4]}


\par 
Having generated our concordance file, we can now create a simple tool
that allows us to search for keywords.
Suppose we kept our concordance information in a file
called \texttt{~/home/concord/master}.
In essence, we'd like to create a command akin to
\textbf{grep}
-$\,$-  but one that searches this file solely according to the third word in the
in the context.
We cannot use
\textbf{grep}
directly since it will find all occurrences of a word no matter where
it occurs in the context.
We need to tell
\textbf{grep}
to ignore all other data.
The filename, bar number, and context fields are separated by tabs.
We can ignore the first two fields by eliminating everything up to
the last tab in the line.
Since words are separated by blank space, the expression \texttt{[\^{} ]+}
will match a word not containing spaces.
In short, the regular expression "\texttt{\^{}.*\textit{tab}[\^{} ]+ [\^{} ]+ }"
will match everything up to the first tab, followed by two additional words.
All we need to do is paste our keyword to the end of this expression.

\par 
Below is a simple one-line script for a command called
\textbf{keyword.}
The user simply types the command
\textbf{keyword}
followed by a regular expression that will allow him/her to search
for a given word in context.
Note that since we've used the extended regular expression character `+' -$\,$-
we must invoke
\textbf{egrep}
rather than
\textbf{grep}
in our script:

\par 

\\
\texttt{\# KEYWORD - A script for searching a master concordance file
\\
\#
\\
\# Usage:  keyword 
\\
\#
\\
egrep "\^{}.*	[\^{} ]+ [\^{} ]+ \$1" ~/home/concord/master}


\par 
Concordances can be used for a number of applications.
One might use a concordance to help identify metaphor or image
related words (such as "light," "darkness," etc.)


\subsection*{Simile}

\par 
One of the most important poetic devices is the
\textit{simile}
-$\,$-  where an analogy or metaphorical link is created between
two things ("My love is like a red red rose.")
In English, similes are often (though not always) signalled
by the presence of the words "like" or "as."

\par 
A simple task involves searching for `like' or `as' in the
lyrics of some input.
For each occurrence of these words, suppose that we would like
to output a line that places the word in
context -$\,$- specifically the preceding and following four words.

\par 
First we transform and isolate the text data using the
\textbf{text}
and
\textbf{extract}
commands:

\par 

\texttt{text inputfile | extract -i '**text' }


\par 
Since the input may contain multiple-stops, we might consider
the precaution of ensuring no more than one word per data record.
For this we can use \textbf{humsed}.
Specifically, we can replace any spaces by a carriage return.
Since the carriage return is interpreted by the shell as
the instruction to begin executing a command,
we need to escape it.
Depending on the shell, the carriage return can be escaped in
various ways.
One way is to precede the carriage return by control-V
(meaning "verbatim").
Another way is to type control-M rather than a carriage return.
In the following command we have used the backslash to escape
a control-M character:

\par 

\texttt{text inputfile | extract -i '**text' | humsed 's/  */$\backslash$\^{}M/g' $\backslash$

| egrep -4 '\^{}|(like)|(as)\$'}



\par 
Having ensured that there is no more than one word per line we
can now search for a line contain
\textit{just}
"like" or "as."
The
\textbf{-4}
option for
\textbf{egrep}
causes any matched lines to be output with four preceding and
four following lines of context.
In addition, an extra line is added consisting of two dashes (\texttt{-$\,$-})
to segregate each pattern output.
That is, for each match, ten lines of output are typically given.
In order to generate our final output, we need to transform
the linear list of words into a horizontal list where each
line represents a single match for "like" or "as."

\par 
The
\textbf{context}
command would enable us to do this.
Unfortunately, however, the output from
\textbf{egrep}
fails to conform to the Humdrum syntax.
In particular, adding \texttt{\^{}$\backslash$*} to the regular expression
will fail to ensure a proper Humdrum output since preceding and
following contextual lines will also be output.

\par 
The
\textbf{hum}
command is a special command that takes non-Humdrum input and
adds sufficient interpretation records so as to make the
input conform to the Humdrum syntax.
Typically, this means simply adding a generic initial exclusive interpretation
(\texttt{**A}) and a spine-path terminator (\texttt{*-}).
If the input contains tabs, then appropriate spines will be added.
If the input contains empty lines, then they will be changed to null data records.

\par 

\texttt{text inputfile | extract -i '**text' | humsed 's/  */$\backslash$\^{}M/g' $\backslash$

| egrep -4 '\^{}|(like)|(as)\$' | hum}



\par 
Now we can make use of the
\textbf{context}
command.
Each context ends with the double-dash delimiters generated by \textbf{egrep}.
The
\textbf{rid}
command can be used to eliminate the interpretations added by \textbf{hum}.

\par 

\texttt{text inputfile | extract -i '**text' | humsed 's/  */$\backslash$\^{}M/g' $\backslash$

| egrep -4 '\^{}|(like)|(as)\$' | hum | context -e '-$\,$-' $\backslash$
\\
| rid -Id}




\subsection*{Word Painting}

\par 
Word painting has a long history in music.
There are innumerable examples where the music has somehow
reflected the meaning of the vocal text.
Suppose we wanted to determine whether words designating height
(e.g., English "high," German "hoch,"
French "haute/haut") tend to coincide with high pitches.

\par 
A simple approach would be to extract those sonorities that
coincide with any of the words high/hoch/haut and determine
the average pitch.
We can then contrast this average pitch with the average pitch
for the repertory as a whole.
Any significant difference might alert us to possible word painting.

\par 
First we translate any pitch data to
\texttt{**semits}
and any
\texttt{**silbe} data to \texttt{**text}.
We will also filter the outputs to ensure that
only \texttt{**semits} and \texttt{**text} are present.

\par 

\texttt{semits * | text | extract -i '**semits,**text'}


\par 
Since a word may be sustained through more than one pitch,
and a pitch may be intoned for more than one word, we should use the
\textbf{ditto}
command to ensure that null tokens are filled-in.

\par 

\texttt{semits * | text | extract -i '**semits,**text' | ditto -s =}


\par 
Next, we can use
\textbf{egrep}
to search for the words of interest:

\par 

\texttt{semits * | text | extract -i '**semits,**text' | ditto -s = $\backslash$

| egrep -i '\^{}$\backslash$*|high|hoch|haut'}



\par 
Notice the addition of the expression \texttt{\^{}$\backslash$*} in the search
pattern.
This expression will match any Humdrum interpretation records and so
ensures that the output conforms to the Humdrum syntax.
We can now isolate the \texttt{**semits} data and pass the output to
\textbf{stats}
in order to determine the average pitch for the words coinciding
with the words high/hoch/haut:

\par 

\texttt{semits * | text | extract -i '**semits,**text' | ditto -s = $\backslash$

| egrep -i '\^{}$\backslash$*|high|hoch|haut' | extract -i '**semits' | stats}



\par 
The average pitch for the entire work can be determined as follows:

\par 

\texttt{semits * | extract -i '**semits' | ditto -s = | rid -GLI $\backslash$

| stats}




\subsection*{Emotionality}

\par 
Musical texts often convey or portray a wide range of emotions.
Some texts celebrate the ecstacy of love or lament the sorrow of loss.
Yet other texts exhibit little emotional content.
Suppose that we wanted to create a tool that would allow us
to estimate the degree of emotional "charge" in the lyrics
of any given vocal work.
A simple approach might be to look for words that are commonly
associated with high emotional content.

\par 
Table 26.4 shows a sample of six words from a study
where 10 people were asked to rate the degree of emotionality
associated with 100 English words.
Participants rated each word on a scale from -10 to +10
where -10 indicates a maximum negative emotional rating
and +10 indicates a maximum positive emotional rating.
The values shown identify the average rating for all 10 participants.
\\\\
\textbf{Table 26.4.  Average Emotionality Ratings for English Words.}

\par 

begin+3.8
river+4.2
friend+5.2
love+8.6
hate-9.7
detest-9.8

Clearly, such a rating system might allow us to create a tool
that would automatically search a large database and identify
those vocal works whose lyrics are most emotionally charged.
One way to generate a crude index of emotionality is to
measure the average ratings for the ten most emotion-laden
words in a given input.

\par 
The
\textbf{humsed}
command provides an appropriate place to start.
In effect, we would take a table (such as Table 26.4) and
use it to create a series of substitutions.
Emotionally-charged words would be replaced by a numerical rating.
Our
\textbf{humsed}
script would have the following form.
Notice that the first substitution is used to eliminate
punctuation marks.
\\\\

\texttt{
s/[.,;:'`"!?]//g
s/begin/+3.8/
s/river/+4.2/
s/friend/+5.2/
s/love/+8.6/
s/hate/-9.7/
s/detest/-9.8/
/[\^{}0-9+-]/s/.*/./}

Also notice that the final command transforms any data records
that contains anything other than a number to a null data token.
In other words, words that are not present in the emotionality list
are not rated.

\par 
In order to process our input,
any syllabic text would first be translated to the \texttt{**text}
representation, and all other spines discarded using \textbf{extract -i}.

\par 

\texttt{text} \textit{inputfile}\texttt{ | extract -i '**text' ...}


\par 
Then we would translate the words using our "emotionality" script,
eliminate everything other than data records, and calculate
the numerical statistics:

\par 

\texttt{text} \textit{inputfile}\texttt{ | extract -i '**text' | humsed -f emotion $\backslash$

| rid -GLId | stats}



\par 
In general, works whose lyrics express predominantly positive emotions
ought to exhibit positive emotionality estimates.
Similarly, works expressing predominantly negative emotions ought
to exhibit negative emotionality estimates.
Of course the process of averaging may be deceptive.
Two sorts of problems may arise.
First, a large number of fairly neutral words will tend to
dilute an otherwise large positive or negative score.
It may be preferable to observe the maximum positive and negative values.
Alternatively, it may be appropriate to limit the average to (say)
the ten most emotionally charged words.
We can do this by sorting the numerical values and using the
\textbf{head}
and
\textbf{tail}
commands to select the highest or lowest values.
In our revised processing, we use
\textbf{sort -n}
to sort the values in numerical order -$\,$- placing the
output in a temporary file.
The UNIX
\textbf{head}
command allows us to access a specified number of lines at the
beginning of a file:
the option
\textbf{-5}
specifies the first five lines.
Similarly, the UNIX
\textbf{tail}
command allows us to access a specified number of lines at the
end of a file.
The ten highest and lowest values are then concatenated together
and piped to the
\textbf{stats}
command:

\par 

\texttt{text} \textit{inputfile}\texttt{ | extract -i '**text' | humsed -f emotion $\backslash$

| rid -GLId | sort -n > temp

head -5 temp > lowest
\\
tail -5 temp > highest
\\
cat highest lowest | stats}


\par 
A second problem with averaging together emotion rating values is
that an emotionally-charged work might include a rough balance of passionate
words expressing both positive and negative emotions.
This might result in an average near zero and be mistaken for
lyrics that exhibit little emotionality.
The
\textbf{stats}
command outputs a variance measure that can be used to gauge
the spread of the data.
However, another way to address this problem is by ignoring the
plus and minus signs in the input.
That is, a rough index of emotionality -$\,$- independent of
whether the emotion is predominantly negative or positive
would simply focus on the most emotionally charged words.

\par 
The plus and minus signs can be eliminated using a simple
\textbf{humsed}
substitution prior to numerical sorting:

\par 

\texttt{humsed 's/[+-]//g'}


\par 
Once again, we could use the
\textbf{head}
command to isolate the 10 or 20 most emotionally charged words.

\par 
Another variant of this approach might be to identify those
words in a text which are most emotionally charged.
Suppose we wanted to determine the location of the most
emotionally charged word.
A combination of
\textbf{sort}
and
\textbf{grep}
can be applied to this task.
First we generate a spine containing the emotional-charge values
taking care to eliminate the signs:

\par 

\texttt{text} \textit{inputfile}\texttt{ | extract -i '**text' | humsed -f emotion $\backslash$

| humsed 's/[+-]//g' > charges}



\par 
Next we assemble this new spine with the original input:

\par 

\texttt{assemble charges} \textit{inputfile}


\par 
We can isolate data records using
\textbf{rid}
and then use
\textbf{sort -n}
to sort according to the numbers present in the first column.
The most emotional charged word will be at the end of the file
(largest number) so we can use
\textbf{tail -1}
to identify the word:

\par 

\texttt{assemble charges} \textit{inputfile}\texttt{ | rid -GLId | sort -n | tail -1}


\par 
Having established what word has been estimated as having the
highest emotional-charge, we can then use
\textbf{grep -n}
to establish the location(s) of this word in the original input file.


\subsection*{Other Types of Language Use}

\par 
Apart from emotionality,
language tends to be used differently in different musical genres.
The contrast between
\textit{aria}
and
\textit{recitative}
provides a classic example.
The aria is intended to be a poetic reflection of a certain emotional
state or reaction whereas the recitative moves the action along by
focusing on concrete circumstances.
Any number of variants on our "emotionality" processing can be
conceived.

For example, we might create another language index related to
the degree of abstraction/concreteness for words.
Words such as Verona, knife and Montague are comparatively concrete,
whereas words such as feud, love and tragedy are more conceptual
or abstract.
We might expect to be able to observe such differences in
recitative versus aria texts.

\par 
Similarly, differences in language use can be found in
folk and popular music.
In the case of the folk \textit{ballad}, a detailed story unfolds.
Differences in language use may be correlated with
the \texttt{!!!AGN} reference record used to identify genres.



\subsection*{Reprise}

\par 
In this chapter we have introduced two text-related representations:
\texttt{**text}
and
\texttt{**silbe}.
We have examined the
\textbf{text}
command (which translates from \texttt{**silbe} to \texttt{**text}).
We have also been exposed to the UNIX
\textbf{fmt}
command (a simple text formatter), the
\textbf{cut}
command (similar to \textbf{extract -f}), and the
\textbf{head}
and
\textbf{tail}
commands.

\par 
In
Chapter 34
we will examine further representations
and processes related to phonetic data.

\\
\begin{itemize}
\item 

\textbf{Next Chapter}
\item 

\textbf{Previous Chapter}
\item 

\textbf{Table of Contents}
\item 

\textbf{Detailed Contents}
\\\\

� Copyright 1999 David Huron
\end{document}
