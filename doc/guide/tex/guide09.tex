% This file was converted from HTML to LaTeX with
% Tomasz Wegrzanowski's <maniek@beer.com> gnuhtml2latex program
% Version : 0.1
\documentclass{article}
\begin{document}



  
  
    
      
      
      
    
  



\\
\\

\section*{Chapter9}


[\textit{Previous Chapter}]
[\textit{Contents}]
[\textit{Next Chapter}]


\section*{Searching with Regular Expressions}



A common task in computing environments is searching through some set of data
for occurrences of a given pattern.
When a pattern is found, various courses of action may be taken.
The pattern may be copied, counted, deleted, replaced, isolated,
modified, or expanded.
A successful pattern match might even be used to initiate further
pattern searches.

\par 
In
Chapter 3
we introduced simple searching using the
\textbf{grep}
command.
We used
\textbf{grep}
to search for strings of characters that match a particular pre-defined
string.
This chapter describes the full power of
\textit{regular expressions}
for defining complex patterns of characters.
Becoming skilled with regular expressions is perhaps
the principal foundation for productive use of Humdrum.
Regular expressions can be used to define patterns in any representation,
and are widely used in many UNIX and Humdrum tools.

\par 
\textit{Regular expression syntax}
provides a standardized method for defining patterns of characters.
Regular expressions are restricted to common text characters
including the upper- and lower-case letters of the alphabet,
the digits zero to nine, and other characters typically found
on computer keyboards.

\par 
Regular expressions will not allow users to define every possible musical
pattern of potential interest.
In particular, regular expressions cannot be used directly
to identify deep-structure patterns from surface-level representations.
However, regular expressions are quite powerful -$\,$- much
more powerful than they appear to the novice user.
Not all users will be equally adept at formulating an appropriate
regular expression to search for a given pattern.
As with the study of a musical instrument,
practise is advised.


\subsection*{Literals}

\par 
The simplest regular expressions are merely literal sequences
of characters forming a character
\textbf{string,}
as in the pattern:

\par 
\texttt{

car

}

\par 
This pattern will match any data string containing the sequence
of letters c-a-r.
The letters must be contiguous, so no character (including spaces)
can be interposed between any of the letters.
The above pattern is present in strings such as "\texttt{carillon}"
and "\texttt{ricercare}" but not in strings such as "\texttt{Caruso}"
or "\texttt{clarinet}".
The above pattern is called a
\textit{literal}
since the matching pattern must be literally identical to the regular
expression (including the correct use of upper- or lower-case).

\par 
When a pattern is found, a starting point and an ending point
are identified in the input string, corresponding to the defined
regular expression.
The specific sequence of characters found in the input string is
referred to as the
\textit{matched string}
or
\textit{matched pattern.}


\subsection*{Wild-Card}

\par 
Regular expressions that are not literal involve so-called
\textit{metacharacters.}
Metacharacters are used to specify various operations,
and so are not interpreted as their literal selves.
The simplest regular expression metacharacter is the period (\texttt{.}).
The period matches
\textit{any single character}
-$\,$-  including spaces, tabs, and other ASCII characters.
For example, the pattern:

\par 
\texttt{

c.u

}

\par 
will match any input string containing three characters, the first of
which is the lower-case `\texttt{c}' and the third of which is the
lower-case `\texttt{u}'.
The pattern is present in strings such as "\texttt{counterpoint}"
and "\texttt{acoustic}" but not in "\texttt{cuivre}"
or "\texttt{Crumhorn}".
Any character can be interposed between the `\texttt{c}' and the `\texttt{u}'
provided there is precisely one such character.


\subsection*{Escape Character}

\par 
A problem with metacharacters such as the period is that sometimes
the user wants to use them as literals.
The special meaning of metacharacters can be "turned-off"
by preceding the metacharacter with
the backslash character ($\backslash$).
The backslash is said to be an
\textit{escape}
character since it is used to release the metacharacter from its
special function.
For example, the regular expression

\par 
\texttt{

$\backslash$.

}

\par 
will match the period character.
The backslash itself may be escaped by preceding it by an additional
backslash (i.e. \texttt{$\backslash$$\backslash$}).


\subsection*{Repetition Operators}

\par 
Another metacharacter is the plus sign (\texttt{+}).
The plus sign means "one or more consecutive instances of the
previous expression."
For example,

\par 
\texttt{

fo+

}

\par 
specifies any character string beginning with a lower-case `\texttt{f}' followed
by one or more consecutive instances of the small letter `\texttt{o}'.
This pattern is present in such strings as "\texttt{food}"
and "\texttt{folly}," but not in "\texttt{front}"
or "\texttt{flood}."
The length of the matched string is variable.
In the case of "\texttt{food}" the matched string consists of
three characters, whereas in "\texttt{folly}" the matched string
consists of just two characters.

\par 
The plus sign in our example modifies only the preceding
letter `\texttt{o}' -$\,$- that is, the single letter `\texttt{o}' is
deemed to be the
\textit{previous expression}
which is affected by the \texttt{+}.
However, the affected expression need not consist of just a single character.
In regular expressions, parentheses (  ) are metacharacters that can be
used to bind several characters into a single unit or sub-expression.
Consider, by way of example, the following regular expression:

\par 
\texttt{

(fo)+

}

\par 
The parentheses now bind the letters `\texttt{f}' and `\texttt{o}' into a single
two-character expression,
and it is this expression that is now modified by the plus sign.
The regular expression may be read as "one or more consecutive
instances of the string `\texttt{fo}'."
This pattern is present in strings like "\texttt{food}" (one instance) and
"\texttt{fofoe}" (two instances).

\par 
Of course we can mix metacharacters together.
The expression:

\par 
\texttt{

(.o)+

}

\par 
will match strings such as "\texttt{polo}" and the first
four letters of "\texttt{tomorrow}."

\par 
Several sub-expressions may occur within a single regular expression.
For example, the following regular expression means "one or
more instances of the letter `\texttt{a}', followed by one or more instances
of the string `\texttt{go}'."

\par 
\texttt{

(a)+(go)+

}

\par 
This would match character strings in inputs such as "\texttt{ago}"
and "\texttt{agogic}," but not in "\texttt{largo}"
(intervening `r') or "\texttt{gogo}" (no leading `\texttt{a}').
Note that the parentheses around the letter `\texttt{a}'
can be omitted without changing the sense of the expression.
The following expression mixes the \texttt{+} repetition operator with the
wild-card (\texttt{.}):

\par 
\texttt{

c+.m+

}

\par 
This pattern is present in strings such as "\texttt{accompany},"
"\texttt{accommodate}," and "\texttt{cymbal}."
This pattern will also match strings such as "\texttt{ccm}"
since the second `\texttt{c}' can be understood to match the period
metacharacter.

\par 
A second repetition operator is the asterisk (\texttt{*}).
The asterisk means "zero or more consecutive instances of the
previous expression."
For example,

\par 
\texttt{

Do*r

}

\par 
specifies any character string beginning with an upper-case `\texttt{D}'
followed by zero or more instances of the letter `\texttt{o}' followed
by the letter `\texttt{r}'.
This pattern is present in such strings as "\texttt{Dorian},"
"\texttt{Doors}" as well as "\texttt{Drum},"
and "\texttt{Drone}."
As in the case of the plus sign, the asterisk modifies only the
preceding expression -$\,$- in this case the letter `\texttt{o}'.
Multi-character expressions may be modified by the asterisk repetition
operator by placing the expression in parentheses.
Thus, the regular expression:

\par 
\texttt{

ba(na)*

}

\par 
will match strings such as "\texttt{ba}," "\texttt{bana},"
"\texttt{banana}," "\texttt{bananana}," etc.

\par 
Incidentally, notice that the asterisk metacharacter can be used to
replace the plus sign (\texttt{+}) metacharacter.
For example, the regular expression \texttt{X+} is the same as \texttt{XX*}.
Similarly, \texttt{(abc)+} is equivalent to \texttt{(abc)(abc)*}.

\par 
A frequent construction used in regular expressions joins the
wild-card (\texttt{.}) with the asterisk repetition character (\texttt{*}).
The regular expression:

\par 
\texttt{

 .*

}

\par 
means "zero or more instances of any characters."
(Notice the plural "characters;"
this means the repetition need not be of one specific character.)
This expression will match \textit{any string}, including nothing at all
(the \textit{null string}).
By itself, this expression is not very useful.
However it proves invaluable in combination with other expressions.
For example, the expression:

\par 
\texttt{

\{.*\}

}

\par 
will match any string beginning with a left curly brace and ending
with a right curly brace.
If we replaced the curly braces by the space character,
then the resulting regular expression would match any string of characters
separated by spaces -$\,$- such as printed words.

\par 
A third repetition operator is the question mark (\texttt{?}), which means
"zero or one instance of the preceding expression."
This metacharacter is frequently useful when you want to specify
the presence or absence of a single expression.
For example, the pattern:

\par 
\texttt{

Ch?o

}

\par 
is present in such strings as "\texttt{Chopin}" and
"\texttt{Corelli}" but not "\texttt{Chinese}" or
"\texttt{cornet}."

\par 
Once again, parentheses can be used to specify more complex expressions.
The pattern:

\par 
\texttt{

Ch?(o)+

}

\par 
is present in such strings as "\texttt{Chorale}," "\texttt{Couperin},"
and "\texttt{Cooper},"
but not in "\texttt{Chloe}" or "\texttt{Chant}."

\par 
In summary, we've identified three metacharacters pertaining to
the number of occurrences of some character or string.
The plus sign means "one or more,"
the asterisk means "zero or more,"
and the question mark means "zero or one."
Collectively, these metacharacters are known as
\textit{repetition operators}
since they indicate the number of times an expression can occur
in order to match.


\subsection*{Context Anchors}

\par 
Often it is helpful to limit the number of occurrences matched
by a given pattern.
You may want to match patterns in a more restricted context.
One way of restricting regular expression pattern-matches is
by using so-called
\textit{anchors.}
There are two regular expression anchors.
The caret (\texttt{\^{}}) anchors the expression to the beginning of the line.
The dollar sign (\texttt{\$}) anchors the expression to the end of the line.
For example,

\par 
\texttt{

\^{}A

}

\par 
matches the upper-case letter `\texttt{A}' only if it occurs at the beginning
of a line.
Conversely,

\par 
\texttt{

A\$

}

\par 
will match the upper-case letter `\texttt{A}' only if it is the last character
in a line.
Both anchors may be used together, hence the following regular expression
matches only those lines containing just the letter `\texttt{A}':

\par 
\texttt{

\^{}A\$

}

\par 
Of course anchors can be used in conjunction with the other
regular expressions we have discussed.
For example, the regular expression:

\par 
\texttt{

\^{}a.*z\$

}

\par 
matches any line that begins with `\texttt{a}' and ends with `\texttt{z}'.


\subsection*{OR Logical Operator}

\par 
One of several possibilities may be matched by making use of the
logical
\textit{OR}
operator, represented by the vertical bar (\texttt{|}).
For example, the following regular expression matches
either the letter `\texttt{x}' or the letter `\texttt{y}' or the
letter `\texttt{z}':

\par 
\texttt{

x|y|z

}

\par 
Expressions may consist of multiple characters, as in the following
expression which matches the string `\texttt{sharp}' or `\texttt{flat}'
or `\texttt{natural}'.

\par 
\texttt{

sharp|flat|natural

}

\par 
More complicated expressions may be created by using parentheses.
For example, the regular expression:

\par 
\texttt{

(simple|compound) (duple|triple|quadruple|irregular) meter

}

\par 
will match eight different strings, including \texttt{simple triple meter}
and \texttt{compound quadruple meter}.


\subsection*{Character Classes}

\par 
In the case of single characters,
a convenient way of identifying or listing a set of possibilities
is to use the
\textit{character class.}
For example, rather than writing the expression:

\par 
\texttt{

a|b|c|d|e|f|g

}

\par 
the expression may be simplified to:

\par 
\texttt{

[abcdefg]

}

\par 
Any character within the square brackets (a "character class") will match.
Spaces, tabs, and other characters can be included within the class.
When metacharacters like the period (\texttt{.}),
the asterisk (\texttt{*}), the plus sign (\texttt{+}), and the dollar
sign (\texttt{\$}) appear within a character class, they lose their special
meaning, and become simple literals.
Thus the regular expression:

\par 
\texttt{

[xyz.+*\$]

}

\par 
matches any one of the characters `\texttt{x},' `\texttt{y},' `\texttt{z},'
the period, plus sign, asterisk, or the dollar sign.

\par 
Some other characters take on special meanings within character classes.
One of these is the dash (\texttt{-}).
The dash acts as a
\textit{range}
operator.
For example,

\par 
\texttt{

[A-Z]

}

\par 
represents the class of all upper-case letters from A to Z.
Similarly,

\par 
\texttt{

[0-9]

}

\par 
represents the class of digits from zero to nine.
The expression given earlier -$\,$- \texttt{[abcdefg]} -$\,$- can be simplified
further to: \texttt{[a-g]}.
Several ranges can be mixed within a single character class:

\par 
\texttt{

[a-gA-G0-9\#]

}

\par 
This regular expression matches any one of the lower- or upper-case
characters from A to G, or any digit, or the octothorpe (\texttt{\#}).
If the dash appears at the beginning or end of the character
class, it loses its special meaning and becomes a literal dash, as in:

\par 
\texttt{

[a-gA-G0-9\#-]

}

\par 
This regular expression adds the dash character to the list of possible
matching characters.

\par 
Another useful metacharacter within character classes is the caret (\texttt{\^{}}).
When the caret appears at the beginning of a character-class list,
it signifies a
\textit{complementary character class.}
That is, only those characters
\textit{not}
in the list are matched.
For example,

\par 
\texttt{

[\^{}0-9]

}

\par 
matches any character other than a digit.
If the caret appears in any position other than at the beginning
of the character class, it loses its special meaning and is treated as
a literal.
Note that if a character-class range is not specified in numerically
ascending order or alphabetic order, the regular expression is considered
ungrammatical and will result in an error.


\subsection*{Examples of Regular Expressions}
\par 
The following table lists some examples of regular expressions and provides
a summary description of the effect of each expression:
\\\\

Amatch letter `A'
\^{}Amatch letter `A' at the beginning of a line
A\$match letter `A' at the end of a line
.match any character (including space or tab)
A+match one or more instances of letter `A'
A?match a single instance of `A' or the null string
A*match one or more instances of `A' or the null string
.*match any string, including the null string
A.*Bmatch any string starting with `A' up to and including `B'
A|Bmatch `A' or `B'
(A)|(B)match `A' or `B'
[AB]match `A' or `B'
[\^{}AB]match any character other than `A' or `B'
ABmatch `A' followed by `B'
AB+match `A' followed by one or more `B's
(AB)+match one or more instances of `AB', e.g. ABAB
(AB)|(BA)match `AB' or `BA'
[\^{}A]AA[\^{}A]match two `A's preceded and followed by characters other than `A's
\^{}[\^{}\^{}]match any character at the beginning of a record except the caret


\textit{Examples of regular expressions.}



\subsection*{Examples of Regular Expressions in Humdrum}
\par 
The following table provides some examples of regular expressions pertinent
to Humdrum-format inputs:
\\\\

\^{}!!match any global comment
\^{}!!.*Beethovenmatch any global comment containing `Beethoven'
\^{}!!.*[Rr]ecapitulationmatch any global comment containing the word
     `Recapitulation' or `recapitulation'
\^{}!(\$|[\^{}!])match only local comments
\^{}$\backslash$*$\backslash$*  match any exclusive interpretation
\^{}$\backslash$*[\^{}*]  match only tandem interpretations
\^{}$\backslash$*[-+vx\^{}]\$  match spine-path indicators
\^{}[\^{}*!]  match only data records
\^{}[\^{}*!].*\$match entire data records
\^{}($\backslash$.<\textit{tab>})*/.\$  match records containing only null tokens (\textit{} means a tab)
\^{}$\backslash$*f\#:  match key interpretation indicating F\# minor


\textit{Regular expressions suitable for all Humdrum inputs.}


\par 
By way of illustration, the next table shows examples of
regular expressions appropriate for processing
\texttt{**kern}
representations.
\\\\

\^{}=match any \texttt{**kern} barline or double barline
\^{}=[\^{}=]match \texttt{**kern} single barlines but not double barlines
\^{}[\^{}=]match any token other than a barline or double barline
;match any \texttt{**kern} note or barline containing a pause
Tmatch any \texttt{**kern} note containing a whole-tone trill
[Tt]match any \texttt{**kern} note containing a whole-tone or
     half-tone trill
-match any \texttt{**kern} note containing at least one flat
[\#]match any \texttt{**kern} note containing a sharp, double-
     sharp, etc.
[\#n-;]match any \texttt{**kern} note containing an accidental,
     including a natural
[A-Ga-g]+match any diatonic pitch letter-name
[0-9]+/.  match \texttt{**kern} dotted durations
[0-9]+/./.[\^{}.]  match only doubly-dotted durations
[Gg]+[\^{}\#-]match any \texttt{**kern} pitch `G' that does not have a sharp
     or flat
(\^{}|[\^{}g])gg(\$|[\^{}g\#-])match only the pitch `gg' (G5)
\{.*r|r.*\{match all phrases that start with a rest
\^{}4[\^{}0-9.]|[\^{}0-9]4([\^{}0-9.]|\$)match \texttt{**kern} quarter durations
\^{}(8|16)[\^{}0-9.]|[\^{}0-9](8|16)[\^{}0-9.]match eighth and sixteenth durations only
(([Ee]+-)|([Gg]+-)|([Bb]+-))(\$|[\^{}-])match any note from E-flat minor chord


\textit{Regular expressions suitable for **kern data records.}

\par 
Note that the above regular expressions assume that comments and
interpretations are not processed in the input.
The processing of just data records can be assured by embedding
each of the regular expressions given above in the expression

\par 
\texttt{

(\^{}[\^{}*!].*\textit{regexp})|(\^{}\textit{regexp})

}

\par 
For example, the following regular expression can be used
to match
\texttt{**kern}
trills without possibly mistaking comments or
interpretations:

\par 
\texttt{

(\^{}[\^{}*!].*[Tt])|(\^{}[Tt])

}

\par 
For Humdrum commands such as
\textbf{humsed},
\textbf{rend},
\textbf{yank},
\textbf{xdelta},
and
\textbf{ydelta},
regular expressions are applied only to data records so there is no
need to use the more complex expressions.
In many circumstances, we will see that it is convenient to use the Humdrum
\textbf{rid}
command to explicitly remove comments and interpretations
prior to processing (see
Chapter 13).


\subsection*{Basic, Extended, and Humdrum-Extended Regular Expressions}

\par 
Over the years, new features have been added to regular expression syntax.
Some of the early software tools that make use of regular expressions
do not support the extended features provided by more recently developed tools.
So-called "basic" regular expressions include the following features:
the single-character wild-card (\texttt{.}), the repetition
operators (\texttt{*}) and (\texttt{?}) but not (\texttt{+}),
the context anchors (\texttt{\^{}}) and (\texttt{\$}), character classes
(\texttt{[...]}), or complementary character classes (\texttt{[\^{}...]}).
Parenthesis grouping is supported in basic regular expressions,
but the parentheses must be used in conjunction with the backslash to
\textit{enable}
this function (i.e.\texttt{ $\backslash$(  $\backslash$)  }).
In
Chapter 3
we introduced the
\textbf{grep}
command;
\textbf{grep}
supports only basic regular expressions.

\par 
"Extended" regular expressions include the following:
the single-character wild-card (\texttt{.}), the repetition
operators (\texttt{*}), (\textbf{?})}
and \texttt{(+)},
the context anchors (\texttt{\^{}}) and (\texttt{\$}),
character classes (\texttt{[...]}), complementary character
classes (\texttt{[\^{}...]}),
the logical OR (\texttt{|}),
and parenthesis grouping.
Extended regular expressions are supported by the
\textbf{egrep}
command;
\textbf{egrep}
operates in the same manner as \textbf{grep},
only the search patterns are interpreted according to
extended regular expression syntax.

\par 
The Humdrum
\textbf{pattern}
command further extends regular expression syntax by providing
multi-record repetition operators that prove very useful
in musical applications.
These Humdrum extensions will be discussed in
Chapter 21.



\subsection*{Reprise}

\par 
Regular expressions provide a powerful method for defining
abstract patterns of alphanumeric characters.
The wild card (.) matches any character.
Repetition operators include "one or more" (+),
"zero or more" (*), and "zero or one" (?).
Context anchors define the beginning of the line (\^{})
or the end of the line (\$).
Character classes ([ ]) specify a choice of possible characters.
Ranges can be specified within character classes ([  -  ])
and complementary classes may be defined ([\^{}   ]).
The logical OR (|) may be used in conjunction with
parentheses to define more complex expressions.

\par 
There are many software tools that make use of regular expressions.
The UNIX
\textbf{grep}
command supports standard or "basic" regular expressions.
The UNIX
\textbf{egrep}
command supports "extended" regular expressions.

\par 
In the next chapter we will explore how regular expressions may be
used in musical applications.

\\
\begin{itemize}
\item 

\textbf{Next Chapter}
\item 

\textbf{Previous Chapter}
\item 

\textbf{Table of Contents}
\item 

\textbf{Detailed Contents}
\\\\

� Copyright 1999 David Huron
\end{document}
