% This file was converted from HTML to LaTeX with
% Tomasz Wegrzanowski's <maniek@beer.com> gnuhtml2latex program
% Version : 0.1
\documentclass{article}
\begin{document}



  
  
    
      
      
      
    
  



\\
\\

\section*{Chapter40}


[\textit{Previous Chapter}]
[\textit{Contents}]


\section*{Conclusion}



In
Chapter 1,
we noted that computers are quite limited in what
they are able to do.
We noted in particular that computers are not good at
\textit{interpreting}
data.
Because so much music scholarship hinges on interpretations,
this would seem to preclude computers from being of much use.
However, as we have seen, there are some things that computers do well,
and when a skilled user intervenes to make a few crucial interpretations,
the resulting computer/human interaction can lead to pretty sophisticated
results.

\par 
The lucid use of Humdrum depends on understanding how each of the
tools works and how it is possible to connect the tools to perform
particular tasks.
The power and creativity of Humdrum truly lies in the hands
of the user.

\par 
As we've seen, there are two parts to Humdrum: the
\textit{Humdrum syntax}
and the
\textit{Humdrum toolkit.}
The Humdrum syntax simply provides a formal framework for
representing any kind of sequential symbolic data.
A number of representation schemes are pre-defined in Humdrum,
but you are free to construct your own representations as
demanded by the tasks.
In several of the tutorial examples in the previous chapters
we saw various
\textit{ad hoc}
representations that were concocted as temporary or
intermediate representations.
You should now feel comfortable with the possibilities of
devising your own representations to assist you in whatever
task you are interested.

\par 
The
\textit{Humdrum Toolkit}
is a set of inter-related software tools.
These tools manipulate text data conforming to the Humdrum syntax.
If the data represents music-related information,
then we can say that the Humdrum tools manipulate music-related information.
Each Humdrum tool carries out a fairly modest process.

\par 
By way of review, we can group the various Humdrum (and UNIX)
tools according to the type of operation.
There are roughly a dozen or so classes of tasks:
\begin{itemize}
\item 
\textbf{\textit{Displaying things.}}
The
\textbf{ms}
command can be used to print or display musical notation.
\item 
\textbf{\textit{Auditing.}}
MIDI sound output can be generated using the
\textbf{midi}
and
\textbf{perform}
commands.
\item 
\textbf{\textit{Searching for things.}}
When searching for things within specified files, appropriate
commands include: \textbf{grep}, \textbf{egrep},
\textbf{yank -m},
\textbf{patt},
\textbf{pattern},
\textbf{correl},
\textbf{simil}
and
\textbf{humdrum -v}.
When searching for files that meet certain conditions,
appropriate commands include
\textbf{grep -l}, \textbf{egrep -l} and \textbf{find}.
Most of these search tools rely extensively on regular
expressions to define patterns of characters.
\item 
\textbf{\textit{Counting things.}}
Appropriate commands include: \textbf{wc}, \textbf{wc -l}, \textbf{grep -c},
\textbf{egrep -c},
\textbf{census}
and \textbf{census -k}.
Eliminating unnecessary or confounding information can be achieved
using
\textbf{rid},
\textbf{extract},
\textbf{grep -v}, \textbf{sed} and
\textbf{humsed}.
\item 
\textbf{\textit{Editing things.}}
Manual editing may be done using any text editor, such as \textbf{emacs}
or \textbf{vi}.
Automated (or "stream") editing may be done
using \textbf{sed} and
\textbf{humsed}.
\item 
\textbf{\textit{Editorializing.}}
E.g., add an editorial footnote to a specified note or passage;
indicate that a passage differs from the composer's autograph.
\item 
\textbf{\textit{Transforming or translating between representations.}}
Appropriate commands include:
\textbf{cents},
\textbf{deg},
\textbf{degree},
\textbf{dur},
\textbf{freq},
\textbf{hint},
\textbf{humsed},
\textbf{iv},
\textbf{kern},
\textbf{mint},
\textbf{pc},
\textbf{pf},
\textbf{pitch},
\textbf{reihe},
\textbf{semits},
\textbf{solfa},
\textbf{solfg},
\textbf{text},
\textbf{tonh},
\textbf{trans}
and
\textbf{vox}.
\item 
\textbf{\textit{Arithmetic transformations of representations.}}
Manipulating numerical values can be done using
\textbf{xdelta},
\textbf{ydelta},
\textbf{recode}
and \textbf{awk}.
\item 
\textbf{\textit{Extracting or selecting information.}}
Appropriate commands include:
\textbf{extract},
\textbf{yank},
\textbf{grep}, \textbf{egrep},
\textbf{yank -m},
\textbf{thru},
\textbf{strophe},
\textbf{rend}
and \textbf{csplit}.
\item 
\textbf{\textit{Linking or joining information.}}
Appropriate commands include:
\textbf{assemble},
\textbf{cat},
\textbf{cleave},
\textbf{timebase},
\textbf{context},
\textbf{ditto}
and
\textbf{join}.
\item 
\textbf{\textit{Generating inventories of things.}}
Appropriate commands include:
\textbf{sort} and \textbf{uniq}.
Once again, unnecessary or confounding information can be eliminated
using
\textbf{rid},
\textbf{extract},
\textbf{grep -v},
\textbf{yank -m},
\textbf{sed} or
\textbf{humsed}.
\item 
\textbf{\textit{Classifying things.}}
Numerical values can be classified using
\textbf{recode};
non-numerical data can be classified using
\textbf{humsed}.
\item 
\textbf{\textit{Labelling things.}}
Appropriate commands include:
\textbf{patt -t},
\textbf{recode},
\textbf{humsed}
and
\textbf{timebase}.
\item 
\textbf{\textit{Comparing whether things are the same or similar.}}
Appropriate commands include:
\textbf{diff}, \textbf{diff3}, \textbf{cmp},
\textbf{correl}
and
\textbf{simil}.
\item 
\textbf{\textit{Capturing data.}}
MIDI data can be input via
\textbf{encode}
and
\textbf{record.}
\item 
\textbf{\textit{Trouble-shooting.}}
Appropriate commands include:
\textbf{humdrum},
\textbf{humdrum -v},
\textbf{proof},
\textbf{proof -k},
\textbf{veritas},
\textbf{midi}
and
\textbf{perform}.
\end{itemize}

\par 
Not all of the existing Humdrum Tools were covered in this book.
Nor were all of the available options described for all of the tools
discussed.
Exploring the
\textit{Humdrum Reference Manual}
is recommended for readers interested in continuing to develop
facility with Humdrum.


\subsection*{Pursuing a Project with Humdrum}

\par 
If you have made it this far in the book, you will now have a fairly
sophisticated knowledge of Humdrum.
With this background, we might review some general principles
and specific tips for making effective use of Humdrum
when pursuing some musicological project.
The following seven questions provide useful guidelines:
\begin{itemize}
\item 
\textit{What do I want my final output to look like?}
\\
Do I want a count or inventory?

\par 

\texttt{115 instances of ...
\\
28 instances of ...}


\par 
Do I want to output a found pattern?

\par 

\texttt{pattern found in line ...
\\
pattern not found ...}


\par 
Do I want a comparison?

\par 

\texttt{file X is the same as file Y
\\
X is similar to Y
\\
X and Y are different}


\par 
\item 
\textit{What materials are available for processing?}

\par 

Use
\textbf{find}
and
\textbf{grep}
to locate useful materials.


\par 
\item 
\textit{What materials do I need to create?}

\par 

Use
\textbf{encode}
to create new data.
Use
\textbf{humdrum}
and
\textbf{proof}
to check the data.
If necessary, define your own Humdrum representation for a given purpose.


\par 
\item 
\textit{How do I transform my data so it is easier to process?}

\par 

Use
\textbf{recode}
and
\textbf{humsed}
to classify data into various classes -$\,$- such as
\textit{up}, \textit{down}, \textit{leap}, \textit{long}, \textit{short}, \textit{difficult},
\textit{easy}, \textit{clarion register}, \textit{dominant},
etc.

\par 
Use translating/transforming commands such as
\textbf{mint},
\textbf{ydelta},
\textbf{pcset},
etc to translate
the data to a different representation.


\par 
\item 
\textit{What data should I eliminate?}

\par 

Use
\textbf{rid},
\textbf{extract},
\textbf{yank},
\textbf{sed},
\textbf{humsed},
\textbf{uniq},
\textbf{uniq -d}
and \textbf{grep -v}
to eliminate selective materials.


\par 
\item 
\textit{What data do I need to coordinate?}

\par 

Use
\textbf{context}
to generate contextual information.
Use
\textbf{assemble},
\textbf{rend}
and
\textbf{cleave}
to link information together.


\par 
\item 
\textit{How do I know my results are worthwhile?}

\par 

Use comparative tests whenever you can.
Use
\textbf{scramble} -r,
\textbf{scramble -t},
\textbf{tac} and
\textbf{reihe} -s
to generate control groups.

\end{itemize}

\\
\begin{itemize}
\item 

\textbf{Previous Chapter}
\item 

\textbf{Table of Contents}
\item 

\textbf{Detailed Contents}
\\\\

� Copyright 1999 David Huron
\end{document}
