% This file was converted from HTML to LaTeX with
% Tomasz Wegrzanowski's <maniek@beer.com> gnuhtml2latex program
% Version : 0.1
\documentclass{article}
\begin{document}



  
  
    
      
      
      
    
  



\\
\\

\section*{Chapter23}


[\textit{Previous Chapter}]
[\textit{Contents}]
[\textit{Next Chapter}]


\section*{Rhythm}



The subject of rhythm touches on nearly every aspect of music.
Musical elements such as pitch, harmony, and dynamics
can all be regarded from the point-of-view of temporal
patterns of events.
A number of complex tasks arise from rhythm-related
In this chapter, two rhythm-related tools are introduced:
\textbf{dur}
and
\textbf{metpos}.


\subsection*{The \textit{**recip} Representation}

\par 
For many types of processing tasks it is helpful to have a
representation that encodes rhythmic information only.
The
\texttt{**recip}
representation is simply a subset of
\texttt{**kern}
that excludes all information apart from the nominal note durations
and common system barlines.
In addition, \texttt{**recip} distinguishes rests from notes by
including the `\texttt{r}' signifier.
Without an accompanying `\texttt{r}' a duration is assumed to pertain
to a note.

\par 
Generating \texttt{**recip} data from \texttt{**kern} is straightforward
using
\textbf{humsed}.
For a single-spine input, the following command will make the
translation:

\par 

\texttt{humsed '/\^{}[\^{}=]/s/[\^{}0-9.r ]//g; s/\^{}\$/./' input.krn $\backslash$

| sed 's/$\backslash$*$\backslash$*kern/**recip/'}



\par 
The first
\textbf{humsed}
substitution eliminates all data other than the numbers 0 to 9,
the period, the lower-case \texttt{r}, and the space (for multiple-stops).
Barlines remain untouched in the output.
The second
\textbf{humsed}
substitution changes any empty lines to null data tokens;
this might be necessary in the case of grace notes.
The ensuing
\textbf{sed}
command is used simply to change the exclusive interpretation
from \texttt{**kern} to \texttt{**recip}.

\par 
A simple type of processing might entail creating an inventory
of rhythmic patterns.
Suppose we wanted to determine the most common rhythmic pattern
spanning a measure.
Using a monophonic \texttt{**recip} input, we could use
\textbf{context}
to amalgamate the appropriate data tokens:

\par 

\texttt{context -b \^{}= -o \^{}= input.recip | rid -GLId | sort $\backslash$

| uniq -c | sort -nr}



\par 
The output for the combined voices of Bach's two-part
Invention No. 5 shows just seven patterns.
The most characteristic patterns are the second one:
\texttt{8r 16 16 8 8 4 4} and the
fourth one:
\\
\texttt{8 16 16 8 8 4 4}.

\texttt{3016 16 16 16 16 16 16 16 16 16 16 16 16 16 16 16
128r 16 16 8 8 4 4
118 8 8 8 16 16 16 16 8 8
88 16 16 8 8 4 4
18 8 8 8 2
18 32 32 32 32 4 2
14 4 16r 16 16 16 16 16 16 16}



\subsection*{The \textit{dur} Command}

\par 
The
\textbf{dur}
command produces
\texttt{**dur}
output from either a \texttt{**kern}
or \texttt{**recip} input.
The \texttt{**dur} representation scheme consists simply of
the elapsed duration of notes and rests, expressed in seconds.
The following example shows a simple \texttt{**dur} representation
(right spine) with a corresponding \texttt{**kern} input:
\\\\

\texttt{**kern**dur
**MM60
=1=1
12g0.3333
12g0.3333
12g0.3333
4g1.0000
4r1.0000r
8g0.5000
8g0.5000
4g1.0000
=2=2
*-*-}

As in the case of \texttt{**recip},
the \texttt{**dur} representation designates rests via
the lower-case \texttt{r} and uses the common system for barlines.
Notice that \texttt{**dur} assumes a metronome indication of
quarter-note equals 60 beats per minute if no other metronome
marking is given.

\par 
Suppose that we wanted to estimate the total duration of some monophonic
passage (ignoring rubato).
We can do this by translating the score to \texttt{**dur},
eliminating everything but notes and rests,
and sending the output to the
\textbf{stats}
command:

\par 

\texttt{dur -d inputfile.krn | rid -GLId | grep -v '\^{}=' | stats}


\par 
The
\textbf{-d}
option for
\textbf{dur}
suppresses the outputting of duplicate durations arising
from multiple-stops.
Note that outputs from
\textbf{dur}
will adapt to any changes of metronome marking found in the input,
so if the work accelerates the durations will be reduced
proportionally.


\par 
The
\textbf{-M}
option will over-ride any metronome markings found in the input
stream.
For example, if we wanted to estimate the duration of a monophonic
passage for a metronome marking of 72 quarter-notes per minute
we could use the command:

\par 

\texttt{dur -M 72 -d input.krn | rid -GLId | grep -v '\^{}=' | stats}



\par 
Of course, the duration of a passage is not the same as
the length of time a given instrument sounds.
Suppose, for example, that we wanted to compare the duration
of trumpet activity in the final movements of Beethoven's symphonies.
We need to make a distinction between the duration of notes
and the duration of rests.
Since the duration values for rests are distinguished by the
trailing letter `r', we can use
\textbf{grep -v}
to eliminate all rest tokens.

\par 

\texttt{extract -i '*Itromp' inputfile.krn | dur -d | rid -GLId $\backslash$

| grep -v '\^{}=' | grep -v r | stats}



\par 
The
\textbf{dur}
command provides a
\textbf{-e}
option that allows the user to echo specified signifiers in the output.
The
\textbf{-e}
option is followed by a regular expression indicating what patterns
are to be passed to the output.
This option allows us to "mark" notes of special interest.
For example, suppose we wanted to determine the longest duration
note for which Mozart had marked a staccato.

\par 

\texttt{dur -e $\backslash$' inputfile | rid -GLId | grep $\backslash$' | sed 's/$\backslash$'//' $\backslash$

| stats}




\par 
The
\textbf{-e}
option ensures that \texttt{**kern} staccato marks (') are passed along
to the output.
The
\textbf{rid}
command eliminates everything but Humdrum data records.
Then
\textbf{grep}
is used to isolate only those notes containing a staccato mark.
The
\textbf{sed}
script is used to eliminate the apostrophe, and finally the numbers
are passed to the
\textbf{stats}
command.
The \texttt{max} value from
\textbf{stats}
will identify the duration (in seconds) of the longest note
marked staccato.


\par 
This same basic pipeline can be used for a variety of similar problems.
Suppose, for example, that we want to determine whether
notes at the ends of phrases tend to be longer than notes
at the beginnings of phrases -$\,$- and if so, how much longer?
In this case, we want to have
\textbf{dur}
echo phrase-related signifiers:

\par 

\texttt{dur -e '\{' inputfile | rid -GLId | grep '\{' | sed 's/\{//' $\backslash$

| stats}

\\
\texttt{dur -e '\}' inputfile | rid -GLId | grep '\{' | sed 's/\{//' $\backslash$

| stats}




\par 
Similarly, do semitone trills tend to be shorter than whole-tone trills?

\par 

\texttt{dur -e 't' inputfile | rid -GLId | grep 't' | sed 's/\{//' $\backslash$

| stats}

\\
\texttt{dur -e 'T' inputfile | rid -GLId | grep 'T' | sed 's/\{//' $\backslash$

| stats}




\par 
Of course, we can also use
\textbf{dur}
in conjunction with
\textbf{yank}
in order to investigate particular musical segments or passages.
How much shorter is the recapitulation compared with the
original exposition?

\par 

\texttt{yank -s 'Exposition' -r 1 inputfile | dur | rid -GLId $\backslash$

| grep -v '=' | stats}

\\
\texttt{yank -s 'Recapituation' -r 1 inputfile | dur | rid -GLId $\backslash$

| grep -v '=' | stats}



\par 
Do initial phrases in Schubert's vocal works tend to be shorter than final
phrases?

\par 

\texttt{yank -m \{ -r 1 lied | dur | rid -GLId | grep -v \^{}= | stats}
\\
\texttt{yank -m \{ -r \$ lied | dur | rid -GLId | grep -v \^{}= | stats}


\par 
How much longer is a passage if all the repeats are played?

\par 

\texttt{thru inputfile | dur | rid -GLID | stats -o \^{}=}


\par 
Recall that the
\textbf{xdelta}
command can be used to calculate numerical differences between
successive values.
If the input to
\textbf{xdelta}
is \texttt{**dur} duration information, then we can determine
rates of change of duration.
Most music exhibits lengthy passages of similar duration notes -$\,$-
as in a sequence of sixteenth notes.
In French overtures, successive notes are often of highly contrasting
durations (longer, very-short, long, etc.).
Using
\textbf{xdelta}
we can identify such large changes of duration.
For example, the following pipeline can be used to determine
the magnitude of the
\textit{differences}
between successive notes.

\par 

\texttt{dur inputfile | xdelta -s \^{}= | rid -GLId | stats -o \^{}=}


\par 
A small \texttt{mean} from
\textbf{stats}
will be indicative of works that tend to have smoother or less
angular note-to-note rhythms.


\subsection*{Classifying Durations}

\par 
We can use the
\textbf{recode}
command to classify durations into a finite set of categories.
Suppose, for example, we wish to create a inventory of
long/short rhythmic patterns.
We might use
\textbf{recode}
with reassignments such as the following:

\texttt{>=0.4long
elseshort}

For a monophonic input,
we can create an inventory of (say) 3-note long/short
rhythmic patterns as follows:

\par 

\texttt{dur inputfile | recode -f reassign -i '**dur' -s \^{}= | $\backslash$

context -n 3 -o = | rid -GLId | sort | uniq -c | sort -n}



\par 
A typical output might appears as follows:

\texttt{230long long long
3422short short short
114long long short
202short short long
38long short long
117short long long
194long short short
114short long short}


\par 
Notice that we might do a similar inventory based on durational
\textit{differences}
rather than on durations.
For example, the
\textbf{xdelta}
command will allow us to distinguish short\textit{er} note relationships
from long\textit{er} relationships.
Our reassignment file would be as follows:

\texttt{==0equal
>0shorter
<0longer}


\par 
And our processing would be:

\par 

\texttt{dur inputfile | xdelta -s \^{}= | recode -f reassign $\backslash$

-i '**Xdur' -s \^{}= | context -n 2 -o = $\backslash$
\\
| rid -GLId | sort | uniq -c | sort -n}




\subsection*{Using \textit{yank} with the \textit{timebase} Command}

\par 
Recall that the
\textbf{timebase}
command can be used to reformat an input so that each data
record represents an equivalent elapsed duration.
For example, in a 4/4 meter, the following command will
format the output so that each full measure consists of precisely
16 data records (not including the barline itself):

\par 

\texttt{timebase -t 16 input.krn}


\par 
Suppose we wanted to isolate all sonorities in a 4/4 work that occur
only on the fourth beat of a measure.
If we use \textbf{timebase}, we can ensure that the fourth beat
always occurs a certain number of data records following the barline.
For example, with the following command, the onset of the fourth beat will
always occur 4 records follow the barline:

\par 

\texttt{timebase -t 4 input.krn}


\par 
We can now use
\textbf{yank -m}
to extract all appropriate sonorities.
The "marker" is the barline and the "range" is 4 records
following the marker, hence:

\par 

\texttt{timebase -t 4 input.krn | yank -m \^{}= -r 4}


\par 
Note that this process will extract only those notes that
begin sounding with the onset of the fourth beat.
Some notes may have begun prior to the fourth beat and yet are
sustained into the beat.
If we want to extract the
\textit{sounded}
sonority, we can use the
\textbf{ditto}
command.
Begin by expanding the work with a timebase that ensures all
notes are present.
For a work whose shortest note is a 32nd note, we can use
an appropriately small timebase value.
Then use the
\textbf{ditto}
command to propagate all sustained notes forward through the successive
sonorities:

\par 

\texttt{timebase -t 32 input.krn | ditto -s \^{}=}


\par 
Now we can yank the data records that are of interest.
Notice that the
\textbf{-r}
(range) option for
\textbf{yank -m}
allows us to select more than one record.
This might allow us, say, to extract only those sonorities that
occur on off-beats.
For example, the following command extracts all notes played
by the horns during beats 2 and 4 in a 4/4 meter work:

\par 

\texttt{extract -i '*Icor' input.krn | timebase -t 16 $\backslash$

| yank -m \^{}= -r 5-8,13-16}




\par 
In some cases, we would like to yank materials that do not
themselves contain explicit durational information.
Suppose, for example, that for a waltz repertory,
we want to contrast those chord functions that tend to occur on the
first beat with those that happen on the third beat.
We will need to have an input that includes both a
\texttt{**harm}
spine encoding the Roman numeral harmonic analysis,
as well as one or more \texttt{**kern} or \texttt{**recip} spines
that include the durational information.
We can use the
\textbf{timebase}
command to expand the output accordingly -$\,$- cuing on the duration
information provided by \texttt{**kern} or \texttt{**recip}.
Having suitable expanded the input, we can dispense with
everything but the \texttt{**harm} spine.
For works in 3/4 meter, the following pipeline would provide
an inventory of chords occurring on the first beat of each bar:

\par 

\texttt{timebase -t 8 input | extract -i '**harm' $\backslash$

| yank -m \^{}= -r 1 | rid -GLId | sort | uniq -c | sort -n}



\par 
And the following variation would provide an inventory of chords
occurring on the third beat of each bar.
(There are 6 eighth durations in a bar of 3/4, therefore
the beginning of the third beat will coincide with the 5th eighth -$\,$-
hence the range \texttt{-r 5}:

\par 

\texttt{timebase -t 8 input | extract -i '**harm' $\backslash$

| yank -m \^{}= -r 5 | rid -GLId | sort | uniq -c | sort -n}




\subsection*{The \textit{metpos} Command}

\par 
The
\textbf{metpos}
command generates a
\texttt{**metpos}
output spine containing
numbers that indicate the metric strength of each sonority.
By "metric position" we mean the position of importance
in the metric hierarchy for a measure.

\par 
The highest position in any given metric hierarchy is given by the
value `1'.
This value is assigned to the first event at the beginning of each
measure.
In duple and quadruple meters, the second level in the metric hierarchy
occurs in the middle of the measure and is assigned the output value `2'.
(In triple meters,
\textbf{metpos}
assumes that the second and third beats in the measure are both assigned
to the second level in the metric hierarchy.)
All other metric positions in the measure (beats, sub-beats,
sub-sub-beats, etc.) are assigned successively increasing numerical
values according to their placement in the metric hierarchy.
In summary, larger \texttt{**metpos} values signify sonorities of
\textit{lesser}
metric significance.

\par 
By way of illustration, consider the case of successive eighth notes
in a 2/4 meter.
The metric hierarchy values for successive eighths are: 1, 3, 2, 3.
In the case of successive sixteenth notes in 2/4, the
metric hierarchy values are: 1,4,3,4,2,4,3,4.
In the case of 6/8 meter, successive sixteenth durations exhibit
a metric hierarchy of: 1,4,3,4,3,4,2,4,3,4,3,4.

\par 
For correct operation, the
\textbf{metpos}
command must be supplied with an input that has been formatted using the
\textbf{timebase}
command.
That is, each data record (ignoring barlines) must represent an
equivalent duration of time.
In addition,
\textbf{metpos}
must be informed of both the
\textit{meter signature}
and the
\textit{timebase}
for the given input passage.
This information can be specified via the command line,
however it is usually available in the input stream via appropriate
tandem interpretations.

\par 
The following extract from Bart�k's "Two-Part Study" No. 121 from
\textit{Mikrokosmos}
demonstrates the effect of the
\textbf{metpos}
command.
The two left-most columns show the original input;
all three columns show the corresponding output from \textbf{metpos}:

\texttt{**kern**kern**metpos
*tb8*tb8*tb8
=16=16=16
*M6/4*M6/4*M6/4
8Gn8b-1
8A8ccn4
8B-8cc\#\}3
8cn\{8f\#4
8c\#\}8gn3
\{8F\#8a4
8G8b-2
8A8ccn4
8B-4b-3
8cn.4
8c\#\}8fn\}3
8r8r4
=17=17=17
*M4/4*M4/4*M4/4
8d2r1
4.d.4
..3
..4
\{2d\_8dd2
.4.dd4
..3
..4
=18=18=18
8d\{1dd\_1
8A.4
8F\#.3
8E.4
8D.2
8BB.4
8D.3
8E\}.4
=19=19=19
*M3/2*M3/2*M3/2
\{8F\#8dd1
8A8ffn4
8c\#8aa3
8A8ff4
8F\#8dd2
8A8ff4
8F\#8dd3
8E8ccn4
8D8b-2
8BBn8gn4
8D8b-3
8E\}8cc4
=20=20=20
*-*-*-}

Notice that
\textbf{metpos}
adapts to changing meter signatures, and
correctly distinguishes between metric accent patterns such as 6/4 (measure 16)
and 3/2 (measure 19).

\par 
The
\texttt{**metpos}
values provide additional ways of
addressing various rhythmic questions.
We might use
\textbf{recode}
for example, to recode the numerical outputs from
\textbf{metpos}
into a smaller set of discrete categories.
For example, we might classify metric positions using the
following reassignment file:

\texttt{==1strong
>=3secondary
elseweak}


\par 
The words `strong', `secondary', and `weak' can then be sought by
\textbf{grep}
or
\textbf{yank -m},
allowing us to isolate points of particular metric stress.
Since
\textbf{metpos}
adapts to changing meters, we can confidently process inputs
that may contain mixtures of meters.


\subsection*{Changes of Stress}


\par 
Once again we can make use of
\textbf{xdelta}
to identify relationships between successive metric position values.
Suppose we had a collection of Hungarian melodies and we wanted to
determine how each degree is approached in terms of metric strength.
That is, we would like to count the number of tonic pitches that
are approached by a weak-to-strong context versus the number of
tonic pitches approached by a strong-to-weak context.
We also want similar measures for supertonic, mediant,
subdominant, etc. scale degrees.

\par 
This task involves creating an inventory where fourteen
different items are possible:
(1) tonic strong-to-weak, (2) tonic weak-to-strong,
(3) supertonic, strong-to-weak, etc.
A suitable inventory will involve creating two spines of
information -$\,$- scale-degree and relative metric strength.

\par 
Assuming that our Hungarian melodies encode key information,
creating a
\texttt{**deg}
spine is straightforward.
Recall that the
\textbf{-a}
option for
\textbf{deg}
avoids distinguishing the direction of approach (from above or below):

\par 

\texttt{deg -a magyar*.krn > magyar.deg}


\par 
Creating a spine encoding relative metric strength will be more
involved.
First we need to expand our input according to the shortest note.
We use
\textbf{census -k}
to determine the shortest duration, and then expand our input
using \textbf{timebase}.

\par 

\texttt{census -k magyar*.krn}
\\
\texttt{timebase -t 16 magyar*.krn > magyar.tb}


\par 
Using
\textbf{metpos}
will allow us to create a spine with the metric position data.

\par 

\texttt{metpos magyar.tb > magyar.mp}


\par 
Note that
\textbf{metpos}
automatically echoes the input along with the new \texttt{**metpos} spine.
At this point, the result might look as follows:

\texttt{!!!OTL: Graf Friedrich In Oesterraaich sin di Gassen sou enge


\texttt{**kern\texttt{**metpos
\texttt{*ICvox\texttt{*
\texttt{*Ivox\texttt{*
\texttt{*M3/4\texttt{*M3/4
\texttt{*k[f\#]\texttt{*
\texttt{*G:\texttt{*
\texttt{*tb16\texttt{*tb16
\texttt{\{8g\texttt{2
\texttt{.\texttt{4
\texttt{8b\texttt{3
\texttt{.\texttt{4
\texttt{=1\texttt{=1
\texttt{8dd\texttt{1
\texttt{.\texttt{4}
etc.


\par 
We want to be able to say that the relationship between the
first eighth-note G and the eighth-note B is "strong-to-weak"
and that the relationship between the eighth-note B and the eighth-note D
is "weak-to-strong."
In order to procede we need to eliminate all of the data records that
contain only a metpos value -$\,$- that is, there is no pitch present in the
\texttt{**kern}
spine.
We can do this using \textbf{humsed};
we simply delete all lines that begin with a period character:

\par 

\texttt{humsed '/\^{}$\backslash$./d' magyar.mp}


\par 
The result is as follows:

\texttt{!!!OTL: Graf Friedrich In Oesterraaich sin di Gassen sou enge


\texttt{**kern\texttt{**metpos
\texttt{*ICvox\texttt{*
\texttt{*Ivox\texttt{*
\texttt{*M3/4\texttt{*M3/4
\texttt{*k[f\#]\texttt{*
\texttt{*G:\texttt{*
\texttt{*tb16\texttt{*tb16
\texttt{\{8g\texttt{2
\texttt{8b\texttt{3
\texttt{=1\texttt{=1
\texttt{8dd\texttt{1}
etc.


\par 
Notice that the successive \texttt{**metpos} values will now
allow us to characterize the changes in stress between successive notes:
2 followed by 3 indicates a strong-to-weak change of metric position,
3 followed by 1 indicates a weak-to-strong change of metric position.
We can use
\textbf{xdelta}
to calculate the differences in metric position values:
positive differences will indicate weak-to-strong changes
and negative differences will indicate strong-to-weak changes.
If both values have the same metric position value, then the
successive notes hold equal positions in the metric hierarchy.
Before using
\textbf{xdelta}
we need to isolate the \texttt{**metpos} spine using \textbf{extract}:

\par 

\texttt{humsed '/\^{}$\backslash$./d' magyar.mp | extract -i '**metpos' $\backslash$

| xdelta -s \^{}=}



\par 
The result is:

\texttt{!!!OTL: Graf Friedrich In Oesterraaich sin di Gassen sou enge
\texttt{**Xmetpos
\texttt{*
\texttt{*
\texttt{*M3/4
\texttt{*
\texttt{*
\texttt{*tb16
\texttt{.
\texttt{1
\texttt{=1
\texttt{-2}
etc.


\par 
Now we can use
\textbf{recode}
to classify the changes of metric position according.
Our reassignment file (named \texttt{reassign}):

\texttt{>0strong-to-weak
<0weak-to-strong
==0equal}


\par 
Appending the appropriate command:

\par 

\texttt{humsed '/\^{}$\backslash$./d' magyar.mp | extract -i '**metpos' $\backslash$

| xdelta -s \^{}= | recode -f reassign -i '**Xmetpos'  -s \^{}= > magyar.xmp}



\par 
Now we can assemble the resulting metric change spine with our original
\texttt{**deg}
spine.
Each data record will contain the scale degree in the first spine and
the change of metric position data in the second spine.
The final task is to create an inventory using
\textbf{rid},
\textbf{sort} and \textbf{uniq}:

\par 

\texttt{assemble magyar.deg magyar.xmp | rid -GLId | grep -v \^{}= $\backslash$

| sort | uniq -c}



\par 
The final result will appear as below.
The first output line indicates that there were three instances
of a tonic pitch approached by a note of equivalent position in
the metric hierarchy.
The second line indicates that there were twenty-five instances
of a tonic pitch approached by a note having a stronger metric position:

\texttt{31equal
251strong-to-weak
301weak-to-strong
32equal
142strong-to-weak
132weak-to-strong
13equal
393strong-to-weak
343weak-to-strong
34equal
264strong-to-weak
174weak-to-strong
135equal
495strong-to-weak
425weak-to-strong
16equal
136strong-to-weak
146weak-to-strong
37strong-to-weak
67weak-to-strong
17-weak-to-strong
3requal
10rstrong-to-weak}


\par 
Instead of scale degree, any other Humdrum spine might be used.
For example, if the input contained functional harmony data (**harm) then
the output inventory would identify how particular chord functions tend
to be approached.
For example, we could establish whether the submediant chord is more
likely to be approached in a strong-to-weak or weak-to-strong rhythmic context.
Similarly, this same technique can be used to determine whether particular
melodic or harmonic intervals tend to be approached using particular stress relationships.

\par 
In addition, our input spine might also be transformed via the
\textbf{context}
command.
Given a **harm spine, for example,
\textbf{context}
could be used to generate two-chord harmonic progressions.
This would permit us to determine, for example, whether a specific progression such
as \textit{ii-V} tends to fall in strong-to-weak or weak-to-strong contexts.




\subsection*{Reprise}

\par 
There are a vast number of issues raised in rhythm-related processing.
In this chapter we have touched on a few of the more basic tasks.
These include identifying the durations of various passages using \textbf{dur};
classifying and contextualizing durations using
\textbf{recode}
and
\textbf{context};
isolating particular rhythmic moments using
\textbf{timebase}
and
\textbf{yank}
\textbf{-m};
determining relative metric positions using
\textbf{metpos};
and characterizing metric syncopation using
\textbf{synco}.

\par 
Processing data that does not explicitly contain duration-related
information (such as \texttt{**harm} or \texttt{**deg}) often requires
some preparation.
It is often useful to maintain a coordinated file
where the spines of interest are linked with duration-related
spines that assist in processing.

\par 
One further topic related to rhythm remains to be discussed.
The
\textbf{accent}
command allows the user to distinguish notes according to
their estimated perceptual importance.
We will consider
\textbf{accent}
in
Chapter 31.

\\
\begin{itemize}
\item 

\textbf{Next Chapter}
\item 

\textbf{Previous Chapter}
\item 

\textbf{Table of Contents}
\item 

\textbf{Detailed Contents}
\\\\

� Copyright 1999 David Huron
\end{document}
