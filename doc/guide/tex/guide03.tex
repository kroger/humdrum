% This file was converted from HTML to LaTeX with
% Tomasz Wegrzanowski's <maniek@beer.com> gnuhtml2latex program
% Version : 0.1
\documentclass{article}
\begin{document}



  
  
    
      
      
      
    
  



\section*{Chapter3}


[\textit{Previous Chapter}]
[\textit{Contents}]
[\textit{Next Chapter}]


\section*{Some Initial Processing}



Now that we have learned some things about Humdrum representations
(and the \texttt{**kern} representation in particular),
let's explore some basic processing tasks.


\subsection*{The \textit{census} Command}

\par 
The Humdrum
\textbf{census}
command provides basic information about an input stream or file.
We can invoke the command by typing the command name followed by
the name of a file.
The command

\par 

\texttt{census india01.krn}


\par 
might produce the following output:

\texttt{HUMDRUM DATA

Number of data tokens:     91
Number of null tokens:     0
Number of multiple-stops:  0
Number of data records:    91
Number of comments:        14
Number of interpretations: 7
Number of records:         112}






\par 
Most commands provide
\textit{options}
that will modify the operation of the command in a particular way.
In UNIX-style commands, options follow after the
command name and are typically specified by a single letter
preceded by a hyphen.
The
\textbf{-k}
option with the
\textbf{census}
command will give further information pertaining to
the Humdrum \texttt{**kern} representation.
With the
\textbf{-k}
option, the output includes the number of notes in the file,
the longest, shortest, highest, and lowest notes, the maximum
number of concurrent notes or voices, the number of rests,
and the number of barlines.
For example, the command:

\par 

\texttt{census -k india01.krn}


\par 
might produce the following
\textit{additional}
output:

\texttt{KERN DATA

Number of noteheads:       78
Number of notes:           78
Longest note:              1
Shortest note:             16
Highest note:              cc
Lowest note:               c
Number of rests:           1
Maximum number of voices:  1
Number of single barlines: 11
Number of double barlines: 1}


\par 
Notice that a distinction is made between the number of notes
and the number of noteheads.
A tied note is considered to be a single "note,"
although it may be notated using two or more noteheads.

\par 
The output from
\textbf{census}
can be restricted to a particular item of information by "piping"
the output to the UNIX
\textbf{grep}
command.


\subsection*{Simple Searches using the \textit{grep} Command}

\par 
The UNIX
\textbf{grep}
command is a popular tool for searching for lines
that match some specified pattern.
Patterns may be simple strings of characters, or may be more complicated
constructions defined using the UNIX
\textit{regular expression}
syntax.
Regular expressions will be described in detail in
Chapter 9.
The command name "\texttt{grep}" is an acronym
for "get regular expression."


\par 
Useful patterns are often literal character strings, such as keywords.
For example, the following command identifies whether the
file \texttt{opus28.krn} contains the word "\texttt{Andante}":

\par 

\texttt{grep 'Andante' opus28.krn}


\par 
Every line containing the specified pattern will be output.
If no match is found, no output is given.


\par 
Using a single command, all files in the current directory can be
searched by substituting the asterisk (shell \textit{wildcard}) in place
of a filename.
The following command identifies all instances where the
word "\texttt{Andante}" occurs;
all files in the current directory are searched:

\par 

\texttt{grep 'Andante' *}


\par 
Once again, every line containing the sought pattern is echoed
in the output.
If more than one pattern is found, each instance of the
pattern will be output on a separate line.
Whenever an asterisk or "wildcard" is used as part of the filename,
\textbf{grep}
causes the \textit{name} of each file to be prepended to the
output for all patterns that are found:

\par 

\texttt{opus28:!! Andante
\\
opus29:!! Andante
\\
opus46:!! Andante
\\
opus91:!! Andante
\\
opus98:!! Andante}


\par 
By default,
\textbf{grep}
distinguishes upper- and lower-case characters,
so the above command will not match strings such
as "\texttt{ANDANTE}".
However, the
\textbf{-i}
option tells
\textbf{grep}
to ignore the case when searching.
E.g.,

\par 

\texttt{grep -i 'Andante' *}



\par 
Sought patterns may occur in any line, including data records
and comments.
The following command will identify the presence of any
double-sharps in the file \texttt{schumann.krn}.

\par 

\texttt{grep '\#\#' schumann.krn}



\subsection*{Pattern Locations Using grep -n}

\par 
If a pattern is found, it is sometimes helpful to know the precise
location of the pattern.
The
\textbf{-n}
option tells
\textbf{grep}
to prepend the
\textit{line number}
for each matching instance.
The following command identifies the line numbers for lines
containing a double sharp for the file \texttt{melody.krn}:

\par 

\texttt{grep -n '\#\#' melody.krn}


\par 
The output might look like this:

\par 

\texttt{1109:\{4g\#\#
\\
1731:16g\#\#
\\
3002:16f\#\#}


\par 
-$\,$-  meaning that double sharps were found in lines 1109, 1731,
and 3002 in the file \texttt{melody.krn}.


\subsection*{Counting Pattern Occurrences Using grep -c}


\par 
In some cases, the user is interested in counting the total number of
instances of a found pattern.
The
\textbf{-c}
option causes
\textbf{grep}
to output a numerical
\textit{count}
of the number of lines containing matching instances.
For example, in the \texttt{**kern} representation, the beginning of
each phrase is marked by the presence of an open curly brace (`\texttt{\{}').
So the following command can be used to count the number of phrases in the
file \texttt{glazunov.krn}:

\par 

\texttt{grep -c '\{' glazunov.krn}


\par 
As noted, the
\textbf{grep}
command will search all lines (including comments)
for matching instances of the specified pattern.
If a curly brace were to appear in a comment or other non-data record,
then our phrase-count would be incorrect.
More carefully constructed patterns require a better knowledge of 
\textit{regular expressions.}
Regular expressions are discussed in
Chapter 9.


\subsection*{Searching for Reference Information}

\par 
As we saw in
Chapter 2,
Humdrum files typically encode
library-type information using reference records.
For example, the
\textbf{composer's name}
is encoded in a \texttt{!!!COM:}
record, and the
\textbf{title}
is encoded via the \texttt{!!!OTL:} record.
In conjunction with the
\textbf{grep}
command, these three-letter codes provide useful tags to search for
pertinent information.
For example, the following command will identify the composer
for the file \texttt{opus24.krn}:

\par 

\texttt{grep '!!!COM:' opus24.krn}


\par 
The output might look like this:

\par 

\texttt{!!!COM: Boulanger, Nadia}



\par 
Once again, a wildcard (i.e., the asterisk) can be used to address
all of the files in the current directory.
Hence the command

\par 

\texttt{grep '!!!COM:' *}



\par 
will produce a list of all composers of files in the current directory.
Similarly, the following command will generate a list of all of
the titles:

\par 

\texttt{grep '!!!OTL:' *}


\par 
The output might look as follows:

\par 

\texttt{foster11:!!!OTL: Oh! Susanna
\\
foster12:!!!OTL: Jeanie with the Light Brown Hair
\\
foster13:!!!OTL: Beautiful Dreamer
\\
foster14:!!!OTL: Gwine to Run All Night (or 'De Camptown Race')
\\
foster15:!!!OTL: My Old Kentucky Home, Good-Night
\\
foster16:!!!OTL: We are Coming, Father Abraam
\\
foster17:!!!OTL: Don't Bet Your Money on De Shanghai
\\
foster18:!!!OTL: Gentle Annie
\\
foster19:!!!OTL: If You've Only Got a Moustache
\\
foster20:!!!OTL: Maggie by my Side
\\
foster21:!!!OTL: Old Folks at Home
\\
foster22:!!!OTL: Better Times are Coming
\\
foster23:!!!OTL: When this Dreadful War is Ended
\\
foster24:!!!OTL: Hard Times Comes Again No More}


\par 
Remember that when a wildcard is used in filenames,
\textbf{grep}
prepends the filename prior to found patterns.
These filename `headers' can be eliminated by selecting the
\textbf{-h}
option for \textbf{grep}:

\par 

\texttt{grep -h '!!!OTL:' *}


\par 
(N.B. Some older versions of
\textbf{grep}
do not support all of the options described here.
Filename headers can be stripped from the output by using the UNIX
\textbf{sed}
command described in
Chapter 14.)

\par 
We might place the resulting list of titles in a separate file using the
UNIX
\textit{file redirection}
construction.
The output of a command can be placed into a file by following the
command with a greater-than sign (>) followed by a filename.
For example, the following command places the output from
\textbf{grep}
in a file called \texttt{titles}:

\par 

\texttt{grep -h '!!!OTL:' * > titles}


\par 
Beware that if the file \texttt{titles} already exists
then it will be over written and its previous contents lost.
With the
\textbf{-h}
option the file \texttt{titles} might contain the following lines:

\par 

\texttt{!!!OTL: Oh! Susanna
\\
!!!OTL: Jeanie with the Light Brown Hair
\\
!!!OTL: Beautiful Dreamer
\\
!!!OTL: Gwine to Run All Night (or 'De Camptown Race')
\\
!!!OTL: My Old Kentucky Home, Good-Night
\\
!!!OTL: We are Coming, Father Abraam
\\
!!!OTL: Don't Bet Your Money on De Shanghai
\\
!!!OTL: Gentle Annie
\\
!!!OTL: If You've Only Got a Moustache
\\
!!!OTL: Maggie by my Side
\\
!!!OTL: Old Folks at Home
\\
!!!OTL: Better Times are Coming
\\
!!!OTL: When this Dreadful War is Ended
\\
!!!OTL: Hard Times Comes Again No More}


\par 

\subsection*{The \textit{sort} Command}

\par 
The UNIX operating system provides a general sorting utility
called \textbf{sort}.
We might use this utility to rearrange the titles in alphabetical order:

\par 

\texttt{sort titles}


\par 
Rather than using an intermediate file, we can directly connect the
\textbf{grep}
and
\textbf{sort}
commands using a UNIX "pipe."
The vertical bar (\texttt{|}) creates a connection between the
output of one command and the input of the next command.
We can combine the above two commands to create an alphabetical
listing of all titles in the current directory:

\par 

\texttt{grep '!!!OTL:' * | sort}


\par 
File redirection can be added at the end of a pipe so the final
output is captured in a file.
In the following case, the alphabetized titles are placed in the
file \texttt{titles}:

\par 

\texttt{grep '!!!OTL:' * | sort > titles}



\subsection*{The \textit{uniq} Command}

\par 
Bach often harmonized a chorale melody more than once.
In the 185 chorales in the original 1784 edition,
several duplicate titles are present.
Suppose you want to create an alphabetical list of titles,
but you want to exclude duplicate titles.
The UNIX
\textbf{uniq}
command provides a useful utility for eliminating duplication.
Without any option,
\textbf{uniq}
simply eliminates any successive repeated lines.
For example, given the input:

\par 

\texttt{1
\\
1
\\
1
\\
2
\\
2
\\
3}


\par 
the 
\textbf{uniq}
command will produce the following output:

\par 

\texttt{1
\\
2
\\
3}


\par 
Note that
\textbf{uniq}
only discards
\textit{successive}
repeated records;
an input such as the following would remain unmodified by the
\textbf{uniq}
command:

\par 

\texttt{1
\\
2
\\
3
\\
1
\\
3
\\
1}


\par 
Another important point about
\textbf{uniq}
is that successive lines must be
\textit{exact repetitions}
in order to be discarded.
For example, if one line has a trailing blank that is not present in
the previous line, then the line is not discarded.

\par 
Returning to our problem of creating a list of unique titles
for J.S. Bach's chorale harmonizations, we can use the following
command pipeline.

\par 

\texttt{grep -h '!!!OTL:' * | sort | uniq}


\par 
Note that our "pipeline" consists of three successive commands
with the outputs connected to the inputs using the UNIX pipe
symbol (\texttt{|}).
The
\textbf{sort}
command is essential in order to collect identical titles
as successive lines before passing the list to
\textbf{uniq}.

\par 
Suppose you wanted to ensure that all of the works in the current
directory are composed by the same composer.
The same command structure can be used, only we would search for reference
records encoding the composer's name:

\par 

\texttt{grep -h '!!!COM:' * | sort | uniq}


\par 
Even if the current directory contains hundreds of works by
one composer (say Beethoven) and just a single work by another composer,
the presence of the odd score will be obvious without
having to look through long lists:

\par 

\texttt{!!!COM: Beethoven, Ludwig van}
\\
\texttt{!!!COM: Stamitz, Carl Philipp}



\par 
Of course we can make similar lists for other types of information
available in reference records.
The \texttt{AIN} reference record encodes instrumentation.
We could make a list of various instrumental combinations
used for scores in the current directory:

\par 

\texttt{grep -h '!!!AIN:' * | sort | uniq}



\subsection*{Options for the \textit{uniq} Command}

\par 
Like \textbf{grep}, the
\textbf{uniq}
command provides several options that modify its behavior.
The
\textbf{-d}
option causes only those records to be output which are
\textit{duplicated}
(i.e. two or more instances).
Conversely, the
\textbf{-u}
option causes only those records to be output that are truly
\textit{unique}
(i.e. only a single instance is present in the input).


\par 
Suppose, for example, that we want to know which of the Bach
chorales are harmonizations of the same tunes -$\,$- that is,
have the same titles.
(Of course the same chorale might be known by two or more titles,
but let's defer this problem until
Chapter 25.)
The
\textbf{-d}
option will only output the duplicate records:

\par 

\texttt{grep -h '!!!OTL:' * | sort | uniq -d}


\par 
The output will identify those titles which appear in two or
more files in the current directory.
The output might look as follows:

\par 

\texttt{!!!OTL: Befiehl du deine Wege
\\
!!!OTL: Christ lag in Todesbanden
\\
!!!OTL: Christus, der ist mein Leben
\\
!!!OTL: Das alte Jahr vergangen ist
\\
!!!OTL: Ein' feste Burg ist unser Gott
\\
!!!OTL: Erbarm' dich mein, o Herre Gott
\\
!!!OTL: Herr, ich habe missgehandelt
\\
!!!OTL: Herr, wie du willst, so schick's mit mir
\\
!!!OTL: Ich dank' dir, lieber Herre
\\
!!!OTL: Jesu, der du meine Seele
\\
!!!OTL: Jesu, meiner Seelen Wonne}


\par 
Having established which titles are duplicates,
a logical next step might be to identify the specific files involved.
We can use
\textbf{grep}
again to search for a specific title.
Without the
\textbf{-h}
option, the output will identify the appropriate filenames.
For example:

\par 

\texttt{grep '!!!OTL: Befiehl du deine Wege' *}


\par 
might produce the following output:


\par 

\texttt{bwv270.krn:!!!OTL: Befiehl du deine Wege}
\\
\texttt{bwv271.krn:!!!OTL: Befiehl du deine Wege}
\\
\texttt{bwv272.krn:!!!OTL: Befiehl du deine Wege}


\par 
Sometimes we would like to have an output that
contains \textit{only} the \textit{filenames} containing the sought pattern.
The
\textbf{-l}
option causes
\textbf{grep}
to output only filenames that contain one or more instances
of the sought pattern:

\par 

\texttt{grep -l '!!!OTL: Befiehl du deine Wege' *}


\par 
The output would appear as follows:

\par 

\texttt{bwv270.krn}
\\
\texttt{bwv271.krn}
\\
\texttt{bwv272.krn}


\par 
As we've already notes, the
\textbf{-u}
option for \textbf{uniq} causes only unique entries in a list to be passed
to the output.
This is often useful in identifying works that differ in some way
from other works in a group or corpus.
For example, in some repertory, you may remember that a particular work
had a different instrumentation than the other works.
But you may not be able to remember what the specific instrumentation was.
Use the
\textbf{-u}
option for
\textbf{uniq}
to produce a list consisting of only those works whose
instrumentation differs from all others:

\par 

\texttt{grep -h '!!!AIN:' * | sort | uniq -u}



\par 
As in the case of the
\textbf{grep}
command,
\textbf{uniq}
also supports a
\textbf{-c}
option which counts the number of occurrences of a pattern.
For example, if we want to count the number of works by each composer
in the current directory:

\par 

\texttt{grep -h '!!!OTL:' * | sort | uniq -c}


\par 
The output might appear as follows:






\par 

\texttt{ 9 !!!COM: Berardi, Angelo
\\
 2 !!!COM: Caldara, Antonio
\\
12 !!!COM: Zarlino, Gioseffo
\\
 2 !!!COM: Sweelinck, Jan Pieterszoon
\\
 4 !!!COM: Josquin Des Pres}
\\


\par 
Notice that the number of instances is prepended to the reference
records.

\par 
Incidentally, if we wanted to rearrange this list in order of
the number of works, we could pass the above output to yet another
\textbf{sort}
command.
Since
\textbf{sort}
sorts from left to right, it will begin sorting according to
the numerical values at the extreme left.
The command

\par 

\texttt{grep -h '!!!COM:' * | sort | uniq -c | sort -n}


\par 
will rearrange the above output as follows:

\par 

\texttt{ 2 !!!COM: Caldara, Antonio
\\
 2 !!!COM: Sweelinck, Jan Pieterszoon
\\
 4 !!!COM: Josquin Des Pres
\\
 9 !!!COM: Berardi, Angelo
\\
12 !!!COM: Zarlino, Gioseffo}
\\


\par 
It is important to understand that the two
\textbf{sort}
commands in our pipeline achieve different goals but
use the same process.
The first
\textbf{sort}
command sorts the composer's names into alphabetical order.
This is done so that the ensuing
\textbf{uniq}
command is able to count successive identical records.
Since the
\textbf{uniq -c}
command prepends numerical counts, the subsequent
\textbf{sort}
sorts first according to the numbers to the left of the
reference records.

\par 
As a final note, we might mention that, like
\textbf{grep}
and \textbf{uniq}, the
\textbf{sort}
command has several options.
One option, the
\textbf{-r}
option, causes the output to be arranged in reverse order.
This can be useful in producing lists that are ordered from
most common to least common.



\subsection*{Reprise}

\par 
In this chapter we have introduced some elementary ways
of processing Humdrum files.
We noted that the
\textbf{census}
command can be used to identify basic statistics about a file.
The
\textbf{-k}
option for
\textbf{census}
provides basic information related to \texttt{**kern} files,
such as the number of notes and rests, the highest and lowest notes,
the number of barlines, etc.

\par 
In this chapter we also introduced simple searching techniques
using the
\textbf{grep}
command;
\textbf{grep}
provides a useful way of locating particular patterns of text
characters in files.
We used
\textbf{grep}
to identify composers, titles, instrumentation and other information.
Most of our examples were limited to searching for Humdrum reference
records.
In later chapters we will use
\textbf{grep}
in more sophisticated searches.
We noted several useful options for \textbf{grep}:
the
\textbf{-c}
option causes a count to be output of the number of instances
of the pattern in each file.
The
\textbf{-i}
option causes
\textbf{grep}
to ignore any distinction between upper- and lower-case characters
when searching for patterns.
The
\textbf{-h}
option causes
\textbf{grep}
to suppress outputting the filenames prior to found patterns
when more than one file is searched.
The
\textbf{-l}
option results in only the filenames being output.
In a later chapter we will encounter
a number of other useful options provided by \textbf{grep}.

\par 
Also discussed in this chapter was the
\textbf{uniq}
command;
\textbf{uniq}
provides a useful utility for eliminating
or isolating duplicate records or lines.
Once again a number of useful options were introduced.
The
\textbf{-c}
option causes
\textbf{uniq}
to prepend a count of the number of duplicate input lines.
The
\textbf{-d}
option results in only duplicate input lines being noted in the
output.
The
\textbf{-u}
option does the reverse:
only those input lines that are unique are passed to the output.

\par 
Finally, we introduced the UNIX
\textbf{sort}
utility.
This command rearranges the order of successive input lines
so they are in alphabetic/numeric order.
The
\textbf{sort}
command provides a wealth of useful options;
however, we mentioned only the
\textbf{-r}
option -$\,$- which causes the output to be sorted in reverse order.

\\
\begin{itemize}
\item 

\textbf{Next Chapter}
\item 

\textbf{Previous Chapter}
\item 

\textbf{Table of Contents}
\item 

\textbf{Detailed Contents}
\\\\

� Copyright 1999 David Huron
\end{document}
