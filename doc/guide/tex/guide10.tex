% This file was converted from HTML to LaTeX with
% Tomasz Wegrzanowski's <maniek@beer.com> gnuhtml2latex program
% Version : 0.1
\documentclass{article}
\begin{document}



  
  
    
      
      
      
    
  



\\
\\

\section*{Chapter10}


[\textit{Previous Chapter}]
[\textit{Contents}]
[\textit{Next Chapter}]


\section*{Musical Uses of Regular Expressions}



Now that you have a better understanding of
regular expressions, let's apply them.
This chapter provides many examples of
how regular expressions may be used to define musically useful patterns.
In subsequent chapters, we'll make frequent use of regular expressions.


\subsection*{The \textit{grep} Command (Again)}

\par 
Although regular expressions are used in a number of Humdrum commands,
they are most frequently used in conjunction with the
\textbf{grep}
command encountered in
Chapter 3.
\textbf{grep}
is a popular software tool that is available from a number of
manufacturers and sources.
Many versions of
\textbf{grep}
differ in the options provided.
For example, the version of
\textbf{grep}
distributed by the GNU Software Foundation provides
no fewer than 19 options.
Some of the most common options for
\textbf{grep}
are identified in Table 10.1.
\\\\
\textbf{Table 10.1}

-ccount the number of lines matching the regular expression
-f \textit{file}search for patterns that are specified in \textit{file}
-iignore differences of upper- and lower-case
-ljust list the names of files containing a matching line
-nprefix each output line with its line number
-hsuppress file-name prefixes (headers) in output when searching more than one file
-vdisplay all lines \textit{not} matching the regular expression
-Llist names of files \textit{not} containing the regular expression


\textit{Common options for the \textbf{grep} command.}

\par 
Many of the predefined Humdrum representations make use of the
"common system" for representing barlines.
The following command counts the number of barlines in the
file \texttt{czech37.krn}.
Note that the caret anchor (\texttt{\^{}}) is used to avoid inadvertent matches
of the equals sign that might appear in Humdrum comments or interpretations.

\par 

\texttt{grep -c \^{}= czech37.krn}


\par 
Recall that the dollar sign (\texttt{\$}) can be used to anchor
an expression to the end of the line.
The following command determines whether numbered measure 9 is present in the
file \texttt{france12.krn};
the dollar sign ensures that measure 9 is not
mistaken for measure 90, 930, etc.

\par 

\texttt{grep \^{}=9\$ france12.krn}


\par 
The asterisk means "zero or more" instances of the preceding
expression.
For example, the following regular expression will match any
reference record or global comment in the file \texttt{clara29}:

\par 

\texttt{grep '\^{}!!!*' clara29}


\par 
Suppose we want to list all of the global comments for all files
in the current directory:

\par 

\texttt{grep '\^{}!!!*' *}


\par 
Notice that the two asterisks serve different functions in the
above command.
The first asterisk means "zero or more instances" and is
part of the regular expression passed to \textbf{grep}.
The second asterisk means "all files in the current directory"
and is expanded by the shell.
The first asterisk is `protected' from the shell by the single quotes.
Otherwise, the first asterisk might be expanded by the shell to a list
of all files in the current directory.

\par 
In regular expressions, the period character (\texttt{.}) matches any single
character.
For example, the expression `\texttt{A.B}' will match strings such as
`\texttt{AXB}' and `\texttt{AAB}' etc.
The following command identifies all eighth-notes containing
at least one flat, and whose pitch lies within an octave of middle C.

\par 

\texttt{grep 8.- *.krn}


\par 
Frequently it is necessary to turn off the special
meanings for metacharacters such as \^{}, \$, and *.
Recall that this can be done by inserting a backslash (\texttt{$\backslash$})
immediately prior to the metacharacter.
In the
\texttt{**kern}
representation the caret signifies an accent.
In a monophonic input, we might count the number of notes
that have a notated accent as follows:

\par 

\texttt{grep -c '$\backslash$\^{}' danmark3.krn}



\par 
In the following command we have used the backslash to escape
the special meaning of the asterisk.
The
\textbf{-l}
option causes
\textbf{grep}
to output only the names of any files that contain a line matching
the pattern.
Hence, the following command identifies those files in the
current directory that encode music in 9/8 meter:

\par 

\texttt{grep -l '\^{}$\backslash$*M9/8' *}



\par 
Recall that square brackets can be used to indicate character classes
where any of the characters in the class can be used to match the
expression.
The following command identifies those files in the current directory
that encode music in either 3/8 or 9/8 meter:

\par 

\texttt{grep -l '$\backslash$*M[39]/8' *}



\par 
One of the most frequently used regular expressions consists of the
period followed by the asterisk (\texttt{.*}).
Recall that this expression will match
\textit{any}
string including the null string (i.e. nothing at all).
This expression commonly appears between two other character strings.
For example, we can identify all files in the current directory whose
instrumentation includes a trumpet:

\par 

\texttt{grep -l '!!!AIN.*tromp' *}

The \texttt{.*} expression is needed since we don't know what other
instruments might be listed following \texttt{AIN} and before \texttt{tromp}.

Instrumentation reference records require that instrument
codes appear in alphabetical order.
This makes it easier to conduct searches for combinations of instruments.
For example, we can identify all scores in the current directory
whose instrumentation includes both trumpet and cornet as follows:

\par 

\texttt{grep -l '!!!AIN.*cornt.*tromp' *}



\par 
There are many variants on the use of the \texttt{.*} expression.
The following command identifies all files that contain
a record having the word \texttt{Drei} followed by the word
"\texttt{Koenige}".
(Notice the use of the
\textbf{-i}
option in order to ignore the case of the letters.)

\par 

\texttt{grep -li 'Drei.*Koenige' *}


\par 
This command will match such strings as:
\textit{Die Heiligen Drei Koenige},
\textit{Drei Koenige,}
\textit{Dreikoenigslied,}
etc.


\par 
The `\texttt{!!!AGN}' reference record is
used to encode genre-related keywords.
The following command lists all files that are ballads.

\par 

\texttt{grep -l '!!!AGN.*Ballad' *}



\par 
List all files that have the word \texttt{Amour} in the title:

\par 

\texttt{grep -li '!!!OLT.*Amour' *}



\par 
List any works that bear a dedication:

\par 

\texttt{grep -l '!!!ODE:' *}



\par 
List those works that are in irregular meters:

\par 

\texttt{grep -l '!!!AMT.*irregular' *}


\par 
The
\textbf{-L}
option for
\textbf{grep}
causes the output to contain a list of files
\textit{not}
containing the regular expression.
For example, we could identify those works that don't bear any dedication:

\par 

\texttt{grep -L '!!!ODE:' *}



\par 
List those works
\textit{not}
composed by Schumann:

\par 

\texttt{grep -L '!!!COM: Schumann' *}



\par 
Identify any works that don't contain any double barlines:

\par 

\texttt{grep -L '\^{}==' *}



\par 
How many works in the current directory are in simple-triple meter?

\par 

\texttt{grep -c '!!!AMT.*simple.*triple' *}



\par 
When searching for more complex patterns it may be necessary
to use
\textbf{grep}
more than once.
Consider, for example, the problem of identifying works
whose titles contain both the words
\texttt{Liebe} and \texttt{Tod}.
The first of the following commands will identify only those titles
that contain \texttt{Liebe} followed by \texttt{Tod},
whereas the second command will identify only those titles that contain
\texttt{Tod} followed by \texttt{Liebe}:

\par 

\texttt{grep '!!!OTL.*Liebe.*Tod' *}
\\
\texttt{grep '!!!OTL.*Tod.*Liebe' *}


\par 
A better solution is to pipe the output between two
\textbf{grep}
commands.
Recall that the vertical bar (`|') conveyes or "pipes" the output
from one command to the input of a subsequent command.
The following command passes all opus-title records
(\texttt{OTL}) containing the word \texttt{Liebe} to a second \textbf{grep},
which passes only those records also containing the word \texttt{Tod}.
Since both
\textbf{grep}
commands process the entire input line,
it does not matter whether the word \texttt{Tod}
precedes or follows the word \texttt{Liebe}:

\par 

\texttt{grep '!!!OTL.*Liebe' * | grep 'Tod'}


\par 
The
\textbf{-v}
option for
\textbf{grep}
causes a "reverse" or "negative" output.
Instead of outputting all records that
\textit{match}
the specified regular expression, the
\textbf{-v}
option causes only those records to be output that do
\textit{not}
match the given regular expression.
For example, the following command eliminates all comments
from the file \texttt{polska24.krn}:

\par 

\texttt{grep -v '\^{}!' polska24.krn}


\par 
Similarly, the following command eliminates all whole-note rests:

\par 

\texttt{grep -v 1r *}



\par 
The
\textbf{-v}
option is especially convenient in pipelines.
For example, the following command identifies all those files
whose instrumentation includes a cornet but not a trumpet:

\par 

\texttt{grep '!!!AIN.*cornt' * | grep -v 'tromp'}



\par 
The following command identifies those works in compound meters
that are not also quadruple meters:

\par 

\texttt{grep '!!!AMT.*compound' * | grep -v 'quadruple'}



\par 
Similarly, the following command identifies those notes that
begin a phrase, but are not rests.

\par 

\texttt{grep '\^{}\{' * | grep -v r}



\subsection*{German, French, Italian, and Neapolitan Sixths}

\par 
In conjunction with the
\textbf{solfa}
command,
\textbf{grep}
can be used to search for various types of special chords.
Suppose, for example, that we wanted to identify occurrences of
augmented sixth chords.
An augmented sixth chord is characterized by an augmented sixth
interval occurring between the lowered sixth scale-degree and
the raised fourth scale-degree.
In
Chapter 4,
we saw that the
\textbf{solfa}
command represents pitches with respect to an encoded tonic pitch.
In the
\texttt{**solfa}
representation, the lowered sixth and raised fourth
degrees will be represented as \texttt{6-} and \texttt{4+} respectively.
First we translate the input to the
\texttt{**solfa}
representation,
and then we search for records matching the appropriate regular expression:

\par 

\texttt{solfa input | grep '6-.*4+'}


\par 
Notice that the expression `\texttt{6-.*4+}' presumes that
the lowered sixth degree is lower in pitch than the raised fourth degree.
For augmented sixth chords, this is a reasonable presumption.
In the unlikely situation that the raised fourth degree is
lower in pitch than the lowered sixth degree, we would need to also search
for the expression `\texttt{4+.*6-}'.
Alternatively, we could use two separate
\textbf{grep}
commands, eliminating the constraint of order:

\par 

\texttt{solfa input | grep '6-' | grep '4+'}


\par 
Augmented sixth chords can be further classified as either
German, French, or Italian sixths.
The German sixth contains the lowered mediant whereas the French sixth
contains the supertonic pitch;
the Italian sixth contains neither:

\par 

\texttt{solfa input | grep '6-.*4+' | grep '3-'}		\# German sixth
\\
\texttt{solfa input | grep '6-.*4+' | grep '2'}		\# French sixth
\\
\texttt{solfa input | grep '6-.*4+' | grep -v '[23]'}	\# Italian sixth


\par 
A similar approach can be used to identify Neapolitan sixth chords.
These chords are based on the lowered supertonic with the
third of the chord (unaltered subdominant) in the bass.


\par 

\texttt{solfa input | grep '4[\^{}-+].*2-' | grep '6-'}	\# Neapolitan sixth


\par 
Depending on the key, Neapolitan chords are sometimes
notated enharmonically as a raised tonic chord.
Suppose we were looking for such enharmonically spelled Neapolitan chords:

\par 

\texttt{solfa input | grep '3+.*1+' | grep '5+'}



\par 
Occassionally, Neapolitan chords are missing the fifth of the chord
(the lowered sixth degree of the scale).
We might search for an example of such a chord:

\par 

\texttt{solfa input | grep '2-' | grep '4' | grep -v '6-'}



\subsection*{AND-Searches Using the \textit{xargs} Command}

\par 
In some cases, we want to identify those files that match
two entirely different patterns (in different records).
Recall that the
\textbf{-l}
option causes
\textbf{grep}
to output the
\textit{filename}
rather than the matching record.
If we could pass along these file names to another
\textbf{grep}
command, we could search those same files for yet another pattern.


\par 
The UNIX
\textbf{xargs}
command provides a useful way of transferring the output from one
command to be used as final arguments
for a subsequent command.
For example, the following command takes each file whose opus title
contains the word \texttt{Liebe} and counts the number of phrases.

\par 

\texttt{grep -l '!!!OTL:.*Liebe' * | xargs grep -c '\^{}\{'}


\par 
In this case the
\textbf{grep -l}
command outputs a list of names of files containing the string
\texttt{Liebe} in an OTL reference record.
The
\textbf{xargs}
command causes these filenames to be appended to the end of the following
\textbf{grep}
command.
The
\textbf{grep -c}
command will thus be applied only to those files already identified by
the previous
\textbf{grep}
as containing \texttt{Liebe} in the title.


\par 
A set of such pipelines can be used to answer more sophisticated
questions.
For example, are drinking songs more apt to be in triple meter?

\par 

\texttt{grep -l '!!!AMT.*triple'  *   | xargs grep -l '!!!AGN.*Trinklied'}
\\
\texttt{grep -l '!!!AMT.*duple'   *   | xargs grep -l '!!!AGN.*Trinklied'}
\\
\texttt{grep -l '!!!AMT.*quadruple' * | xargs grep -l '!!!AGN.*Trinklied'}



\par 
Similarly, the following commands determine whether files whose
titles contain the word
\textit{death}
are more apt to be in minor keys:

\par 

\texttt{grep -li '!!!OTL.*death' * | xargs grep -c '\^{}$\backslash$*[a-g][\#-]*:'}
\\
\texttt{grep -li '!!!OTL.*death' * | xargs grep -c '\^{}$\backslash$*[A-G][\#-]*:'}



\par 
Note that the
\textbf{xargs}
command can be used again and again to
continue propagating file names as arguments to subsequent searches.
For example, the following command outputs the key signatures
for all works originating from Africa that are written
in 3/4 meter:

\par 

\texttt{grep -l '!!!ARE.*Africa' * | xargs grep -l '\^{}$\backslash$*M3/4' $\backslash$

| xargs grep '\^{}$\backslash$*k$\backslash$['}




\par 
Similarly, the following command outputs the names of all files in
the current directory that encode 17th century organ works containing
passages in 6/8 meter:

\par 

\texttt{grep -l '!!!ODT.*16[0-9][0-9]/' | xargs grep -l $\backslash$

 '!!!AIN.*organ' | xargs grep -l '$\backslash$*M6/8'}




\par 
Using the
\textbf{-L}
option allows us to form even more complex criteria by excluding
certain works.
For example, the following variation of the above command outputs the
names of all files in the current directory that encode 17th century organ
works that do not contain passages in 6/8 meter:

\par 

\texttt{grep -l '!!!ODT.*16[0-9][0-9]/' | xargs grep -l $\backslash$

 '!!!AIN.*organ' | xargs grep -L '$\backslash$*M6/8'}




\subsection*{OR-Searches Using the \textit{grep -f} Command}

\par 
In effect, the above pipelines provide logical \textbf{AND}
structures:
e.g. identify works composed in the 17th century AND written for organ AND
containing a passage in 6/8 meter.
The
\textbf{-f}
option for
\textbf{grep}
provides a way of creating logical \textbf{OR} searches.
With the
\textbf{-f}
option, we specify a file containing the patterns being sought.
For example, we might create a file called \texttt{criteria}
containing the following three regular expressions:

\par 

\texttt{!!!ODT.*16[0-9][0-9]/
\\
!!!AIN.*organ
\\
$\backslash$*M6/8}



We would invoke
\textbf{grep}
as follows:

\par 

\texttt{grep -l -f criteria *}


\par 
The
\textbf{-f}
option tells
\textbf{grep}
to fetch the file \texttt{criteria} and use the records in this
file as regular expressions.
A match is made if any of the regular expressions is found.


\par 
The output would consist of a list of all files in the current directory
that encode works composed in the 17th century OR written for organ OR
in 6/8 meter.
The
\textbf{-f}
option is more typically used to specify several variations of the
same idea.
For example, suppose we were searching for D major
triads in
\texttt{**pitch}
data.
We could use a file containing the following regular expressions:

\par 

\texttt{[Dd].*[Ff]\#.*[Aa]
\\
[Dd].*[Aa].* [Ff]\#
\\
[Ff]\#.*[Aa].*[Dd]
\\
[Ff]\#.*[Dd].*[Aa]
\\
[Aa].*[Dd].*[Ff]\#
\\
[Aa].*[Ff]\#.*[Dd]}


\par 
Depending on the application,
it may be easier to construct such pattern files than to use a
lengthy pipeline.
That is:

\par 

\texttt{grep -f Dmajor *}


\par 
may be less cumbersome than:

\par 

\texttt{grep [Dd] * | grep [Ff]\# | grep [Aa]}


\par 
The
\textbf{-f}
option can be combined with
\textbf{-L}.
For example, suppose we wanted to identify all works in the current
directory that are not in the keys of C major, G major, B-flat major
or D minor.
Our regular expression file would contain the following regular expressions:

\par 

\texttt{\^{}$\backslash$*[CGd]:}
\\
\texttt{\^{}$\backslash$*B-:}


\par 
The corresponding command would be:

\par 

\texttt{grep -L -f criteria *}



\par 
Another way of thinking of the
\textbf{-f}
option is that it allows us to define equivalences.
Consider, for example, the task of counting all of the notes in a
\texttt{**kern} melody that belong to a particular whole-tone pitch set.
Let's create two files, one called \texttt{whole1} and the
other called \texttt{whole2}.
The file \texttt{whole1} might contain the following regular expressions:

\par 

\\
\texttt{[Cc]([\^{}-\#Cc]|\$)
\\
[Dd]([\^{}-\#Dd]|\$)
\\
[Ee]([\^{}-\#Ee]|\$)
\\
[Ff]\#([\^{}\#]|\$)
\\
[Gg]-([\^{}-]|\$)
\\
[Gg]\#([\^{}\#]|\$)
\\
[Aa]-([\^{}-]|\$)
\\
[Aa]\#([\^{}\#]|\$)
\\
[Bb]-([\^{}-]|\$)}


\par 
Notice that the regular expressions have been carefully defined.
The first regular expression defines a pattern consisting of
either an upper- or lower-case letter `C' followed either by
a character that is neither a sharp (\#) nor a flat (-) nor another letter `C',
nor is followed by the end of the line (\$).

\par 
Recall that parenthesis grouping (...) is part of the
\textit{extended}
regular expression syntax.
Therefore, we should use the
\textbf{egrep}
rather than the
\textbf{grep}
command with the above expressions.
We can count the number of notes in a monophonic \texttt{**kern}
input that belong to this whole-tone set:

\par 

\texttt{egrep -c -f whole1 debussy}


\par 
If the file \texttt{whole2} contains regular expressions for the
complementary pitch set, we could similarly count the number of
pitches that belong to this alternative set:

\par 

\texttt{egrep -c -f whole2 debussy}




\subsection*{Reprise}

\par 
The
\textbf{grep}
command is usually thought of as a way to find particular patterns
in a file or input stream.
However, the various options for
\textbf{grep}
(such as -v, -l, and -L) allow
\textbf{grep}
to be used for other purposes.
It can be used to isolate data, to count occurrences of patterns,
to eliminate unwanted lines, to identify files for processing,
and to avoid files that contain certain information.

\par 
We have seen how the
\textbf{xargs}
command can be used to carry out \textbf{AND}-searches where
each work must conform to multiple criteria.
We have also seen how the
\textbf{-f}
option for
\textbf{grep}
can be used to permit \textbf{OR}-searches where a work needs to
conform only to one of a set of possible criteria.

\par 
Although this chapter has focussed principally on the
\textbf{grep}
command, the ensuing chapters will show that regular expressions
are used by a wide variety of commands.
In
Chapter 33,
many more powerful examples will be discussed in
conjunction with the
\textbf{find}
command.

\\
\begin{itemize}
\item 

\textbf{Next Chapter}
\item 

\textbf{Previous Chapter}
\item 

\textbf{Table of Contents}
\item 

\textbf{Detailed Contents}
\\\\

� Copyright 1999 David Huron
\end{document}
