% This file was converted from HTML to LaTeX with
% Tomasz Wegrzanowski's <maniek@beer.com> gnuhtml2latex program
% Version : 0.1
\documentclass{article}
\begin{document}



  
  
    
      
      
      
    
  



\\
\\

\section*{Chapter31}


[\textit{Previous Chapter}]
[\textit{Contents}]
[\textit{Next Chapter}]


\section*{Repertories and Links}



In initiating a research project, we often begin by
selecting a suitable repertory for study.
A common approach is to focus on a particular composer,
period, style, or culture.
For example, a researcher might focus on 17th century canons, or on
Beethoven's compositions prior to Opus 20, or on Ojibway songs transcribed
in the 1900s.
Depending on the research task, the user may wish to locate works
that conform to highly complex criteria -$\,$-
such as solo Baroque flute works written in compound
meters, or slow Russian symphonic movements that don't include any
wind instruments and are written in minor keys.

\par 
In this chapter, we discuss how to search entire file-systems
for Humdrum scores that conform to user-specified criteria.
With large musical databases,
automated methods for locating specific
repertories become important.
As we will see, in most cases, one or two commands are all
that is necessary to assemble
a repertory list of works conforming to complex selection criteria.


\subsection*{The \textit{find} Command}

\par 
The UNIX
\textbf{find}
command traverses through a file hierarchy,
and finds all files that match certain conditions.
The
\textbf{find}
command takes the following syntax:

\par 

\texttt{find} \textit{  }


\par 
The \textit{PATH} is a directory from which the search
\textit{commences.}
All files in the specified directory are examined including all files in the
subdirectories, sub-subdirectories, and so on.

\par 
The path

\par 

\texttt{/}


\par 
means the "root" directory containing all files on a computer system.
Even single-user systems are apt to have several thousand files subsumed
under the root directory.

\par 
The path

\par 

\texttt{/scores}


\par 
means all files under the \texttt{scores} directory, whereas

\par 

\texttt{/scores/bach}


\par 
means all files under the \texttt{scores/bach} directory.
The period character:

\par 

\texttt{ .}


\par 
tells
\textbf{find}
to commence searching from the current directory.

\par 
Since
\textbf{find}
searches all files under the given path, its operation
may be quite slow when there are thousands of files to search.
It's wise to restrict the search by choosing a reasonable starting point.
For example, specifying the path \texttt{/scores/bach/chorales}
may save a great deal of time compared with the path \texttt{/scores}.
Although we won't discuss them in this book, the
\textbf{find}
command provides
a number of options that help to restrict the depth of searches or
otherwise "prune" the search.
When first trying
\textbf{find}
it's a good idea to limit the searches to small segments of the file system.


\par 
When searching through the specific \textit{PATH},
\textbf{find}
is able to carry out a wide variety of possible tests on each file.
One simple action is to test whether the file-name conforms to a
given regular expression.
Consider, for example, the goal of identifying all files representing
pitch-class
(\texttt{**pc})
information.
The Humdrum convention is to identify these files by adding the \texttt{.pc}
extension to the filename -$\,$- such as \texttt{opus24.pc}.
The following
\textbf{find}
command will traverse through the \texttt{/scores}
directory (and all sub-directories) searching for files that contain
the \texttt{pc} file extension:

\par 

\texttt{find /scores -name *.pc}


\par 
The above command uses the
\textbf{-name}
option followed by the appropriate regular expression.
This command is unusual in that it has no
explicit \textit{action}.
In such cases, the implied action is to print or display
the name of all files whose names match the regular expression.

\par 
Note that regular expressions may be literal strings.
This means we can locate a specific named file.
For example, the following command will locate
all files named \texttt{findme}:

\par 

\texttt{find / -name findme}


\par 
An example of an explicit \textit{ACTION} might be to delete files conforming
to a particular criterion.
For example, the following command
searches the \texttt{/scores} path for files whose names contain
the \texttt{.tmp} extension.
Any matching file is then deleted using the UNIX
\textbf{rm}
command:

\par 

\texttt{find /scores -name *.tmp -exec rm "\{\}" ";"}


\par 
This command illustrates a number of features of the
\textbf{find}
command.
The search begins from the
\textit{path}
\texttt{/scores}.
The
\textit{option}
\texttt{-name *.tmp} identifies the search condition.
The \texttt{-exec}
flag identifies the \textit{action} as that of executing a command.
The arguments between \texttt{-exec} and the semi-colon are treated as the
command to be executed.
Each time a file is found with the \texttt{.tmp}
extension, the \texttt{-exec} action is executed.
The paired curly braces \{\}
have a special significance to
\textbf{find.}
The braces are replaced by the filename found to match the regular expression.
For example, if the first file found is named \texttt{xyz.tmp} the
braces will be replaced by the string \texttt{xyz.tmp}.
The quotation marks around the braces and around the semi-colon are
necessary in order to prevent the UNIX shell interpreter
from substituting inappropriate information before passing the command
line to \textbf{rm}.

\par 
Note that the
\textbf{-name}
option defaults to filenames only; it does not apply to directory names.
The above command is equivalent to the more explicit form:

\par 

\texttt{find /scores -type f -name *.tmp -exec rm "\{\}" ";"}


\par 
The
\textbf{-type}
option can be used to match 
regular files (\textbf{f}), directories (\textbf{d}), or network files (\textbf{n}).
By way of example, the following command deletes all directories
whose names have the \texttt{.tmp} extension.

\par 

\texttt{find /scores -type d -name *.tmp -exec rmdir "\{\}" ";"}



\subsection*{Content Searching}

\par 
For most music research applications,
we are interested in identifying files on the basis of their contents.
That is, we'd like to know what's inside the file before we take any action.


\par 
The
\textbf{grep}
command is especially useful in determining whether certain
items of information are present in a file.
For example, the following
command identifies all files in the path \texttt{/scores}
that contain passages in 7/8 meter:

\par 

\texttt{find /scores -type f -exec grep -l '$\backslash$*M7/8' "\{\}" ";"}


\par 
Recall that the
\textbf{-l}
option for
\textbf{grep}
causes the output to consist only
of names of files that contain the sought regular expression.
Note that the
\textbf{-type f} option has been specified in order to ensure that the
\textbf{grep}
command is only executed for files.


\par 
The structure of the above command can be used to search for all
sorts of pertinent musical information.
For example, recall that the \texttt{*IC} tandem interpretation is used to encode
instrument classes such as strings, voice, percussion, etc.
The following command
searches all files in the path \texttt{scores} and generates a list of
those files that encode scores containing one or more brass instruments:

\par 

\texttt{find scores -type f -exec grep -l '$\backslash$*ICbras' "\{\}" ";"}



\par 
The following command identifies all files in the path \texttt{/scores},
that contain passages in the key of C major:

\par 

\texttt{find /scores -type f -exec grep -l '$\backslash$*C:' "\{\}" ";"}



\par 
The following command identifies all files in the path \texttt{/scores},
that contain passages in any minor key:

\par 

\texttt{find /scores -type f -exec grep -l '$\backslash$*[a-g][\#-]*:' "\{\}" ";"}



\par 
Humdrum reference records are ideal targets for such searches
since reference records encode information such as the composer's name,
composer's dates, title of work, date of composition, movement number,
instrumentation, meter classification, and so on.
For example, the following command identifies all files in the
path \texttt{/scores} that are composed by Franck:

\par 

\texttt{find /scores -type f -exec grep -l '!!!COM.*Franck' "\{\}" ";"}



\par 
The following command identifies all files in the path \texttt{/scores}
that are written in compound meters:

\par 

\texttt{find /scores -type f -exec grep -l '!!!AMT.*compound' "\{\}" ";"}



\par 
The following command identifies all files beginning from the current directory
that are rondos:

\par 

\texttt{find . -exec grep -il '!!!AFR.*rondo' "\{\}" ";"}


\par 
Recall that the
\textbf{-i}
option for
\textbf{grep}
makes the pattern-match insensitive to upper- or lower-case.


\par 
The following command identifies all files in the path \texttt{non-western}
that have been designated as having heterophonic textures:

\par 

\texttt{find non-western -exec grep -il '!!!AST.*heterophony' "\{\}" ";"}



\par 
In the path \texttt{/scores/jazz},
we might want to identify all files
that contain the style-designation "bebop:"

\par 

\texttt{find /scores/jazz -exec grep -il '!!!AST.*bebop' "\{\}" ";"}



\par 
The following command identifies all files in the path \texttt{18th-century},
that include French horns and oboes:

\par 

\texttt{find 18th-century -exec grep -il '!!!AIN.*cor.*oboe' "\{\}" ";"}



\par 
Of course, more complex regular expressions can be also be defined.
For example, the following command identifies all works composed
between 1805 and 1809:

\par 

\texttt{find / -exec grep -l '!!!ODT.*180[5-9]' "\{\}" ";"}


\par 
There is no restriction on the complexity of the regular expression.
The following command identifies all works composed between 1812
and 1840:

\par 

\texttt{find / -exec egrep -l '!!!ODT.*18(1[2-9])|([23][0-9])|(40)' $\backslash$

"\{\}" ";"}



\par 
Often the
\textbf{find}
command can be used to answer research questions more directly.
Suppose we wanted to determine whether
German drinking songs more likely to be in triple meter.
There are over four thousand German folksongs encoded in Helmut Schaffrath's
\textit{Essen Folksong Collection}.
These works contain genre-related tags encoded as "\texttt{AGN}"
reference records.
One of the genres distinguished is "Trinklied" (drinking song).

\par 
In order to answer our question, we need to search the file system
for all works that have the "Trinklied" designation, and then generate
an inventory of meter classifications (available in "AMT" records).

\par 

\texttt{find /scores -type f -exec grep -l '!!AGN.*Trinklied' "\{\}" $\backslash$

";" | grep '!!!AMT.*' | sort | uniq -c}



\par 
For the entire database, the output is as follows:

\par 

\texttt{ 1 !!!AMT: compound duple
\\
 4 !!!AMT: irregular
\\
14 !!!AMT: simple quadruple
\\
 5 !!!AMT: simple triple}


\par 
There are just 24 drinking songs in the Essen collection and only
five are in triple meters.
The proportion of drinking songs in triple meters turns out to be no different
than the distribution of triple meters in general for German folksongs.
In other words, according to the
\textit{Essen Folksong Collection},
it is not the case that German drinking songs are
more likely to be in triple meters.


\subsection*{Using \textit{find} with the \textit{xargs} Command}


\par 
As we saw in
Chapter 10,
the
\textbf{xargs}
command can be used to propagate file names from command to command within
a pipeline.
Using
\textbf{xargs}
in conjunction with
\textbf{find}
provides a powerful means for finding works that conform to highly
complex criteria.
For example, the following command identifies all files in the path
\texttt{/corelli} that contain a change of meter signature:

\par 

\texttt{find /corelli -type f -name '*' | xargs grep -c '\^{}$\backslash$*M[0-9]' $\backslash$

| grep -v ':[01]\$'}



\par 
The output specifies each filename followed by a colon, followed by
the number of meter signatures in the corresponding file.
For example,
in the following output, the third movement from Opus 1, No. 5 by
Corelli is identified as containing 6 meter signatures at different
points in the score:

\par 

\texttt{/corelli/opus1n5c.krn:6
\\
/corelli/opus1n9a.krn:3
\\
/corelli/opus1n9b.krn:2
\\
/corelli/opus1n9d.krn:2}



\par 
Similarly, the following command identifies all works that contain
a change of key signature:

\par 

\texttt{find /scores -type f -name '*' | xargs grep -c '\^{}$\backslash$*k$\backslash$[' $\backslash$

| grep -v ':[01]\$'}




\par 
As a further example of the use of \textbf{xargs},
consider the following extension of the above pipeline.
The
\textbf{grep -v}
command causes only those files containing more than one key signature to be passed.
The
\textbf{sed}
command eliminates the colon and the number appended to the filenames.
The ensuing
\textbf{grep -c}
counts the number of meter signatures in each file.
The final
\textbf{grep -v}
passes only those filenames containing 2 or more meter signatures.

\par 

\texttt{find / -type f -name '*' | xargs grep -c '\^{}$\backslash$*k$\backslash$[' | $\backslash$

grep -v ':[01]\$' | sed 's/:.*\$//' | $\backslash$
\\
xargs grep -c '\^{}$\backslash$*M[0-9]' | grep -v ':[01]\$'}



\par 
In summary, the above pipeline identifies all scores that contain both a
change of key signature as well as a change of meter signature.


\par 
The
\textbf{xargs}
command can also be used to process a list of files -$\,$- where the list
has been stored in a file.
For example, suppose we used the
\textbf{find}
command to locate all scores in compound meters written for woodwind quintet:

\par 

\texttt{find . -name '*' | xargs grep -l '!!!AMT:.*compound' $\backslash$


| xargs grep -l '!!!AIN: clars cor fagot flt oboe' > scorelist}


\par 
The resulting list of files can be used for further processing.
For example, we might search these files for any scores containing changes
of key:

\par 

\texttt{cat scorelist | xargs grep -c '\^{}$\backslash$*[A-Ga-g][\#-]*:' | grep -v ':[01]\$'}


\par 
The output identifies all scores in compound meters written for woodwind quintet
that contain changes of key.


\subsection*{Repertories As File Links}

\par 
Rather than applying commands to files stored in a list,
it is often helpful to have all of the files accessible in one location.
That is, we might create a directory containing only those
score-files that meet our selection criteria.
it is often helpful to have all of the files accessible in one location.
We might simply make copies of the files in a special directory.
However, UNIX systems make it possible to create "links"
to files in other directories without having to make duplicate copies
of already existing files.

\par 
Suppose you wanted to make a directory of all scores containing
vocal parts.
The following command creates a file
(\texttt{vocalfiles}) listing all files
in the path \texttt{/scores} that contain one or more vocal parts:

\par 

\texttt{find /scores -exec grep -l '!!!AIN.*vox' "\{\}" ";" > vocalfiles}


\par 
The contents of \texttt{vocalfiles} might look like the following:

\par 

\texttt{/scores/bach/cantatas/cant140.krn
\\
/scores/bach/chorales/chor217.krn
\\
/scores/bach/chorales/midi/chor368.hmd}
\\
etc.


\par 
We can create an appropriate new directory using the
\textbf{mkdir}
command.

\par 

\texttt{mkdir vocal}


\par 
Next, edit the file containing the list of filenames as follows.
Insert
\textbf{ln -s}
prior to each filename, and append the directory
name \texttt{vocal} at the end of each line.

\par 

\texttt{ln -s /scores/bach/cantatas/cant140.krn vocal
\\
ln -s /scores/bach/chorales/chor217.krn vocal
\\
ln -s /scores/bach/chorales/midi/chor368.hmd vocal}
\\
etc.


\par 
(The \textbf{-s} option for
\textbf{ln}
is used to create a so-called "symbolic" link.)

\par 
Using the
\textbf{chmod}
command, we can make this file executable, and then we can execute it:

\par 

\texttt{chmod +x vocalfiles
\\
./vocalfiles}


\par 
We now have a new directory whose files contain scores with vocal parts.



\subsection*{Reprise}

\par 
The
\textbf{find}
command provides a convenient way to traverse through
an entire file-system looking for files that conform to specific
criteria.
In musicological tasks, the
\textbf{find}
command is especially well suited to assembling a repertory of scores
that exhibit some characteristic(s) of interest.
Multiple selection criteria can be accommodated by using one or more
pipes in conjunction with the
\textbf{grep}
command.

\par 
For convenience, it is often helpful to create a new directory
that holds all of works selected for a study repertory.
On UNIX systems, file "links" can be created,
so that there is no need to make multiple copies of the same score.
This means that several concurrent directory structures can be created
without duplicating files.
For example, a given score may be accessed in one directory
structure via \textit{composer}, in another directory via \textit{instrumentation},
in a third directory via \textit{genre}, and so on.

\\
\begin{itemize}
\item 

\textbf{Next Chapter}
\item 

\textbf{Previous Chapter}
\item 

\textbf{Table of Contents}
\item 

\textbf{Detailed Contents}
\\\\

� Copyright 1999 David Huron
\end{document}
