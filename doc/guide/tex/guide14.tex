% This file was converted from HTML to LaTeX with
% Tomasz Wegrzanowski's <maniek@beer.com> gnuhtml2latex program
% Version : 0.1
\documentclass{article}
\begin{document}



  
  
    
      
      
      
    
  



\\
\\

\section*{Chapter14}


[\textit{Previous Chapter}]
[\textit{Contents}]
[\textit{Next Chapter}]


\section*{Stream Editing}



Most computer users are familiar with editing an electronic document using an
interactive word-processor or text editor.
\textit{Stream editors}
are non-interactive editors that automatically process
a given input according to a user-specified set of editing instructions.
A stream editor can be used, for example, to automatically transform a
document from British spelling to American spelling.
Stream editors are especially useful when processing large numbers of
documents -$\,$- such as a series of files encoding some musical repertory.
In this chapter we will introduce two stream editors:
\textbf{sed}
and
\textbf{humsed}.


\subsection*{The \textit{sed} and \textit{humsed} Commands}

\par 
The
\textbf{humsed}
command is simply a Humdrum version of the UNIX
\textbf{sed}
stream editor.
The syntax and operation of
\textbf{sed}
and
\textbf{humsed}
are virtually identical.
However,
\textbf{humsed}
will modify only Humdrum data records,
whereas
\textbf{sed}
will modify any type of record, including Humdrum comments and
interpretations.
Both stream editors provide operations for
\textit{substitution},
\textit{insertion},
\textit{deletion},
\textit{transliteration},
\textit{file-read} and
\textit{file-write}.
When used in combination, these operations can completely
transform an input stream or document.


\subsection*{Simple Substitutions}

\par 
The most frequently used stream-editing operation is substitution.
Both
\textbf{humsed}
and
\textbf{sed}
designate substitutions by the lower-case letter \texttt{s}.
Substitutions require two strings: the
\textit{target string}
to be replaced, and the
\textit{replacement string}
to be introduced.
The syntax for substitutions is as follows:

\par 

\textbf{s}\textit{}


\par 
No spaces are permitted between these elements.
The delimiter can be any character;
however, the same delimiter character
must be used throughout the operation.
The following substitution command causes occurrences of the letter `A'
to be replaced by the letter `B':

\par 

\texttt{s/A/B/}


\par 
Since the slash character (/) appears immediately following the \texttt{s},
it becomes the delimiter for the rest of the operation.
In this case no option has been given at the end of the substitution.
Since the delimiter can be any character, the above command is functionally
identical to the following:

\par 

\texttt{sxAxBx}


\par 
If it is necessary to use the delimiter character (as a literal)
within either the target string or the replacement string it can be
escaped using the backslash character.

\par 
There are two ways to execute a substition operation such as given above.
One way is to give the substitution as a command-line argument to
\textbf{sed} or \textbf{humsed}:

\par 

\texttt{humsed s\%A\%B\% filename}


\par 
Alternatively, the operation can be placed in a file
(for example, named \texttt{revise}):

\par 

\texttt{s\%A\%B\%}


\par 
Then the stream editor can be invoked to execute the operations
contained in this file using the
\textbf{-f}
option:

\par 

\texttt{humsed -f revise }\textit{inputfile}


\par 
By default the output will be displayed on the screen.
Using file-redirection (>) the output can be placed in
some other file.
Note that you should never redirect the output to the same file
as the input -$\,$- this will destroy the original input file.
If necessary, send the output to a temporary file, and then use
the UNIX
\textbf{mv}
command to rename the output.


\par 
Suppose that you had encoded a musical work in the \texttt{**kern}
representation.
Having finished the encoding, you realize that what you thought were
\textit{pizzicato}
marks are really
\textit{spiccato}
marks.
In the
\texttt{**kern}
representation, pizzicatos are indicated by
the double quote (\texttt{"}) whereas spiccatos are represented by
the lower-case letter \texttt{s}.
We can change all pizzicato marks to spiccato marks using the following
command:

\par 

\texttt{humsed 's/"/s/g'} \textit{inputfile}


\par 
Since the double quote is interpreted as a special character
by the UNIX shell, we have escaped the entire substitution
operation by placing it in single quotes.
(Alternatively, we could place a backslash immediately before the
double-quote character.)
Note also the presence of the \texttt{g} option at the end of
the string.
Permissible options include any positive integer or the letter \texttt{g}.
Without any option, the
\textbf{sed}
and
\textbf{humsed}
substitute (s) operation will replace only the
\textit{first}
occurrence of the string in each data record.
The \texttt{g} option specifies a "global" substitution,
in that all occurrences on a given data record are replaced.
If the option consisted of the number `3', then only the
third instance of the target string would be replaced on each line.


\subsection*{Selective Elimination of Data}

\par 
The
\textit{target string}
in substitution operations is actually a regular expression.
This means that we can specify patterns using the full power
of regular expression syntax.
Frequently, it is useful to eliminate certain kinds of information
from a file.
For example, we can eliminate all sharps and flats from
a \texttt{**kern}-format file as follows:

\par 

\texttt{humsed s/[\#-]//g} \textit{inputfile}


\par 

\par 
Suppose we wanted to eliminate all beaming information in a score.
In the \texttt{**kern} representation, open and closed beams
are represented by \texttt{L} and \texttt{J} respectively;
partial beams are represented by \texttt{K} and \texttt{k}.

\par 

\texttt{humsed s/[JLkK]//g} \textit{inputfile}



\par 
Alternatively, we might want to eliminate all data except
for the beaming information:

\par 

\texttt{humsed s/[\^{}JLkK]//g} \textit{inputfile}



\par 
Sometimes we need to restrict the circumstances where the
data are eliminated.
For example, we might want to eliminate all measure numbers.
Eliminating all numbers from a \texttt{**kern} file will have
the undesirable consequence of eliminating all note durations as well.
Most
\textbf{humsed}
operations can be
\textit{preceded}
by a regular expression delineated by slashes.
This tells
\textbf{humsed}
to execute this substitution only if the data record matches
the leading regular expression.
For example, the following command eliminates measure numbers
but not note durations:

\par 

\texttt{humsed /\^{}=/sX[0-9]*XXg} \textit{inputfile}


\par 
The operation may be interpreted as follows:
look for lines that match a pattern where the first character
in the line is an equals sign;
if you find this pattern look for zero or more instances of
any number between zero and nine, and replace that by an
empty string; do this substitution for all numbers on the
current data record.


\par 
Incidentally, Humdrum provides a
\textbf{num}
command that can be used to insert numbers in data records.
The
\textbf{num}
command supports an elaborate set of options, but is not
used often, so we won't describe it here.
The following command renumbers all of the barlines in an input so
that the first measure begins with the number 72.
(Refer to the
\textit{Humdrum Reference Manual}
for details regarding \textbf{num}.)

\par 

\texttt{humsed /\^{}=/sX=[0-9]*X=Xg} \textit{inputfile}\texttt{ | num -n \^{}= -x == -p = -o 72}


\par 
Suppose we wanted to eliminate all octave numbers from a
\texttt{**pitch}
representation.
In this case we want to delete all numbers except when they
occur in conjunction with a barline.
Our substitution should occur only when the current
record does not match a leading equals sign:

\par 

\texttt{humsed /\^{}[\^{}=]/s\%[0-9]\%\%g} \textit{inputfile}



\par 
Suppose we wanted to determine which of two MIDI performances
exhibits more dynamic range -$\,$- that is, which performance has a
greater variability in key-down velocities.
Recall from
Chapter 7
that MIDI data tokens consist of three
elements separated by slashes (/).
The third element is the key velocity.
First, we want to eliminate key-up data tokens.
These tokens can be distinguished by the minus sign associated with
the second data element.
An appropriate substutition is:

\par 

\texttt{s\%[0-9][0-9]*/-[0-9][0-9]*/[0-9]* *\%\%g}


\par 
(That is, replace by nothing any data that matches the following:
a numerical digit followed by zero or more digits, followed by
a slash, followed by a minus sign, followed by a digit, followed by
zero or more digits, followed by a slash, followed by zero or
more digits, followed by zero or more spaces.)

\par 
Having isolated only the key-down data tokens, we now need to
eliminate everything but the third data element,
the MIDI key-down velocities:

\par 

\texttt{s\%[0-9][0-9]*/[0-9][0-9]*/\%\%g}



\subsection*{The \textit{stats} Command}

\par 
We can determine the range or variance of these velocity values by
piping the output to the
\textbf{stats}
command.
The
\textbf{stats}
command calculates basic statistical information for any input
consisting of a column of numbers.
A sample output from
\textbf{stats}
might appear as follows:

\texttt{n:124
total:5700
mean:45.9677
min:9
max:102
S.D.:232.37}


\par 
The value \texttt{n} indicates the total number of numerical values
found in the input;
the \texttt{total} specifies the sum of these numbers;
the \texttt{mean} identifies the average;
the \texttt{min} and \texttt{max} report the minimum and maximum
values encountered, and the \texttt{S.D.} represents the standard deviation.
The standard deviation provides a useful way of characterizing
which performance has greater variability in key-down velocities.

\par 
Assuming that the above two stream-editing substitutions are kept in a file
called \texttt{revise} we can compare the dynamic range for the
two performances as follows:

\par 

\texttt{extract -i '**MIDI' perform1 | grep -v \^{}= | humsed -r revise $\backslash$

| rid -GLId | stats}

\\
\texttt{extract -i '**MIDI' perform2 | grep -v \^{}= | humsed -r revise $\backslash$

| rid -GLId | stats}



\par 
The
\textbf{extract}
command has been added to ensure that we only process
\texttt{**MIDI}
data;
the
\textbf{grep}
command ensures that possible barlines are eliminated,
and the
\textbf{rid}
command eliminates comments and interpretations prior to passing
the data to the
\textbf{stats}
command.


\subsection*{Eliminate Everything But ...}

\par 
A common use for
\textbf{humsed}
is to eliminate signifiers that are not of interest.
Stream editors like
\textbf{sed}
and
\textbf{humsed}
can be used to dramatically simplify a representation.

\par 
Did Monteverdi use equivalent numbers of sharps and flats?
Or did he favor one accidental over the other?
A simple way to determine this is to throw away everything
but the sharps and flats.
We can generate an inventory of just sharps and flats:

\par 

\texttt{humsed 's/[\^{}\#-]//g' montev* | rid -GLId | sort | uniq -c}


\par 
In some tasks, we might wish to transform a \texttt{**kern}-format file so that
only pitch-related information is preserved:

\par 

\texttt{humsed 's/[\^{}a-gA-G\#-]//g'} \textit{inputfile}


\par 
In extreme cases, we may wish to eliminate all Humdrum data from an input.
The following command replaces all data tokens by null tokens:

\par 

\texttt{humsed 's/[\^{}     ][\^{}     ]*/./g'} \textit{inputfile}


\par 
(That is, globally substitute all instances of the string
not-a-tab followed by zero or more instances of not-a-tab
characters, by a single period character.)
This sort of command can be useful in generating a file that
maintains the
\textit{structure}
but not the
\textit{content}
of some document.
Incidentally, neither the
\textbf{sed}
nor the
\textbf{humsed}
commands support extended regular expressions, so we are not able
to use the \texttt{+} metacharacter in the above substitution.


\subsection*{Deleting Data Records}

\par 
Sometimes it is useful to delete entire data records rather than
simply eliminating certain kinds of information.
The \texttt{d} operation causes lines to be deleted.
Normally, it is preceded by a regular expression that
identifies which records should be eliminated.
Deleting barlines can be done using the following command:

\par 

\texttt{humsed /\^{}=/d} \textit{inputfile}


\par 
Note that this is functionally equivalent to:

\par 

\texttt{grep -v \^{}=} \textit{inputfile}


\par 
In the general case,
\textbf{humsed /.../d}
is preferable to \textbf{grep -v}.
Remember that
\textbf{humsed}
only manipulates Humdrum data records;
it never touches comments or interpretations.
The
\textbf{grep}
command has no such restriction.
Consider, for example, the following command to eliminate
grace notes (acciaccaturas) from a \texttt{**kern}-format file.

\par 

\texttt{humsed '/q/d'} \textit{inputfile}


\par 
By contrast, the command:

\par 

\texttt{grep -v q} \textit{inputfile}


\par 
would also eliminate any comments or interpretation records
containing the letter `q'.


\par 
Suppose that we wanted to know whether a melody still evokes
a certain key perception even if we eliminate all the tonic pitches.
First we translate the representation to scale degree
and assemble this file with the original \texttt{**kern}
representation for the melody.

\par 

\texttt{deg} \textit{input}\texttt{ > temp}
\\
\texttt{assemble} \textit{input}\texttt{ temp | humsed '/1\$/d' | midi | perform}



\par 
Of course deleting all of the tonic notes will disrupt the
original rhythm.
An alternative is to replace all tonic pitches by rests:

\par 

\texttt{deg }\textit{input}\texttt{ > temp}
\\
\texttt{assemble} \textit{input}\texttt{ temp | humsed '/1\$/s\%[A-Ga-g\#-]*\%r\%' | midi $\backslash$

| perform}




\par 
Perhaps we might want to eliminate all the pitch information,
and simply listen to the rhythmic structure of a work.
That is, we might change all of the pitches in a work to
a single pitch -$\,$- in the following case, middle C:

\par 

\texttt{humsed 's/[A-Ga-g\#-]*/c/' | midi | perform}



\subsection*{Adding Information}

\par 
The substitute command can also be used to add information
to points in a Humdrum input.
For example, we might wish to add an explicit breath-mark (\texttt{,})
to the end of each phrase in a \texttt{**kern}-format input:

\par 

\texttt{humsed s/\}/\},/g} \textit{inputfile}


\par 
Any occurrence of the ampersand (\texttt{\&}) in the replacement string
of a substitution is a standard stream-editing convention which means
"the matched string."
Suppose we want to add a tenuto mark to every quarter-note in a work.
The following substitution seeks the number `4' followed by
any character that is not a digit or period.
This pattern is replaced by itself (\&) followed by a tilde (~),
the
\texttt{**kern}
signifier for a tenuto mark:

\par 

\texttt{humsed s/4[\^{}0-9.]/\&~/g} \textit{inputfile}



\subsection*{Multiple Substitutions}

\par 
Some tasks may require more than one substitution command.
Multiple operations can be invoked by separating each
operation by a semicolon.
In the following example, we change all \texttt{**kern} quarter-notes
to eighth-note durations:

\par 

\texttt{humsed 's/4[A-Ga-g]/8\&/g; s/84/8/g'} \textit{inputfile}


\par 
The first substitution finds strings that match the number `4'
followed by an upper- or lower-case letter from A to G.
The matched string is then output preceded by the number `8'.
This operation will change all quarter notes and rests
to eighty-fourth durations.
The ensuing substitution operation changes `84' to `8'
and so completes the transformation.


\subsection*{Switching Signifiers}

\par 
In some situations, we will want to switch two or more
signifiers -$\,$- make all A's B's and all B's A's.
These sorts of tasks require three substitutions and
involve creating a unique temporary string.
For example, the following command
changes all
\texttt{**kern}
up-bows to down-bows and vice versa.

\par 

\texttt{humsed 's/u/ABC/g; s/v/u/g; s/ABC/v/g'} \textit{inputfile}


\par 
The first substitution changes down-bows (`\texttt{u}') to the
unique temporary string \texttt{ABC}.
(In the \texttt{**kern} representation \texttt{ABC} is an illegal
pitch representation, so it is bound to be a unique character string.)
The second substitution changes up-bows (\texttt{v}) to down-bows.
The third substitution changes occurrences of the temporary
string \texttt{ABC} to up-bows.


\subsection*{Executing from a File}

\par 
When several instructions are involved in stream editing,
it can be inconvenient to type multiple operations on the command line.
It is easier to place the editing instructions in a file,
and use the
\textbf{-f}
option (with either \textbf{sed} or
\textbf{humsed})
to execute
from the file.
Consider, for example, the task of rhythmic diminution,
where the durations of notes are halved.
We might create a file called \texttt{diminute} containing the
following operations:

\par 

\texttt{s/[0-9][0-9]$\backslash$*/\&XXX/g
\\
s/64XXX/128/g
\\
s/32XXX/64/g
\\
s/16XXX/32/g
\\
s/8XXX/16/g
\\
s/4XXX/8/g
\\
s/2XXX/4/g
\\
s/1XXX/2/g
\\
s/0XXX/1/g}


\par 
Each substitution command is applied (in order) to every line or
data record in the file.
The first substitution adds the unique string \texttt{XXX} to
every number.
The ensuing substitutions transform these numbers to appropriate
diminution values.
We can execute these commands as follows:

\par 

\texttt{humsed -f diminute} \textit{inputfile}



\subsection*{Writing to a File}

\par 
A useful feature of
\textbf{humsed}
is the "write" or \texttt{w} operation.
This operation causes a line to be written to the end of a specified file.
Suppose, for example, we wanted to collect all seventh chords
into a separate file called \texttt{sevenths}.
With a
\texttt{**harm}-format
input, the appropriate command would be:

\par 

\texttt{humsed '/7/w sevenths'} \textit{inputfile.hrm}


\par 
Each line containing the number 7 wll be written to a file named
\texttt{sevenths}.

\par 
Similarly, we could copy all sonorities containing pauses to
the file \texttt{pauses}.

\par 

\texttt{humsed '/;/w pauses'} \textit{inputfile}


\par 
Of course there are other ways of achieving the same goal:

\par 

\texttt{yank -m ';' 0} \textit{inputfile}\texttt{ > pauses}


\par 
Or even:

\par 

\texttt{grep ';'} \textit{inputfile}\texttt{ | grep -v '\^{}[!*]' > pauses}




\par 
In some cases, a stream editor can be used to eliminate or modify data
that will confound subsequent processing.
For example, suppose we wanted to count the number of phrases that
begin on the subdominant and the number of phrases that end on the subdominant.
The
\textbf{deg}
command will allow us to identify subdominant pitches (via the number `4').
Since we would like to maintain the phrase indicators, we will
avoid the
\textbf{-x}
option for \textbf{deg}.
However, the
\textbf{-x}
option will pass
\textit{all}
of the non-pitch related signifiers,
including the duration data which encodes numbers.
Hence, we will not be able to distinguish the subdominant (`4') pitch
from a \texttt{**kern} quarter-note (`4').
The problem is resolved by first eliminating all of the duration
information (numbers) from the original input:

\par 

\texttt{humsed 's/[0-9.]//g' input.krn | deg | egrep -c '(\{.*4)|4.*\{)'}
\\
\texttt{humsed 's/[0-9.]//g' input.krn | deg | egrep -c '(\}.*4)|4.*\})'}



\par 
In texts for vocal works, identify the number of notes per syllable.

\par 

\texttt{extract -i '**kern'} \textit{input}\texttt{ | humsed 's/X//g' > tune}
\\
\texttt{extract -i '**silbe'} \textit{input}\texttt{ | humsed 's/[a-zA-Z]*/X/' > lyrics}
\\
\texttt{assemble tune lyrics | cleave -i '**kern,**silbe' -o '**new' $\backslash$

> combined}

\texttt{context -b X -o '[r=]' combined | rid -GLId | awk '\{print NF\}'}



\par 
Identify the number of notes per word rather than per syllable.

\par 

\texttt{extract -i '**kern'} \textit{input}\texttt{ > tune}
\\
\texttt{extract -i '**silbe'} \textit{input}\texttt{ | humsed 's/\^{}[\^{}-].*[\^{}-]\$/BEGIN\_END/; s/-.*[\^{}-]\$/END/; s/\^{}[\^{}-].*-/BEGIN/' > lyrics}
\\
\texttt{assemble tune lyrics | cleave -i '**kern,**silbe' -o '**new' $\backslash$

> combined}

\texttt{context -b BEGIN -e END -o '[r=]' combined | rid -GLId $\backslash$

| awk '\{print NF\}'}




\subsection*{Reading a File as Input}

\par 
Another useful feature is the
\textbf{humsed}
"read" or \texttt{r} operation.
Whenever a leading regular expression is matched,
a file is read in at that point.
Suppose, for example, that we want to annotate a file with Humdrum
comments identifying the presence of cadential 6-4 chords.
First, we might create a file -$\,$- \texttt{comment.6-4} -$\,$-
containing the following Humdrum comment:

\par 

\texttt{!! A likely cadential 6-4 progression.}


\par 
We can use the Humdrum
\textbf{pattern}
command (to be described in
Chapter 21),
as follows:

\par 
File \texttt{template}:

\texttt{
 .*}
\\
\texttt{Ic}
\\
\texttt{\^{}$\backslash$.	*}
\\
\texttt{=	*}
\\
\texttt{V[\^{}I]}


\par 
Command:

\par 

\texttt{pattern -f template} \textit{inputfile}\texttt{ > output}
\\
\texttt{humsed 'cadential-64/r comment.6-4' output > commented.output}


\par 



\subsection*{Reprise}

\par 
The
\textbf{sed}
and
\textbf{humsed}
commands provide stream editors that can automatically edit
a data stream.
We've seen that multiple operations can be carried out,
either from the command line or from a file containing
editing instructions.
It should be noted that the
\textbf{sed}
and
\textbf{humsed}
commands provide many more editing facilities than those
discussed in this chapter.
Some 25 operations are provided by
\textbf{sed}
and \textbf{humsed}.
For example, segments of text can be stored in various buffers,
the contents of these buffers modified, and the results placed anywhere
in the output text.
Markers can be set at particular points and conditional branch
statements executed.
Stream-editing scripts have been written to execute programs
of considerable complexity.
However, for most tasks, the simple substitute (\textbf{s}) and delete (\textbf{d})
operations are the most useful.
For further information about stream editing using
\textbf{sed}, refer to the book on
\textbf{sed}
and
\textbf{awk}
written by Dale Dougherty
(listed in the bibliography).

\\
\begin{itemize}
\item 

\textbf{Next Chapter}
\item 

\textbf{Previous Chapter}
\item 

\textbf{Table of Contents}
\item 

\textbf{Detailed Contents}
\\\\

� Copyright 1999 David Huron
\end{document}
