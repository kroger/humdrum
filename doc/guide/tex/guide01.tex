% This file was converted from HTML to LaTeX with
% Tomasz Wegrzanowski's <maniek@beer.com> gnuhtml2latex program
% Version : 0.1
\documentclass{article}
\begin{document}



  
  
    
      
      
      
    
  


\section*{Chapter1}


[\textit{Contents}]
[\textit{Next Chapter}]


\section*{Humdrum:  A Brief Tour}



Humdrum is a general-purpose software system intended to assist
music researchers.
Humdrum's capabilities are quite broad, so it is difficult
to describe concisely what it can do.
This chapter provides a brief tour through Humdrum:
the goal is to give you a quick glimpse of Humdrum's capabilities
and to help sketch some of the big picture.
Don't worry if you don't understand everything in this chapter.
All of the topics mentioned here will be covered more thoroughly
in later chapters.


\subsection*{What Can Humdrum Do?}

 
Although Humdrum facilitates exploratory investigations,
it is best used when the user has a clear problem or question in mind.
For example, Humdrum allows users to pose and answer questions
such as the following:
\begin{itemize}
\item 
In the music of Stravinsky, are dissonances more common in strong
metric positions than in weak metric positions?
\item 
In Urdu folk songs, how common is the so-called "melodic arch" -$\,$-
where phrases tend to ascend and then descend in pitch?

\item 
What are the most common fret-board patterns in guitar riffs by Jimi Hendrix?
\item 
Which of the Brandenburg Concertos contains the B-A-C-H motif?
\item 
Which of two different English translations of Schubert lyrics best
preserves the vowel coloration of the original German?
\item 
Did George Gershwin tend to use more syncopation in his later works?
\item 
After the V-I progression, which harmonic progression is most
apt to employ a suspension?
\item 
How do chord voicings in barbershop quartets differ from chord voicings
in other repertories?
\item 
In what harmonic contexts does H�ndel double the leading-tone?
\end{itemize}

 
Although the Humdrum tools may take just seconds to compute an
answer for any of the above problems, only a few of these questions are
\textit{easy}
to answer using Humdrum.
The primary impediment to a quick solution is the user's skill in
interconnecting the right tools for the task at hand.
The purpose of this book is to help you develop that skill.


\subsection*{The Humdrum Syntax and the Humdrum Toolkit}

 
Humdrum consists of two distinct components:
the
\textit{Humdrum Syntax}
and the
\textit{Humdrum Toolkit.}
The
\textit{Humdrum Syntax}
is a grammar for representing information.
Over the course of this book we'll encounter a variety of representation
schemes that conform to this syntax.
The syntax itself doesn't represent anything:
it merely provides a common framework for representing all sorts
of information.
Within the syntax an endless number of representation schemes can be defined.
Theoretically, any type of sequential symbolic data may be
accommodated -$\,$- such as medieval square notation,
MIDI data,
acoustic spectra,
piano fingerings,
changes of emotional states,
Indian tabla notations,
dance steps,
Schenkerian graphs,
concert programs -$\,$- or even industrial chemical processes.

 
A number of representation schemes are pre-defined in Humdrum;
however, users are free to concoct their own task-specific representations
as necessary.
For example, if you are interested in Telugu notation or Dagomba dance,
you can design your own pertinent representation scheme within the
Humdrum syntax.
Humdrum representations may be meticulously crafted,
or they may be invented in a matter of seconds.
It is common to generate intermediate
or "throw-away" representations that are
used only for a single research task.

 
The
\textit{Humdrum Toolkit}
is a set of more than 70 inter-related software tools.
These tools manipulate ASCII data (text) conforming to
the Humdrum syntax.
If the ASCII data represents music-related information,
then we can say that the Humdrum tools manipulate music-related information.

 
What kinds of manipulations can be done with Humdrum?
The various Humdrum tools can be grouped roughly into the
following sixteen types of operations.
\begin{itemize}
\item 
\textbf{Visual display.}
E.g., display a score beginning at measure 128;
output the libretto from Act II, Scene 5;
print the string parts for the Coda -$\,$- including the Roman
numeral harmonic analysis.
\item 
\textbf{Aural display.}
E.g., play the bass trombone part slowly beginning at measure 70;
play just the opening two measures from all of the works in a given repertory.
\item 
\textbf{Searching.}
E.g., search for instances of a motive; locate any deceptive cadences;
find all of the works that are composed for a given combination
of traditional Japanese instruments.
\item 
\textbf{Counting.}
E.g., how often do augmented intervals occur in Hungarian folk songs?
What proportion of phrases do not begin with a pick-up or anacrusis?
\item 
\textbf{Editing.}
E.g., change all up-stems to down-stems in measure 88 of the second horn part.
\item 
\textbf{Editorializing.}
E.g., add an editorial footnote to a specified note;
indicate that a passage differs from the composer's autograph.
\item 
\textbf{Transforming or translating between representations.}
E.g., transpose from one key to another;
calculate the harmonic intervals between two parts;
represent a score according to scale-degrees;
reconstitute chords so they are represented in set \textit{normal form}.
\item 
\textbf{Arithmetic transformations of representations.}
E.g., calculate the semitone spacings between successive notes,
or determine points where parts cross in pitch.
\item 
\textbf{Extracting or selecting information.}
E.g., extract the second verse; exclude the development section;
isolate the third phrase;
grab the second chord in each measure;
select the brass parts; take the second endings when repeating the trio;
choose the Dresden manuscript version.
\item 
\textbf{Linking or joining information.}
E.g., assemble instrumental parts into a full score;
tag notes with their harmonic function;
coordinate heart-rate data from a listener with the musical score.
\item 
\textbf{Generating inventories.}
E.g., list all the types of embellishment (non-chordal) tones from the
most common to the least common;
what chord functions are absent from a work?
\item 
\textbf{Classifying.}
E.g., classify all chords as "open" or "closed"
position; identify all secondary dominants;
classify all intervals as either unisons, steps or leaps;
classify various piano fingerings as either easy,
moderate, difficult, or impossible.
\item 
\textbf{Labelling.}
E.g., mark musical sections; label themes; identify French,
Italian and German sixth chords;
mark appropriate words in a vocal text as either "passionate,"
"apathetic," or "neutral."
Mark sonorities as falling on either strong or weak metric positions.
\item 
\textbf{Comparison.}
E.g., determine whether the Amsterdam and Manchester manuscripts
for a work have identical pitches in the third movement;
determine whether motets by John Dunstable are more
similar to motets by Thomas Morley or by Lionel Power.
\item 
\textbf{Capturing Data.}
E.g., import live or recorded MIDI data;
import data from a notation program.
\item 
\textbf{Trouble-shooting.}
E.g., identify any transgressions of notational conventions;
check whether a score has been tampered with;
get help when you're stuck.
\end{itemize}

 
If the above functions sound vague, that's because the corresponding
Humdrum tools are similarly broad in their function.
For example, the Humdrum
\textbf{patt}
tool can be used to locate Landini
cadences, label statements of 12-tone rows, or search for
piano fingering patterns in Liszt.
All three tasks involve searching for
a sequential pattern that might have concurrent features as well.

 
When using computers in research activities,
it is important to maintain a keen awareness of their limitations.
For example, computers are typically not good at
\textit{interpreting}
data.
Because much music scholarship hinges on interpretations,
this would seem to preclude computers from being of much use.
However, there are some things that computers do well,
and when a skilled user intervenes to make a few crucial interpretations,
the resulting computer/human interaction can lead to pretty sophisticated
results -$\,$- as we will see.

 
Although computers are typically not good at interpreting data,
interpretive software will become increasingly important
as computational musicology continues to develop.
Humdrum provides a handful of tools that
\textit{interpret}
data in various ways.
For example, the
\textbf{key}
command implements the Krumhansl and Kessler method
for estimating the key of a music passage.
The
\textbf{melac}
command implements Joseph Thomassen's model of melodic accent.
Other interpretive tools characterize syncopation
or implement Johnson-Laird's model of rhythmic prototypes.

 
The names of some of the Humdrum tools will be readily recognizable
by musicians.
Humdrum tools such as
\textbf{key,}
\textbf{pitch,}
\textbf{record,}
\textbf{tacet,}
\textbf{trans}
and
\textbf{reihe}
may evoke fairly accurate ideas about what they do.
Ironically, the most recognizable tools are typically the
least useful tools in the toolkit -$\,$- because they are so specialized.
The most powerful Humdrum tools have names such as
\textbf{cleave,}
\textbf{humsed,}
\textbf{simil,}
\textbf{recode,}
\textbf{context,}
\textbf{patt}
and
\textbf{yank.}

 
By itself, each individual tool in the Humdrum Toolkit is
quite modest in its effect.
However, the tools are not intended to be self-sufficient.
They are designed to work in conjunction with each other,
and with existing standard UNIX commands.
Like musical instruments, their potential usefulness is greatly increased
when they are combined with other tools.
Musical problems are typically addressed by linking together
successive Humdrum (and UNIX) commands to form one or more command
\textit{pipelines.}
Although each individual tool may have only a modest effect,
the tools' combined capacity for solving complex problems is legion.

 
Now that we've sketched an overview of Humdrum,
we can consider in greater detail the two principal components of
Humdrum:
the \textit{Humdrum Syntax} and the \textit{Humdrum Toolkit}.


\subsection*{Humdrum Syntax}

 
Humdrum data are organized somewhat like the tables of a spread-sheet.
As with a spread-sheet, you can define or label your own types of data.
As with a spread-sheet, the different
\textit{columns}
can be set up to represent whatever type of data you like.
The \textit{rows} of data, however, have a fixed meaning -$\,$-
they represent successive moments in time;
that is, time passes as you move down the page.

 
By way of example, consider Table 1.1.
This table shows something akin to a piano-roll.
The diatonic pitches are labelled in columns from C4 (middle C) to C5.
Each row represents the passage of a sixteenth duration.
The table entries indicate whether the note is \textit{on} or \textit{off}.
The table encodes an ascending scale.
As the table stands, it's not clear whether each note has
a duration of an eighth note or whether two successive sixteenth notes
are sounded for each pitch:
\textbf{Table 1.1}


\textbf{C4D4E4F4G4A4B4C5}
1st 16thONoffoffoffoffoffoffoff
2nd 16thONoffoffoffoffoffoffoff
3rd 16thoffONoffoffoffoffoffoff
4th 16thoffONoffoffoffoffoffoff
time5th 16thoffoffONoffoffoffoffoff
6th 16thoffoffONoffoffoffoffoff
7th 16thoffoffoffONoffoffoffoff
8th 16thoffoffoffONoffoffoffoff
9th 16thoffoffoffoffONoffoffoff
10th 16thoffoffoffoffONoffoffoff
11th 16thoffoffoffoffoffONoffoff
etc.

Table 1.2 shows another example where different kinds of information
are combined in the same table.
Here the last column represents a combination of trumpet valves:
\textbf{Table 1.2}


\textbf{Pitch\textbf{Duration\textbf{Valve Combination}}}
1st noteC4quarter0
2nd noteB3eighth2
3rd noteG4eighth0
4th noteF4eighth1
5th noteG4eighth0
6th noteA4quarter1-2
7th noteG4eighth0
8th noteAb4quarter2-3

Humdrum representations can be very similar to the data
shown in Tables 1.1 and 1.2.
With just a few formatting changes, either table 
can be transformed so that it conforms to the Humdrum syntax.
For example, Table 1.3 recasts Table 1.2 so that it conforms to
the Humdrum syntax.
Just four changes have been made:
(1) the left-most column has been given a heading name so that
\textit{every}
column has a label,
(2) each column heading is preceded by two asterisks,
(3) the columns have been left-justified so successive columns
are separated by a tab, and
(4) each column has been terminated with the combination
of an asterisk and hyphen.
\textbf{Table 1.3}   A Humdrum Equivalent to Table 1.2


\texttt{**Note**Pitch**Duration**Valve Combination
1st noteC4quarter0
2nd noteB3eighth2
3rd noteG4eighth0
4th noteF4eighth1
5th noteG4eighth0
6th noteA4quarter1-2
7th noteG4eighth0
8th noteAb4quarter2-3
*-*-*-*-}

It does not matter what characters appear in the table -$\,$- numbers, letters,
symbols, etc.
(although there are some restrictions concerning the use of spaces and tabs).
The table can have as many columns as you like, and can be as long
as you like.
Unlike spreadsheet columns, Humdrum "columns" can exhibit
complicated "paths" through the document;
columns can join together, split apart, exchange positions,
stop in mid-table, or be introduced in mid-table.
Humdrum also allows subsidiary column headings that can
clarify the state of the data.
Subsidiary headings can also be added anywhere in mid-table.
Finally, Humdrum also provides ways of adding running commentaries;
comments might pertain to the whole table, to a given row or column,
to a given data cell, or to a particular item of information within a cell
(such as a single letter or digit).
Since the
\textit{columns}
in Humdrum data can roam about the table
in a semi-flexible way, they are referred to as
\textbf{spines.}
We'll see how these devices are used in later chapters.

 
The most common Humdrum files encode musical notes in the various cells of the table;
the most common use of a spine is to represent a single musical part
or instrument.

 
Some twenty or more representation schemes are pre-defined in Humdrum,
but remember that users are always free to concoct their own
representations as necessary.
As in a spread-sheet, the spines or columns can be used to represent
whatever you like.
In the following chapter, we'll look at the most commonly used of
the pre-defined Humdrum representations -$\,$- the
\textit{kern}
representation.
This representation gets its name from
the German word for \textit{core};
it is a scheme intended to represent the basic or core musical information
of notes, durations, rests, barlines, and so on.


\subsection*{Humdrum Tools}

 
The Humdrum software is not a program that you invoke like
a word-processor or notation editor.
Humdrum is not a big program that you start-up when you want
to do music research.
Instead, Humdrum provides a toolbox of
\textit{utilities}
-$\,$-  most of which can be accessed at any time from anywhere
in the system.

 
Any data that conforms to the Humdrum syntax can be manipulated using
the Humdrum software tools.
Since the tools can be interconnected with each other
(and can also be interconnected with non-Humdrum tools)
there are a lot of ways to manipulate Humdrum data.
Much of this book will deal with how the tools can be
interconnected to do musically useful things.

 
For the remainder of this chapter, we will describe a few Humdrum tools
and illustrate how they might be used.
Once again, the goal in this chapter is to give you an initial taste
of Humdrum.
Don't worry if you don't understand everything at this point.


\subsection*{Some Sample Commands}

 
One group of tools is used to extract or select sections of data.
Vertical spines of data can be extracted from a Humdrum file using the
\textbf{extract}
command.
For example, if a file encodes four musical parts, then the
\textbf{extract}
command might be used to isolate one or more given parts.
The command

 

\texttt{extract -f 1 filename}



 
will extract the first or left-most column or spine of data.
Often it is useful to extract material according to the encoded content
without regard to the position of the spine.
For example, the following command will extract all spines containing
a label indicating the tenor part(s).

 

\texttt{extract -i '*Itenor' filename}


 
Instruments can be labelled by "instrument class" and so can be
extracted accordingly.
The following command extracts all of the woodwind parts:

 

\texttt{extract -i '*ICww' filename}


 
Any vocal text can be similarly extracted:

 

\texttt{extract -i '**text' filename}



 
Or if the text is available in more than one language,
a specific language may be isolated:

 

\texttt{extract -i '*LDeutsch' filename}


 
Segments or passages of music can be extracted using the
\textbf{yank}
command.
Segments can be defined by sections, phrases, measures,
or other any user-specified marker.
For example, the following command extracts the section labelled "Trio"
from a minuet \& trio:

 

\texttt{yank -s Trio -r 1 filename}



 
Or select the material in measures 114 to 183:

 

\texttt{yank -n = -r 114-183 filename}


 
Or select the second-last phrase in the work:

 

\texttt{yank -o \{ -e \} -r '\$-1' filename}


 
Don't worry about the complex syntax for these commands;
the command formats will be discussed fully in the ensuing chapters.
For now, it is important only that you get a feel for some of
the types of operations that Humdrum users might perform.



 
Two or more commands can be connected into a
\textit{pipeline.}
The following command will let us determine whether there are any
notes in the bassoon part:

 

\texttt{extract -i '*Ifagot' filename | census -k}



 
The following pipeline connects together four commands:
it will play (using MIDI) the first and last measures from
a section marked "Coda" at half the notated tempo from a
file named \texttt{Cui}:

 

\texttt{yank -s '*>Coda' Cui | yank -o \^{}= -r 1,\$ | midi | perform -t .5}


 
Some tools translate from one representation to another.
For example, the
\textbf{mint}
command generates melodic interval information.
The following command locates all tritones -$\,$- including
compound (octave) equivalents:

 

\texttt{mint -c filename | egrep -n '((d5)|(A4))'}


 
Incidentally, Humdrum data can be processed by many common commands
that are not part of the Humdrum Toolkit.
The
\textbf{egrep}
command in the above pipeline is a common computer utility
and is not part of the Humdrum Toolkit.

 
Depending on the type of translation, the resulting data can be
searched for different things.
The following command identifies French sixth chords:

 

\texttt{solfa file | extract -i '**solfa' | ditto | grep '6-.*4+' | grep 2}


 
Locate all sonorities in the music of Machaut where the seventh
scale degree has been doubled:

 

\texttt{deg -t machaut* | grep -n '7[\^{}-+].*7'}


 
Count the number of phrases that end on the subdominant pitch:

 

\texttt{deg filename | egrep -c '(\}.*4)|(4.*\})'}



 
The following command identifies all scores
whose instrumentation includes a tuba but not a trumpet:

 

\texttt{grep -sl '!!!AIN.*tuba' * | grep -v 'tromp'}


 
Some tasks may require more than one command line.
For example, the following three-line script locates any parallel fifths
between the bass and alto voices of any input file:

 

\texttt{echo P5 > P5
echo '=	*' >> P5; echo P5 >> P5
extract -i '*Ibass,*Ialto' file | hint -c | pattern -s = P5}


 
More complicated scripts can be written to carry out more sophisticated
musical processes.
In later chapters we'll encounter some scripts that contain 10
or more lines of commands.



\subsection*{Reprise}

 
In this chapter we have seen that Humdrum consists of two parts:
(1) a representation
\textit{syntax}
that is similar to tables in a spread-sheet,
and (2) a set of utilities or
\textit{tools}
that manipulate Humdrum data
in various ways.
The tools carry out operations such as displaying, performing,
searching, counting, editing, transforming, extracting, linking,
classifying, labelling and comparing.
The tools can be linked together to carry out a wide variety of tasks.
Even one-line commands can carry out sophisticated operations.
A few lines of Humdrum can also be used to write programs of some complexity.
Users can write their own programs using the Humdrum tools,
or they can add new tools that augment the functioning of Humdrum.

\end{document}
